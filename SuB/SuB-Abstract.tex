\documentclass[dvipsnames,12pt]{article}%[trans] for handouts
\usepackage[letterpaper,margin=1in,foot=0in]{geometry}
 
 \usepackage[T1]{fontenc}
 \usepackage{textcomp}
 \usepackage{wrapfig}
 \usepackage{ragged2e}
 \usepackage{booktabs}
% \usepackage{newtxmath,newtxtext}
 \usepackage{fontspec,xltxtra,xunicode}
 \usepackage{comment}
 \defaultfontfeatures{Mapping=tex-text}
 %\setromanfont[Mapping=tex-text]{Hoefler Text}
 %\setsansfont[Scale=MatchLowercase,Mapping=tex-text]{Gill Sans}
 %\setmonofont[Scale=MatchLowercase]{Andale Mono}
 
\usepackage{wrapfig}
 \setmainfont{Cambria}
\makeatletter
\def\@xfootnote[#1]{%
	\protected@xdef\@thefnmark{#1}%
	\@footnotemark\@footnotetext}
\makeatother

 

 \pagenumbering{gobble}
 \usepackage{subcaption}
 \captionsetup{compatibility=false}
 % \usepackage{subfigure}
 \usepackage{amsthm}

 \usepackage[normalem]{ulem}
 \usepackage{amssymb}
 \usepackage{multirow}
 \usepackage{mathrsfs}
 \usepackage{pifont}
 \usepackage{ stmaryrd }
 \usepackage{mathtools}
 \usepackage{tikz}
 \usepackage{qtree}
 \usepackage{tikz-qtree}
 %\usepackage{tipa}
 \usetikzlibrary{decorations}
 \usetikzlibrary{decorations.pathreplacing}
 \usepackage{textcomp}
 \usepackage[normalem]{ulem}
 \usepackage{url}
 \usepackage[all]{xy}
 \usepackage{multicol}
 \usepackage{hanging}
 \usepackage{booktabs}
 \usepackage{setspace}
 \usetikzlibrary{shapes,backgrounds}
 \usepackage{geometry}
 
 
 
 % \newtheorem{definition}{Definition}
 %\newtheorem{theorem}{Theorem}
 
 
 \usepackage{enumerate}
 %\usepackage{gb4e} \let\eachwordone=\sl
 \usepackage{expex}	
	 \lingset{%exnoformat={\bf (X},labelformat={\bf A)},
	 	 numoffset=-.2em,labeloffset=0.1em,interpartskip=.3ex,textoffset=.4em,aboveglftskip=0.1em,belowexskip=0.1em}
 
 
 
 %\newcommand{\denote}[1]{\mbox{$[\![\mbox{#1}]\!]$}}
 \newcommand{\concat}{\mbox{$^\frown$}}
 \newcommand{\ph}{\varphi}
 \newcommand{\vsep}{\vspace{8pt}}
 \newcommand{\linesep}{\rule{6.5in}{.5pt}}
 \def\attop#1{\leavevmode\vtop{\strut\vskip-\baselineskip\vbox{#1}}}
 \newcommand{\denote}[1]{\mbox{$[\![\mbox{#1}]\!]$}}
 \newcommand{\exref}[1]{~(\ref{#1})}
 \newcommand{\glem}[1]
 {\MakeUppercase{\scriptsize{\textbf{#1}}}}
  \newcommand{\textbc}[1]
  {\MakeUppercase{\footnotesize{\textbf{#1}}}}

 
 %\input{setup}
 
 %\singlesidestandardsetup
 
 %\parindent 1em
 
 %\input{psfig-scale}
 
 
 \newcommand{\verteq}{\rotatebox{90}{$\,=$}}
 \newcommand{\equalto}[2]{\underset{\scriptstyle\overset{\mkern4mu\verteq}{#2}}{#1}}
 
 
 \newcommand{\secsep}{\hrulefill}
 % \renewcommand*{\marginfont}{\small}
 \usepackage{qtree}
 \qtreecenterfalse
 
% \renewcommand{\baselinestretch}{1.2} 
 \usepackage{xcolor}
 \definecolor{dgreen}{rgb}{0.,0.6,0.}
 
 \newcommand{\la}{\langle}
 \newcommand{\ra}{\rangle}
 \newcommand{\lamda}{\lambda}
 \definecolor{ochre}{cmyk}{0, .42, .83, .20}
 \definecolor{forest}{cmyk}{.67, .23, .67, .18}
 \definecolor{maroon}{cmyk}{0, 1, .07, .5}
 \definecolor{peri}{cmyk}{.2,.2,0,0}
  \definecolor{plum}{cmyk}{.48,.85,.29,.2}
 \usepackage{framed}

%\lingset{textoffset=.0em}
 \usepackage{enumitem}
 \gathertags
 
 \usepackage[nonumberlist]{glossaries}
	 \newglossary*{gloss}{List of abbreviations}

 % abbreviations:
 \newglossaryentry{I}{
 	name = \textbf{\textcolor{blue}{I}} ,
 	description = \textsc{primary},
 	type=gloss,	}
  \newglossaryentry{II}{
 	name = \textbf{\textcolor{ochre}{II}} ,
 	description = \textsc{primary},
 	type=gloss,	}
  \newglossaryentry{III}{
 	name = \textbf{\textcolor{forest}{III}} ,
 	description = \textsc{primary},
 	type=gloss,	}
  \newglossaryentry{IV}{
 	name = \textbf{\textcolor{violet}{IV}} ,
 	description = \textsc{primary},
 	type=gloss,	}
 
 

\begin{document}
	\begin{center}
		\textbf{{\large  Negation, reality status \& the Yolŋu verbal paradigm:}\\Towards a formal account of the porousness of tense and modality}\end{center}
	
	
	\vspace{0.1em}
\noindent	This work presents the first formal analysis of negation-based asymmetries in reality-status marking across Yolŋu (Australian, Pama-Nyungan) verbal paradigms, based on original fieldwork on two Yolŋu varieties. I show that innovative language varieties have recruited a verbal paradigm---previously used for encoding tense distinctions---reanalysing it as a series of polarity-sensitive mood/reality-status inflections.

%On the basis of newly collected field data from two Yolŋu varieties, this work presents a first formal analysis of negation-based asymmetries in reality-status marking (attested in the functional and typological literatures, see Miestamo 2005:96-109) in view of understanding meaning variation \& change across Yolŋu verbal paradigms.


\noindent\textbf{Yolŋu Matha }is a Pama-Nyungan language family spoken in Northeastern Arnhem Land (Northern Australia), encircled by a number of genetically unrelated languages (non-Pama-Nyungan). Within Yolŋu, there is a striking amount of variation in the grammatical expression of TMA categories. I compare the relationship between negation and temporal expression in two closely-related varieties (Djambarrpuyŋu \texttt{[djr]} and Wägilak \texttt{[rit]}) in view of developing a theory of variation -- contemporary and historical -- in Yolŋu Matha. All verb stems in both of these languages inflect for four cognate TMA categories (numbered \textbf{I-IV} following Wilkinson 1991, Lowe n.d.) The distribution of these inflections differs significantly between the two languages: while Wägilak draws a three-way tense distinction, Djambarrpuyŋu exhibits ``cyclic tense'': a typologically extremely rare system of temporal marking reported only in a small number of languages in Western Arnhem Land (cf. Comrie 1985). The figure below schematises the ``discontinuous'' temporal intervals that license the \textbf{I} and \textbf{III} inflections in Djambarrpuyŋu (where $ t* $ marks \textit{now} -- the time of utterance which partitions time into a \textsc{past} and \textsc{future}).

\begin{wrapfigure}{l}{.6\linewidth}
\begin{tikzpicture}[scale=.8]
% draw horizontal line   
	\draw[<->, line width=.5mm] (0,0) -- (12,0);
%\draw[<->, line width=.5mm] (0,0) -- (11.5,0);	
%draw rex
\shade[left color=blue!15!white, right color=green!15!white] (0,0.02) rectangle (4.8,1.5);
%	\fill[green!10!white] (2.5,0.02) rectangle (4.8,1.5);
\fill[blue!10!white] (4.8,0.02) rectangle (6.8,1.5);
\fill[green!10!white] (6.8,0.02) rectangle (9.5,1.5);
	\fill[orange!10!white] (9.5,0.02) rectangle (12,1.5);

% draw nodes
\draw (1.25,0) node[below=3pt] {\textbf{}} node[above=10pt] {\textsc{\textbf{III}}};
\draw (3.675,0) node[below=3pt] {\textbf{}} node[above=10pt] {\textbf{I}};
\draw (5,0)   node[circle,fill,label=below:$\big\lfloor{\sl today}$] {} node[below=3pt] {\textbf{}} node[above=3pt] {};
\draw (6.8,0) node[diamond,shade,outer color=black, inner color  = ochre,label=below:$\boldsymbol{t*}$] {} node[below=3pt] {\textbf{}} node[above=3pt] {\textsc{}};
\draw (5.8,0) node[below=3pt] {\textbf{}} node[above=10pt] {\textsc{\textbf{III}}};	
\draw (8.15,0) node[below=3pt] {\textbf{}} node[above=10pt] {\textsc{\textbf{I}}};	
	\draw (10.75,0) node[below=3pt] {\textbf{}} node[above=10pt] {\textsc{\textbf{II}}};	
\draw (9.5,0)   node[circle,fill,label=below:${\sl today}\big\rfloor$] {} node[below=3pt] {\textbf{}} node[above=3pt] {};


%%%braces

%\draw [decorate,decoration={brace,amplitude=4pt},xshift=-0pt,yshift=35pt]
%(0.5,0.5) -- (4.5,0.5) node [black,midway,yshift=0.35cm] 
%{\footnotesize metricality};
%
%\draw [decorate,decoration={brace,amplitude=4pt},xshift=-0pt,yshift=40pt]
%(3.5,0.5) -- (9,0.5) node [black,midway,yshift=0.35cm] 
%{\footnotesize cyclicity};

\end{tikzpicture}
\end{wrapfigure} 


Djambarrpuyŋu past-tensed predications are marked with \textbf{III} unless reference time is \textbf{before today} and judged to be in the \textbf{recent past}, in which case they receive the \textbf{I}-inflection (obligatory for the same-day future.)


\begin{wraptable}[6]{r}[-2em]{5em}\vspace{-5ex}
	\begin{tabular}{ccc}
		&\multicolumn{2}{c}{\textsc{\textbf{inflections}}} \\
		& \textsc{--neg} & \textsc{+neg}\\\midrule
		&	\gls{I} & \multirow{2}{*}{\gls{II}}\\
		& \gls{II} \\\midrule
		&	\gls{III} & \multirow{2}{*}{\gls{IV}}\\
		& \gls{IV} \\\bottomrule
	\end{tabular}
\end{wraptable}\noindent\textbf{\textit{Negative asymmetry.}} In a number of Arnhem languages, clausal negation \textit{triggers the appearance of irrealis}-type modal markings. This morphosemantic asymmetry is exhibited in Djambarrpuyŋu; only two of the four inflectional classes that are available in positive clauses are also available in negative clauses. These distributional facts are exhibited below in (1‑4) and summarised in the table to the right.\\%[-.4em]



%\vspace{-.2em}



\textbf{Djambarrpuyŋu (positive)}\hspace{.22\textwidth}\textbf{Djambarrpuyŋu (negative)}\vspace{.4em}
	\pex~[glwidth=.48\textwidth]\a\begingl
	\gla goḏarr ŋarra dhu \textbf{nhäŋu} mukulnha\deftaglabel{djr-fut}//
	\glb tomorrow 1s \textsc{fut} see.\gls{II} aunt.\textsc{acc}//
	\glft`I'll see my aunt tomorrow.'//\endgl b.\hspace{.3em}\begingl
	\gla bäyŋu ŋarra dhu \textbf{nhäŋu} mukulnha//
	\glb \textsc{neg} 1s \textsc{fut} see.\gls{II} aunt.\textsc{acc}//
	\glft`I won't see my aunt tomorrow.'//\endgl\xe
	
	\pex~[glwidth=.48\textwidth]\a\begingl\gla ŋarra ga \textbf{nhäma} mukulnha//
	\glb  1s \textsc{ipfv.\gls{I}} see.\gls{I} aunt-\textsc{acc} now\deftaglabel{djr-prs}//
	\glft`I see my aunt (right now).'//\endgl b.\hspace{.3em}\begingl\gla bäyŋu ŋarra gi \textbf{nhäŋu} mukulnha//
	\glb \textsc{neg} 1s \textsc{ipfv}.\gls{II} see.\gls{II} aunt.\textsc{acc}//
	\glft`I don't see my aunt (right now).'//\endgl\xe


\pex~[glwidth=.48\textwidth  ]\a\begingl\gla  ŋarra \textbf{nhäŋal} mukulnha gäthur\deftaglabel{djr-pst}//
\glb 1s see.\gls{III} aunt-\textsc{acc} today//
\glft`I didn't see my aunt this morning.'//\endgl b.\hspace{.3em}\begingl\gla bäyŋu ŋarra \textbf{nhänha} mukulnha gäthur//
\glb \textsc{neg} 1s see.\gls{IV} aunt.\textsc{acc} today//
\glft`I didn't see my aunt this morning.'//\endgl\xe

		
\pex~[glwidth=.48\textwidth,belowexskip=.5em  ]\a\begingl\gla ŋarra \textbf{nhäma} mukulnha barpuru\deftaglabel{djr-pst}//
\glb 1s see.\gls{I} aunt-\textsc{acc} yesterday//
\glft`I didn't see my aunt yesterday.'//\endgl b.\hspace{.3em}\begingl\gla bäyŋu ŋarra \textbf{nhäŋu} mukulnha barpuru//
\glb \textsc{neg} 1s see.\gls{II} aunt.\textsc{acc} yesterday//
\glft`I didn't see my aunt yesterday.'//\endgl\xe

\noindent\textbf{The status of negation.} As a consequence of Djambarrpuyŋu's grouping of negation with other irrealis modalities (only the future marker \textit{dhu} shown here, others include circumstantial modal \textit{balaŋ} and conditional subordinator \textit{ŋuli}), I argue that the erstwhile future category (\textbf{II}) has been reanalysed as a species of \textsc{irrealis} marker (cf. De Haan 2012). Crucially, this diverges from the Wägilak data (\getref{wag}), where the verbal inflection encodes a three-way tense distinction, insensitive to clausal polarity. Djambarrpuyŋu, along with neighbouring varieties, exists in a state of stable language contact with the non-Pama-Nyungan languages of Western Arnhem Land. An apparent Sprachbund, these languages exhibit the \textsc{negative asymmetry}, along with a number of other features (incl. `cyclic' tense and aspectual auxiliation, not analysed in this presentation, see Waters 1989:275-89) that are absent in the (conservative) Yolŋu varieties of the east and south (incl. Wägilak). Consequently, this contact situation has led to the wholesale restructuring of Djambarrpuyŋu's conventions of TMA expression.

\vspace{-1em}
\begin{wrapfigure}{l}{.48\linewidth}
\begin{minipage}{\linewidth}\vspace{-1em}
\pex[numoffset=-.8em,labeloffset=-.7em]\textbf{Wägilak}\deftagex{wag}\a\begingl\gla goḏarr ŋarra \textbf{nhäŋu}-('ma') mukulnha//
\glb tomorrow 1s see.\textsc{\gls{II}}-\textsc{neg} aunt.\textsc{acc}//
\glft`I will (not) see my aunt tomorrow.'//\endgl

\a\begingl\gla \textbf{nhäma}(-'ma') rra yakuthi mukulnha//
\glb see.\textbf{\gls{I}}-(\textsc{neg}) 1s now aunt.\textsc{acc}//
\glft`I'm (not) looking at my aunt currently.'//\endgl

\a\begingl\gla gätha ŋarra \textbf{nhäwala}-('ma') mukulnha//
\glb today 1s see.\textbf{\gls{III}}-(\textsc{neg}) aunt.\textsc{acc}//
\glft`I saw (didn't see) my aunt this morning.'//\endgl
\xe\end{minipage}\end{wrapfigure}

\noindent 

\noindent\textbf{Analysis.}  I examine the mechanisms by which lexical material, cognate with the tense-prominent inflectional system of other Yolŋu languages (and, by hypothesis, proto-Yolŋu, cf. Bowern 2009), has been recruited by the speakers of these varieties to encode a polarity-sensitive modal distinction. %These findings evince the status of both negation and futurity as irrealis categories.

Functional explanations for the \textsc{negative asymmetry} generally emphasise the fact that negated predicates `[belong] to the realm of the non-realized', a domain that is associated with irrealis marking (Miestamo 2005: 225). This is also the realm to which future predications belong, providing a likely pathway along which the proto-Yolŋu future tense category has seen its distribution expand (generalise) in Djambarrpuyŋu. I relate this to formal treatments of natural language negation as a modal (necessity) operator (e.g. Ripley 2009, Wansing 2001). A modal treatment of negation can predict the emergence of a grammar where negation beh\-aves morphosyntactically similarly to other modal particles, all encoding different `flavours' of irrealis marking. Consequently, The analysis defended here treats \textbf{II} and \textbf{IV} as encoding temporal information in addition to a presupposition of \textit{nonfactivity} (see also Krifka's 2016 treatment of (antifactive) negative categories in Daakie (Vanuatuan)).

\noindent\textbf{Consequences.} Bhat's (1999) influential study of `the prominence of tense, mood and aspect' seeks to devise a linguistic typology based on the nature of the contrasts made in the morphosyntax of different languages (i.e. tense-/mood-/aspect-prominence). % degree to which the expression of these categories is privileged in the morphosyntax of different languages.%of these three categories' and examines the permeable boundaries between them.% He consequently assumes a heuristic where languages can be categorised as ``tense-'', ``aspect-'' or ``mood-prominent'', depending on how they have grammaticalised these features (1999:8-9). %He further notes cross-linguistic correlations between these categories and other features exponed in a given grammar. 
Given the variation in TMA realisation that forms the basis for Bhat's typological work, a natural prediction is that, diachronically, languages will be able to `shift between' these categories. Work on semantic change has indeed demonstrated the existence of ``grammaticalisation pathways'' between these categories (\textit{e.g.} \textsc{perf}~$ \to $~\textsc{pst}; %(Schaden 2009, Condoravdi \& Deo 2014)
root modalities~$ \to $~\textsc{fut}; see Bybee \textit{et al.} 1991). Consequently, formal theories of TMA ought to be able to shed light on the possibility for reanalysis of these forms; they ought to \textit{predict} the porousness of tense, mood and aspect \textit{qua} grammatical categories (cf. Boneh 2016). The current study of the emergence of mood-based semantics for verbal inflections in Yolŋu contributes to this enterprise, given evidence that Djambarrpuyŋu has ``repurposed'' Yolŋu tense marking as providing modal information.
As a result, this work has additional consequences for our conceptions of modal expression in natural language as well as the relation that negation and future-tense operators may bear to modal displacement more broadly.



  %Given that \textsc{cyclic tense} and the \textsc{negative asymmetry} are features of the non-Pama-Nyungan languages of Western Arnhem Land, it is likely that the Westernmost Yolŋu varieties (incl. Djambarrpuyŋu) have innovated these features as a result of sustained contact with these languages (cf. Waters 1989).% While, like other non-Western varieties, the Wägilak sentence data are suggestive of a Bhatian tense-prominent system, Djambarrpuyŋu resists classification.

%Whereas \textbf{I}/\textbf{III} appear to encode \textsc{realis} categories, \textbf{II}/ \textbf{IV} encode \textsc{irrealis}. How have speakers of these varieties have reanalysed the semantics of tense inflections to encode this complex system?


\vfill

\noindent\footnotesize\textbf{Sel. References.} Bhat (1999) \textit{Prominence of Tense, Mood \& Aspect} • Boneh (2016) On becoming a tense prominent system. \textit{FoDS1} • Bowern (2009) Conjugation class stability. \textit{ICHL19} • Bybee, Perkins \& Pacuglia (1991) \textit{Evolution of Grammar} • Comrie (1985). \textit{Tense} • de Haan (2012) Irrealis: fact or fiction? \textit{Lang. Sci. 34} • Heath (1981) \textit{Ritharrŋu} • Krifka (2016) Realis \& nonrealis modalities in Daakie. \textit{SALT26} • Lowe (n.d.) \textit{Grammar lessons in Gupapuyŋu} • Ripley (2009) \textit{Negation in natural language} • Waters (1989) \textit{Djinang \& Djinba} • Wilkinson (1991) \textit{Djambarrpuyŋu}.



\end{document}

