
\synctex=1

\documentclass[11pt,dvipsnames]{article}
\usepackage[lmargin=55pt,rmargin=55pt,tmargin=70pt,bmargin=70pt,marginparwidth=110pt,marginparsep=5pt,a4paper]{geometry}
\usepackage{amssymb}
\usepackage{hyperref}
%\usepackage[tiny,compact]{titlesec}
\usepackage{graphicx}
\graphicspath{ {figs/} }

\usepackage{Sanremo,lettrine}
\usepackage{booktabs}
\usepackage{abstract}
\renewcommand{\abstractname}{}    % clear the title
\renewcommand{\absnamepos}{empty} % originally center



\usepackage{wrapfig}
\usepackage{textcomp}
\usepackage{bold-extra}
\usepackage{tikz}
\usepackage{qtree}
\usepackage{tikz-qtree}
\usepackage{expex}
				
				
 \usepackage[nonumberlist]{glossaries}
\newglossary*{gloss}{List of abbreviations}

% abbreviations:
\newglossaryentry{I}{
	name = \textbf{\textcolor{blue}{I}} ,
	description = \textsc{primary},
	type=gloss,	}
\newglossaryentry{II}{
	name = \textbf{\textcolor{ochre}{II}} ,
	description = \textsc{primary},
	type=gloss,	}
\newglossaryentry{III}{
	name = \textbf{\textcolor{forest}{III}} ,
	description = \textsc{primary},
	type=gloss,	}
\newglossaryentry{IV}{
	name = \textbf{\textcolor{violet}{IV}} ,
	description = \textsc{primary},
	type=gloss,	}

	\newglossaryentry{mod}{
	name = \textsc{mod} ,
	description = modal operator,
	type=gloss,	} 
\newglossaryentry{refl}{
	name = \textsc{r/r} ,
	description = reflexive-reciprocal marker,
	type=gloss,	} 
\newglossaryentry{pl}{
	name = \textsc{pl} ,
	description = plural,
	type=gloss,	} 


\newglossaryentry{irr}{
	name = \textsc{irr} ,
	description = irrealis (modality)marker,
	type=gloss,	} 
\newglossaryentry{comp}{
	name = \textsc{comp} ,
	description = complementiser,
	type=gloss,	} 
\newglossaryentry{malk}{
	name = \textsc{\textit{mälk}} ,
	description = Skin name (cultural `subsection'),
	type=gloss,	} 
\newglossaryentry{cplv}{
	name = \textsc{cplv} ,
	description = completive aspect,
	type=gloss,	} 
\newglossaryentry{priv}{
	name = \textsc{priv} ,
	description = privative case , 
	type = gloss , }

\newglossaryentry{sn}{
	name = SN ,
	description = standard negation/negator, 
	type = gloss , }

\newglossaryentry{indef}{
	name = \textsc{indef} ,
	description = indefinite indexical \citet[280]{Wilkinson1991}, 
	type = gloss , }

%		\newglossaryentry{obls}{
%		name = \textsc{obls} ,
%		description = standard negation/negator, 
%		type = gloss , }



\newglossaryentry{NP}{
	name = \textsc{NP} ,
	description = noun phrase,
	type = gloss , }

\newglossaryentry{TFA}{
	name = \textsc{tfa} ,
	description = temporal frame adverbial,
	type = gloss , }

\newglossaryentry{proh}{
	name = \textsc{proh} ,
	description = prohibitive,
	type = gloss , }
\newglossaryentry{recip}{
	name = \textsc{recip} ,
	description = reciprocal,
	type = gloss , }

\newglossaryentry{perf}{
	name = \textsc{perf} ,
	description = perfect aspect,
	type = gloss , }
\newglossaryentry{neg}{
	name = \textsc{neg} ,
	description = negator,
	type = gloss , }
\newglossaryentry{all}{
	name = \textsc{all} ,
	description = allative case,
	type = gloss , }
\newglossaryentry{abl}{
	name = \textsc{abl} ,
	description = ablative case,
	type = gloss , }


\newglossaryentry{dp}{
	name = \textsc{dp} ,
	description = discourse particle,
	type = gloss , }

\newglossaryentry{acc}{
	name = \textsc{acc} ,
	description = accusative case,
	type = gloss , }
\newglossaryentry{nom}{
	name = \textsc{nom} ,
	description = nominative case,
	type = gloss , }
\newglossaryentry{mvtawy}{
	name = \textsc{mvtawy} ,
	description = `movement away' \citep{Wilkinson1991} -- perhaps \textsc{vend} or sth?,
	type = gloss , }

\newglossaryentry{add}{
	name = \textsc{add} ,
	description = additive particle,
	type = gloss , }

\newglossaryentry{neu}{
	name = \textsc{neu} ,
	description = ``neutral'' verbal inflection \citep{Kabisch-Lindenlaub2017,McLellan1992},
	type = gloss , }

\newglossaryentry{mvttwd}{
	name = \textsc{mvttwd} ,
	description = `movement toward' \citep{Wilkinson1991} -- perhaps \textsc{itv} or sth?,
	type = gloss , }

\newglossaryentry{cfact}{
	name = \textsc{cfact} ,
	description = counterfactual,
	type = gloss , }
\newglossaryentry{pres}{
	name = \textsc{pres} ,
	description = present tense,
	type = gloss , }
\newglossaryentry{erg}{
	name = \textsc{erg} ,
	description = ergative case,
	type = gloss , }
\newglossaryentry{dm}{
	name = \textsc{dm} ,
	description = ``discourse clitic'' \citep{McLellan1992},
	type = gloss , }
\newglossaryentry{intens}{
	name = \textsc{intens} ,
	description = intensifier,
	type = gloss , }
\newglossaryentry{negex}{
	name = \textsc{negex} ,
	description = negative existential/quantifier,
	type = gloss , }
\newglossaryentry{red}{
	name = \textsc{redup} ,
	description = reduplicant,
	type = gloss , }
\newglossaryentry{negq}{
	name = \textsc{negq} ,
	description = negative quantifier (existential) $\nexists$,
	type = gloss , }

\newglossaryentry{comit}{
	name = \textsc{comit} ,
	description = comitative case,
	type = gloss , }

\newglossaryentry{vblzr}{
	name = \textsc{vblzr} ,
	description = `\textit{-Thu-} verbalizer' (derivational suffix),
	type = gloss , }

\newglossaryentry{instr}{
	name = \textsc{instr} ,
	description = instrumental case,
	type = gloss , }
\newglossaryentry{temp}{
	name = \textsc{temp} ,
	description = temporal case (see \citealt[585]{Wilkinson1991}),
	type = gloss , }
\newglossaryentry{prop}{
	name = \textsc{prop} ,
	description = proprietive case,
	type = gloss , }

\newglossaryentry{kinprop}{
	name = \textsc{prop} ,
	description = proprietive case -- kinship augment,
	type = gloss , }

\newglossaryentry{perl}{
	name = \textsc{perl} ,
	description = perlative case,
	type = gloss , }

\newglossaryentry{inch}{
	name = \textsc{inch} ,
	description = inchoative,
	type = gloss , }
\newglossaryentry{seq}{
	name = \textsc{seq} ,
	description = sequential,
	type = gloss , }
\newglossaryentry{abs}{
	name = \textsc{abs} ,
	description = absolutive case,
	type = gloss , }

\newglossaryentry{prom}{
	name = \textsc{prom} ,
	description = prominence marker ($\approx$ focus),
	type = gloss , }
\newglossaryentry{nmlzr}{
	name = \textsc{nmlzr} ,
	description = nominaliser (derivation),
	type = gloss , }

\newglossaryentry{tr}{
	name = \textsc{tr} ,
	description = transitiviser (derivation),
	type = gloss , }

\newglossaryentry{tfa}{
	name = \textsc{tfa} ,
	description = temporal frame adverbial,
	type = gloss , }


\newglossaryentry{emph}{
	name = \textsc{emph} ,
	description = (em)phatic particle \textcolor{red}{Wilk91},
	type = gloss , }

\newglossaryentry{ana}{
	name = \textsc{ana} ,
	description = ``anaphoric reference'' \citet{McLellan1992};\citet[248]{Wilkinson1991},
	type = gloss , }

\newglossaryentry{hab}{
	name = \textsc{hab} ,
	description = ``habitual (aspect)'',
	type = gloss , }

\newglossaryentry{caus}{
	name = \textsc{caus} ,
	description = causative,
	type = gloss , }

\newglossaryentry{foc}{
	name = \textsc{foc} ,
	description = focus marker
	type = gloss , }
\newglossaryentry{loc}{
	name = \textsc{loc} ,
	description = locative case,
	type = gloss , }
\newglossaryentry{per}{
	name = \textsc{per} ,
	description = pergressive case,
	type = gloss , }
\newglossaryentry{excl}{
	name = \textsc{excl} ,
	description = exclusive (1ns-pronoun),
	type = gloss , }
\newglossaryentry{pst}{
	name = \textsc{pst} ,
	description = past tense,
	type = gloss , }
\newglossaryentry{incl}{
	name = \textsc{incl} ,
	description = inclusive (1ns-pronoun),
	type = gloss , }
\newglossaryentry{dist}{
	name = \textsc{dist} ,
	description = distal (demonstrative),
	type = gloss , }
\newglossaryentry{prox}{
	name = \textsc{prox} ,
	description = proximal (demonstrative),
	type = gloss , }
\newglossaryentry{med}{
	name = \textsc{med} ,
	description = medial (demonstrative),
	type = gloss , }
\newglossaryentry{texd}{
	name = \textsc{endo} ,
	description = endophoric  demonstrative (Wilkinson's ``textual deictic'' \citeyear[e.g. 254]{Wilkinson1991}),
	type = gloss , }
\newglossaryentry{obl}{
	name = \textsc{obl} ,
	description = oblique case,
	type = gloss , }
\newglossaryentry{dat}{
	name = \textsc{dat} ,
	description = dative case,
	type = gloss , }
\newglossaryentry{ds}{
	name = \textsc{ds} ,
	description = different subject (subordinate clause),
	type = gloss , }

\newglossaryentry{ipfv}{
	name = \textsc{ipfv} ,
	description = imperfective (aspect),
	type = gloss , }
\newglossaryentry{imp}{
	name = \textsc{imp} ,
	description = imperative,
	type = gloss , }

\newglossaryentry{pfv}{
	name = \textsc{pfv} ,
	description = perfective (aspect),
	type = gloss , }

\newglossaryentry{fut}{
	name = \textsc{fut} ,
	description = future (tense),
	type = gloss , }

\newglossaryentry{assoc}{
	name = \textsc{assoc} ,
	description = associative,
	type = gloss , }


\newglossaryentry{hyp}{
	name = \textsc{hyp} ,
	description = hypothetical (modality),
	type = gloss , }


\usetikzlibrary{positioning,decorations.pathmorphing,arrows.meta,decorations.text,decorations.pathreplacing}
\tikzset{snake it/.style={decorate, decoration=snake}}
\usetikzlibrary{calc, shapes, backgrounds,angles,quotes,tikzmark}
\usepackage{afterpage}
\usepackage{verbatim}
\usepackage{array}
\usepackage{multirow}
%\usepackage{hanging}
\usepackage{supertabular}
\newcommand{\specialcell}[2][c]{%
	\begin{tabular}[#1]{@{}c@{}}#2\end{tabular}}
\usepackage{mathtools}
\usepackage[all]{xy}
\usepackage{ot-tableau}

\usepackage{paralist} 
\usepackage[labelsep=period,labelfont=bf]{caption}
\usepackage{subcaption}
\usepackage{fancyhdr} 
\usepackage{sectsty}
%\allsectionsfont{\sffamily\mdseries\upshape} 
\usepackage{float}
\usepackage[nottoc,notlof,notlot]{tocbibind} 
\usepackage[titles,subfigure]{tocloft} 
\usepackage{setspace}
%\usepackage[colorinlistoftodos]{todonotes}
\usepackage{xcolor}

\definecolor{blech}{rgb}{.78,.78.,.62}
\definecolor{ochre}{cmyk}{0, .42, .83, .20}
\definecolor{shadecolor}{cmyk}{.08,.08,.1,.12}
\definecolor{forest}{cmyk}{.57, .13, .57, .08}
%\usepackage[explicit]{titlesec}
%\usepackage{type1cm}
%\usepackage{xcolor}

\usepackage{xltxtra} % Loads fontspec, xunicode, metalogo, fxltx2e, and some extra customizations for XeLaTeX
%\defaultfontfeatures{Mapping=tex-text} % to support TeX conventions like ``---''
\usepackage{pifont}
\defaultfontfeatures{Mapping=tex-text}
\setmainfont{Cambria}
\usepackage{soul}

\usepackage[sort]{natbib}
\bibliographystyle{apa}
\bibpunct[:]{(}{)}{,}{a}{}{,}

%\usepackage{gb4e} \let\eachwordone=\it %\let\eachwordthree=\sf



\pagestyle{fancy}
\fancyhf{}
\rhead{\footnotesize %Josh Phillips
	\hspace{2cm}\textbf{\thepage}}
\rfoot{}


%\RequirePackage{expex}
%\makeatletter
%\def\everyfootnote{%
%	\keepexcntlocal
%	\excnt=1
%	\lingset{exskip=1ex,exnotype=roman,sampleexno=,
%		labeltype=alpha,labelanchor=numright,labeloffset=.6em,
%		textoffset=.6em}
%}
%\renewcommand{\@makefntext}[1]{%
%	\everyfootnote
%	\parindent=1em
%	\noindent
%	\footnotemark\enspace #1%
%}
%\resetatcatcode
%	
%	\makeatletter

\def\@xfootnote[#1]{%
	
	\protected@xdef\@thefnmark{#1}%
	\@footnotemark\@footnotetext
	\makeatother
}
\resetatcatcode






\renewcommand{\headrulewidth}{0pt} 
\newcommand{\rowgroup}[1]{\hspace{-1em}#1}
\usepackage{stmaryrd}
\newcommand{\denote}[1]{\mbox{$[\![\mbox{#1}]\!]$}}
\newcommand{\denotn}[1]{\mbox{\llbracket\mbox{#1}\rrbracket}}

\newcommand{\mcom}[1]
{\marginpar{\color{black}\raggedleft\raggedright\hspace{0pt}\linespread{0.9}\footnotesize{#1}}}
\newcommand{\cb}[1]
{\marginpar{\color{orange}\raggedleft\raggedright\hspace{0pt}\linespread{0.9}\footnotesize{#1}}}
\newcommand{\hk}[1]
{\marginpar{\color{purple}\raggedleft\raggedright\hspace{0pt}\linespread{0.9}\footnotesize{#1}}}
\newcommand{\note}[1]{{ }\mcom{Note}\textbf{#1}}


\newcommand{\glem}[1]
{\MakeUppercase{\scriptsize{\textbf{#1}}}}

\newcommand{\exem}[1]
{\textit{\textbf{#1}}}

\newcommand{\xmark}{\ding{55}}

%\newcommand{\gls}{\textsc}

\usepackage{framed}
\usepackage{wrapfig}
\usepackage{enumitem}
\begin{document}
\noindent\textbf{{Dissertation Committee Meeting}\hfill 16 September 2019}\\
\textit{At the intersection of temporal and modal expression}\\


\subsubsection*{agenda}
\begin{itemize}
	\item committee composition
	\item schedule to completion
	\item the current state of the dissertation
	
\end{itemize}


\subsection*{schedule to completion}


\begin{description}[labelwidth=3cm,align=right, leftmargin=3.2cm]
{\color {gray!95}\item[4 Dec 2017] Prospectus defense
\item[Winter 2017] LSA \textit{otherwise}
\item[Spring 2018] Fieldwork grant applications\\Field planning\\Oxford Handbook ch.\\LSP volume ch.\\Yolŋu studies (2)
\item[Summer 2018] NT fieldtrip (I -- 6w)
\item[Fall 2018] SLE \textit{bambai}\\
FoDS \textit{bambai; NEC}\\
NELS \textit{otherwise}\\
\textsc{\textbf{chapter} -- lit review}: tense(lessness) \& intensionality (comprehensive bullet points)\\
\textit{Job applications}
%\item[Winter 2018] 
\item[Spring 2019] \textsc{\textbf{chapter --} description }of the `facts' of TMA (and related categories) in \texttt{djr}\\Field planning\\NT fieldtrip (II -- 9w)
\item[Summer 2019] NELS proceedings paper (\textit{otherwise})\\
Manuscript \textit{(otherwise)}
}
\item[Fall 2019] \textsc{\textbf{chapters} -- analysis} Yolŋu intensionality:
\begin{itemize}[nosep,topsep=1pt]
	\item \textcolor{gray!97}{Elements of aspect \& present-reference}
	\item Cyclicity
	\item Reality status \& negation
\end{itemize}
NELS (\textit{NEC})\\FoDS (\textit{Yolŋu irrealis})\\
\textsc{\textbf{Chapter} -- } comparison \& diachrony \\
Yolŋu studies  (3)\\
\textit{Job applications}
\item[Winter 2019] LSA (\textit{NEC}; grammaticalisation workshop)
\item[Spring 2020] \textsc{\textbf{Chapters}: (less) contentful}
\begin{itemize}[nosep]
	\item Lang. Bkgrd
	\item Intro/Conclusion
	\item Kriol -- \textit{bambai} (????)
\end{itemize} 
\item[16 Mar 2020] Dissertation submission deadline
\end{description}



\subsection*{the dissertation}

\begin{abstract}
	\noindent	Among its aims, (the presently relevant component of) my dissertation seeks to understand:
	
	\begin{itemize}
		\item \textbf{The proper semantics for (\textit{sc.} meaning contribution of) Yolŋu inflectional categories \&}
		\item \textbf{How temporal relations are encoded and understood in Yolŋu.}
	\end{itemize}
\end{abstract}






\subsection*{Temporal expression}

\begin{minipage}{.65\textwidth}
	
	
	\begin{itemize}
		\item \textbf{Temporal remoteness} (``metrical/graded'' tense) has received a number of treatments in the recent literature \citep[e.g][]{Klecha2016,Cable2013,Bohnemeyer2018}.
		\begin{itemize}
			\item Grammaticalisation of finegrained markers of temporal location  (\citealp[84]{Comrie1983},\citealp{Dahl1983})
			\item It is likely not a unified phenomenon semantically; authors show different ways in which their object languages encode temporal remoteness.
			\item All provide evidence for an unmarked tense marker which is blocked in particular situations by \textsc{MaxPresupp} or similar principles.
		\end{itemize}
		\item Djambarrpuyŋu does indeed seem to have a grammatical reflex for temporal remoteness (\getref{MetPst}-\getref{MetFut})
		
	\end{itemize}
\end{minipage}
\hfill\begin{minipage}{.3\textwidth}
	
	\textbf{\textsc{Contributions of Gikũyũ \textsc{pst }TRMs}} \citep[257]{Cable2013}
	\begin{tabbing}
		\textsc{impst}~~~~\=$ t_e\circ\textsc{imm}(t_u) $\\
		\textsc{nrpst}\>$ t_e\circ\textsc{day}(t_u)$\\
		\textsc{recpst}~~\>$t_e\circ\textsc{rec}(t_u)  $\\
		\textsc{rempst}~~\>$ t_e\prec t_u $
		
	\end{tabbing}
	
	\vfill
	
	\textsc{\textbf{The inventory in \texttt{DJR}}}
	
	Future~~  
	\tikz[remember picture] \node[coordinate,yshift=0.5em] (n1) {}; \\
	Today~~~
	\tikz[remember picture] \node[coordinate] (n2) {};\\
	Today~~~
	\tikz[remember picture] \node[coordinate] (n3) {};\\
	Recent~~
	\tikz[remember picture] \node[coordinate] (n4) {};\\
	Remote
	\tikz[remember picture] \node[coordinate] (n5) {};\\
	
	
	\begin{tikzpicture}[overlay,remember picture]
	\path (n2) -| node[coordinate] (np) {} (n1);
	\draw[thick,decorate,decoration={brace,amplitude=3pt}]
	(n1) -- (n2) node[midway, right=4pt] {\textsc{nonpast}};
	
	
	\path (n3) -| node[coordinate] (p) {} (n5);
	\draw[thick,decorate,decoration={brace,amplitude=3pt}]
	(n3) -- (n5) node[midway, right=4pt] {\textsc{past}};
	
	\end{tikzpicture}
	
	
	
\end{minipage}
\subsubsection*{Cyclicity}
\gathertags
\begin{itemize}
	
	\item What are the forms associated with these five categories?
	
	\item \textbf{I} is obligatory in \textsc{present} (\getfullref{today.I})  \textbf{and} \textsc{recent past} \getfullref{MetPst.I}) contexts
	
	\item \textbf{III} is to refer to an earlier event on the day of speech (\getfullref{today.III}) as well as in the \textsc{remote past} (\getfullref{MetPst.III}).
	
	\gathertags
	
	\begin{itemize}
		\item Times compatible with \textbf{I} and \textbf{III} are \textbf{discontinuous}.
		
		\textbf{Cyclic tense} \citep[88]{Comrie1983} has been reported only in languages spoken in this area of Arnhem Land.
	\end{itemize}
	
	\begin{figure}[H]\centering\caption{Temporal expression in the Yolŋu Matha varieties of Central Arnhem, demonstrating two descriptive phenomena: (a) cyclicity --- the interspersion/discontinuity of \textbf{I} and \textbf{III} forms and (b) metricality --- the (subjective) division of the past domain between these two forms.\\$\lfloor{\sl today}\big)$ indicates the boundaries of the privileged interval {\sl today}. $\boldsymbol{t*}$ is utterance time}\label{TempSchem}
		\begin{tikzpicture}[scale=.85]
		% draw horizontal line   
		\draw[<->, line width=.5mm] (0,0) -- (12,0);
		
		%draw rex
		\shade[left color=blue!15!white, right color=green!15!white] (0,0.02) rectangle (4.8,1.5);
		%	\fill[green!10!white] (2.5,0.02) rectangle (4.8,1.5);
		\fill[blue!10!white] (4.8,0.02) rectangle (6.8,1.5);
		\fill[green!10!white] (6.8,0.02) rectangle (9.5,1.5);
		\fill[orange!10!white] (9.5,0.02) rectangle (12,1.5);
		
		% draw nodes
		\draw (1.25,0) node[below=3pt] {\textbf{}} node[above=10pt] {\textsc{\textbf{III}}};
		\draw (3.675,0) node[below=3pt] {\textbf{}} node[above=10pt] {\textbf{I}};
		\draw (5,0)   node[circle,fill,label=below:$\lfloor{\sl today}$] {} node[below=3pt] {\textbf{}} node[above=3pt] {};
		\draw (7,0) node[diamond,shade,outer color=black, inner color  = ochre,label=below:$\boldsymbol{t*}$] {} node[below=3pt] {\textbf{}} node[above=3pt] {\textsc{}};
		\draw (5.8,0) node[below=3pt] {\textbf{}} node[above=10pt] {\textsc{\textbf{III}}};	
		\draw (8.15,0) node[below=3pt] {\textbf{}} node[above=10pt] {\textsc{\textbf{I}}};	
		\draw (10.75,0) node[below=3pt] {\textbf{}} node[above=10pt] {\textsc{\textbf{II}}};	
		\draw (9.5,0)   node[circle,fill,label=below:${\sl today}\big)$] {} node[below=3pt] {\textbf{}} node[above=3pt] {};
		
		
		%%%braces
		
		\draw [decorate,decoration={brace,amplitude=4pt},xshift=-0pt,yshift=35pt]
		(0.5,0.5) -- (4.5,0.5) node [black,midway,yshift=0.35cm] 
		{\footnotesize metricality};
		
		\draw [decorate,decoration={brace,amplitude=4pt},xshift=-0pt,yshift=40pt]
		(3.5,0.5) -- (9,0.5) node [black,midway,yshift=0.35cm] 
		{\footnotesize cyclicity};
		
		\end{tikzpicture}\end{figure}
	
	
	\item Descriptions (particularly of the neighbouring `Maningrida' lanugage family') have adopted a schema like the one in Table \ref{GlaswegianTR} (originally due to \citet{Glasgow1964}).
	\item \citet{Wilkinson1991} and other Yolŋuists discuss but seem uncommitted to this style of analysis (they've treated it largely as a type of polysemy, \textit{pers. comm.})
	
	\begin{table}[H]\centering\onehalfspacing
		\begin{tabular}{@{}llll@{}}\toprule
			
			&                 & \multicolumn{2}{c}{\textsc{frame}}          \\ 
			&                 & \multicolumn{1}{c}{\textbf{today}}         & \multicolumn{1}{c}{\textbf{before today}}      \\\midrule
			\multirow{2}{*}{\textsc{\rotatebox[origin=c]{90}{infl}}} & \textbf{\phantom{I}I}    & now           & yesterday/recently \\
			& \textbf{III} & earlier today & long ago           \\ \bottomrule%(l){2-4} 
		\end{tabular}
		\caption{A \citet{Glasgow1964}-style analysis of \textbf{past-time restrictions} introduced by the verbal inflections, adapted for the Dhuwal(a) data. \textbf{I} and \textbf{III} inflections correspond to Eather's \textbf{contemporary} and \textbf{precontemporary} ``tenses'' (``precontemporary'' is Eather's \citeyearpar[166]{Eather2011} relabelling of Glasgow's ``remote'' tense.)}\label{GlaswegianTR}
	\end{table}
\end{itemize}

\begin{itemize}
	\item  A formal tool for relating a reference interval to a related interval comes from \citet{Condoravdi2014}.
	\item  In order to capture the meaning component of the \textsc{Perfect} aspect they define a relation \textsc{Nonfinal instantiation} that holds between a property and two intervals $i,j$:
	$$\textsc{NfInst}(P,j,i)\leftrightarrow\exists k [\textsc{Inst}(P,k)\wedge k\sqsubseteq j\wedge k\prec i]$$
	
	such that this relation holds when we can find some interval $k$ contained in $j$, \textbf{preceding} the reference interval $i$, in which $P$ is instantiated.
	
	\item \textbf{A first tilt}
	\begin{itemize}
		
		
		\item Adapting from a treament of the \textsc{Perfect} in \citet{Condoravdi2014}:
		$$\llbracket\textbf{III}\rrbracket^{g,c}=\lambda P\lambda i_c.\exists j\big[i_c\sqsubseteq_{\text{final}}j\wedge\textsc{NfInst}(P,j,i)\big]$$
		
		\item  Which may for current purposes be equivalent to a simpler denotation...(?)
		$$\llbracket\textbf{III}\rrbracket^{g,c}=\lambda P\lambda j_c.\exists k\big[k\sqsubseteq_{\text{nonfinal}}j_c\wedge\textsc{Inst}(P,k)\big]$$
		
	\end{itemize}


\begin{figure}[H]
	\centering
	\begin{tikzpicture}
	\draw[<->, line width=.5mm] (0,0) -- (12,0);
	\draw[line width=.8mm,densely dotted] (5,1.8) -- (5,-1.8); 
	\draw[line width=.8mm,densely dotted] (9,1.8) -- (9,-1.8);
	\draw[line width=2mm,nearly transparent] (7.35,1.8) -- (7.3,-1.8) node [at end,yshift=-2mm] {\textit{\textsc{now}}};
	
	
	\filldraw[nearly transparent,green](3,0)		ellipse [x radius=2cm,y radius=1cm];
	\draw (2.25,.25) node[color=blue] {\textbf{$ j $}};
	\filldraw[nearly transparent,blue,xshift=0.2cm](2,0)		ellipse [x radius=1.2cm,y radius=.6cm];
	\draw (4.25,.25) node[color=forest] {\textbf{$ i $}};
	\filldraw[nearly transparent,green,xshift=.28cm](6,0)		ellipse [x radius=1.2cm,y radius=.8cm];
	\draw (7.3,.25) node[color=black] {\textbf{$ i $}};
	\filldraw[nearly transparent,blue,xshift=.15cm](6,0)		ellipse [x radius=1.05cm,y radius=.5cm];
	\draw (6,.25) node[color=blue] {\textbf{$ j $}};
	
	
	\fill[very nearly transparent,ochre] (9.5,1) -- (9.5,-1) -- (11.5,-1) -- (11.5,1); 
	
	
	
	\draw [decorate,decoration={brace,amplitude=6pt},xshift=-0pt,yshift=40pt]
	(5.1,0.5) -- (9,0.5) node [black,midway,yshift=0.35cm] 
	{\footnotesize \textsc{today}};
	
	\draw [decorate,decoration={brace,amplitude=6pt},xshift=-0pt,yshift=40pt]
	(0,0.5) -- (4.9,0.5) node [black,midway,yshift=0.35cm] 
	{\footnotesize \textsc{before}};
	
	\draw [decorate,decoration={brace,amplitude=4pt},xshift=-0pt,yshift=20pt]
	(3.5,0.5) -- (4.8,0.5) node [black,midway,yshift=0.35cm] 
	{\footnotesize \textit{barpuru}};
	
	\draw [decorate,decoration={brace,amplitude=4pt},xshift=-0pt,yshift=20pt]
	(0.5,0.5) -- (3,0.5) node [black,midway,yshift=0.35cm] 
	{\footnotesize \textit{baman'}};
	
	\draw [decorate,decoration={brace,amplitude=4pt},xshift=-0pt,yshift=20pt]
	(9.25,0.5) -- (11.8,0.5) node [black,midway,yshift=0.35cm] 
	{\footnotesize \textit{goḏarr'}};
	
	\end{tikzpicture}
	\caption{Appealing to `nonfinal instantiation' to provide a unified entry for the temporal reference of \textbf{III}}
	
	
\end{figure}
	
	\item What this genre of analysis would buy us is a situation in which $i$ is identified either as the time-of-speech (roughly \textbf{now}) or some constrained (recent) period \textit{prior to the day-of-speech}.
	
	
	\item \textbf{III} is then licensed when the property which is denoted by the verb that it inflects is instantiated within $j$ (a superinterval of $i$ that shares its right boundary) but not in $i$ itself.
	
	
	
	
	\item An implication of this initial treatment would be that the temporal work that \textbf{III} is not really that of an absolute tense marker (taken by, e.g. \citealt{Klein2009} to be the relation of utterance time to a reference time. Here eventuality time is directly built in to the semantics.)
	\
	\item It's likely possible to maintain a pronominal treatment of tense in the style of \citet{Partee1973} (roughly, $\llbracket\textsc{pst}\rrbracket=\lambda t:t\prec\textbf{now}.t$), but how to do or what the implications are aren't immediately clear to me as I get this handout together.
	
	\item The temporal contribution of future-tense operator \textit{dhu} might be simply analysed as placing a presupposition on the temporal location of $ i $ (or at least on the \textsc{inst} relation) such that $ \tau(\varepsilon)\succ \textsc{now} $
	\begin{itemize}
		\item This is likely to be truth-conditionally insufficient given that it says nothing about the modal (necessity) contribution of \textit{dhu}.
	\end{itemize}
	
\end{itemize}

\subsubsection*{A potential functional explanation}
\begin{itemize}
	
	\item The \textsc{today} and \textsc{nontoday} frames in the descriptive lit correspond to two different discourse modes: \textbf{\textit{conversational}} and \textbf{\textit{narrative}} respectively.
	\item  In \textbf{\textit{conversation}}, where we might be less concerned with remote displacement, we might expect to be concerned with the immediate past as distinguished from the non-past. These presuppositions are grammaticalised as \textbf{III} (the Glaswegian ``remote''/``precontemporary'') and \textbf{I} (the Glaswegian ``contemporary'') respectively.
	\item  Conversely, in \textbf{\textit{narration}}, which concerns the past almost exclusively, a distinction between states-of-affairs that hold in (relative) here-and-now as against the remote past is (arguably) more relevant than past/nonpast.
\end{itemize}
\subsection*{Interactions with modality}
\begin{itemize}
	\item The picture becomes more complicated when we admit data describing \textbf{non-instantiated} events (which includes negated, modalised, and generic/habitual predications)
	\item clausal negation \textit{triggers the appearance of irrealis}-type modal markings
	
	
	
	
	\begin{figure}[h]\caption{The effect of negation as a licensing condition for verbal inflections}\centering	\begin{tikzpicture}[scale=.85]
		% draw horizontal line   
		\draw[<->, line width=.5mm] (0,0) -- (12,0);
		
		%draw rex
		\shade[left color=RoyalPurple!15!white, right color=orange!15!white] (0,0.02) rectangle (4.8,1.5);
		%	\fill[green!10!white] (2.5,0.02) rectangle (4.8,1.5);
		\fill[RoyalPurple!10!white] (4.8,0.02) rectangle (6.8,1.5);
		\shade[left color=orange!10!white, right color=Green!10!white] (6.8,0.02) rectangle (9.5,1.5);
		\fill[orange!10!white] (9.5,0.02) rectangle (12,1.5);
		
		% draw nodes
		\draw (1.25,0) node[below=3pt] {\textbf{}} node[above=10pt] {\textsc{\textbf{IV}}};
		\draw (3.675,0) node[below=3pt] {\textbf{}} node[above=10pt] {\textbf{II}};
		\draw (5,0)   node[circle,fill,label=below:$\lfloor{\sl today}$] {} node[below=3pt] {\textbf{}} node[above=3pt] {};
		\draw (7,0) node[diamond,shade,inner color=ochre,outer color=black,label=below:$\boldsymbol{t*}$] {} node[below=3pt] {\textbf{}} node[above=3pt] {\textsc{}};
		\draw (5.8,0) node[below=3pt] {\textbf{}} node[above=10pt] {\textsc{\textbf{IV}}};	
		\draw (7.5,0) node[below=3pt] {\textbf{}} node[above=10pt] {\textsc{\textbf{II}}};
		\draw (9,0) node[below=3pt] {\textbf{}} node[above=10pt] {\textsc{\textbf{I}}};	
		\draw (10.75,0) node[below=3pt] {\textbf{}} node[above=10pt] {\textsc{\textbf{II}}};	
		\draw (9.5,0)   node[circle,fill,label=below:${\sl today}\big)$] {} node[below=3pt] {\textbf{}} node[above=3pt] {};
		
		
		\end{tikzpicture}\end{figure}
	
	\item The basic picture is given in \ref{negneut}:
	
	
	\begin{table}[h]\centering
\caption{neutralisation of inflectional system under negation}\label{negneut}
	\begin{tabular}{ccc}
		&\multicolumn{2}{c}{\textsc{\textbf{inflections}}} \\
		& \textsc{--neg} & \textsc{+neg}\\\midrule
		&	\gls{I} & \multirow{2}{*}{\gls{II}}\\
		& \gls{II} \\\midrule
		&	\gls{III} & \multirow{2}{*}{\gls{IV}}\\
		& \gls{IV} \\\bottomrule
	\end{tabular}
	\end{table}
	
	\item This provides support for the semanticisation of an \textsc{instantiation} relation in the verbal inflections
	
	\item I assume that it is not a coincidence that \textbf{II} is licensed in both \textsc{future} predications and the negations/modalisations of \textbf{I}-predications.
	
	\item Note, importantly, that today-nonpast utterances \textbf{still receive I}-marking. I expect that the analysis will build in something to do with perceptual access and/or a \textsc{plan} type operator.
	
	The story for this will be that things that are \textbf{currently not happening} can be thought of as perceptually-supportable facts of the world. 
	
	
	\item Functional explanations for the \textsc{negative asymmetry} generally emphasise the fact that negated predicates `[belong] to the realm of the non-realized', a domain that is associated with irrealis marking (Miestamo 2005: 225). 
	
	
	
	
\end{itemize}



\newpage
\section*{data}
\small


\pex \textbf{Temporal remoteness (past)}
\a\deftagex{MetPst}\deftaglabel{I}\begingl\glpreamble\textsc{Recent past} with \textbf{I}//
\gla yo barpuru-ny ŋarra \textbf{marrtji}(*-na) shop-lil//
\glb yes, yesterday{\sc-prom} 1s fo-\textbf{I/*III} shop-{\sc all}//
\glft`Yes, I went to the store yesterday.'//\endgl


\a
\deftaglabel{III}\begingl\glpreamble\textsc{Remote past} with \textbf{III}//
\gla yo ŋarra marrtji-\textbf{na} ŋunhawala ŋäthil baman'//
\glb yes 1s go-\textbf{III} \textsc{dist.all} before long.ago//
\glft`Yes, I went there long ago.'//
\endgl
\xe

\pex\deftagex{MetFut}\textbf{Temporal remoteness (future)}

\a\deftaglabel{I}\begingl\gla yalala ŋarra dhu nhokal lakara-\textbf{m}//
\glb later 1s \textsc{fut} 2s\textsc{.obl} tell-\textbf{I}//
\glft `Later (today) I'll tell you.' \trailingcitation{\citep[373]{Wilkinson1991}}//\endgl

\a\begingl\deftaglabel{II}\gla Barpuru goḏarr ŋarra dhu nhä(\textbf{-ŋu/$^*$-ma})//
\glb funeral tomorrow 1s \gls{fut} see(-\textbf{II}/$^*$-\textbf{I})//
\glft `I'll see the funeral tomorrow'\trailingcitation{[AW~20180730]}//\endgl
\xe


\pex\textbf{Cyclicity (the \textsc{hodiernal} ``frame'')}\deftagex{today}
\a\deftaglabel{I}\begingl\glpreamble\textsc{Present} with \textbf{I}//
\gla ŋarra ga nhä-\textbf{ma} warrkun' (dhiyaŋ bala)//
\glb 1s \textsc{ipfv}-\textbf{I} see.\textbf{I} bird \textsc{endo}.\textsc{erg} \textsc{mvtawy}//
\glft`I'm looking at a bird (now)'//
\endgl

\a\deftaglabel{III}\begingl\glpreamble\textsc{Today past} with \textbf{III}//
\gla ŋe gäthur ŋarra ŋanya nhä-\textbf{ŋal} (*nhäma) goḏarr dhiyal//
\glb	yes, today 1s 3s{\sc.acc} see-\textbf{III} (*see.\textbf{I}) morning {\sc prox-loc}//
\glft`Yes, I saw him here this morning'\trailingcitation{\citep{Wilkinson1991}}//\endgl

\xe

\pex\textbf{Vagueness in the TFA domain}\deftagex{tfa}
\a\begingl\gla ga \textbf{(yawungu)} \textbf{ŋuriŋi-ny} \textbf{bala} ga dhuwal ḏumurru'-ŋu-y, + bäyŋu-n yolŋu walal wukirri waŋara-n ga dhärra//
\glb and (yesterday) \textbf{\gls{texd}.\gls{erg}}-\gls{prom} \gls{mvtawy} \gls{ipfv}.\textbf{I} \gls{prox} big-\textit{ŋu}-\gls{erg} \gls{negq}-\gls{seq} people 3p school empty-\gls{seq} \gls{ipfv}.\textbf{I} stand.\textbf{I}//
\glft`Last week there was nobody at school.'\trailingcitation{\citep[256]{Wilkinson1991}}//\endgl

\a\deftaglabel{II}\begingl Compatibility of \textsc{future} \textbf{II} with \textit{ŋuriŋi bala}\glpreamble//
\gla ga \textbf{ŋuriŋi-n} \textbf{bala} dhu boŋguŋ, bäyŋu-n goḻ, + waŋara-n dhu gi dhärri//
\glb and \textbf{\gls{texd}.\gls{erg}}-\gls{seq} \gls{mvtawy} \gls{fut} tomorrow  \gls{negq}-\gls{seq} school empty-\gls{seq} \gls{fut} \gls{ipfv}.\textbf{II} stand.\textbf{II}//
\glft`And next (week), there'll be nobody at school, it'll be empty.'\trailingcitation{\citep[256]{Wilkinson1991}}//\endgl

\a	\begingl\gla ḏirramu-wal yothu-wal bäpa-'mirriŋu-y rrupiya \textbf{barpuru} djuy'yu-\textbf{n} märr \textbf{barpuru} ga barpuru \textbf{buna}-ny dhiyal-nydja//
\glb man-\gls{obl} kid-\gls{obl} father-\gls{kinprop}-\textsc{erg} money yesterday send.\textbf{I} somewhat yesterday and yesterday arrive.\textbf{I}-\textsc{prom} \gls{prox}.\gls{erg}-\gls{prom}//
\glft`The father sent money to the boy recently and it arrived here yesterday'\trailingcitation{\citep[343]{Wilkinson1991}}//\endgl
\xe

\pex\deftagex{ḏaŋgay}\begingl\glpreamble\textbf{Productive derivation of temporal frame from nominal}//
\gla bala ŋayi yaryu'\textasciitilde{yaryu}-n \textbf{ḏaŋga-y} \textbf{wäŋa-y}//
\glb \textsc{mvtawy} 3s wade\textasciitilde{\textsc{red}}-\textbf{I} \textbf{fine-\textsc{erg}} \textbf{place-\textsc{erg}}//
\glft`Then he went along the water's edge (hunting) while it was fine out (not raining).'\trailingcitation{\citep[159]{Wilkinson1991}}//\endgl\xe


\pex\textbf{Compatibility of \textit{dhiyaŋ bala} with non-present reference}\deftagex{dhiyaŋ}
\a\begingl\gla \textbf{dhiyaŋ} \textbf{bala} napurr bäpi nhä-ŋal gäthur//
\glb \textbf{\gls{prox}.\gls{erg}} \textbf{\gls{mvtawy}} 1p.\gls{excl} snake see-\textbf{III} today//
\glft`We saw a snake today'\trailingcitation{\citep[256]{Wilkinson1991}}//\endgl

\a\begingl\gla \textbf{dhiyaŋ bala} walal dhu buna, yalala//
\glb \textbf{\gls{prox}.\gls{erg}}~\textbf{\gls{mvtawy}} 3p \gls{fut} arrive.\textbf{I} later//
\glft`They're coming later today.'\trailingcitation{\citep[256]{Wilkinson1991}}//\endgl
\xe



\pex \begingl\glpreamble \textbf{\textit{Ŋuriŋi bala} `at (some other) time'}//
\gla Way, marŋgi nhe (ŋarra-kalaŋa-w bäpa-'mirriŋu-w-nydja [\textbf{ŋunhi} [ŋayi dhiŋga\textbf{-ma}-ny \textbf{ŋuriŋi} \textbf{bala} dhuŋgara-y]])//
\glb hey know 2s 1s-\gls{obl}-\gls{dat} father-\gls{kinprop}-\gls{dat}-\gls{prom} \textbf{\gls{texd}} 3s die-\textbf{I}-\gls{prom} \gls{texd}-\gls{erg} then year-\gls{erg}//
\glft`Hey, did you know my father, who died last year?'\trailingcitation{\citep[343]{Wilkinson1991}}//\endgl\xe

\bibliographystyle{apa}\bibliography{../FullBiblio.bib}


\end{document}