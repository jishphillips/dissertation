\synctex=1

\documentclass[10pt]{article}
\usepackage[margin=70pt,marginparwidth=60pt,a4paper]{geometry}
\usepackage{amssymb}
\usepackage{hyperref}
%\usepackage[tiny,compact]{titlesec}
\usepackage{graphicx}

\usepackage{booktabs}


\usepackage{wrapfig}
\usepackage{textcomp}
\usepackage{bold-extra}
\usepackage{tikz}
\usepackage{qtree}
\usepackage{tikz-qtree}
\usepackage{expex}


\usetikzlibrary{positioning,decorations.pathmorphing,arrows.meta,decorations.text}
\tikzset{snake it/.style={decorate, decoration=snake}}
\usetikzlibrary{calc, shapes, backgrounds,angles,quotes,tikzmark}
\usepackage{afterpage}
\usepackage{verbatim}
\usepackage{array}
\usepackage{multirow}
%\usepackage{hanging}
\usepackage{supertabular}
\usepackage{mathtools}
\usepackage[all]{xy}
\usepackage{ot-tableau}

\usepackage{paralist} 
\usepackage[labelsep=period,labelfont=bf]{caption}
\usepackage{subcaption}
\usepackage{fancyhdr} 
\usepackage{sectsty}
%\allsectionsfont{\sffamily\mdseries\upshape} 
\usepackage{float}
\usepackage[nottoc,notlof,notlot]{tocbibind} 
\usepackage[titles,subfigure]{tocloft} 
\usepackage{setspace}
%\usepackage[colorinlistoftodos]{todonotes}
\usepackage{xcolor}

\definecolor{blech}{rgb}{.78,.78.,.62}
\definecolor{ochre}{cmyk}{0, .42, .83, .20}
%\usepackage[explicit]{titlesec}
%\usepackage{type1cm}
%\usepackage{xcolor}

\usepackage{xltxtra} % Loads fontspec, xunicode, metalogo, fxltx2e, and some extra customizations for XeLaTeX
%\defaultfontfeatures{Mapping=tex-text} % to support TeX conventions like ``---''
\defaultfontfeatures{Mapping=tex-text}
\setmainfont{Cambria}

\usepackage[sort]{natbib}
\bibliographystyle{apa}
\bibpunct[:]{(}{)}{,}{a}{}{,}

%\usepackage{gb4e} \let\eachwordone=\it %\let\eachwordthree=\sf


\makeatletter
\def\@xfootnote[#1]{%
	\protected@xdef\@thefnmark{#1}%
	\@footnotemark\@footnotetext}
\makeatother

\pagestyle{fancy}
\fancyhf{}
\rhead{\footnotesize %Josh Phillips
	\hspace{2cm}\textbf{\thepage}}
\rfoot{}

\renewcommand{\headrulewidth}{0pt} 
\newcommand{\rowgroup}[1]{\hspace{-1em}#1}
\usepackage{stmaryrd}
\newcommand{\denote}[1]{\mbox{$[\![\mbox{#1}]\!]$}}
\newcommand{\denotn}[1]{\mbox{\llbracket\mbox{#1}\rrbracket}}

\newcommand{\mcom}[1]
{\marginpar{\color{black}\raggedleft\raggedright\hspace{0pt}\linespread{0.9}\footnotesize{#1}}}
\newcommand{\cb}[1]
{\marginpar{\color{orange}\raggedleft\raggedright\hspace{0pt}\linespread{0.9}\footnotesize{#1}}}
\newcommand{\hk}[1]
{\marginpar{\color{purple}\raggedleft\raggedright\hspace{0pt}\linespread{0.9}\footnotesize{#1}}}
\newcommand{\note}[1]{{ }\mcom{Note}\textbf{#1}}


\newcommand{\glem}[1]
{\MakeUppercase{\scriptsize{\textbf{#1}}}}

\newcommand{\exem}[1]
{\textit{\textbf{#1}}}

\usepackage{framed}
\usepackage{wrapfig}

\title{Cross-linguistic approaches to temporal reference\\\small\em Annotated Bibliography \& Research notes}

\date{}
\begin{document}
	
\section*{44 \textit{nyi}}
\textbf{Lit. Production Centre, Elcho (1980)}\\
\textbf{Written by} Marrŋanyin\\
\textbf{Illustrated by }Steven Bunbatju.

	
	
Yolŋu walal gan marrtjin bala raŋilil ŋarirriw. Marrtjin walal gan, bala walal nhäŋal nyumukuṉiny winyiwinti ŋayi gan gorruŋal garrwar dharparŋur, yurr ŋayi gan ŋunhi ŋäthin ŋathaw.

Walalnydja ŋanya ŋunhi märraŋala, bala gäŋal walal ŋanya balan raŋilila.

Ga waŋgany yolŋu waŋan bitjarr, ``walal, dhuwal nyumukuṉiny winyiwinyi limurr märraŋal, ga djäga limurr dhu ŋamatham.'' Ga ŋayi waŋgany bulu yolŋu waŋan, ``Ŋe, nyumukuṉiny dhuwali winyiwinyi, limurr dhu gäman ŋanya wäŋalila.''

Ga wiripu walal marrtjin ŋarirriw ga wiripuny mala gan nhinan warraw'ŋur yolŋu walal ga winyiwinyi''

Ŋunhi walal marrtjin ŋarirriw, ga bumarnydja walal dharrwa ŋarirri. Walalnydja gan djäma yindin gurtha ŋarirriw'nha ŋunhalnydja warraw'ŋurnydja.

Gäŋala walal marrtjin ŋarirriny, bala walal yan bathara gurthalila. Bala walal gan galkurra nhinanan warraw'ŋura.

Bala walal märraŋala ŋunhi ŋarirriny mala, bala walal marrtji nyaŋ'thurra.

Ga gäŋalnydja walal wäŋalilnydja ŋarirrin mala ga nyumukuṉiny winyiwinyi.

Ga walalnydja nhäŋala ŋunhi nyumukuṉiny winyiwinyi bala walal waŋanan, ``Walal yolthu dhuwal nhäŋal?''\\Ga walanydja waŋan, ``Napurr dhuwal nhäŋal, ŋayi gan gorruŋal dharpaŋur.''\\
``Go, ŋarra nhäma.''\\
``Ŋay'! Nhäŋun.''\\
Ŋunhi walal gan nhäŋalnydja ŋunhi nyumukuṉiny winyiwinyi, bala walal ŋorranan.

\section*{Yuwalk yolngu ga buthurumirr}
\textbf{NT Dep't of Employment, Ed \& Training (2006)}\\
\textbf{Dhawu: } Fred Munyirinyirwuŋ\\
\textbf{Biḏi'yunawuynydja:} Ranhdhakpuywuŋ

\pex[glftpos=right]\a\begingl\gla yo! djamarrkuḻi, dhuwal nhumalaŋ dhawu. Dhuwal balanda mala limurruŋ guŋga'yu-nha-mirr mala. Ŋunha walal gunharr'yun walala-ŋgu-wuy wäŋa ga gurruṯu-mirri-ny mala. Ga dhuwanna walal ŋuyulkŋur marrtji räli guŋga'yunnaraw limurruŋ.//
\glb yes children \textsc{prox} 2p\textsc{.dat} story \textsc{prox} balanda group 1p\textsc{.incl} help-IV-\textsc{prop} group \textsc{dist} 2p leave-\textbf{I} 3p\textsc{-dat-?} home and family\textsc{-prop-prom} group and \textsc{prox?} 3p ? go\textbf{-I} hither help-\textbf{?} 1p.\textsc{incl}//
\glft `Children, this is your story. These balanda are our teachers. They have left their homes and their families, and they have come here to help us for our own sake.'//\endgl
\a\begingl\gla Yo! Marrtji-ny walal dhuwal räli-ny, walal dhu guŋga'yun limurruny. Limurr dhu nyumukiṉiny gakal märram walala-ŋgu-n. Yalalaŋumiriw limurruŋ ŋunhi ŋali dhu dhawaṯ'thurr wukirri-ŋur-nydja bala djämawnha ḻarruŋ//
\glb yes go.\textbf{I}\textsc{-prom} 3p \textsc{prox} hither-\textsc{prom} 3p \textsc{fut} help\textbf{.I} 1p.\textsc{incl.prom} 1p.\textsc{incl} \textsc{fut} small? ? get 3p-\textsc{dat-seq} later~on 1p\textsc{.incl} \textsc{text} \textsc{comp?} \textsc{fut} finish-\textbf{II}  write\textsc{-loc-prom} then work-\textsc{dat-IV} search\textbf{-II?}//
\glft`Yes, they have come here to help us. We must gain skills from them. Later when we finish school then we will look for work.'//\endgl
\a\begingl\gla Buku-djulŋi-mirr nhänha-mirr walal yol mala nhuma dhuwal, dharaŋan-mirr walal gi. Yaka nhuma dhu ga warku'yunmirr. Dhuwal limurr raypirri'mirr mala ga buthurumirr mala. Yolŋu limurr dhuwal.//
\glb face-happy-\textsc{prop} look.IV-\textsc{prop} 3p who group 2p \textsc{prox} understand-\textsc{prop} 3p \textsc{ipfv.II?} \textsc{neg} 2p \textsc{fut} \textsc{ipfv.I} mock-IV\textsc{-prop} \textsc{prox} 1p\textsc{.incl} discipline\textsc{-prop} group and ear\textsc{-prop} group Yolŋu 1p\textsc{.incl} \textsc{prox}//
\glft`Please look at yourselves and recognise who you are. You must not tetase. We must be self-disciplined and listeners. We are Yolŋu.'//\endgl

\a\begingl\gla Ŋuli nhuma dhu buthuru-mirri-yirr, nhuma-ny dhu marŋgi-thirra. Bala nhuma dhu djäl-mirri-yirra ŋunhi nhä nhuna-ny dhu ga ditjay nokal marŋgi'kum.//
\glb \textsc{comp} 2p \textsc{fut} ear-\textsc{prop-?} 2p\textsc{prom} \textsc{fut} know\textsc{-inch\textbf{-II}} then 2p fut want\textsc{-prop-?} \textsc{texd} what 2p\textsc{.acc-prom} \textsc{fut} and teacher?\textsc{-erg} ? know\textsc{-caus}//
\glft`If you listen, you will learn. Then you will be interested in what the teachers are teaching you.'//\endgl

\a\begingl\gla Yo! Gärri-ny dhu ga dhuwal nokal muḻkurr-ŋur waŋga'waŋganygal, dhuwal yäku birrka'yun. Ga birrka'yurrnydja gi mirithin yan. Ga yän warray ga yuwalk yän nhe du märram walalaŋgun guŋga'yunawuy.//
\glb yes enter-\textbf{II-prom} \textsc{fut} and \textsc{prox} ? head-\textsc{loc} one\textasciitilde{\textsc{redup}}-? \textsc{prox} name tru\textbf{-I} and try\textbf{-II-\textsc{prom}} \textsc{ipfv.\textbf{II}}  \textsc{intens-?} only and only maybe? and true only 2s ? get 3p-\textsc{dat-seq} help-IV-?//
\glft`Yes! Each of you keep this in your mind, and keep on keeping this in your mind. Only then will you get the best assistance from the teachers.'//\endgl

\a\begingl\gla Manymak, yaka gulyurr wukirri-ŋur, marrtjin gi yan. Nhä nokal dhu ga ḏalthirr birrka'yurr mirithi märrmaw', yolŋuw balandaw romgu. Yo! Birrka'yurrnydja walal gi mirithin yän.//
\glb good \textsc{neg} pound?\textbf{-II} write-\textsc{loc} go-\textsc{seq} \textsc{ipfv} only what ? \textsc{fut} \textsc{ipfv.I} hard-\textsc{inch.I} try.\textbf{II} \textsc{intens-II} two\textsc{-dat} yolŋu\textsc{-dat} balanda\textsc{-dat} culture\textsc{-dat} yes try\textbf{-II-prom} 3p \textsc{ipfv.II} \textsc{intens-II-seq} only//
\glft`Therefore, don't stop attending school. Keep coming. Even if it is hard for you, keep on trying both ways, Yolŋu and Balanda. Yes! Keep trying as hard as you can!'//\endgl

\xe
	
\end{document}