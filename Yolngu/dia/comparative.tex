
\subsection{Yan-nhaŋu \& Goḻpa}

KL describes \gls{guf} as a ``modality-based language'' (176) against which she contrasts Golpa: `There are several (more and less strong) arguments against a modality-based analysis of the Golpa verb system' \citep[179ff]{Kabisch-Lindenlaub2017}

KL ex 365,55,77 --- reanalysis of III by Goḻpa speakers as a general past marker? (Cf. Yannhaŋu), see pg 209ff (table 26), 159ff for forms (Table 15ff).
No negation efffects ex 28, 119


\pex\textbf{Recent past receiving \textsc{past} (cog. \textbf{III}) marking in Golpa}

\a\begingl\gla Yawungu ŋarra bath\textbf{-ana}//
\glb yesterday 1s cook\textbf{-\gls{pst}}//
\glft `Yesterday I cooked'\trailingcitation{\citep[100]{Kabisch-Lindenlaub2017}}//\endgl
\a\begingl\gla Ŋarra-ŋayu ma ŋurru\textbf{-nha} djulŋi-yu\textbf{-nha} barpuru//
\glb 1s-\gls{prom} \gls{ipfv} sleep-\textbf{\gls{pst}} good-\gls{inch}-\textbf{\gls{pst}} yesterday//
\glft`I was sleeping well yesterday.' \trailingcitation(ibid.)//\endgl\xe

\pex \begingl\glpreamble \textbf{Future reference receiving \textsc{neutral} (cog. \textit{I}) marking in Golpa}//
\gla Nhaŋu ŋarra-ŋayu wurruku djinidhal gara\textbf{-ma} ga baŋu-ŋayu duy’tj\textbf-{un} munhamurru godarr’//
\glb \gls{prox} 1s-\gls{prom} \gls{fut} now go-\textbf{\textsc{neu}} and here-\gls{prom} return-\textbf{\textsc{neut}} tomorrow morning//
\glft `I'll leave now and come back here tomorrow morning.'\trailingcitation{(\textit{ibid.})}//
\endgl\xe



\subsection{Wangurri (Dhaŋu-Djaŋu)}

Mally's thesis came out almost the same time as Mel's (there're signs that they were speaking/comparing also and they were both at Sydney universities): a big point of difference which is likely the language (rather than the linguist) is that Mally describes the cognate to \textbf{III} as the \textbf{PFV} and \textbf{doesn't report cyclicity.} She \textit{does} argue for a very mood-central conception of the verbal paradigm. My inclination is that this has some intersections with the evidential status and more accurately the \textbf{illocutionary force} of an utterance given its inflectional status.

\begin{itemize}
	\item Mally notes that the mood-based neutralisation described of Dhuwal doesn't obtain in Wangurri. Here all (5) inflections can occur under negation. It is however possible that there is some variation along this parameter (see McLellan's ex. 34)
	
	\pex\begingl\glpreamble Cooccurrence of \textsc{neg} and \textbf{III} in Wangurri//
	\gla Ŋarru \textbf{ŋangawal} nhän banha banha-ya bäpi-m waŋa-yi-\textbf{na}-m//
	\glb But \textsc{neg} 3s that that-\gls{ana} snake-d die-\gls{inch}-\textbf{III}-d//
	\glft`But that snake did not die.'\trailingcitation{\citep[178]{McLellan1992}}//\endgl\xe
	
	\item It's not actually clear to me that there isn't some sort of metricality/cyclicty between cognates of \textbf{I} (her \textsc{1/neu}) and \textbf{III} (her \textsc{2/pfv}).
	\pex\a
	\begingl\gla ŋangawul-nha nhän barpuru  gayŋa ŋuwaly-man-ma ŋarra//
	\glb \gls{neg}-\gls{prom} 3s yesterday \gls{ipfv}.\textbf{I} well:\textsc{caus}-\textbf{I}-\gls{prom} go.\textbf{I}//
	\glft`He kept not being well yesterday.'\trailingcitation{\citep[196]{McLellan1992}}//\endgl
	\a \begingl\gla waŋgany-dhu barpuru ŋanapiliny nhawun guŋnharru-ma-n-a//
	\glb one-\gls{erg} yesterday 1p.\textsc{excl.acc} \gls{dp} separate-\gls{caus}o\textbf{I}-\gls{prom}?//
	\glft`One left us yesterday.'\trailingcitation{\citep[250]{McLellan1992}}//\endgl 
	\xe
	\begin{itemize}
		\item Another problem with this is the \textsc{ipfv} auxiliary cooccurs with all inflections: \textit{gayŋa.\textbf{I}, gayŋi.\textbf{II}, gayŋan.\textbf{III}, gayŋarra.\textbf{IV}}
		
	\end{itemize}
\item Nice examples of remote past \textbf{III} in the \textit{Israel Text} (pg. 263-4)
	\item Possibly only one circumstantial modal \textit{ŋarru} that conflates the functions of \textit{dhu} and \textit{balaŋu}. Note that \textit{ŋarru }is also glossed as `but' in \lastx. This means it is almost certainly cognate with \textit{yurru}: the Miwatj \textsc{fut}. (And in fact, \textit{dhu} is also almost certainly a grammaticalisation of this.)
	\item \textbf{NB} -- the \textsc{hab} form described by McLellan is \textit{bayiŋ}. This is the \textsc{erg}-inflection of the demonstrative \textit{banha} (hence \textit{\textbf{bayiŋu}}). This has maybe be whence \textit{balaŋu} or \textit{bäyŋu} in Dhuwal? (184)\\
	\textit{bilaŋ garri/bayin warri} could also be source constructions for balaŋu?? (see 214)
	\item \textit{bayiŋ} `\textsc{hab'} also has some circ modal uses., see pg. 256: \textit{nhunu bayiŋ ŋarray} `You should come'
	\item \textit{wilak=mak}, \textit{bitjan ḻinygu, yäna, ḻinygu/bili} (197)\\
	\item \textit{Bili, tout court} is used to end stories. I"m just convinced that this is an illocutionary force marker. An instruction to upload to the cg. 
	\item p248,9 gives examples (, 16) of \textbf{I} receiving future reference.
\end{itemize}


\subsection{Wuḻaki \& Djinaŋ}


\mcom{Waters focuses his work on Djuwiŋ Djinaŋ. I have access to Ganalbingu speakers which could help to fill some of the big gaps in his data on this language. Additional judgments from Yirritjiŋ Djinaŋ are also available via the Wuḻaki men and Margaret hopefully.}
Djinaŋ-Djinba look to have floresced a little in verbal inflectional domain. There seems to be solid attested cyclicity/metricality in the djr/guf style and then a bit of extra stuff. Unsure what happens under negation. 

\subsection{Ganalbingu (Djinba)}
\subsection{Ritharrŋu}

A likely close relative of Dhuwal-Dhuwala, Ritharrŋu -- the southernmost Yolŋu variety -- is described by Heath as having `basic...semantically straightforward' \textsc{past, present} and \textsc{future} tense categories \citeyearpar[74]{Heath1980}.



\pex\textbf{Ritharrŋu \textsc{future} (cog. II) licensed in complements of propositional attitudes}
\a\begingl\gla djäl-thirri nhanŋu rra ḻan\textbf{-ŋu} nya rra//
\glb want-\gls{inch}.\gls{pres} 3s.\gls{dat} 1s spear-\gls{fut} 3s.\gls{acc} 1s//
\glft `I want to spear him' \trailingcitation{\citep[105]{Heath1980}}//\endgl

\a\begingl\gla marŋgi nhe \textbf{waŋi} nhe ritharrŋu-ŋuru'//
\glb know 2s speak.\gls{fut} 2s Ritharrŋu-\gls{abl}//
\glft`You know how to speak Ritharrŋu.'\trailingcitation{\citep[105]{Heath1980}}//\endgl
\xe


\pex\textbf{Relative temporal ordering?}

\begingl\gla guyupa-na-thaŋ' ŋay ḏul'-maṉ ḏali//
\glb die-\gls{pst}-\gls{temp} 3s burn-\gls{caus}.\gls{pres} 3p//
\glft`When someone dies, they burn [grass].'\trailingcitation{\citep[97]{Heath1980}}//\endgl\xe

\subsubsection{Data from Salome's elicitation}

Two elicitation sessions led by Salome Harris, May 2019 in Ngukurr on the basis of my TMA questionnaire which she translated into Kriol with Anthony Daniels (i have a copy of their translation in an email from S'lomes on 20 May 19). My transcriptions (for RNPW) were checked (and corrected extensively) by her.
At least one recording (RNPW) is archived w AIATSIS already
\begin{itemize}
	\item Roy Natilma \& Peter (Djudja) Wilfred [20190520~RNPW]
	\item Andy Lukuman (wäg) \& David Wilfred (rit) [20190522~ALDW]
\end{itemize}

\pex \textbf{Present (progressive)}
\a\begingl\gla nhäma-nu ŋarra mukul'nha-ya yaŋun'thu-ya bala//
\glb see.\textbf{I}-\gls{seq} 1s aunt.\gls{acc}-\gls{prom} \gls{prox}-\gls{erg}-\gls{prom} \gls{mvtawy}//
\glft`I'm looking at my aunt right now.'\trailingcitation{[RN~20190520]}//\endgl
\a\begingl\gla nhäma-'ŋirri' ŋarra mukul'nha-ya//
\glb see.\textbf{I}-\textit{only} 1s aunt.\gls{acc}-\gls{prom}//
\glft`I'm looking at my aunt right now.'\trailingcitation{[AL~20190522~57s]}//\endgl
\xe


\pex \textbf{Present (habitual)}\deftagex{hab}
\a\begingl\gla guḻku'mirri ŋarra nhäma mukulŋ'nha-y//
\glb frequent-\gls{prop} 1s see.\textbf{I} aunt.1s.\gls{acc}-\gls{prom}//
\glft`I see my aunt frequently (every day).'\trailingcitation{[RN~20190520]}//
\endgl
\a\begingl\gla nhäma-'ŋirri' ŋarra mukul'nha-ya (muŋuy')//
\glb see.\textbf{I}-\textit{only} 1s aunt.\gls{acc}-\gls{prom} always//
\glft`I see my aunt all the time (every day).'\trailingcitation{[AL~20190522~1min30s]}//\endgl

\a\begingl\gla napu nya ŋuli bartjun-dja ŋuŋ'ŋara-dhi ḻuka-nu ŋay ŋaŋ'gun-ŋu ŋay wurpan\deftaglabel{emu}//
\glb 1p.\gls{excl} 3s.\gls{acc} \gls{dist}|\gls{hab}? spear-\gls{prom} \gls{dist}.\gls{loc}-\gls{temp} eat.\textbf{I}-\gls{seq} 3s bathe.\textbf{I}-\gls{comp} 3s emu//
\glft`We throw spears at it right there in the waterl; as it drinks, as it bathes, the emu.'\trailingcitation{\citep[137]{Heath1978a}}//\endgl

\xe

\pex\textbf{future (same-day)}
\a\begingl\gla  ripurru'mirri ŋarra nhäŋu mukulŋ'nha-ya//
\glb afternoon 1s see.\textbf{II} aunt.1s.\gls{acc}-\gls{prom}//
\glft`I'll see my aunt this afternoon'\trailingcitation{[RN~20190520]}//\endgl\xe

\textit{ripurru-} can also mean `yesterday'

\pex\textbf{future (tomorrow)}
\a\begingl\gla  guḏarrpuy ŋarra nhäŋu mukulŋ'nha-ya rraku//
\glb tomorrow 1s see.\textbf{II} aunt.1s.\gls{acc}-\gls{prom} 1s.\gls{dat}//
\glft`I'll see my aunt tomorrow'\trailingcitation{[RN~20190520]}//\endgl\xe

\textit{guḏarr} can also mean `morning'

\pex\textbf{past (same-day)}
\a\begingl\gla gätha ŋarra nhäwala mukulŋ'nha-ya//
\glb today 1s see.\textbf{III} aunt.1s.\gls{acc}-\gls{prom}//
\glft`I saw my aunt today (this morning.)'\trailingcitation{[RN~20190520]}//\endgl
\xe

\pex\textbf{past (yesterday)}
\a\begingl\gla ripurrumirri ŋarra nhäwala mukulŋ'nha-ya//
\glb yesterday 1s see.\textbf{III} aunt.1s.\gls{acc}-\gls{prom}//
\glft`I saw my aunt today (this morning.)'\trailingcitation{[RN~20190520]}//\endgl
\a\begingl\gla  nhäwala ŋarra mukul'ya gäthura-ya//
\glb see.\textbf{III} 1s aunt-\gls{prom} yesterday-\gls{prom}//
\glft`I saw my aunt yesterday.'\trailingcitation{[DW~20190522~5min11s]}//\endgl\xe


\pex\textbf{past (distant)}
\a\begingl\gla ŋarra yothu-ganyaŋ', nhänha ŋarra ŋuli mukulŋ'nha-ya//
\glb 1s child-\gls{dim} see.\textbf{IV} 1s \gls{hab}? aunt.1-\gls{prom} yesterday-\gls{prom}//
\glft`When  I was a kid, long ago, I saw my aunt.'\trailingcitation{[RN~20190520]}//\endgl
\a\begingl\gla  baman'dja nhäwala ŋarra mukulŋ'nhaya, yothu'thi-ya-ŋ//
\glb before.\gls{prom} see.\textbf{III} 1s aunt.1-\gls{acc}-\gls{prom} yesterday-\gls{prom} child.\gls{temp}-\gls{prom}-1//
\glft``When  I was a kid, long ago, I saw my aunt.'\trailingcitation{[DW~20190522~6min17s]}//\endgl\xe

\pex \textbf{past (habitual, distant)}

\a\begingl\gla ŋarra yothu-ganyaŋ', nhänha ŋarra ŋuli mukulŋ'nha-ya//
\glb 1s child-\gls{dim} see.\textbf{IV} 1s \gls{hab}? aunt.1-\gls{acc}-\gls{prom}//
\glft`When  I was a kid, long ago, I saw my aunt.'\trailingcitation{[RN~20190520~2min15]}//\endgl
\a\begingl
\gla  nhänha'ŋirri ŋarra yaku'yu-na bili yothuganyaŋ'dhu-ya rra//
\glb see.\textbf{IV} 1s thusly-\textbf{III} \gls{cplv} young.\gls{dim}-\gls{temp}-\gls{prom} 1s//
\glft `When I was a kid I'd seem my aunt all the time.'\trailingcitation{[RN~20190520~2min25]}//\endgl
\xe

\begin{framed}\noindent
Heath doesn't report any \textit{ŋuli} particle in Rit so far as i can tell, it's \textit{possible} that this is a borrowing?  In the other recording the habitual past elicitaiton also receives \textbf{III} marking though it's not clear that the meaning is totally understood (there's no temp freq adverbial, for example. Note that Heath notes that the difference between \textit{wänina} and \textit{wäninya}, two forms of `go.\textsc{pst}' (`went') may differ in ±\textsc{habitual}, pg. 75.) \textit{Ŋuli} \textit{is} the form distal demonstrative in Rith though (where in djr it's \textit{ŋunhi}, although both \textit{ŋuli} and \textit{ŋunhi} seem to show up as a subordinator and \textsc{hab }is only available with \textit{nuli}.) We're in the presence of something here. Note that the subordinating enclitic in rit (also present in \getfullref{hab.emu}) is \textit{\textdblhyphen{ŋu}}.
\end{framed}
\pex \textbf{negative present}

\a\begingl
\gla  nhäma'may'-nu rra yaŋun'dhu-ya bala mukulŋ'nha-ya//
\glb see.\textbf{I}-\gls{neg}-\gls{seq} \gls{prox}-\gls{temp}-\gls{prom} \gls{mvtawy} aunt.1\gls{acc}-\gls{prom}//
\glft`I'm not looking at my aunty right now.'\trailingcitation{[RN~20190520~2min42]}//\endgl
\a AL translates this with \textbf{III} and a now-TFA (cf. my proposed lexical constraint on aktionsart for djr. verbs)
\xe

\pex\textbf{neg pres hab}

\a\begingl\gla  yaka ŋarra nhäma'may' mukulŋ'nha-ya guḻku'-mirri-ya//
\glb \gls{neg} 1s see.\textbf{I} aunt.1-\gls{acc}-\gls{prom} frequent-\gls{prop}-\gls{prom} //
\glft`I don't see my aunt every day.'\trailingcitation{[RN~20190520~2min54s]}//\endgl\xe


\pex\textbf{neg fut sameday}

\a\begingl\gla  yaka rra nhäŋu'may mukulŋ'nha-ya gäthura-ya//
\glb \gls{neg} 1s see.\textbf{II}-\gls{neg} aunt.1-\gls{acc}-\gls{prom} today-\gls{prom} //
\glft`I won't see my aunty today.'\trailingcitation{[RN~20190520~3min3s]}//\endgl

\a\begingl\gla  yakaŋu, mukul nhäŋu-'may' ŋarra; gäŋu nhe wäyindja//
\glb\gls{negq} aunt see.\textbf{II}-\gls{neg} 1s get.\textbf{II} 2s meat-\gls{prom}//
\glft`No, I won't see aunty (today), you get the meat.'\trailingcitation{[DW~20190522~12min45s]}//\endgl

\xe


\pex\textbf{neg fut tomorrow}

\a\begingl\gla  yaka ŋarra nhäŋu'may mukulŋ'nha-ya goḏarrpuy-ya (bulu)//
\glb \gls{neg} 1s see.\textbf{II} aunt.1-\gls{acc}-\gls{prom} tomorrow-\gls{prom} again //
\glft`I won't see my aunt tomorrow (either).'\trailingcitation{[RN~20190520~3min12s]}//\endgl\xe


\pex\textbf{neg pst sameday}

\a\begingl\gla  gäthura-ya (bulu) ŋarra nhäwala'may' mukulŋ'nha-ya//
\glb today-\gls{prom} again 1s see.\textbf{III}-\gls{neg} aunt.1-\gls{acc}-\gls{prom}//
\glft`I haven't seen my aunty today (either).'\trailingcitation{[RN~20190520]}//\endgl

\a\begingl\gla nhäwala'may' ŋarra(na) gäthura mukulŋ'-nha-ya//
\glb see.\textbf{III} 1s-\gls{seq} today aunt.1-\gls{acc}-\gls{prom}//
\glft`I  didn't see my aunt today.'\trailingcitation{[AL~20190522~14min18s]}//\endgl
\xe

\pex\textbf{neg past distant}

\a\begingl\gla baman'dja yothuthinaŋ ŋarra nhäwala'may' ŋarra mukulnhaya//
\glb before-\gls{prom} child.\gls{temp}.\gls{seq} 1s see.\textbf{III}-\gls{neg} 1s aunt.1-\gls{acc}.\gls{prom}//
\glft`When I was a kid, I never saw my aunt.'\trailingcitation{AL~20190522~17min}//\endgl
\a\begingl\gla yothu-ganyaŋ'tja rra, nhäwala'may ŋarra, mukulŋ'nha-ya//
\glb child-\gls{dim}-\gls{prom} 1s see.\textbf{III}-\gls{neg} 1s aunt.1-\gls{acc}-\gls{prom}//
\glft`when i was a kid, i never saw my aunt.'\trailingcitation{[RN~20190520]}//\endgl
\xe

\pex\textbf{neg past hab}

\a\begingl\gla nhänha'maynydja rra muka mukulŋ'nha-ya, yothu-thaŋ'tja//
\glb see.\textbf{IV}-\gls{neg}-\gls{prom} 1s ok aunt.1-\gls{acc}-\gls{prom} child-\gls{temp}-\gls{prom}//
\glft`.'\trailingcitation{[RN~20190520]}//\endgl\xe

\pex\textbf{deontic modality}

\a\begingl\gla balijiman'djana waŋana: helmet'muru nhe. wäŋa-ŋura-nu nhe nhini.//
\glb policeman-\gls{prom}-\gls{seq} say.\textbf{III} helmet-\gls{priv} 2s home-\gls{loc}-\gls{seq} 2s sit.\textbf{II}//
\glft`The policeman said ``You don't have a helmet. You must stay home.'''\trailingcitation{[DW~20190522~36min15]}//\endgl\xe

For epistemic modalities, the same strategy as djr seems to apply (i.e. an adverbial \textit{bärri} that, like djr (guf) \textit{mak(u)} doesn't exert any licensing conditions on the inflection)