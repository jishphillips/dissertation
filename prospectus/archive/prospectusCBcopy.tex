 \documentclass[12pt]{article}
 \usepackage[margin=1.2in,letterpaper]{geometry}
 \usepackage[T1]{fontenc}
 \usepackage{textcomp}
 \usepackage{ragged2e}
 \usepackage{booktabs}
 \usepackage{bold-extra}
 \usepackage{fancyhdr}
 \usepackage{fontspec,xltxtra,xunicode}
 \usepackage[labelsep=period,labelfont=bf]{caption}
 \defaultfontfeatures{Mapping=tex-text}
 \setmainfont{Cambria}
 \newcommand{\HRule}{\rule{\linewidth}{0.5mm}}
 %\setromanfont[Mapping=tex-text]{Hoefler Text}
 %\setsansfont[Scale=MatchLowercase,Mapping=tex-text]{Gill Sans}
 %\setmonofont[Scale=MatchLowercase]{Andale Mono}
 
 % \usepackage[margin=.7in,nohead,right=1.5in]{geometry}
 \def\singlesidestandardsetup{
 	%\textwidth 6in
 	%\oddsidemargin 0in
 	%\evensidemargin 0in
 	%\topmargin -.5in
 	
 	%\textheight 9.5in
 	%\columnwidth \textwidth
 	\parindent 1em
 	%\parskip 8pt
 	\pagenumbering{roman}
 	\parindent 1em
 	\parskip 10pt
 	
 }
 \usepackage{marginnote}
 \xdef\marginnotetextwidth{\the\textwidth}
 
 \usepackage{wrapfig}
 %\newcommand{\longsquiggly}{\xymatrix@C=1.915em{{}\ar@{~>}[r]&{}}}

\fancypagestyle{plain}{%
	\fancyhf{} % clear all header and footer fields
	\fancyfoot[C]{\sffamily\fontsize{9pt}{9pt}\selectfont\thepage} % except the center
	\renewcommand{\headrulewidth}{0pt}
	\renewcommand{\footrulewidth}{0pt}}
\pagestyle{plain}

 
 \newcommand{\mcom}[1]
 {\marginpar{\raggedleft\raggedright\hspace{0pt}\linespread{0.9}\footnotesize{#1}}}
 \newcommand{\cb}[1]
 {\marginpar{\color{orange}\raggedleft\raggedright\hspace{0pt}\linespread{0.8}\footnotesize{#1}}}
 \newcommand{\hk}[1]
 {\marginpar{\color{purple}\raggedleft\raggedright\hspace{0pt}\linespread{0.8}\footnotesize{#1}}}
 \newcommand{\note}[1]{{ }\mcom{Note}\textbf{#1}}
 
 
 \newcommand{\glem}[1]
 {\MakeUppercase{\scriptsize{\textbf{#1}}}}
 
 \newcommand{\specialcell}[2][c]{%
 	\begin{tabular}[#1]{@{}c@{}}#2\end{tabular}}
 
 \makeatletter
 \def\@xfootnote[#1]{%
 	\protected@xdef\@thefnmark{#1}%
 	\@footnotemark\@footnotetext}
 \makeatother
 
 
 \newcommand{\xmark}{\ding{55}}
 
 
 
 \author{Josh}
 \date{\today}
 
 \usepackage{subfigure}
 \usepackage{amsthm}
 \usepackage[normalem]{ulem}
 \usepackage{amssymb}
 \usepackage{multirow}
 \usepackage{mathrsfs}
 \usepackage{pifont}
 \usepackage{mathtools}
 \usepackage{tikz}
 \usepackage{qtree}
 \usepackage{tikz-qtree}
 %\usepackage{tipa}
 \usetikzlibrary{decorations}
 \usepackage{textcomp}
 \usepackage[normalem]{ulem}
 \usepackage{url}
 \usepackage[all]{xy}
 \usepackage{multicol}
 \usepackage{hanging}
 \usepackage{booktabs}
 \usepackage{setspace}
 \usetikzlibrary{shapes,backgrounds}
 \usepackage{geometry}
 
 
 
 \newtheorem{definition}{Definition}
 \newtheorem{theorem}{Theorem}
 
 
 \usepackage{enumerate}
 %\usepackage{gb4e} \let\eachwordone=\sl
 \usepackage{expex}
 
 
 
 %\newcommand{\denote}[1]{\mbox{$[\![\mbox{#1}]\!]$}}
 \newcommand{\concat}{\mbox{$^\frown$}}
 \newcommand{\ph}{\varphi}
 \newcommand{\vsep}{\vspace{8pt}}
 \newcommand{\linesep}{\rule{6.5in}{.5pt}}
 \def\attop#1{\leavevmode\vtop{\strut\vskip-\baselineskip\vbox{#1}}}
 \newcommand{\denote}[1]{\mbox{$[\![\mbox{#1}]\!]$}}
 \newcommand{\exref}[1]{~(\ref{#1})}
 
 \newcounter{nextsec}
 \newcommand\nextsection{%
 	\setcounter{nextsec}{\thesection}%
 	\stepcounter{nextsec}%
 	\thenextsec%
 }
 \newcommand\nextsubsection{%
 	\setcounter{nextsec}{\expandafter\parsesub\thesubsection\relax}%
 	\stepcounter{nextsec}%
 	\thesection.\thenextsec%
 }
 \def\parsesub#1.#2\relax{#2}
 \def\parsesubsub#1.#2.#3\relax{#3}
 %\input{setup}
 
 %\singlesidestandardsetup
 
 %\parindent 1em
 
 %\input{psfig-scale}
 
 
 \newcommand{\verteq}{\rotatebox{90}{$\,=$}}
 \newcommand{\equalto}[2]{\underset{\scriptstyle\overset{\mkern4mu\verteq}{#2}}{#1}}
 
 
 \newcommand{\secsep}{\hrulefill}
 \renewcommand*{\marginfont}{\small}
 \usepackage{qtree}
 \qtreecenterfalse
 
 \renewcommand{\baselinestretch}{1} %% this is the linespacing
 
 \newcommand{\la}{\langle}
 \newcommand{\ra}{\rangle}
 \newcommand{\lamda}{\lambda}
 \usepackage{framed}

\begin{document}
\begin{center}
 	\thispagestyle{empty}
 	{\Large	\textsc{dissertation prospectus}}
\vfill
\HRule\vspace{.33cm}


\textbf{{\huge The intersection of temporal \& modal interpretation:}\\
{\Large a view from Arnhemland (northern Australia)}}\\\textit{working title}

\HRule
\vfill
{\small \textbf{Josh Phillips}}
\vfill
\textit{\textbf{Committee}}\\
\begin{tabular}{cl}
Claire Bowern (c.) & Yale U.\\
Ashwini Deo (??) & Ohio State U.\\
Hadas Kotek & New York U.\\
Lisa Matthewson (??) & U. of British Columbia\\
María Piñango & Yale U.\\
Judith Tonnhauser (??) & Ohio State U.\end{tabular}
\vfill
\sc Department of Linguistics
	
	Yale University
	
	\today

	\cb{Just include people who are confirmed. ie, me, Maria, Hadas, at present, plus an external person (Lisa Matthewson would be good)}
\end{center}\newpage
\tableofcontents
\section{Motivation}
\textsc{Displacement} --- a stated universal of human language ---  permits us to make assertions that are embedded in different times, locations and possible worlds (e.g. Hockett's `design features of human language' 1960). Prevailing trends in linguistic work --- descriptive, pedagogical, theoretical --- assume a categorical distinction between subtypes of verbal inflection: \textit{viz.} the \textsc{temporal} and \textsc{modal} domains. Whether or not these basic claims are intended as heuristic, they quickly unravel upon close inquiry into cross-linguistic data; a challenge for linguistic theory, and one that a growing body of literature is identifying)\cb{delete ()s}. This ought to\cb{will} become clear in section \ref{phen} of this prospectus.

The empirical focus of the dissertation proposed\cb{del} are\cb{agreement problem} the verbal inflectional systems of a set of languages in the Arnhemland\cb{Change to Arnhem Land throughout} linguistic area of Northern Australia. Arnhemland is `linguistically dense' --- an area of close historic and contemporary contact between unrelated languages. Represented in these languages are various typological curiosities\cb{minimizes the importance - rephrase?} in this domain which have evaded an adequate, unified account. Consequently, I operate under the belief\cb{too weak -- not just your beliefe. Be more direct -- this is an active area of inquiry, we know there are problems with the current explanation, and these languages have systems that crucially provide us with the means of discriminating between theories.} that understanding these systems will help us to nuance a theory of temporomodal displacement.

Crucially, in this work I seek to consider the contribution of studying language change (specifically meaning change) in better understanding the cognitive apparatus that permits for the interpretation of temporomodal devices. It is a starting assumption in this dissertation that `diachronically consecutive grammars are not characterised by radical discontinuities or unpredictable leaps, but that change consists of gradual discrete steps constrained by properties of grammar' (Deo 2006: 5). By hypothesis, then, the investigation of these `steps' and the inference of these `constraints', represent a significant potential source of insight into the linguistic expression and evaluation of event structure, time and possibility.
%(or perhaps amphichronic, in the sense of Kiparsky 2006) approaches to linguistics represent an fount of insight into the relationship between 

This prospectus is organised into four sections: this statement of motivation, followed by an introduction of particular phenomena in Arnhem Land languages in §2. The following section (§3) seeks to identify and situate this work within the broader scholarship of the semantics and interpretation of verbal inflectional categories. The final sections draft a chapter structure and schedule to completion of the proposed dissertation (§§4-5). Following a conclusion which rehearses the prospective contribution of this work in §6, additional data is provided in appendix.

\cb{I think you need a clearer statement here of what you'l be doing. I know you can give that, because you've given me good elevator pitches in the past. Just write one down. And incorporate here the why Kriol and Yolngu Matha that we talked about this afternoon.}

\section{Some phenomena}\label{phen}

As indicated above, Australian data have not been explicitly accounted for in the elaboration of formal semantic work. Temporal and modal phenomena in these languages appear to pose some problems for our models. The proposed dissertation would seek to marshal synchronic and diachronic data to nuance our understanding of the interpretation of temporal and modal operators. §\ref{rop} summarises a diachronically-informed account of the emergence of modal meaning from a temporal frame adverbial. §\ref{yol} provides an comparative overview of the verbal paradigms of some Yolngu languages which are problematic for standard model-theoretic conceptualisations of tense and modal semantics. These phenomena are examples of the work that will be developed through the proposed dissertation. Data and existing descriptive and analytic work are described in this section

\subsection{Kriol. The emergence of {\small\textsc{APPR}}}\label{rop}

Australian Kriol is a contact language spoken through many communities in Northern Australia, including much of southern Arnhem Land. The Ngukurr variety is generally considered to be the `birthplace' of this language, a result of radical language contact between English-based pidgins and a number of Arnhemland substrates in the early twentieth century.


Recent work has shown the apparent recruitment of a temporal frame adverbial \textit{bambai} `$<$ by-and-by' (a lexical item present in many Pacific contact languages) as a marker of so-called \textsc{apprehensional} modality (\textit{see} Angelo \& Schultze-Berndt 2016; Phillips forthcoming.) Apprehensionals are a grammatical category widely represented in Australian (in addition to Austronesian, Amazonian etc.) languages. In the only published work dedicated to a treatment of apprehensionals, Lichtenberk (1995) describes these markers as dually encoding (a) an assertion of the possibility (his ``epistemic downtoning'') and (b) information about negative speaker affect vis-à-vis their prejacent. The sentence pair in (\nextx) shows the possible temporal or modal contribution of \textit{bambai}. 

\pex\textbf{Context:} I've invited a friend around to join us for dinner. They reply:	
\a\begingl\deftagex{pres}\deftaglabel{seq}
\gla yuwai! \textbf{bambai} ai gaman jeya!//
\glb yes! \textit{\textbf{bambai}} 1s come there//
\glft `Yeah! I'll be right there!'//
\endgl
\a\deftaglabel{m}\begingl\deftaglabel{appr}
\gla najing, im rait! \textbf{bambai} ai gaan binijim main wek!//
\glb no 3s okay \textit{\textbf{bambai}} 1s \textsc{neg.mod} finish 1s work//
\glft`No, that's okay! (If I did,) I mightn't (be able to) finish my work!'//
\endgl
	\xe

In (a), \textit{bambai} displaces the reference time of the prejacent slightly forward; the speaker has undertake to join for dinner in the near-immediate future of speech time. In (b), however, the speaker asserts that, in the event that they join for dinner, they may fail to complete their work (a negative outcome).

Similarly, as shown in the sentence pair (\nextx), this apprehensional meaning also appears in past irrealis (i.e. counterfactual) contexts.\cb{punctuation in free translations.}

\pex\a\begingl
\gla ai\textdblhyphen{}bin wotji muvi en \textbf{\textit{bambai}} aibin silip$\sim$silip//
\glb 1s\textdblhyphen\textsc{pst} watch film and \textit{\textbf{bambai}} 1s\textdblhyphen\textsc{pst} sleep$\sim$\textsc{red}//
\glft`I watched a film last night \textbf{then shortly afterwards}, fell asleep'//\endgl

\a \deftagex{sjv}\deftaglabel{A}\begingl
\gla ai\textdblhyphen{}bin dringgi kofi nairram \textbf{bambai} ai bina silip$\sim$silip-bat la wek//
\glb 1s\textdblhyphen{\sc pst} drink coffee night \textit{\textbf{bambai}} 1s {\sc pst:irr} sleep{\sc$\sim$red-ipfv} {\sc loc} work//
\glft`I had coffee last night \textbf{otherwise} I would've slept at work' \hspace*{\fill}(AJ~23022017)//\endgl
\xe

The apprehensional reading shown in both (b) examples above appears to have emerged out of what I term the `subsequential' reading of \textit{bambai}, the latter of which is shared by \textit{bambai}'s many cognates in related Pacific contact varieties (and the English etymon). These apprehensional readings appear to emerge in the contexts summarised in table \ref{triggers} below.



\begin{table}[h]\caption[Semantic operators]{Semantic operators\footnotemark{} that give rise to modalised readings of \textit{bambai}}\label{triggers}\small\centering
	\begin{tabular}{ll|l}\toprule
		\glem{Gloss} & \textbf{Morph} & \textbf{\textit{Example}} \\\midrule\midrule
		\textsc{irrealis} & \textit{garra} & \specialcell[l@]{\textit{ai\textbf{rra} dringgi kofi \textbf{bambai} mi gurrumuk}\\`I'll have a coffee or I might fall asleep'}\\\midrule
		\textsc{prohibitive} & \textit{kaan} & \specialcell[l@]{\textit{ai \textbf{kaan} dringgi kofi \textbf{bambai} mi nomo silip}\\`I won't have a coffee or I mightn't sleep'}\\\midrule
		\textsc{counterfactual} & $\underset{\textsc{pst:irr}}{bina}$ & \specialcell[l@]{\textit{ai \textbf{bina} dringgi kofi nairram \textbf{bambai} aibina silip}\\`I had a coffee last night or I might've fallen asleep'}\\\midrule
		\textsc{imperative} & $\varnothing$ & \specialcell[l@]{\textit{yumo jidan wanpleis \textbf{bambai} mela nogud\footnotemark}\\`Youse sit still or we might get cross'}\\\midrule
		\textsc{prohibitive} &  $\underset{\textsc{impr}\!\!}{\varnothing}$[\textit{nomo}] & \specialcell[l@]{\textit{\textbf{nomo} krosim det riba, \textbf{bambai} yu flodawei}\\`Don't cross the river or you could be swept away!'}\\\midrule
		\textsc{generic} & $\varnothing$ &\specialcell{\textit{im gud ba stap wen yu confyus, \textbf{bambai} yu ardim yu hed}\\`It's best to stop when you're confused or you'll get a headache'}\\\midrule
		\textsc{negative} & $\underset{\textsc{gen}}{\varnothing}$[\textit{nomo}] & \specialcell{\textit{ai \textbf{nomo} dringgi kofi enimo \textbf{bambai} mi fil nogud}\\`I don't drink coffee anymore or I feel unwell'}\\\midrule\midrule
		{\sc conditional}&\textit{if}&\specialcell[]{\textit{\textbf{if} ai dringgi kofi \textbf{bambai} ai kaan silip}\\`If I have coffee, then I mightn't sleep'}\\\bottomrule
		
		
	\end{tabular}
\end{table}
\footnotetext[9]{This does not entail the claim that these operators are in any way semantic primitives.}
\footnotetext[10]{This example due to Dickson (2015:168 [KM~20130508]).}

 The formal machinery proposed in (\nextx) goes part of the way to providing a unified treatment of these two uses.



\pex\a\deftagex{ssqIsem}\textbf{Subsequential Instantiation} (intensionalised)\\$\text{{\sc subseqInst}}(P,t,w)\leftrightarrow\exists t^\prime:t^\prime\succ{}t\wedge\text P(t^\prime)(w)\wedge\mu(t,t^\prime)\leq s_c$


A subsequentiality relation {\sc subseqInst} holds between a predicate $P$, reference time $t$ and reference world $w$ iff the $P$ holds in $w$ at some time $t^\prime$ in the future of $t$.\\Additionally they assert that the temporal distance $\mu(t,t^\prime)$ between reference and event time must be below some contextually provided standard of `soon-ness' $s_c$. %modulo vagueness??


\a $\denote{bambai}^{t,w}\!\underset{\text{def}}{=}\lambda f\lambda g\lambda P.\exists w\,^\prime\in\boldsymbol{Best}(f,g,t,w)\wedge\text{{\sc subseqInst}}(P,t,w\,^\prime)$

\textit{bambai} asserts that there exists some world $w^\prime$ in a set of worlds that are optimal with respect to a contextually-determined modal base $f$ and ordering source $g$ in the reference context $\la t,w\ra$. It additionally asserts that the {\sc subsequential instantiation} relation holds between that world $w^\prime$, the prejacent $P$, and the reference time $t$.
\xe


My second qualifying paper proposed that that temporal frame (sc. `subsequential') uses of \textit{bambai} differ from apprehensional uses in terms of the \textsc{conversational background} $f,g$ that is selected (cf. Kratzer 1981 a.o). These conversational backgrounds must be retrieved from context, both linguistic and nonlinguistic.

With the entry in (\lastx), we can formalise the intuition that, when predicating into an unsettled timeline \denote{bambai $p$} represents an epistemic claim. We model this by claiming that under these conversational backgrounds, \textit{bambai} has selected an \textit{epistemic} modal base $f_{\text{epist}}$ and a stereotypical ordering source. These conversational backgrounds are formalised in (\nextx), adapting liberally from Kratzer (2012: 37,39-40 a.o.)%\mcom{It is essential to find some way of intersecting the (negation of???) the antecedent with the modal base otherwise this is literally just $\lozenge$. What we have here however does give \denote{bambai $P$}, we also need $\denote{Q bambai P}$}

\pex\a $\bigcap f_{\text{epist}}(w)(t)=\{w^\prime\mid w^\prime\text{ is compatible with what S knows in $w$ at $t$\}}$
\a $g_{\text{s'typ}}(w)=\{p\mid p\text{ will hold in the `normal' course of events in }w\}$.
\a$g(w)$ then induces an ordering $\leq_{g(w)}$ on the modal base:

\hspace{-.45cm}$$\forall w^\prime\!,w^{\prime\prime}\!\in\bigcap f_{\text{epist}}(w)(t):w^\prime\!\boldsymbol{\leq_{g(w)}}\!\!w^{\prime\prime}\leftrightarrow\{p:p\in g(w)\wedge ^{\prime\prime}\in p\}\subseteq\{p:p\in g(w)\wedge w^\prime\in p\}$$\\
\hspace{.35cm}
For any worlds $w^\prime$ and $w^{\prime\prime}$, $w^\prime$ is `at least as close to an ideal' than $w^{\prime\prime}$ with respect to $g_{\text{s'typ}}(w)$ (\textit{i.e.} it is at least as close `normal course of events') if all the propositions of $g(w)$ true in $w^{\prime\prime}$ are also true in $w^\prime\!$.
\a \textbf{\textit{Best}}$(f_{\text{epist}},g_{\text{s'typ}},t,w)$ then returns just that subset of worlds that are both consistent with what the Speaker knows at $t$ in $w$ that are closest to the normal unfolding course of events in $w$.
\xe

The so-called subsequential TFA use of \textit{bambai}, then, is maintained when the discourse context fails to R-implicate that the Speaker is making a non-modalised claim (cf. Grice's Quality$_2$)\footnote{\textit{I.e.} \textit{`Do not say that for which you lack adequate evidence'} (Grice 1991: 27, a.o.)}. In these cases the intensional contribution of \textit{bambai} can be captured by claiming that it quantifies (trivially) over a \textit{metaphysical} modal base and a \textit{totally realistic} ordering source (adapted partially from Kratzer 2012.)\footnote{This component of the analysis has additionally benefited greatly from Deo (2017).}\cb{be a little more specific in the footnote - e.g. owes inspiration from Deo 2017? Or uses the technical suggestion of Deo 2017? etc.}

\pex
\a $\bigcap f_{\text{meta}}(w)(t)=\{w^\prime\mid w^\prime\simeq_t w\}$
\a $ g_{\text{real}}(w)=\{w\}$
\a \textbf{\textit{Best}}$(f_{\text{meta}},g_{\text{real}},t,w)$ then simply returns a set of worlds which are historical alternatives to $w$ at $t$ (\textit{i.e.} those that best comply with all the propositions that uniquely characterise $w$).
\xe


\subsubsection{selecting a conversational background}
Ultimately, the interpretation of \textit{bambai} must be anaphoric on discourse context, apprehensional readings restricted to propositions made in the `nonfactual speaker mood' (cf. Roberts 1989). 

\pex\textsc{\textbf{The omniscience restriction.}} It is notable that in the apprehensional cases presented above, those where predication into an unsettled timeline has been triggered by one of the operators presented in Table \ref{triggers} (\textit{p.}\pageref{triggers} above), modalisation with respect to a non-settled property cannot reasonably select for the set of conversational backgrounds presented in (\lastx). Such an operation would require the participants to be able to retrieve all propositions that are true in and characteristic of worlds with respect to a vantage point in the future or to be able to calculate all the ramifying consequences of eventualities that might have obtained in the past. This condition allows us to unify the modalised and non-modalised readings of \textit{bambai}.
\xe


The emergence of new data shows that the omniscience restriction as formulated above is too strong; it fails to predict the felicity of the subsequential reading in (\getfullref{pres.seq}). The interpretation of \textit{bambai} then seems to provide evidence for the necessity of considering both extralinguistic factors (i.e. world knowledge and notions of the \textit{common ground} (see Stalnaker 1979) and intersentential dependencies and speaker mood (i.e. discourse structuring, \textit{e.g.} Kamp 1979; Heim 1981; Roberts 1989, 1998.) The insights from  information structure and dynamic semantics literatures can provide a vital framework to help us to understand the heuristic devices which allow speakers to identify an antecedent to \textit{bambai} and to disambiguate these readings.

\subsubsection{a diachronic perspective}\cb{punctuation in headings}
In addition to the synchronic analysis described above, \textit{bambai} provides a case study of the clear emergence of modal meaning from an erstwhile temporal adverbial in particular supporting contexts. I suggest that a discussion of the apparent semanticisation of apprehensional readings can be best explained in terms of invited inference theory (e.g. Eckhardt\cb{spelling?} xxxx, Deo 2015a). In addition to these insights, however the unified denotation that is proposed above also suggests the existence of a conceptual link between (relative) future marking and epistemic modalities, in keeping with observations made elsewhere on the semantics of the future tense (e.g. Copley 2001, Palffy-Muhoray 2016 i.a.)\cb{the heading is a diachronic perspective, but this para talks about the conceptual link -- which isn't the same. Maybe flesh this out a bit more, in terms of the debate we talked about between Condoravdi, Matthewson, etc? Start with that earlier, then talk about the contribution that Yolngu brings?}

\subsection{W. Yolŋu Matha}\label{yol}\cb{you'll need to introduce this a bit more.}

The verbal inflectional paradigms of contemporary Yolŋu languages can be reconstructed to proto-Yolŋu (e.g. Bowern 2009). Notwithstanding this demonstrated cognacy, there is significant cross-linguistic variation reported in the distributions and `meanings' associated with each of the inflectional categories. Where eastern and southern language varieties are described as having `basic tense categories' that are `semantically straightforward' (Heath 1980 on Ritharrŋu), an adequate treatment of the morphosemantics of tense marking in the related Yolŋu languages spoken in western Arnhem land appears to be much more elusive. \cb{researchers draw on a variety of vocabulary related to tense, aspect, and reference points} Consider to begin, the minimal pair in (\nextx) below.\cb{from Wangurri}

\pex
\a\begingl
\gla nhän \textbf{gayŋa} ŋirrima-ḻi \textbf{ŋarra}//
\glb 3s \textsc{ipfv} home-\textsc{all} go.I//
\glft`they went home' \textbf{\textit{or}} `they're going home'//
\endgl 
\a\begingl
\gla nhän \textbf{ŋarru} ŋirrima-ḻi \textbf{ŋarra}//
\glb 3s \textsc{irr} home-\textsc{all} go.I//
\glft`they will go home'\hfill(\texttt{[dhg]} McLellan 1992:154)//
\endgl
\xe

In (\lastx), the verbal inflection alone (glossed here as \textbf{I}) fails to disambiguate tense altogether: it provides no information on whether the event described (\textit{viz.} it \textsc{go} home) obtains in an interval preceding, subsequent to or overlapping with the speech time. This information must be provided by context or by aspectual/modal auxiliary.

Additionally, Western Yolŋu varieties exhibit a phenomenon which Comrie refers to as `cyclic tense' --- an ostensibly areal feature and crosslinguistic rarity\footnote{Little typological work has been done on cyclical and metrical tense systems. Nevertheless, in his important (1979) cross-linguistic survey of tense systems, Comrie identifies these systems as uncommon, pointing only to Burarra (\texttt{[bvr]} Maningrida: W Arnhem), and perhaps Kisksht (\texttt{[wac]} Chinookan: Columbia River) and Bamiléké (\texttt{[ybb]} Niger-Congo: Cameroon).}\cb{footnote point -- for now, check if there's anything written in detail on these languages. I remember trying to find something on Bamileke and failing.} that it shares with the neighbouring, unrelated Maningrida language family. Exponence of \textit{cyclic tense} is discussed in more detail in §\ref{tense§} below, although refers to a phenomenon where tenses have `discontinuous time reference', ostensibly arising from `the combination of two oppositions, one an absolute cut-off point between today and earlier than today, the other between recent and remote within each of these two time frames' (89). The example that follow in (\nextx) demonstrate this discontinuity: three past tense events; the event in (a), temporally intermediate to the other two is inflected with the primary verb form whereas those in (b,c) are inflected in the tertiary form.


\pex
\a\deftagex{pasts}\begingl\gla yo barpuru-ɲ ŋarra ŋaɲa nhaa-\textbf{ma}-ɲ (*nhaaŋal)//
\glb	yes, yesterday{\sc-prom} 1s 3s{\sc.acc} see-\textbf{I}-{\sc prom}//
	\glft`Yes, I saw him yesterday'//\endgl
\a\begingl\gla ŋe gaathur ŋarra ŋaɲa nhaa-\textbf{ŋal} goḏarr dhiyal//
\glb	yes, today 1s 3s{\sc.acc} see-\textbf{III} morning {\sc prox-loc}//
	\glft`Yes, I saw him here this morning'//\endgl
	\a\begingl
	\gla maarrma ga-\textbf{n} malwan-dja dhaara-\textbf{n} yindi maṉḏa-ɲ//
	\glb two {\sc ipfv-\textbf{III}} Hibiscus-{\sc prom} stand-\textbf{III} big 3d-{\sc prom}//
	\glft`Two big Hibiscus flowers were growing there' (at some place in the speaker's youth)//\hfill(Wilkinson 1991: 339)
	\endgl
\xe


All Yolŋu languages are described as having (at least) four cognate verbal inflection classes (`forms'). Existing labels for these verbal inflections are summarised in Table \ref{metacomp} below. This prospectus adopts the semantically agnostic terminology adopted in Wilkinson (1991) and Lowe (1996), who enumerate the four forms, eschewing semantically-motivated labelling/metalanguage. The descriptive inadequacy of other scholars' heuristic metalinguistic approaches will be shown in detail in what follows. %cryptic
 


% Please add the following required packages to your document preamble:
% \usepackage{booktabs}
\begin{table}[h]
	\centering
	\caption{Existing classifications of inflectional Yolŋu classes}
	\label{metacomp}
	\begin{tabular}{@{}clll@{}}
		\toprule
		\multicolumn{2}{c}{\textbf{\textsc{Dhuwal(a)}}}                                                                                                                      & \multicolumn{1}{c}{\textbf{\textsc{Wangurri}}}                                    & \multicolumn{1}{c}{\textbf{\textsc{Ritharrŋu}}}                             \\ \midrule
		\textbf{\begin{tabular}[c]{@{}c@{}}Wilkinson/Lowe\\ {\texttt{[djr]/[guf]}}\end{tabular}} & \textbf{\begin{tabular}[c]{@{}l@{}}Morphy\\ {[}\texttt{dwu}{]}\end{tabular}} & \textbf{\begin{tabular}[c]{@{}l@{}}McLellan\\ {[}\texttt{dhg}{]}\end{tabular}} & \textbf{\begin{tabular}[c]{@{}l@{}}Heath\\ {[}\texttt{rit}{]}\end{tabular}} \\\midrule
		I                                                                                     & \textsc{unm}                                                                 & \textsc{neu}                                                                   & \textsc{pres}                                                               \\
		II                                                                                    & \textsc{pot}                                                                 & \textsc{irr}                                                                   & \textsc{fut}                                                                \\
		III                                                                                   & \textsc{perf}                                                                & \textsc{pfv}                                                                   & \textsc{pst}                                                                \\
		IV                                                                                    & \textsc{pst} \textsc{non-indic  }                                                     & \textsc{hab.pfv}                                                               & \textsc{pst.pot  }                                                          \\ \bottomrule
	\end{tabular}
\end{table}



\subsubsection{Descriptive work on the Yolŋu verbal paradigm}
While some important attempts have been made to describe the functional contribution of Yolŋu verbal inflections, these ultimately amount to listed distributions. Table \ref{formdesc} below summarises the functions of the verb forms below and their interactions with auxiliaries that appear to be shared by documented western Yolŋu languages \texttt{[djr]/[guf]/[dhg]}.

% Please add the following required packages to your document preamble:
% \usepackage{s}
\begin{table}[h]
	\centering
	\caption{Preliminary summary of distribution of verbal inflectional forms and their interaction with auxiliaries \mbox{(adapting from Wilkinson 1990 and Lowe 1996 on Dhuwal/a)}}
	\label{formdesc}
	\begin{tabular}{@{}clllllll@{}}\toprule
		\textbf{}    & $\varnothing$                                                                                  & \textit{ŋuli} \textsc{`hab/hyp'}                                                                   & \textit{dhu} \textsc{`fut'}                                                                    & \textit{yaka/bäyŋu} `\textsc{neg}'             & \textit{balaŋ} \textsc{`irr'}   &  \\\midrule\midrule
		\denote{\textbf{I}}   & \begin{tabular}[c]{@{}l@{}}•\textsc{pres}\\•\textsc{pst} (*today)\end{tabular}                                  & \textsc{pres.hab}                                                                & \begin{tabular}[c]{@{}l@{}}•\textsc{fut} today\\ •\textsc{fut} indefinite\end{tabular}         & $^?*$                  & ?           &  \\\midrule
		\denote{\textbf{II}}  & \textsc{imper}                                                                                        &                                                                                & \begin{tabular}[c]{@{}l@{}}\textsc{fut} definite\\ (\textit{incl}. tomorrow)\end{tabular} & $\supset$\denote{I}    &             &  \\\midrule
		\denote{\textbf{III}} & \begin{tabular}[c]{@{}l@{}}•\textsc{pst} today\\•\textsc{pst} unspecific\\•$\psi$ states$^{§6.5.1.2}$\end{tabular} &  *                                                                              & *                                                                          & *                           & *           &  \\\midrule
		\denote{\textbf{IV}}  &                                                                                                 & \begin{tabular}[c]{@{}l@{}}•distant \textsc{pst.hab}\\•counterfactual\end{tabular} & *?                                                                         & $\supset$\denote{III} & \textsc{pst.irr}\\ \bottomrule
	\end{tabular}
\end{table}

\subsubsection{Questions in the semantics of Yolŋu verbal inflection}
One of the immediate implications of the data presented in (\getfullref{pasts}) above---and the temporal discontinuity of the availability of the primary and tertiary inflectional forms---is the apparent insufficiency of presuppositional-indexical treatments (sc. the best candidate for a `standard formalism') of morphological tense for predicting this distribution. This treatment is provided in (\nextx) below, adapting liberally from Kratzer 1998.\cb{liberally adapted in what way?}
\pex\a $\denote{\textsc{\textbf{pst}}}^{g,c}=\lambda t:t\prec t_0.t$\\\textsc{past} is only defined if $c$ provides a $t$ that precedes speech time $t_0$.\\If defined then \denote{\textsc{pst}}$=t$
\a$\denote{\textsc{\textbf{prs}}}^{g,c}=\lambda t:t\circ t_0.t$\\\textsc{present} is only defined if $c$ provides an interval $t$ that contains speech time $t_0$.\\If defined then \denote{\textsc{prs}}$=t$
\a$\denote{\textsc{\textbf{fut}}}^{g,c}=\lambda t:t\succ t_0.t$ \\\textsc{past} is only defined if $c$ provides a $t$ that follows speech time $t_0$.\\If defined then \denote{\textsc{fut}}$=t$
\xe

I describe these treatments as presuppositional-indexical given that they contain two formal components:\cb{you'll need to unpack this a bit more; good practice at writingsemantics for a general audience.}
\begin{itemize}
	\item a presuppositional component that restricts the time of property instantiation relative to evaluation time, and
	\item an identity function $\la i,i\ra$; the instantiation time is provided contextually (i.e. it is not liguistically specified/is not in the denotation of the tense marker.)
\end{itemize}

In order to make a purely presuppositional-indexical treatment of tense work for Yolŋu, the denotation of the primary inflectional form would then have to resemble something like (\nextx) below.


\pex $\denote{\textbf{I}}=\begin{cases}t\in today\leftrightarrow t\succcurlyeq t_0\\
t\notin today \leftrightarrow t\prec t_0\wedge\mu(t,t_0)<s_c
\end{cases}$
\xe
The defense of an preliminary analysis like that given in (\lastx) entails:
\begin{enumerate}[(a)]
	\item motivating the introduction of a privileged interval \textit{today} into Yolŋu temporal ontology;
	\item motivating the joint grammaticalisation of these disjoint presuppositions (\textbf{cyclicality}); and
	\item understanding whether and how a contextual standard is retrieved in order to predict in which past contexts the verb is inflected with \textbf{I} in lieu of \textbf{III} (\textbf{metricality}).
	
\end{enumerate}

Consequently, an analysis that treats the verbal inflectional categories in Yolŋu as `morphological tense' is unwieldy and likely fails to capture a deeper generalisation about the interpretation of these categories by Yolŋu speakers. The following questions remain:
\begin{enumerate}[(i)]
	\item What is the proper semantics for Yolŋu inflectional categories?
	\item How are temporal relations encoded and understood in Yolŋu?
\end{enumerate}
\subsubsection{A treatment of `cyclic tense'}

In her treatment of the Wangurri verb system (1992), McLellan claims that `[t]he concern of Wangurri is not to locate a process in time, but in reality' (153) as a governing principle behind the inflectional system that is `basic to [a verb's] finiteness.' For McLellan, then, the distinction between reality and irreality: whether `a process or state is based in reality' or not is crucial for clausal interpretation.

These intuitions are wanting of a formal treatment but form a primary source of data for an understanding about the intersections between temporal and modal interpretation.

Ritter and Wilstchko's \textit{parametric substantiation hypothesis} (2009, 2014) suggests that the category \textsc{infl} varies cross-linguistically. They propose that different languages variably associate this projection with temporal, spatial or participant marking. This type of analysis challenges seeks to account for the optionality or absence of tense morphology across languages.

\section{Work at the intersection of tense \& modal semantics}
\begin{itemize}
	\item Condoravdi 2002 (TMA stacking)
\item	von Fintel (and/or Kratzer) i.a. on the past-subjective stuff
\item Roberts and discourse subordination
\item Rullman \& Matthewson forthcoming `towards a theory of modal-temporal interaction'

\item Tonnhauser
\item Matthewson On the (non)future orientation of modals
\item Brenda \textsc{Laca} On modal tense and tensed modals  2008\textsc{ms}
\item Iatridou 2000 grammatical ingredients of counterf \textit{LI}
\item Hacquard 2006 PhD \textit{aspects of mod}
\item Eide 2003 Modals and Tense \textit{SuB}
\item Xling sems of tense mood mod (ed.	) \texttt{P294.5.C76X 200}
\item Arregui PhD dissertation \textit{acc of poss wolds: role of T and Asp}
\item Abusch Circ and temp dep in counterf modals \textit{NLS20}
\end{itemize}	
	\subsection{Diachrony}
	\subsection{Tense relative, cyclic and metrical}\label{tense§}
\begin{itemize}
\item 	Comrie 1979

Accords cyclic tense systems `marginal status in the theory' given their rarity (i.e. that they challenge a stated universal)
\item 	Hyman 1980 (Yɛmba relative tense)
\item Silverstein 1974 (Chinook tense evolution --- esp Kiksht cyclicity §6.3)
\item Matthewson 2006 temp semantics in a supp tenseless language
\item Tohauser 2015 X-linguistic Temp Ref, 2011 Temp ref in guaraní (tenseless lang)
\end{itemize}

	
	
\subsection{The semantics landscape in Australia}
\begin{itemize}
\item the effect of NSM on semantic theory in australia.
\item Verstraete 2006 --- past and irrealis
\item tense and time (eg AJL special vol, ALS special session)


\end{itemize}

\section{A chapter outline}
\section{$\boldsymbol{\langle \mathcal{T},\mathcal{I},\prec,\sqsubseteq,\mathcal{Q},t\!*\rangle}$}
\newpage
\section{Appendix: Yolŋu data}
\subsection{Primary inflection}
\pex\begingl\deftagex{prI}
\glpreamble\textsc{Present}//
\gla ŋarra marrtji-n dhiyaŋu-n bala//
\glb 1s go.I-\textsc{seq} \textsc{prox.erg-seq} away//
\glft`I am going now'\hfill(256)//
\endgl\xe

\pex\begingl\deftagex{pstI}
\glpreamble\textsc{Past}//
\gla yo, barpuru-ny ŋarra ŋanya nhä-ma-ny//
\glb yes yesterday\textsc{-prom} 1s 3s.\textsc{acc} see-1-\textsc{prom}//
\glft yeah, I saw him yesterday\hfill(339)//
\endgl\xe

\pex\a\begingl\deftagex{habI}
\glpreamble\textsc{present habitual}//
\gla ŋunhi ŋilinyu ŋuli ga warkthu-n maṉḏa waŋgany-ŋur//
\glb \textsc{texd} 1d\textsc{.incl} \textsc{hab} \textsc{ipfv.I} work-I 3d one-\textsc{loc}//
\glft`us two, who are working in the one work'\hfill(348)//\endgl
\a\begingl\glpreamble\textsc{Past habitual}//
\gla ŋarra ŋuli ga rur'yun munhawumirri yan bitjan~bili//
\glb 1s \textsc{hab} \textsc{ipfv.I}//
\glft `I always get up early in the morning'\hfill(348)//
\endgl\xe

\pex\begingl\deftagex{futI}\glpreamble \textsc{today Future}//
\gla yalala ŋarra dhu nhokal lakara-m//
\glb later 1s \textsc{fut} 2s.\textsc{obl} tell-I//
\glft`I'll tell you later (today)'\hfill(346)//\endgl\xe

\subsection{Secondary inflection}
\pex\deftagex{imprII}\begingl\glpreamble\textsc{Imperative}//
\gla yaka waŋi!//
\glb \textsc{neg} talk.II//
\glft`don't speak!'\hfill(360)//\endgl\xe

\pex\textsc{Future}\deftagex{futII}
\a\begingl\glpreamble\textsc{without \textit{dhu}}//
\gla ŋayi boŋguŋ nhini ŋäku ŋarra-ny ŋunhal yirrkala//
\glb 1s tomorrow sit.II hear;II 1s\textsc{-acc} \textsc{dist.loc} \textsc{name}//
\glft`They will be there at Yirrkala listening to me (in several weeks' time)'\hfill(340)//
\endgl
\a\begingl\glpreamble\textsc{with \textit{dhu}}//
\gla goḏarr'nha ŋarra dhu nhuŋu dhäwu-ny lakara-ŋ//
\glb tomorrow\textsc{-seq} 1s \textsc{fut} 2s\textsc{.dat} story\textsc{-prom} tell-2//
\glft`I'll tell you the story tomorrow'\hfill(346)//\endgl
\a\begingl\gla yalala-ŋu-mirri-y ŋula~nhätha ŋarra dhu nhokal lakara-ŋ//
\glb later-ŋu\textsc{-prop-erg} sometime 1s \textsc{fut} 2s\textsc{.obl} tell-2//
\glft`I'll tell you the story tomorrow'\hfill(346)//\endgl\xe

\pex\begingl\deftagex{negII}\glpreamble \textsc{Negative recent past}//
\gla bäyŋu ŋayi gi nhini barpuru//
\glb \textsc{neg} 3s \textsc{ipfv.II} sit.II yesterday//
\glft`They weren't staying here yesterday'\hfill(357)//\endgl\xe


\subsection{Tertiary inflection}
\pex\textsc{Past}\deftagex{pstIII}
\a\deftaglabel{today}\begingl\glpreamble\textsc{today}//
\gla ŋe gaathur ŋarra ŋaɲa nhaa-\textbf{ŋal} goḏarr dhiyal//
\glb	yes, today 1s 3s{\sc.acc} see-\textbf{III} morning {\sc prox-loc}//
\glft`Yes, I saw him here this morning'\hfill(339)//\endgl
\a\deftaglabel{dist}\begingl\glpreamble\textsc{distant/non-specific}//
\gla maarrma ga-\textbf{n} malwan-dja dhaara-\textbf{n} yindi maṉḏa-ɲ//
\glb two {\sc ipfv-\textbf{III}} Hibiscus-{\sc prom} stand-\textbf{III} big 3d-{\sc prom}//
\glft`Two big Hibiscus flowers were growing there' (at some place in the speaker's youth)//\hfill(339)
\endgl\xe


\pex\textsc{$\Psi$ uses}


`The description of such states rel. to the moment of speech appears to invoke a general temporariness to the state. it is quite poss with a pred such as \textit{rirrikthu-N} `be sick, in pain' for sb to be seen as in that state for a long period of time. Then \textbf{III} is not required. the use of\textbf{ I} in this context however would appear to be relative to `nowadays' rather than the actual moment of speech... perfective.' (\textbf{I}  cooccurs with \textit{gana} to reduce affectedness.)

\a\begingl\gla ŋarra dhuwal/dhika djawaryu-rr/rerrikthu-rr/djanŋarrthi-n//
\glb 1s \textsc{prox/indefp} be.tired-III/be.sick-III/be.hungry-III//
\glft`I'm (a bit) tired/sick/hungry'\hfill(278)//\endgl
\a\begingl\gla bili djawar'yu-\textbf{rr}-a//
\glb \textsc{complv} be.tired-III//
\glft`They're already tired'\hfill(365)//\endgl
\xe
\subsection{Quarternary inflection}
\pex\deftagex{nuliIV}\textsc{Cooccurrences with \textit{ŋuli} `hyp/hab'}\a\begingl\glpreamble\textsc{Past habitual}\deftaglabel{hab}//
\gla ga ŋayi-ny \textbf{ŋuli} yarrupthu-na-n ganybu-lil-a dharpu-nha-lil ga linyu-ny \textbf{ŋuli} ḏuwatthu-na-n...//
\glb  and 3s-\textsc{foc} \textsc{hab} go.down-IV-\textsc{seq} net\textsc{-all-seq} sew-IV\textsc{-all} and 2d\textsc{.incl.foc} \textsc{hab} go.up-IV\textsc{-seq}//
\glft`And then she  would go and sew fishing nets and we two  would go up...'\hfill(350)//
\endgl
\a\deftaglabel{hyp}\begingl\glpreamble\textsc{`Hypothetical' (counterfactual)}//
\gla ŋäthil ŋarra \textbf{ŋuli} \textbf{balaŋ} ḻiya-ŋamaŋamayu-n-mi-nya bala ŋarra \textbf{balaŋ} waŋa-nha-n//
\glb earlier 1s \textsc{hyp} \textsc{irr} have.idea-1-\textsc{r/r}-\textbf{IV} then 1s \textsc{irr} speak-\textbf{IV}-\textsc{seq}//
\glft`Had I thought of it before, I would've spoken'//
\endgl\xe
\end{document}
