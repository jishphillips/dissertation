{\sf\begin{center}
	Abstract\\
	
	\mbox{}\\
	
	{\Large At the Intersection of Temporal \& Modal Interpretation:}\\
	
	{\large Essays on Irreality}\\
	
	\mbox{}\\
	
	Josh Phillips
	
	2021
\end{center}\doublespacing
\pagenumbering{gobble}
\noindent This work is chiefly concerned with the semantics of linguistic categories including tense, modality and negation and the relationships between them. In particular, how do they interact in order to ``displace'' discourse and to talk about situations remote from the time \& place where they're produced? What gets conventionally encoded in linguistic expressions (semantics)? And what's the role of discourse context and extralinguistic factors (pragmatics) in performing these operations?

The current thesis contains three connected (but independent) components; each explores different sets of data in view of understanding particular types of displacement phenomena --- that is, how, in a given discourse context, reference is established to different possible worlds and different times. In other words, we are concerned with the interactions between temporal reference, modal reference and negation/polarity, and the linguistic phenomena that these give rise to. Methodologically, these projects also engage with diachronic considerations in view of explaining variation and change across spatially and temporally separate language varieties. This is motivated by the desiderata formulated by the \textsc{amphichronic program} --- that is, I assume that studying changes in language use over time has something to teach us about synchronic systems and \textit{vice versa}, all in the service of developing an understanding of human language as a cognitive system.


Each of these three component ``essays'' considers data from a number of languages spoken in Aboriginal Australia --- particularly Yolŋu Matha and Australian Kriol --- on the basis of both published and original data, collected on-site in the \textit{Top End} and in consultation with native speakers. While there is a rich tradition of Australian language description, little Australian language data has been brought to bear on the development of formal theories of meaning. 

Data from these languages promise to challenge and enrich the methodological and theoretical toolbox of formal semantics. Equally, it is a general contention throughout this work that formal perspectives hold exceptional promise in terms of better understanding the range of linguistic diversity exhibited across Australian languages and developing cross-linguistic typologies of the expression of grammatical categories.

\textbf{‘The emergence of apprehensionality in Australian Kriol’} considers the semantics of the adverb \textit{bambai} in Australian Kriol, a creole language spoken by indigenous populations across northern Australia. Derived from English archaism \textit{by-and-by}, % cognates of \textit{bambai} are found across contact varieties in the south Pacific. 
Kriol has retained the ``temporal frame'' use that is found in other South Pacific contact varieties (roughly `soon afterward'), while  also having developed an identifiable ``apprehensional'' use. Apprehensionals---an understudied, if cross-linguistically well-documented category---are taken to modalize their prejacent while implicating their speaker's negative attitude vis-à-vis the possibility described in the prejacent. This essay proposes an unified analysis of the meaning contribution of \textit{bambai}, analyzing the item as unambiguous and claiming that, synchronically, the apprehensional reading ``emerges'' reliably in discourse contexts where the truth of its prejacent is \textit{not presumed settled} as a result of standard assumptions about pragmatic reasoning. Diachronically, it is shown that a similar set of processes led to the generalisation and conventionalization of \textit{bambai}'s meaning components.

\textbf{‘The semantics of the Negative Existential Cycle’} represents a semantic treatment of another little-theorized but cross-linguistically attested cyclic change as it is instantiated in a number of Australian (Pama-Nyungan) language (sub)families. The \textit{Cycle} involves the recruitment of a ``special'' nominal negative element which diachronically displaces an older sentential negator. In this essay, the \textsc{privative}---a nominal case marking described in many Australian languages---is analysed as a negative quantifier. The \textit{Cycle}, then, is understood as the progressive generalisation in the quantificational domain of a negative quantifier: privatives scope over nominalized event descriptions and ultimately over full sentences, at which stage they have encroached into the domain of ``standard'' negation.

\textbf{‘Reality status \& the Yolŋu verbal paradigm’} contains a description of and formal proposal for strategies of expressing temporal and modal categories in Western Dhuwal(a), a Yolŋu language of northern Arnhem Land. Crucially, this language exhibits a number of puzzling phenomena --- in particular, \textit{cyclic tense} and the \textit{neutralization of reality status marking in negative sentences}. As a consequence of these phenomena, the four inflectional categories that constitute \textsc{wd}'s verbal paradigm have been treated as unanalyzable from a compositional perspective. Further, neither of these phenomena has received attention in the formal semantic literature. Consequently, this essay represents the first formal proposal for the semantics \textsc{wd} inflectional paradigm (as instantiating a cyclic \textsc{tense} system and an \textsc{irrealis} mood which is licensed by negation) as well as the first formal analysis of these two typological phenomena.


}

\pagebreak