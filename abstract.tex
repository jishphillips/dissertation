This project presents a theory of the temporal and modal expression and the interaction of tense and modality devices on the basis of original data elicited in a number of language varieties spoken in Arnhem Land, Northern Australia.

The primary empirical focus of the dissertation is on Western Dhuwal(a), a Yolŋu language with four verbal inflections that are licensed in a number of contexts not straightforwardly predicted by existing accounts of tense, aspect or modality. On the basis of new data collected from native speakers during fieldwork in Northern Australia, I develop a formal semantic analysis of how this morphological material---in concert with both contextual information and other lexical and grammatical items---contributes to temporomodal expression and displacement in discourse. 

Of particular note for theories of temporal and modal expression are two related phenomena: \textbf{cyclic tense} and \textbf{negation-based mood asymmetries}. Both of these have been documented in previous scholarship, although neither has been explained by existing formal semantic frameworks.

 \textsc{Cyclicity}---a phenomenon named by \citet[91]{Comrie1983}%, who explicitly accords it `a marginal status in the overall theory [of tense]' on the basis of its rarity
 ---refers to cross-linguistically uncommon tense marking systems in which the temporal intervals compatible with given markers are discontinuous. 
While current theories of temporal expression that have been brought to bear on similar phenomena (accounts based on interactions with situation and/or viewpoint aspect, sequence-of-tense effects and pragmatic shifting), these fail to predict the distribution of Western Dhuwal(a) verbal inflections. I propose a unified, interval-semantic analysis which can be shown to capture the temporal contributions of each marker, a finding that is additionally supported by the language's interval-denoting demonstrative inventory. Here I argue that, while the temporal semantics of these markers ostensibly diverges sharply from that of tense morphology cross-linguistically, this phenomenon in fact points to the semanticisation of universal pragmatic and discourse-structural norms (cf. \citealt{Culioli1980}).
 
Additionally, in most contexts, irrealis, future and negative operators trigger a different set verbal inflections (`\textsc{negation-based mood asymmetry}' following \citealp{Miestamo2005}). I propose an analysis which treats these semantic categories, as they are instantiated in W Dhuwal(a), as a natural class. A consequence of this is a treatment of W Dhuwal(a) inflection as encoding information about tense, mood and assertoric force; a finding that I argue speaks to the porous boundaries between these categories, and one that the inventory of formal semantics can insightfully and elegantly account for.

Consequently, the analysis defended in this dissertation shows that these two properties of Western Dhuwal(a) morphosemantics are epiphenomenal on an inflectional system that grammaticalises interactions between tense and mood; these \textit{prima facie} surprising distributions fall out naturally from a compositional, unified semantics for each of the four inflectional categories.

Finally, on the basis of data from a number of other Yolŋu language varieties, I show that these tense and modal phenomena are innovations that point to a history of contact-induced change between western varieties of Yolŋu and the unrelated languages of Western Arnhem. These innovative varieties are a consequence of the reanalysis of the semantic contribution of an older tense-based paradigm. These meaning change phenomena rebalance the division of labour between pragmatic reasoning and a number of pragmatic operators, giving rise to the diachronic reorganisation (and synchronic variation) in the means of temporal and modal expression across Yolŋu Matha. This semantic change pathway provides additional, diachronic support to theories of conceptual connections between the temporal and modal domains.