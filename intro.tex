\lettrine{D}{isplacement} --- a stated universal and distinctive feature of human language ---  permits us to make assertions that are embedded in different times, locations and possible worlds (\textit{e.g.} Hockett's `design features of human language' 1960:90). Linguistic work --- descriptive, pedagogical, theoretical --- has traditionally assumed a categorical distinction between subtypes of verbal inflection: \textit{viz.} the \textsc{temporal} and \textsc{modal} domains. Whether or not these basic claims are intended as heuristic, they quickly unravel upon close inquiry into cross-linguistic data; a challenge for linguistic theory, and one that a growing body of literature is identifying (\textit{e.g. }Condoravdi 2002, Laca 2008, Rullman \& Matthewson to appear \textit{i.a.}).% This will become clear in section \ref{phen} of this prospectus.

The \textbf{empirical focus} of the dissertation proposed here is the tense-mood-aspect (TMA) systems of a set of languages in the Arnhem Land linguistic area of Northern Australia. Arnhem Land is `linguistically dense' --- an area of close historic and contemporary contact between unrelated languages (see map in Figure \ref{map}). The verbal systems of many of these languages have evaded an adequate, unified account and exhibit various features that have been identified elsewhere as typologically rare (and certainly sharply diverge from better described Indo-European systems).

Consequently, given how resistant these data have been to description and analysis with existing linguistic apparatus, no theory neatly accounting for the inflectional range or making predictive generalisations; a better understanding of these systems will help us to nuance the way we think about categories like `tense' and `modality' --- a theory of temporomodal displacement. The potential \textbf{theoretical contribution} of this dissertation, then, bears broadly on \textit{intensionality}: our notional categories of tense, mood, modality, aspect, evidentiality, conditionals \textit{etc.} Further, as will be shown in §2, the role of pragmatics/information structure and their interactions with semantics are crucial for understanding how these categories are expressed and interpreted: how intensional meanings are generated, how communication permits for the displacement of times and worlds.

% a theoretical framework that can provide a (compositional) analysis of the data, whether small or large changes to existing work or a new proposal remains to be seen. This is relevant for anyone who works on anything intensional, including modality, tense, aspect, evidentiality, conditionals (counterfactuals, unconditionals, etc), to name a few


Additionally, in this work I seek to consider the contribution of studying \textbf{language change} (specifically meaning change) to a better understanding of the cognitive apparatus that permits for the interpretation of temporomodal devices (\textit{sc. `what is it that speakers are doing in order to `displace' discourse?}). A starting point in the assumption that `diachronically consecutive grammars are not characterised by radical discontinuities or unpredictable leaps, but that change consists of gradual discrete steps constrained by properties of grammar' (Deo 2006: 5). By hypothesis, then, the investigation of these `steps' between subsequent stages of a grammar with respect to its verbal semantics--and the inference of `constraints' on these changes--represent a significant potential source of insight into the linguistic expression and evaluation of event structure, time and possibility.

\section{Methodologies, conventions etc.}