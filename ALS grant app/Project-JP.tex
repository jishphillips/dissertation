 \documentclass[12pt]{article}
 \usepackage[margin=1.2in,a4paper]{geometry}
 \usepackage[T1]{fontenc}
 \usepackage{textcomp}
 \usepackage{ragged2e}
 \usepackage{booktabs}
		  \usepackage{epltxfn} %expex footnotes
 \usepackage{bold-extra}
 \usepackage{fancyhdr}
 \usepackage{fontspec,xltxtra,xunicode}
 \usepackage[labelsep=period,font=small,labelfont=bf,small]{caption}
 \defaultfontfeatures{Mapping=tex-text}
 \setmainfont{Cambria}
 \newcommand{\HRule}{\rule{\linewidth}{0.5mm}}
 %\setromanfont[Mapping=tex-text]{Hoefler Text}
 %\setsansfont[Scale=MatchLowercase,Mapping=tex-text]{Gill Sans}
 %\setmonofont[Scale=MatchLowercase]{Andale Mono}
 

 \usepackage{Sanremo,lettrine}
 \renewcommand\LettrineFontHook{\Sanremofamily}
 
 % \usepackage[margin=.7in,nohead,right=1.5in]{geometry}
 \def\singlesidestandardsetup{
 	%\textwidth 6in
 	%\oddsidemargin 0in
 	%\evensidemargin 0in
 	%\topmargin -.5in
 	
 	%\textheight 9.5in
 	%\columnwidth \textwidth
 	\parindent 1em
 	%\parskip 8pt
 	\pagenumbering{roman}
 	\parindent 1em
 	\parskip 10pt
 	
 }
 \usepackage{marginnote}
 \xdef\marginnotetextwidth{\the\textwidth}
 
 \usepackage{wrapfig}
 %\newcommand{\longsquiggly}{\xymatrix@C=1.915em{{}\ar@{~>}[r]&{}}}

\fancypagestyle{plain}{%
	\fancyhf{} % clear all header and footer fields
	\fancyfoot[C]{\sffamily\fontsize{9pt}{9pt}\selectfont\thepage} % except the center
	\renewcommand{\headrulewidth}{0pt}
	\renewcommand{\footrulewidth}{0pt}}
\pagestyle{plain}

 
 \newcommand{\mcom}[1]
 {\marginpar{\raggedleft\raggedright\hspace{0pt}\linespread{0.9}\footnotesize{#1}}}
 \newcommand{\cb}[1]
 {\marginpar{\color{orange}\raggedleft\raggedright\hspace{0pt}\linespread{0.8}\footnotesize{#1}}}
 \newcommand{\hk}[1]
 {\marginpar{\color{purple}\raggedleft\raggedright\hspace{0pt}\linespread{0.8}\footnotesize{#1}}}

\newcommand{\mmp}[1]
{\marginpar{\color{green}\raggedleft\raggedright\hspace{0pt}\linespread{0.8}\footnotesize{#1}}}

 \newcommand{\note}[1]{{ }\mcom{Note}\textbf{#1}}
 
 \newcommand{\glem}[1]
 {\MakeUppercase{\scriptsize{\textbf{#1}}}}
 
 
 %%%%cancel margin notes
 \renewcommand{\mcom}[1]{}
 \renewcommand{\mmp}[1]{}
 \renewcommand{\cb}[1]{}
 \renewcommand{\hk}[1]{}
 
 \newcommand{\specialcell}[2][c]{%
 	\begin{tabular}[#1]{@{}c@{}}#2\end{tabular}}
 
 \makeatletter
 \def\@xfootnote[#1]{%
 	\protected@xdef\@thefnmark{#1}%
 	\@footnotemark\@footnotetext}
 \makeatother
 
 
 \newcommand{\xmark}{\ding{55}}
 
 
 
 \author{Josh}
 \date{\today}
 
 \usepackage{subfigure}
 \usepackage{amsthm}
 \usepackage[normalem]{ulem}
 \usepackage{amssymb}
 \usepackage{multirow}
 \usepackage{mathrsfs}
 \usepackage{pifont}
 \usepackage{mathtools}
 \usepackage{tikz}
 \usepackage{qtree}
 \usepackage{tikz-qtree}
 %\usepackage{tipa}
 \usetikzlibrary{decorations.pathreplacing}
 \usepackage{textcomp}
 \usepackage[normalem]{ulem}
 \usepackage{url}
 \usepackage[all]{xy}
 \usepackage{multicol}
 \usepackage{hanging}
 \usepackage{booktabs}
 \usepackage{setspace}
 \usetikzlibrary{shapes,backgrounds}
 \usepackage{geometry}
 
 
 
 \newtheorem{definition}{Definition}
 \newtheorem{theorem}{Theorem}
 
 
 \usepackage{enumerate}
  \usepackage{enumitem}
 %\usepackage{gb4e} \let\eachwordone=\sl
 \usepackage{expex}
 
 
 
 %\newcommand{\denote}[1]{\mbox{$[\![\mbox{#1}]\!]$}}
 \newcommand{\concat}{\mbox{$^\frown$}}
 \newcommand{\ph}{\varphi}
 \newcommand{\vsep}{\vspace{8pt}}
 \newcommand{\linesep}{\rule{6.5in}{.5pt}}
 \def\attop#1{\leavevmode\vtop{\strut\vskip-\baselineskip\vbox{#1}}}
 \newcommand{\denote}[1]{\mbox{$[\![\mbox{#1}]\!]$}}
 \newcommand{\exref}[1]{~(\ref{#1})}
 
 \newcounter{nextsec}
 \newcommand\nextsection{%
 	\setcounter{nextsec}{\thesection}%
 	\stepcounter{nextsec}%
 	\thenextsec%
 }
 \newcommand\nextsubsection{%
 	\setcounter{nextsec}{\expandafter\parsesub\thesubsection\relax}%
 	\stepcounter{nextsec}%
 	\thesection.\thenextsec%
 }
 \def\parsesub#1.#2\relax{#2}
 \def\parsesubsub#1.#2.#3\relax{#3}
 %\input{setup}
 
 %\singlesidestandardsetup
 
 %\parindent 1em
 
 %\input{psfig-scale}
 
 
 \newcommand{\verteq}{\rotatebox{90}{$\,=$}}
 \newcommand{\equalto}[2]{\underset{\scriptstyle\overset{\mkern4mu\verteq}{#2}}{#1}}
 
 
 \newcommand{\secsep}{\hrulefill}
 \renewcommand*{\marginfont}{\small}
 \usepackage{qtree}
 \qtreecenterfalse
 
 \renewcommand{\baselinestretch}{1} %% this is the linespacing
 
 \newcommand{\la}{\langle}
 \newcommand{\ra}{\rangle}
 \newcommand{\lamda}{\lambda}
 \usepackage{framed}

\begin{document}
\noindent	\large \textbf{\textsc{Application: }ALS research grant}
	
\noindent	\normalsize \textit{Josh Phillips --- Yale University}
	
\section{Project title}
\textbf{\textit{Modal \& temporal interpretation in Arnhem Land}}

\section{Project summary}

Displacement --- a stated universal and distinctive feature of human language ---  permits us to make assertions that are embedded in different times, locations and possible worlds (\textit{e.g.} Hockett's `design features of human language' 1960). Linguistic work --- descriptive, pedagogical, theoretical --- has traditionally assumed a categorical distinction between subtypes of verbal inflection: \textit{viz.} the \textsc{temporal} and \textsc{modal} domains. Whether or not these basic claims are intended as heuristic, they quickly unravel upon close inquiry into cross-linguistic data; a challenge for linguistic theory, and one that a growing body of literature is identifying (\textit{e.g. }Condoravdi 2002, Laca 2008, Rullman \& Matthewson 2018 \textit{i.a.}).% This will become clear in section \ref{phen} of this prospectus.
	
My dissertation research seeks to provide a thorough description and analysis of tense-mood-aspect (TMA) systems of a set of languages in the Arnhem Land linguistic area of Northern Australia, with particular focus given to Yolŋu Matha varieties and Australian Kriol. Arnhem Land is `linguistically dense' --- an area of close historic and contemporary contact between unrelated languages. The verbal systems of many of these languages have evaded an adequate, unified account and exhibit various features that have been identified elsewhere as typologically rare (and certainly sharply diverge from better described Indo-European systems).


\section{Research goals}

Given how resistant TMA data in these Arnhem land languages have been to description and analysis with existing linguistic apparatus, no theory neatly accounting for the inflectional range or making predictive generalisations; a better understanding of these systems will help us to nuance the way we think about categories like `tense' and `modality' --- a theory of temporomodal displacement. Consequently, I hope that the data and analysis proposed here will make a contribution to linguistic theory bearings broadly on \textit{intensionality}: our notional categories of tense, mood, modality, aspect, evidentiality, conditionals \textit{etc.} Further, the role of pragmatics/information structure and their interactions with semantics are crucial for understanding how these categories are expressed and interpreted: how intensional meanings are generated, how communication permits for the displacement of times and worlds.


In terms of collection of new data, the bulk of the work will be undertaken in the field in Arnhem Land. I have already undertaken some field work in Ngukurr (SE Arnhem) and Ramingining (N central Arnhem) and expect to return Ramingining (and potentially Milingimbi and Maningrida) for approximately two months in early 2019 between Feburary and April. I have been working with native speakers of Australian Kriol and a number of varieties of Yolŋu Matha. 

Methodologically, my work primarily takes the form of semantic fieldwork (see Matthewson 2004, Bochnak \& Matthewson 2015 for details on established methodology) -- contextually rich elicitation tasks that target particular linguistic structures that bear on the empirical domain under investigation. These elicitation sessions take the form of interviews with native speakers which are recorded and transcribed in ELAN. This novel data is to be supplemented with primary linguistic data provided by other linguists who have work, published or otherwise, on these languages (e.g. Wilkinson 1991 and pers. comm.; McLellan 1992).

\section{Research Expertise}

I am a fifth-year PhD candidate at Yale University. I have already been to Arnhem land communities three times, totalling several months, to undertake linguistic work. This work has led to numerous refereed conference papers and a published article in \textit{Studies in Language 42(2)}. This work  bears on questions of language change and the semantics of grammatical categories.

My dissertation director, Professor Claire Bowern, has extensive field experience in both Northern Arnhem Land and the Dampier Peninsula. Further, she is a highly experienced advisor and mentor who has published in Australian historical and typological linguistics in a wide variety of venues. I have also spoken extensively to other linguists working with Yolŋu languages, including Melanie Wilkinson, Marilyn McLellan and Margaret Carew, who have provided valuable insights into these languages and the dynamics of the relevant language communities. Other mentors include Hadas Kotek and Ashwini Deo, both of whom work on and have published widely on questions of semantic theory (in addition to language change on a number of minority languages).

\section{Research Approvals}

This project has been reviewed by the Yale Ethics Review Board. Given that the methodology does not collect personal identifying information and works with consenting adults, it has been exempted from review (common for projects of this type which investigate speakers' linguistic competence.)

Research trips will be conducted with approval from the Northern Land Council. The elicitation work itself is undertaken with informed consent and in close association with elder Yolŋu, many of whom have had experience working with linguists and on literature production. 

\section{Research Expenses Summary}

The expenses for this project include accommodation, transport and consultant payments. Equipment is on loan from the Yale linguistics department.

\section{Research Timeframe}

There are no particular time constraints imposed on this research.

I expect to complete the fieldwork component by mid-2019 and to have written the PhD dissertation in mid-2020.

\section{Potential Alternate Funding}

The Macmillan Centre at Yale University provides research funding for doctoral candidates. I have applied for this grant and been awarded up to US\$15,000 for expenses relating to field travel for two field trips. This grant is subject to a number of constraints. I have used approximately one-third of this funding so far for a preliminary fieldtrip conducted in July 2018.

The attached budget considers only the forthcoming fieldtrip.


\section*{Budget}

\begin{tabular}{l|l|p{.6\textwidth}}\toprule
Item&Am't (AUD)&Notes\\\midrule
consultant payments&\$2,000 &50h @ \$40ph (excluded from school grant)\\
flights&\$5,000 &Flights from NY to Syd, Drw and Arnhem Land (including internal flights). Covered by school grant\\
accommodation&\$8,100 &45d @ \$180pn (see Bula'bula Art Centre). covered by school grant\\\bottomrule
	
\end{tabular}


\end{document}