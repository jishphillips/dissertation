\synctex=1

\documentclass[10pt]{article}
\usepackage[lmargin=30pt,rmargin=115pt,tmargin=70pt,bmargin=70pt,marginparwidth=110pt,marginparsep=5pt,a4paper]{geometry}
\usepackage{amssymb}
\usepackage{hyperref}
%\usepackage[tiny,compact]{titlesec}
\usepackage{graphicx}

\usepackage{booktabs}


\usepackage{wrapfig}
\usepackage{textcomp}
\usepackage{bold-extra}
\usepackage{tikz}
\usepackage{qtree}
\usepackage{tikz-qtree}
\usepackage{expex}


\usetikzlibrary{positioning,decorations.pathmorphing,arrows.meta,decorations.text}
\tikzset{snake it/.style={decorate, decoration=snake}}
\usetikzlibrary{calc, shapes, backgrounds,angles,quotes,tikzmark}
\usepackage{afterpage}
\usepackage{verbatim}
\usepackage{array}
\usepackage{multirow}
%\usepackage{hanging}
\usepackage{supertabular}
\usepackage{mathtools}
\usepackage[all]{xy}
\usepackage{ot-tableau}

\usepackage{paralist} 
\usepackage[labelsep=period,labelfont=bf]{caption}
\usepackage{subcaption}
\usepackage{fancyhdr} 
\usepackage{sectsty}
%\allsectionsfont{\sffamily\mdseries\upshape} 
\usepackage{float}
\usepackage[nottoc,notlof,notlot]{tocbibind} 
\usepackage[titles,subfigure]{tocloft} 
\usepackage{setspace}
%\usepackage[colorinlistoftodos]{todonotes}
\usepackage{xcolor}

\definecolor{blech}{rgb}{.78,.78.,.62}
\definecolor{ochre}{cmyk}{0, .42, .83, .20}
\definecolor{shadecolor}{cmyk}{.08,.08,.1,.12}
%\usepackage[explicit]{titlesec}
%\usepackage{type1cm}
%\usepackage{xcolor}

\usepackage{xltxtra} % Loads fontspec, xunicode, metalogo, fxltx2e, and some extra customizations for XeLaTeX
%\defaultfontfeatures{Mapping=tex-text} % to support TeX conventions like ``---''
\defaultfontfeatures{Mapping=tex-text}
\setmainfont{Cambria}

\usepackage[sort]{natbib}
\bibliographystyle{apa}
\bibpunct[:]{(}{)}{,}{a}{}{,}

%\usepackage{gb4e} \let\eachwordone=\it %\let\eachwordthree=\sf


\makeatletter
\def\@xfootnote[#1]{%
	\protected@xdef\@thefnmark{#1}%
	\@footnotemark\@footnotetext}
\makeatother

\pagestyle{fancy}
\fancyhf{}
\rhead{\footnotesize %Josh Phillips
	\hspace{2cm}\textbf{\thepage}}
\rfoot{}

\renewcommand{\headrulewidth}{0pt} 
\newcommand{\rowgroup}[1]{\hspace{-1em}#1}
\usepackage{stmaryrd}
\newcommand{\denote}[1]{\mbox{$[\![\mbox{#1}]\!]$}}
\newcommand{\denotn}[1]{\mbox{\llbracket\mbox{#1}\rrbracket}}

\newcommand{\mcom}[1]
{\marginpar{\color{black}\raggedleft\raggedright\hspace{0pt}\linespread{0.9}\footnotesize{#1}}}
\newcommand{\cb}[1]
{\marginpar{\color{orange}\raggedleft\raggedright\hspace{0pt}\linespread{0.9}\footnotesize{#1}}}
\newcommand{\hk}[1]
{\marginpar{\color{purple}\raggedleft\raggedright\hspace{0pt}\linespread{0.9}\footnotesize{#1}}}
\newcommand{\note}[1]{{ }\mcom{Note}\textbf{#1}}


\newcommand{\glem}[1]
{\MakeUppercase{\scriptsize{\textbf{#1}}}}

\newcommand{\exem}[1]
{\textit{\textbf{#1}}}

\usepackage{framed}
\usepackage{wrapfig}

\title{Cross-linguistic approaches to temporal reference\\\small\em Annotated Bibliography \& Research notes}

\date{}
\begin{document}
\maketitle 
\tableofcontents

\setcounter{section}{-1}
\part{Semantics of TMA}
\section{Formalising tense}
\begin{itemize}
	\item Recall from the prospectus that there are multiple basic approaches to tense in the compositional semantics literature (von Stechow 2009 suggests there are three main approaches)
	\item Prior takes tense as a (sentential) operator, truth of a sentence is to be understood relative to a time. (1957, 1967)
	\begin{description}
		\item[Quantificational approaches] as in tense (predicate) logic: famously Prior 1967 (also \textit{``indefinite approach''}). $$\llbracket\boldsymbol{P}\rrbracket=\lambda t.\lambda P_{\langle i,t\rangle}.\exists t^\prime[t^\prime\prec t\wedge P(t^\prime)]$$
		
		\begin{itemize}
			\item Fails to compose with temporal adverbials (yields weak TCs, see Dowty 82:23)
		\end{itemize}
		\item[Referential approaches] as in Partee 1973 and assumed in Heim 1994 apud Abusch (also ``definite approach'' ($t$ is provided by $g$)) $$\llbracket\textsc{past}\rrbracket^{s*}=\lambda t:t\prec s\!*.t$$\mcom{Acc. Dowty 82:29 also Marion Johnson 77 PhD (Kikũyũ), Cooper ms}
		(von Stechow points out that later revisions by Partee would need to change the $\prec$ relation between those times to $\subseteq$ for accomplishment-type predicates.)
		
		
		The original tense-as-quantifier approach can be refined for definiteness, or ``restricted'': vStechow's formalism for a contextually-retricted past (e.g. Musan 2002). $C$ is a function from times to truth values, apparently, containing a contextual set of salient times.  $$\llbracket\boldsymbol{P}\rrbracket_{\langle\langle i,t\rangle,\langle i,\langle\langle i,t\rangle,t\rangle\rangle\rangle}=\lambda C\lambda t\lambda Q.\exists t^\prime[C(t^\prime)\wedge t^\prime\prec s\!*\wedge Q(t^\prime)]$$
		
		\begin{itemize}
			\item `explicit rep of temp variables in the representational language' -- variables $ t_e, t_r $ are present in the \textsc{object language} (Ramchand 2007:1699 following Partee 1984, Kamp \& Reyle 1993...)
			
			\item Note that a problem with purely referential approachs comes apparently from Ogihara who notes that contextual restriction must necessarily occur, but still necessarily involves existential quantification over some restricted set of times. (i.e. if `i didn't turn the oven off' picks out some interval of time in which there was no oven-off period the non-event is still being asserted of some subinterval of that....)
		\end{itemize} 

	
		\item[Predicational approaches] `Predicates of times' (Most notably Dowty 1979 Ch. 7, \textit{et seq.} also Stechow 95 from \textit{SALT V})
				$$AT(t_1,\phi)(t)\leftrightarrow \forall t.\phi(t_1) \text{ is true.} $$
	$$			PAST(t_1)(t)=1\leftrightarrow t_1\prec t	$$
	\begin{itemize}
\item Temporal adverbials as (sets of sets of) intervals (parallel to NPs in PTQ):
$\llbracket\textit{on-Thursday}\rrbracket=\lambda T\exists t[t\subseteq{Thursday}\wedge T(t)]$ set of all sets that contain an interval on-Thursday 
\item Inflectional tense becomes an agreement phenomena under this type of `syntactic' based analysis. TNS is parasitic on Temporal adverbials. He shows the need two extra rules to introduce tense for sentences w/o TmpAdvs (28)
\item Higginbotham (2001:69) suggests for Dowty's ``model-theoretic approach'' (e.g. 1982), tense is a species of modality. (I.e. tense, appropriately?, as an intensional operator, see also Ogihara 1994.)
	\end{itemize}


	\end{description}
\subsection{Subordination \& Sequence of Tense}
\item  \textbf{`A touchstone for the adequacy of the semantics for tense is the behaviour of tenses in subordinate constructions'} (von Stechow)

	\begin{itemize}
		\item Relative clauses
		
		Observations from Ogihara 1989 on behaviour of relatives under matrix future in English leads to principle: \textit{The semantic tense of a relative Cl is obligatorily bound by a higher tense (can also be bound by a higher attitude predicate)}
		\item Under propositional attitudes
		
		Requires intensionalisation of $\mathcal L_\lambda$ (basic type $\langle s,\sigma\rangle$)
		\item Adjunct (\textbf{before-/after-}clauses)
	\begin{itemize}
		\item  If a main clause contains past we invariably find past in the adjunct clause
		\item Future main clauses cannot have future in adjunct clause
		\item \textit{before/after} as bivalent functions (type $\langle i,\langle i,t\rangle\rangle$)
		
			\end{itemize}	
	\end{itemize}


\item Higginbotham 2001 points out the variable interpretations of a sentence like: \textit{John saw a woman who was ill}
\begin{itemize}
	\item $ t_{see} = t_{ill} $
	\item $ t_{ill} \prec t_{see} $
	\item $ t_{see} \not\mathcal R t_{ill} $ -- `2 years ago J saw a woman who was ill last year'
\end{itemize}
\item Notes different treatment of RCls and Se complements (\textsuperscript\#2 years ago, Gianni said that Maria was ill last year)


\item \textbf{Klein (1992, 1994)} is taken to `break with tradition' (Bohnemeyer 2007) in understanding \textbf{(deictic) tense} to relate $ t_{utt} $ to $ t_{top} $ rather than to $ \tau(e) $. 
\begin{itemize}
	\item `Topic times are the times for which propositions arek, depending on their illocutionary force, asserted to be true, questioned for their truth, ``requested'' to be made true...\textbf{the times w/r/t which propositions are evaluated}' (Bohnemeyer \& Swift 2004: 279)
	\item `$ t_r $s can be understood as $ t_{top} $ in their capacity of being determined in context...time intervals anaphorically tracked across clause boundaries' (Bohn 2007: 924, see min. triplet).
	\item For Klein (viewpoint) \textbf{aspect} relates $ t_{top} $ to $ \tau(e) $
	\item A consequence of this is that \textbf{relative tense} and \textbf{viewpoint aspect} are the same phenomenon in formal terms (contested by Bohnemeyer 2007, see below.)
\end{itemize}
\end{itemize}
\subsubsection{Interval-based logics}
\begin{itemize}
	\item \textbf{Bennett \& Partee 2004 [1978[1972]]} may be the first formal semantic treatment of tense and aspect that explicitly argues for a definition of truth holding \textit{at an interval in time} \textbf{(§4.3)}
	\begin{itemize}
		\item \textbf{notation:}
		
		$ [t_1,t_2] $ all the $ t\in\mathbb R $ between $ t_1,t_2 $ (incl)\\
		$ (t_1,t_2] $ all the $ t\in\mathbb R $ after $ t_1$ up to $t_2 $\\
		$ (t_1,t_2) $ all the $ t\in\mathbb R $ between $ t_1,t_2 $ (excl)\\
	\item Definitions of: $ \sqsubseteq,\sqsubset_{init},\sqsubset_{fin},\prec_T $, (initial-, final-) endpoint for $ I $, (initial/final) bound for $ I $
	\textit{B\&P} give truth conditions for various Tns-Asp arrays, e.g. their `past perfect tense' \textit{John had eaten the fish by $ \alpha $} : true at $ I $iff... (pg98\textit{ff})
	\end{itemize}
\item Appealing to intervals,\textbf{ Dowty (e.g. 1986:42)} provides a formal, interval-based definition of the Vendlerian aspectual classes (given that these are taken to vary with respect to the \textit{subinterval property})
\item \textit{since $ t $} temporal adverbials serve to identify/precise the \textsc{xnow} (evidence for this is the cooccurrence of \textsc{pres perf} with \textit{since}-phrases, see also Dowty 1982:27)
\begin{itemize}
	\item One consequence being that any sentence with in the \textsc{prog} ``tense''is a stative sentence (45)
	\item D also notes that these aspectual classes are taken to range over sentences not nec. predicates (44)
	\item \textsc{Temporal discourse interpretation principle} gives a formalism similar to Klein/Kamp/Partee regarding ``immediate follows the ref time of the previous sentence $ S_{i-1} $ (45). The remainder of the paper investigates how \textsc{tdip} interacts with each of the Vendlerian classes (+ progressives).
	\item ``...we do not understand the perceived temporal ordering of discourse simply by virtue of the times that the discourse asserts events to occur or states to obtain, but rather \textit{also in terms of the additional larger intervals where we sometimes assume them to occur and obtain}.'' (59)
\end{itemize}
\end{itemize}
\subsection{Situation types \& lexical aspect}
\begin{itemize}

	\item Smith 1997 on the distinctions between situation aspect and viewpoint aspect
	\item Also Dowty 1986 speaks in some detail about how to formalise (in a gen. semx framework) the different aspectual classes (wht he calls verbal aspect). This also comprises a major motivation for the intro of an interval semantics (following Bennett \& Partee's work on Eng temp encoding.)
	\item Binnick 1991
	\begin{itemize}
		\item Saurer's system introduces events (1981) -- here an untensed sentence (ie an event description) denotes an set of events $\denote{walks}=\lambda x\textasciicircum e\,walk'(e)(x) $ (\textit{walks} denotes tokens of the event-type \textit{walk.})
		\item (Pre-Krifka) introduces function $ time:\mathcal E\to\mathcal T $ for the time of occurrence of an event
	\end{itemize}
\item A composition rule: \textit{The TMA ordering principle} from Woisetschläger's dissertation 1977:24 `if $ a,b $ are syntactic formatives representing a verbal cat and $ a $ is closer to stem, $ a $'s translation rule precedes $ b $'s. 

He's writing on english auxiliaries so perhaps this need not be taken to seriously.
\end{itemize}

\setcounter{section}{-1}
\newpage\part{Tenselessness and notions of tense as a universal}
\begin{quotation}
	Distinguishing temporal reference and tense makes precise the observation that every language realizes clauses with past temporal reference, but not all languages have past tenses that convey that the topic time precedes the utterance time. \hfill(Tonhauser 2015: 134, also see Jespersen 1933: 230)
\end{quotation}
\section{\textbf{Tonhauser} \textit{(Annu. Rev. Linguist.)}}
\begin{description}
\item[Tense] `grammaticalized marking of location in time' as playing a central role in analyses of temporal reference.'

`well-studied languages that have been the empirical focus of formal research on meaning are tensed languages'

DeCaen 1996: `about half the world's languages are tenseless'

\item[The debate:] \textit{whether tenseless languages can receive truly tenseless analyses...} the range of linguistic and contextual means that affect temporal reference.
\end{description}

\subsection{Tense speak (§ 2)} 

Modifications to the classic Reichenbach schema ($t_s,t_r,t_e$)
\begin{itemize}

	\item \textit{Utterance time}
	\item \textit{Topic time} (e.g. Klein 1994) -- the time the utterance is about.
	\item \textit{Eventuality time}
\end{itemize}

\textit{Tenses are paradigmatic expressions that indicate a temporal relation between topic time and utterance time} (Klein 1994).
 
\mcom{Yolŋu varieties ought not be considered tenseless.}A \textit{tenseless language}, then, is one that does not have paradigmatic expressions that convey a temporal relation between $t_{topic}$ and $t_{utt}$
\begin{itemize}
	\item Non-triviality of diagnosing tenselessness (Shaer \textit{CLS} 1997, 2003; Bittner \textit{JoS} 2005; Lin \textit{Ox. HBk of T\&A }2012)
	\item Guaraní \textit{kuri} `back.then' and \textit{-ta} \textsc{`prosp'} had been treated as tense morphs in earlier literature. Tonhauser argues against this analysis in (2011), noting that they're `not part of the tense paradigm.' (and indeed can cooccur in sentences like (\nextx)
	\begin{itemize}
		\item \textit{kuri} $N=32$ in \textit{The Little prince}
		\item \textit{-ta} acceptable in matrix clauses with past temporal reference $t_e\succ t_{topic}$)\mcom{To what extent does \textit{dhu\textasciitilde yurru} have these properties also?}
	\end{itemize}
	\pex\begingl\glpreamble\textbf{Context:} When I cleaned her wound I tried not to show her how bad it looked.//
	\gla o-ñe-hundí\textbf{-ta} chugui la i-po \textbf{kuri}//
	\glb A3\textsc{-je}-lose\textsc{\textbf{-prosp}}  pron.\textsc{non.ag.3.abl} the B3-hand \textbf{back.then}//
	\glft`She was going to lose her hand'//\endgl\xe
\end{itemize}

\subsection{constraints on temporal reference (§3)}

\begin{itemize}
\item \textbf{Context:} • Utterance situation, • linguistic context, • info. structure of preceding discourse.

Topic time as an implicit temporal anaphor (i.e. the context dependent analyses of Partee, also Mitchell 1986, Condoravdi \& Gawron 1996)

``constraining the set of times to which the anaphoric topic time may be resovled by requiring that only contextually salient times are possible antecedents'' (136)

\begin{itemize}
	\item Tonhauser complains that some theorists have assumed that tenses are responsible for introducing the temporal anaphor that is interpreted as topic time.
	
	It's not clear to me that this is necessarily a problem, if the semantic contribution of tense marking is only modelled as an identity function $\langle i,i\rangle$ with a contextual restriction presupposition on the topic time (w/r/t speech time). 
	
	What she's talking about may be older analyses for which the type of a tense morpheme is actually a variable over times $i$. And in fact in §3.2 she does seem to espouse this view.
\end{itemize}
\item \textbf{Tense (i.e. morphology)} Tenses `constrain the temporal location of the anaphoric topic time relative to the utterance time.'

\begin{itemize}


\item possibilities of `mixed tense' systems (where tense marking is optional, or even degraded in particular constructions: Navajo, Korean, Japanese, Mohawk, Gujarati...)
\end{itemize}


\item \textbf{Temporal adverbials} All reported languages have adverbial constructions that constrain possible topic times. It is demonstrably not the case that tenseless languages `make up for' their tenselessness with increased usage of these constructinos.
\end{itemize}

\subsubsection*{Tensing the tenseless (§3.4)}

\begin{itemize}
	\item the notion of Tº being the head of a finite clause: universally assigns \textsc{nom}
	\item to saturate a proposition (Partee 1992, vF\&Matt 2008)
	\item T introduces Topic Time \textbf{(Matthewsonian perspective)}
	
	``a tensed analysis of a tenseless language is empirically motivated if the language exhibits temporal reference restrictions comparable to those exhibited by some tensed language...'' (139)
\end{itemize}

\subsubsection*{Tonhauser's perspective: \textit{resolving temporal ref in discourse} (§4)}

\begin{itemize}
	\item \textbf{Aspect} taken to have discourse structuring properties (perfective marking implicates narrative advancing as against imperfective marking)
		\begin{itemize}
		\item Are the temporal relations associated with aspect \textit{entailments} (Bittner on Mandarin, Greenlandic) or \textit{default readings} (Bohnemeyer on Yucatec)?
		\item Further questions arise about whether perfectivity (gram. aspect) is the source of the phenomenon or whether it's telicity (incl. Aktionsart). This diff is probably reconcilable (Bohnemeyer \& Swift)\end{itemize}
	\item Default temporal ref associated with aspect marking (Carlota Smith \textit{et al}): overridable by tense marking, adverbials, context...
	\item \textbf{Remoteness}: Cable on Gĩkũyũ current past vs near past ($\approx$±hodiernal) marking as \textit{not} tense marking (they do not constrain the relation between speech- and topic-times. Rather speech- and \textbf{eventuality-time}).
\end{itemize}
\section{Bohnemeyer \textit{LD\&C \texttt{ms.}} Elicitation \& documentation of tense \& aspect \& 2018 SULA presentation on metricality}
\begin{itemize}
	\item Bhat 2009 on tense- v. mood- v. aspect-centrality as a linguistic variable
	\item viewpoint aspect specifying the nature of $R(\tau(\varepsilon),t_\textit{top})$ (perfective vs. imperfective \& perfect vs. prospective)
	\item situation aspect (incl. aktionsart) as a set of properties: \textbf{durativity, telicity, dynamicity}
	\item tense as constraining $R(t_\textit{top}, t_\textit{utt})$
	\item apud klein 1994, ``tense and aspect are independent semnatic relations although their expression is often conflated in natural languages''(+6).\\
	One consequence is that relative/`anaphoric' tenses may then not be needed in view of the theoretic dissociation of tense and vpt aspect
	\item Bohnemeyer 2014 contests the disposal of anaphoric tense: ``constraiing the temproal relation between $t_\textit{top}$ and some contextual reference time rather than the relation between $t_\textit{top},\tau(\varepsilon)$ as vpt. asp. do' $\Rightarrow$ addition of another $t_{r}$ to take care of.
		\begin{itemize}
			\item tenses ought to be able to combine w aspectual markers (if ``syntagmatic combinations of T and A'' are allowed for)
			\item pg 22 (unclear argumentation)
		\end{itemize}
	\item ``cyclical notions of time'' (B critiques this Farris 87 (`remembering the future, anticipating the past' in \textit{comp. stud. soc. hist 29(3)}); León-Portilla 88 \textit{tiempo y realidad en el pensamiento maya})
	\begin{itemize}
		\item B hypothesises that these are misconstruals and that a society that construed time as nonlinear would be nonfunctional.
	\end{itemize}
\textbf{See §\ref{TRM} for review of treatments of metricality/remoteness}


	\end{itemize}

\section{Tonhauser \textit{Linguist. Philos. }2009 (Guaraní temporal reference)}
\begin{itemize}


\item As is the case for St'át'imcets, there's something of a \textsc{fut/nfut} distinction in Guaraní.
\item Tonhauser argues against a tensed analysis of Guaraní (cf. Mathewson's covert \textsc{nfut} morpheme in all finite clauses)
\item She provides a dynamic semantics formalism for a tensed (Mathewsonian) and tenseless analysis of Guaraní.
\item Tonhauser's analyis says that zero-TMA-marked verbs in Guaraní receive \textit{imperfective} OR \textit{perfective} aspect interpretations:
\begin{itemize}
	\item \textit{I.e.} $t_\varepsilon\circ t_r$
	\item As opposed to perfect $(t_\varepsilon\prec t_r)$
	\item or modalised?/prospective interpretations $(t_\varepsilon\succeq t_r)$
\end{itemize}
\item A mathewsonian covert tense morpheme has a denotation like:
\denote{\textsc{tense$_i$}}$^{g,c}=g(i)$ and presupposes that no part of $g(i)\succ t_c$

\subsection{\textsc{`tns'}: Critiques from Guaraní (§3)}
\begin{itemize}
	\item \textbf{Problem 1: }One of tonhau's critiques is that an analysis like this as applied to guarani would need to allow \textsc{nonfut} marked clauses to receive \textit{absolute future} reference (ie $t_s\prec t_\varepsilon$). See the below example, where the matrix clause cannot have \textsc{nfut} reference \textbf{if we are to maintain the basic cross-linguistic generalisation that, in matrix clauses, $\boldsymbol{t*=t_s}$}.
	
	\pex\begingl\glpreamble\textbf{Context:} It's morning and the spkr is talking about a goose walking past her//
	\gla{} [ja'ú\textbf{-ta-re} ko gánso ko'ẽro] \textbf{a-juka} ko ka'arú-pe//
	\glb{} A1p\textsc{.incl}-eat\textsc{\textbf{-prosp-for}} this goose tomorrow A1s-kill this afternoon-at//
	\glft`Since we are going to eat this goose tomorrow, I will kill it this afternoon'//\endgl\xe
	
	\item\textbf{Subordinate clauses} readily receive future reference.

\pex\a\begingl\glpreamble\textbf{context:} M's wedding is tomorrow, she invited p to sing but doesn't know whether she'll come. J says.//
\gla i-katu o-purahei ko'ẽro//
\glb b3-possible a3-sing tomorrow//
\glft`I'ts possible she'll sing tomorrow'//\endgl
\a\begingl\glpreamble\textbf{context:} to play a trick on M, we plan to call him to ask directions to his house//
\gla mario oi'mo'ã-ta ja-ju-ha//
\glb mario a3-think-\textsc{prosp} a1p.\textsc{incl}-come-\textsc{nom}//
\glft`mario's going to think we're coming'//\endgl\xe

\item What this is taken to mean then, is that if all of these clauses have covery NFUT marking, then it can't be the case that it's an \textit{absolute tense} (i.e. that $t*=t_s$ in all clauses.)

\item \textit{-ta} \textsc{`prosp'} presupposes an epistemic modal base with stereoypical ordering source or a circumstantial modal base with an ordering source that specifies the relevant agent's intentions. If defined:

$\denote{-ta} = \lambda P_{\langle w,\langle i,\langle i,t\rangle\rangle\rangle}\lambda w \lambda t_r\lambda t_\varepsilon.\forall w^\prime\in\textbf{best}(mb,os,\langle w,t_r\rangle)\to t_r\prec t_\varepsilon \wedge P(w^\prime, t_\varepsilon,t_\varepsilon)$
\item\textbf{Problem 2:} Tonhauser identifies an additional problem though, where this treatment of the prospective and its interactions with zero-marked \textsc{nfut} clauses ought to also allow for an (unavailable) reading of (3b) where \textit{mario's going to think that we came} $t_\varepsilon\prec t_\text{thinking},t_s$

{\small (This is because -- even if the subordinate clause takes the eventuality time of the \textsc{prosp}-marked matrix clause as the evaluation time, the unmarked subordinate clause ought to enforce that the coming-time is at or earlier than that time.)}
\item \textbf{Problem 3:} ? \textit{something to do with the times available for \textsc{nfut} interpretation in subordinate clauses (utterance/matrix eventuality don't seem to constrain a subordinate NFUT??).}
\end{itemize}
\item T suggests a SoT type rule as an escape for problems 2\&3: \textit{the \textsc{nonfut} morpheme of a finite subordinate clause is not interpreted under identity with a \textsc{nonfut} morpheme in the matrix clause}).

SoT leads to the noninterpretation of \textsc{nonfut} in non-matrix clauses in Guaraní.
\subsection{The tenseless analysis}
\item Problems discussed above do not arise because there is no postulated \textsc{nfut} morph that is constraining temporal ref.
\item\mcom{Again it needs to be know how obligatory \textit{dhu} is in I-inflected Yolŋu clauses for future reference.} Challenge arising is the need to account for the apparent need for \textsc{prosp} suffix \textit{-ta} to permit future ref. 
\begin{itemize}
	\item \textbf{empirical motivation that `absolute future reference times...are not contextually available in guaraní' (except where they are introduced by some construction)}
	\item reichenbachian primer (options for constraining temporal reference)
	
\begin{itemize}



\item \textit{J danced} \quad $t_D\subseteq t_r\prec t_s$\\
perfective aspect constrains dance time to a subint of reference time.\\
past tense tells us that the ref time precedes speech time.

\item \textit{J has danced} $t_D\prec t_r=t_s$\\
perfect aspect : dance time anterior to ref time\\
present tense tells us that the ref time is at utt. time.

\item There're convincing proposals that \textit{will} is constraining the relation between eventuality time (this is kind of intuitive, how else do we get \textit{you will have})
\end{itemize}
\item See also Bittner's work on W Greenlandic temporal reference -- this is a language that descriptive work had suggested had a rich metrical tense system. The current lit suggests that the Greendlandic \textbf{exclusively uses the eventuality time option} (i.e. has no morphologised \textit{absolute} tense.)
\begin{itemize}
	\item `if only past and present antecedednt ref times are contextually available, temporal ref is contextually restricted to be non-future' (285)
	\item `future discourse in Guaraní is realised almost exclusively...by reference to past and present antec ref times in whose future $t_e$ is located.'\mcom{Authors use \textit{The Little Prince} and a book of translated short stories for a corpus study of future expression. \textbf{Is it worth working on a translation of \textit{The Little Prince??!}}}
	\item `...almost exclusively by prospective AM markers, poss and nec. modals and prosp moods'
	\pex\begingl\gla \textbf{i-katu} ne-re-mosã-i-rõ o-ho ndéhe-gui ha o-kaný\textbf{-ne}//
	\glb b3-\textsc{poss} \textsc{neg}-A2s-tie-\textsc{neg}-if a3-go 2s,o\textsc{-abl} and a3-hide-\textsc{might}//
	\glft`It's possible, if you don't tie him, he will get away from you and might get lost'\\
	`But if you don't tie him, he will wander off somewhere and get lost.'\trailingcitation{[English TL]}//\endgl\xe
\end{itemize}

\end{itemize}
		%%%%%%%%%
		\iffalse \item Notably, in complex sentences there are sentences in which an tensed analysis would need to be able to say that \textit{-ta} can scope out of a subordinate clause to the matrix (274) OR that evaluation time \textbf{is not} utterance time (a violation of the x-linguistic \textit{matrix clause rule}):
		\pex\begingl\gla Ja'ú-ta-re ko gánso ko'ẽro, a-juka ko ka'arú-pe//
		\glb a1p\textsc{.incl}-eat-\textsc{prosp}-for this goose tomorrow a1s-kill this afternoon-at//
		\glft`Since we're eating this goose tomorrow, I'll kill it this afternoon'//\endgl\xe
		
		\item \textbf{Also,} interactions with subordinate clauses demonstrate that \textsc{nonfut} doesn't have absolute reference. (\nextx examples show embedding under a modal and a prop att.)
		\pex\a \begingl\gla i-katu o-purahei ko'ẽro//
		\glb b3-possible a3-sing tomorrow//
		\glft`It's possible that she'll sing tomorrow'//\endgl
		
		
		\a\begingl\gla mario oi-mo'ã-ta ja-ju-ha//
		\glb mario a3-thing-\textsc{prosp} a1p.\textsc{incl}-come\textsc{-nom}//
		\glft`Mario is going to think we're coming'//\endgl\xe
		\fi%%%%%%%%%%%%%%%%

\item For Tonnhauser's analysis, the \textbf{matrix clause rule }is reformulated as follows:
\pex The final translation of a matrix clause translated as $\phi$ of type $\langle\omega,\langle\iota,\langle\iota,\tau\rangle\rangle\rangle$ is $\exists t(\phi(w_0,t_{rt},t))$ of type $\tau$\xe

So: $$\begin{matrix}\textit{a-jahu}&\Rightarrow&\lambda w\lambda t^\prime\lambda t[AT(t^\prime,\textit{bathe}(sp,w,t))]\\
&=&\exists t(AT(t_{rt},\textit{bathe}^\prime(sp,w_0,t)))\end{matrix}$$

And dynamising this: the result of updating $\sigma$ with \textit{a-jahu} is:

$$\{i\in\sigma[t]:\langle g_i(t_{rt}),g_i(t)\rangle\in f_i(\sqsubset_{nf})\wedge g_i(sp),g_i(w_0),g_i(t)\rangle\in f_i(\textit{bathe}^\prime) \}$$

\mcom{The formalism 4 the first conjunct seems to be a needlessly confusing way of saying that $t_{rt}\sqsubset_{\textbf{nf}} t$}I.e. a new information state containing those possibilities (worlds??) $i$ in $\sigma$ as updated with (temporal) DR $t$. In $i$, $g$ assigns the speaker to \textit{bathe} at $w_0,t$. It also enforces that $t_{rt}$ is a non-final subinterval of $t$.
\item There is no restriction on the identity of $t_{rt}$ which is why no abs. tense surfaces. In principle can be any time in the domain of possibilities in $\sigma$. Whereas an adverbial would restrict the domain as:

\mcom{Evidence provided that calendrical adverbials constrain $t_\varepsilon$}$$\llbracket\textit{kuehe}\rrbracket\Rightarrow\lambda P\lambda w\lambda t^\prime \lambda t[t^\prime\sqsubseteq\textit{yesterday}^\prime\wedge P(w,t^\prime,t)]$$

\item \textbf{constructions that `can' introduce a future ref time:} when a conjunct in guaraní describes a future eventuality, subsequent conjuncts receive (successive) future reference.

\item Also for the goose example \textbf{(is this hacky?)} the adverbial suffix \textit{-re} `for' is taken to introduce a reference time $t''$ (i.e. modelled as introducing a conjunct $\exists t''(t'\prec t \wedge \exists t'''(P(w',t'',t''')))$)

The final translation of the whole sentence is given as

$$\begin{matrix}\exists t\forall w'(w'\in\textbf{best}(f,g,\langle w_0,t_{rt}\rangle)\to t_{rt}\prec t\wedge\\ \textit{eat}'(\xi,g,w',t,)\wedge\\ t\sqsubseteq\textit{tomorrow}'\wedge\\ t''(t_{rt}\prec t''\wedge\\\exists t'''(t'''\sqsubseteq\textit{this.afternoon}'\wedge\\ t'''\subseteq t''\wedge\\\textit{kill}'(\textit{sp},g,w',t''')) )\end{matrix}$$


\mcom{Hm, no I dont think it's hacky but don't totally understand why two additional times had to be introduced? Ostensibly one to serve as reference for the eventuality but it feels unclear what that's buying.}That is, the sentence updates a $\sigma$ to those possibilities where in the best $w'$, there's a $t\succ t_{rt}$ tomorrow at which the gooese is eaten as well as a time $t''\succ t_{rt}$ which includes the goose-killing event at $t'''$.

\item\textbf{Subordinate clauses}
\end{itemize}

\section{Matthewson 2006 \textit{Ling. \& Philos.} (Temporal semantics in St'át'imcets)}

\begin{itemize}
	\item \textbf{Proposal.} All clauses have a phonologically null \textsc{nfut} morph in Tº.
	\item Future, counterfactual/``past future'' meanings arise by combination of \textsc{nfut} and an Abuschian \textit{\textsc{woll}} operator (phonologically realised as \textit{kelh}).
		\begin{itemize}
			\item \textbf{Consequence.} `...may reveal a universal semantic fact: \textit{that future is never itself a tense}, but rather involves another element (modal/temporal ordering predicate)'
			\item  depending on whether $t_r$ is pst or pres (both within the range of \textsc{nfut}), \textit{kelh} `\textsc{woll}' receives `ordinary' vs. `pst' future readings.
			\item Evidence of the modal contribution (i.e. quantification over $\mathcal W$) \textbf{in addition to} obligatory futurate meanings of \textit{kelh} (§4)
		\end{itemize}
	\item Basic definitions (Klein 94 apud Reichenbach): Tense encodes $\mathcal R(t_u,t_r)$, while aspect encodes $\mathcal R(t_t,t_\varepsilon)$
	\item \textbf{methodology}
	\begin{itemize}
		\item translations
		\item truth judgments in context
		\item felicity judgments in context
		\item speaker comments
	\end{itemize}
	\item Otherwise uninflected sentences can receive present/past readings.
	\item Aktisonsart strongly biases interpretation (statives get present, achs/accs get past, activities don't display this pref). These are all shown to be contextually defeasible \& restrictable by TFA \textit{(although infelicitous w future reference)}.
	\begin{itemize}
		\item enclitic \textit{tu7} compels past ref. (\textsc{compl} Asp?)
	\end{itemize}
	\item Future reference usually achieved by marking \textit{kelh} (\textsc{nfut} unavail with \textit{kelh} in matrix clauses). Also \textit{cuz'} \& some (grammaticalising?) motion Vbs have some type of prospective function.
	\begin{itemize}
		\item Interesting... it's not 100\% clear how grammaticalised this is or whether some motion predicates just don't have the \textsc{nfut} restriction. (678)
	\end{itemize}
	\item Kratzerian presupp/index approach (Tº valued by $g$) \Tree [.TP$_{\langle s,t\rangle}$ [.T$_i$ ] [.AspP$_{\langle i,\langle s,t\rangle\rangle}$ Asp vP ] ]


	i.e. $\llbracket\textit{Mary walked}\rrbracket^{g,c}=\lambda w\exists e:g(i)\prec t_c.W(e)(w)\wedge\textbf{ag}(m)(e)(w)\wedge\tau(e)\subseteq g(i)$
	\item On this approach matthewson's \textsc{tense} presupposes $\nexists i^\prime\sqsubseteq i,i^\prime \succ g(i)$ and denotes $g(i)$.
	\item \textbf{\textit{kelh} as \textsc{woll}} (as opposed to \textsc{irr/\textsc{mod}}) (although does involve quantification over $\mathcal W$)
	\begin{itemize}
		\item \textsc{woll} is a predicate of times (temporal precedence)
		
		$$\llbracket\textsc{kelh}\rrbracket=\lambda P_{\langle i,\langle s,t\rangle\rangle}\lambda t\lambda w.\exists t^\prime[t\prec t^\prime\wedge P(t^\prime)(w)] $$
		\item therefore Matthewson (provisionally?) posits a \textit{kelh}P between the TP and AspP--vP layers.
		
		\item Binary splits \textsc{±nonfut} are often debatably ±\textsc{irr} (see Chung \& Timberlake 1985)
		\item \textbf{Chung \& Timberlake 1985:} \textit{\textsc{irr}ealis categories} sometimes or always include
		\begin{itemize}
			\item \textsc{cond, counterf, imper, fut, q, neg, obl, desid, pot}, warnings
			\item \mcom{There minimally ought to be some reference to their 1985 chapter in motivating the whole dissertation (important early work on `TMA correlations')} Chung \& Timberlake `the nonfuture is used for events in the past, ongoing events in the present and future events that are imminent in the present' (on Takelma, data from Sapir; similar data from Dyirbal) (cf. Yidiɲ ±\textsc{pst})
			\item `A consequence of these correlations is that temporal distinctions may be expressed by morphosyntactic categories that have wider modal or aspectual functions'
			\item Lakhota future marker \textit{kta} is used in `unrealized and potential events'
			\item Chamorro (ir)realis encodes (non)futurity, progressive encodes presentness...
			\item Chinook metricality (see Silverstein 74)
			\item `whereas there is basically one way for an event to be actual, there are numerous ways than an event can be less than completely actual...different types of non-actuality' (241)
			\item \textbf{`languages differ significantly as to which events are evaluated as (non)actual'}
			\begin{itemize}
				\item evaluation w/r/t \textbf{source} ($\approx$ ``judge'', located as Spkr for matrix clauses, subordinate clauses:subject)
				\item As an example, Attic Gk inflects counterfactuals in realis terms: one way of looking at this is that `since cfact events are definitely not actual (but only hypothetically possible) they are definite in their modality and in Anc. Gk. expressed in the realis.' (255)
			\end{itemize}
		\end{itemize}
		\item \textbf{\textit{kelh} as general \textsc{irr}?} Matthewson tests optionality \textit{kelh} in negative, yes-no, conditional contexts and finds that it obligatorily contributes future meaning. (Also infelic in imperatives) (685-6) 
		\item  \textbf{\textit{kelh} as modal aux?} In nonfutures, \textit{might/possibly} is translated as \textit{k'a} or perhaps with evidential \textit{-an'}
	\end{itemize}
\end{itemize}

\section{Bohnemeyer 2009 (Yucatec temporal anaphora)}
\begin{itemize}
	\item[\textbf{background} (§§1-2)] Yucatec has a system in which verbs (\textbf{i.e. process \& state-change predicates \textit{unlike stative predicates}}) obligatorily inflect for:
	\begin{itemize}
		\item \textbf{status} (some combo of viewpoint aspect, modality, illocution:) \textit{incompletive, completive, subjucntive, extrafocal, imperative}
		\item \textbf{stem class} \textit{active, inactive, inchoative, positional, transitive} (complementation patterns and lexical semantics)
	\end{itemize}
	There are also preverbal `Asp/Mod markers' which mark various -- formally prefixes \& (\textsc{pfv,ipfv}), predicates (\textsc{prog,term,prosp} and \textsc{oblig,nec,desid,assur,pen} and \textsc{rem.fut, rem.pst, rec.pst, imm.pst, prox.fut})\\
	
	
	
	\item To motivate tenselessness, Bohnemeyer gives: \textbf{15 sentences} with different aspect-mood marking and shows that (with some differential felicity caveats) all fifteen can appear embedded in \textbf{3 discouse contexts with present, future and past \textsc{topic times.}}
	
	\item Anticipating the debate between Tonhauser \& Matthewson, Bohnemeyer formulates the \textbf{Modal commitment constraint} which states that \textit{realisation of events in the (relative or absolute) future cannot be asserted, denied, questioned, presupposed as fact. Assertions, questions and presuppositions regarding the future realisation of events (or the failure thereof) \textbf{require specification of a modal attitude on the part of the speaker} (109)}\mcom{this happens to be really similar to my omniscience constraint from the \textit{bambai} QP although it is formulated as a \textbf{language-specific constraint}, where as mine was probably naïvely ambitious.}
	\begin{itemize}
		\item I.e. any discussion of future events requires the explication of a \textbf{modal attitude} (i.e. necesity, desire, agreement, prediction)
		\item In Mayan the modals that encode these attitudes cannot cooccur with the perfective.
		\item They can cooccur with \textit{terminative} marking
 as well as past-tensed adverbials. Bohnemeyer suggests that this is because these markers `have stative meanings: they [target] not the realisation of the event but its result state.'
 \begin{itemize}
 	\item e.g. Remote future 
 	\end{itemize}	\end{itemize}
\item \textbf{Pronominal anaphora} (113ff) -- determination of topic time in context (empirically demonstrated that there is no greater reliance on adverbials.) \\ Aspect, world knowledge \& discourse advancing
\begin{itemize}
	\item progressive clause as contained within previous (`aspect markers stipulate that the perception event is included in this topic time')
	\item \textit{il} `see' marked for \textsc{ipfv} -- idiomatically interpreted as the realisation of previously unknown facts (\textsc{pfv} interpretation!)
\end{itemize}
\pex \textbf{Natural Temporal reference point} (117)\\
 A time interval $t$ is an NTRP in a given discourse iff $t$ is identified in that discourse as either:\\
 \textbf{(a)} the coding time of some utterance or\\
 \textbf{(b)} a calendrical time interval or\\
 \textbf{(c)} an event time (the ``runtime'' of an event described in discourse).\xe
\end{itemize}

A (the?) consequence of this constraint/formulation (NTRP) is that \textbf{only perfective clauses introduce NTRPs} (the fact that some $e$ is predicated as in progress at $t^\prime$ fails to qualify $t^\prime$ as an NTRP.)

This feeds into the discourse coherence principle (`preferred topic time selection') next formulated.

\pex \textbf{Preferred topic time selection} (118)\\
The topic times selected in a given discourse context are preferred to be identical to or include NTRPs identified in the same discourse context.\xe

\begin{itemize}
\item binding implicatures triggetred by nonpfv clauses are satisfied by calendrical adverbials or salient ref points in context. Where these are absent, coding time takes over as the NTRP.
\item topic time includes coding time and is itself included in the run time of a \textsc{prog} event description, a results stat marked by a \textsc{term} description or a prestate marked by a \textsc{prosp} description. Or some state characterising the realisation of the event in possible worlds for a \textsc{mod} description. or some state that characterises the distance from $t_e$ to $t_r$ with \textsc{metric} marking. 
\end{itemize}
\section{Ritter \& Wiltshko 2009/14 \textit{NLLT} (composition of \textsc{infl)}}
\begin{itemize}
	\item Universality of an \textsc{inflº}\mcom{this better be motivated} that is variably associated with \textbf{temporal, spatial or participant} marking

	\begin{quotation}
		``Languages differ in the morphosyntactic categories they make use of and as a consequence differ in their formal organization of meaning''\hfill(Sapir~1921)
	\end{quotation}
	\item `Ritter \& Wiltschko 2009 hypothesise that \textsc{infl} requires deictic substantive content \textit{i.e.} content whose denotation is determined by the utterance context, including not only tense but also location and person' (2014:1336).
	\item ``there is no principled reason why UG should privilege temporality as its anchoring category (R\&W 2005)'' (1339)
	\item Optionality of Halkomelem past marker \textit{(-lh)}.\\
	Absence of past marker in Blackfoot
	\item Parallelism (Abney 1987) between \textbf{\textit{n}} and \textbf{\textit{v}} categories and between \textbf{D} and \textbf{I} categories.
	\begin{itemize}
\item 		\textbf{D} and \textbf{I} as `anchoring categories': `locating the individual in time and space' (1334)
\item `linking' domains CP and KP
\item `thematic' domains \textbf{\textit{v}P} and \textbf{\textit{n}P}
	\end{itemize}
	\item `This parallelism...would be coincidental if there was no universally predetermined order in the project of functional categories'
	\item \textbf{Parametric substantiation hypothesis}
	\begin{enumerate}[a.]
		\item UG makes available a set of hierarchically organised functional cats: \textit{the universal spine}
		\item languages vary in the substantive content associated with functional cats.
	\end{enumerate}
\end{itemize}
\subsection{Motivating the IP (§2)}
	\begin{itemize}
		\item  In the \textit{aspects} era, an exocentric constituent \textbf{S} was taken to dominate NP, Aux, VP
		\item But \textit{cf.} sentential complementation (missing a generalisation?)
		\item GB reanalyses tense (+agr) features (sc. verbal inflection) as the head of the phrase
		\item A consequence of the `historical accident' (owing to Pollockian Romançocentrism) was the relabelling of \textbf{I\textsc{nfl}} as \textbf{T}. (Creating issues for the extension of T to eg Halkomelem \& Blackfoot while preserving any substantial content for the category (i.e. $\lambda\mathcal R.t_\varepsilon\mathcal R t_\text{utt}$))
	\item \textbf{Halkomelem}, privative \textsc{prox/dist} distinction encoded by auxiliary:
	\pex\begingl\gla \textbf{í$|$lí} qw'eyílex tútl'ò//
	\glb \textsc{\textbf{prox$|$dist}} dance he//
	\glft`He is/was dancing here$|$there'//\endgl\xe
	\item \textbf{Blackfoot}, privative \textsc{(non)local} distinction encoded by ``order'' suffix \textit{-hp} which emerges only when predicating of Spkr or Addr.\mcom{\color{violet}At this point stop and have a think about what the content of a R\&W-style \textsc{Infl} for Yolŋu could be. Presumably syntacticians are gonna wanna say that the four inflections reside in \textbf{Iº}}
	\end{itemize}
\subsection{Formalising \textit{anchoring} (§§2.3-3)}
	\begin{itemize}
		\item \textsc{infl} relates \textbf{the event situation to the utterance situation}\mcom{the $s$-typing suggests a probably translatability to reality-status}
	\item `The abstract argument in SpecIP\mcom{wtf is a specifier} is in fact a pronominal situation variable which--in absence of a proper antecedent--is interpreted deictically' (1343)
	\item At \textsc{infl} is an abstract \& unvalued \textbf{coincidence} feature: ``the defn of spatial, temporal and identity relations in terms of `central' versus `noncentral' (or `terminal') coincidence'' (Hale 1986:238)\mcom{\color{violet}	}
	
	 \textsc{Coin}cidence might be positively valued iff two temporal intervals, two spatial regions or two participant roles/stages overlap (1345).\\`The core deitic categories tense, location and person can all serve [to value [$u$\textsc{coin}]]' (1346).
	 
	 Speaker comment `\textit{he} is the past tense of \textit{you}' -- taken by R\&W to capture an intuition about the shared noncoincidence of $s_\varepsilon\&s_{utt}$ encoding of both \textsc{pst} and \textit{3}.
	
	\Tree [.IP Utt-sit [.I $\lbrack u\textsc{coin}\rbrack$ $\varepsilon$v-sit ]  ] 
	\item Halkomelem \textit{-lh}: 
	\begin{itemize}
		\item doesn't participate in SOT (clauses embedded under \textit{-lh}-clauses don't obligatorily receive \textit{-lh}-marking)
		\item can attach to nominals \textit{sílá\textbf{-lh}} \texttt{grandpa\textbf{-pst}} `dead grandpa'
		\end{itemize}
	\end{itemize}
\subsection*{INFL without substantive content}

\begin{itemize}
	\item[\textbf{Subordination}] `infinitives obligatorily lack \textit{m-}tense...$u$\textsc{coin} is valued by embedding predicate.' (``predicate valuation'')
	\item `relevant literature identifies two subtypes of infinitive: • simultaneous and • fut-irr 
	\item Predicate type selects for one of these readings (AspVb in case of simul. vs. desiderative/directive in case of fut-irr)\mcom{Test Yolŋu embedding under aspectual  and desiderative/directiv constructions}
	\item R\&W's analysis has a +\textsc{coin} feature inherited by infinitives embedded under AspV, and $-$\textsc{coin} inherited under directives.
	\item[\textbf{Halkomelem}] ``all embedded clauses are nominalised''\mcom{\textbf{nb} that this is true of Yolŋu as well by an analysis where \textbf{IV} is some nominalised form (which is probably the most parsimonious analysis given the nominal marking that \textbf{IV}-forms take.)}
	\item Infinitives embedded under desiderative predicates (e.g. \textit{stl'í} `want, like') can still take spatial auxiliaries, but the futurate meaning is lost here (\textit{p}1356) -- \textcolor{red}{it is unclear to me how this isn't a problem for their analysis. (See pdf notes)}
	\item \textcolor{blue}{Table 1 on p1357 is a useful decomposition of the `valuation strategy' and `pro-sit' antecedent for finite vs. nonfinite embedded clauses.}
	\item There seems kind of almost to be an underdefended generalisation that \textit{lí} `\textsc{dist'} can cooccur w fut-oriented and \textit{í} \textsc{`prox'} w AspV but this is under-shown empirically.
	
	\item[\textbf{Blackfoot}] R\&W conclude that `predicate valuation is not available in this language'\mcom{Is this surprising? Why not?} Instead AspV and \textit{try}-class verbs take a bare vP (rather than IP) complement whereas future-oriented verbs require irrealis marking (what's this showing?)
	\item \textit{complex predication: }Blackfoot Asp and \textit{try-}type Vbs are \textit{preverbs} in the verbal complex (possibly formally identical to some sort of derivational prefix?) $\Rightarrow$ if these are monoclausal structures then there's no intervening Iº to be evaluated/to instantiate (\textsc{+coin}) predicate valuation.
	\item For future oriented verbs \textit{want, tell...} the embedded predicate obligatorily takes an \textsc{irr} prefix: \textbf{i think }the claim now is that this serves as m-marking of a \textsc{$-$coin} feature (so the predicate itself is doing no valuation.)? Although given the apparent claim that \textit{-hsi}, the dependent clause marker is there carrying a \textsc{+coin} feature i'm not sure about this.
\end{itemize}
	\subsection*{Valuation of \textsc{Infl} in imperatives \& counterfactuals}
	\begin{itemize}
		\item[Imperatives] -- who cares
		\item[Counterfactuals] past tense marking in [English] counterfactuals (protases) as an instance of ``fake meaning'' --- e.g. \textit{*(if) I had a car...}.
		\item R\&W take this as evidence that in \textsc{cfact} contexts past tense marking does not value I. (Is rather \textit{past agreement} like the imparfait du subjonctif, and is agreeing with counterfactual content in COMP (which apparently...exists now?).) 
	\end{itemize}
	


\section{Bittner 2011, 2006 (Greenlandic \texttt{[kal] }time \& modality)}
\begin{itemize}
	\item Bittner argues `that temporal and modal discourse anaphora can be just as precise in a language [without] anaphoric tenses or anaphoric modals' (2011: 147)
	\item Eng. \textit{will, would} translatable by `prospect-oriented' attitude states (expectation, desire, intent, need, anxiety, considered (im)possibility...)
	
	\item From her website, my emphasis:
	
	
	 \textit{In my Temporality book (Bittner 2014), I propose that every language has one or two grammatical paradigms of obligatory TAM categories (tense, aspect, or mood) interpreted as a \textbf{centering system that keeps track of top-ranked discourse referents}. Grammatical tense keeps track of top-ranked times; grammatical aspect, of top-ranked eventualities (events or states); and grammatical mood, of top-ranked modalities. The proposed theory predicts six types of languages: tense-based, aspect-based, mood-based, tense-aspect-based, aspect-mood-based, or tense-mood-based.}

\end{itemize}

\subsection{2006 \textit{J. Semant. 22}: Tenseless future -- viewed from Greenland}
\begin{itemize}
	\item Proposes to weigh in in favour of Ben Shaer's (incomplete, see B. 06:350) SULA treatment of \texttt{kal} as tenseless \textit{contra} descriptive accounts of a complex metric tense system. (\textbf{future reference retrieved from context, aspect marking})
	\item \texttt{kal} vbl inflections `do not park temporal location' (\textsc{nfut}) `no temporal ambig arises in actual use,\mcom{strong claim here: `no temp ambiguity' means temp interp would have to be priveleged vs other types of morphosyn-meaning underspec} bc the rel temp interp is predictable based on the aspect and the context' (344)
	\item Factual moods \textsc{ind, fct, que}: `can be reported [as such] only if it has already happened'
	\item[$\bigstar$] It's not actually obvious that some of Bittner's examples are actually unambiguous (8b', 10'...) but this is isn't really a problem per se for her tenselessness hypothesis, she just maybe puts her argument a bit too strong.

\end{itemize}
\begin{shaded}
\textbf{\texttt{kal} verbal inflection as a fused \textit{mood--aspect--centering} system which contrasts the following mood families:}
\begin{description}
		\item[factual] \textsc{ind, que, fct}
		\item[nfactual] \textsc{irr, non}
		\item[prosp] \textsc{opt, imp, hyp}
		\item[circ] \textsc{elab, hab}
\end{description}
\end{shaded}

\begin{itemize}
	\item  a consequence of this is the ``subsum[ption of] future discourse under stative discourse' by analysing what had elsewhere been described as bound future morphology as future-oriented (propositional) attitudes:
	
	\begin{itemize}
		\item Traditional approaches gave \textit{-ssa, -niar, -jumaar} as future markers. Diffs were vaguely described as respectively holding possible \textbf{modal} meanings, \textbf{incipency} meanings or \textbf{vague/indefinite fut} meanings
	\end{itemize}
\item Shaer blows fortescues treatment out of the water with examples containing what had prev been described as past and future morphology in the same word (\nextx)
\pex\begingl\glpreamble \textit{\texttt{kal} probable past prospective}//
 \gla atur-\textbf{sima-ssa}-va-a//
 \glb use\textit{-sima-ssa}-\textsc{ind$_{tr}$}-3s$>$3s//
 \glft`He must have used it.'\trailingcitation{(349, Fortescue 80:267)}//
 \endgl\xe

\item Bittner's text shows a wide variety of strategies for encoding future discoure a ``natural class'' (352) consisting of:
\begin{description}
\item[Prospective statives]  (future-oriented mental states/prop. atts to \textit{de se} prospects) `be.likely, expect...'
\item[Prospective inchoatives] (that subtype of AspV when composed with realis \textsc{(ind)} marking entails that a result state (for expected processes) has optained at speechtime.) 
\item[Prospective matrix moods] indicating request/wish-status of prejacent `(please)+\textsc{imper/hort}'
\end{description}
This is her \textsc{Prospectivity thesis} for `\texttt{kal} translations of future auxiliaries (from Gmc)' (354)

\item A selection of her that prospective statives are not tense markers (but rather a type of predicate)\mcom{B has tried to make her def of tense as ``liberal'' as poss, but then there are these predicational approaches (§0) that exist in the lit, assoc. w Dowty...}
\begin{itemize}
	\item There are more than twenty lexicalisations
	\item They are not paradigmatic/can cooccur
	\pex\begingl\gla...qarsursaq uummati-n-nut apissigul-lu-gu tuqqutigi\textbf{-nioa-ssa}-va-t//
	\glb...hook heart-2s-\textsc{dat} have.$\bot$.go~all~way~in-\textsc{ela$_\top$}-3s$_{\bot}$ die.from-be.intended.be.desired-\textsc{ind$_{tr}$-2s$>$3s}//
	\glft`...[I] intend \& desire that you die from having the hook go all the way into your heart.'\trailingcitation{(355-6)}//\endgl\xe
	\item they evidently ``survive nominalisation''
	\item TO for prop attitudes (future TO requires a prosp stat)\mcom{The TO data she uses on 373 is v v unclear/doesn't seem to do exactly what she says it does.}
\end{itemize}
\subsection{2011 in \textit{Tense across languages}}

\begin{itemize}
	\item `Instead of tense,[\texttt{kal}] has a grammatical system of mood inflections that distinguish currently verifiable facts (decl, interr or factual mood) from current prospects (imper, opt, hyp moods)
	\item \textbf{``futurity is a species of fact''}\mcom{in (\nextx) there is an unsegmented line in the gloss which is either a typo or suggests that \textit{-pu} (the factuality marker) is not pronounced. If it's not a typo this is pretty bad form... The generalisation does seem to be repeated in exx. further down.}
	\pex\begingl\gla Aani ajugaa-ssa-pu-a//
	\glb Ann win-exp$^{>}$\textsc{-dec$ _{T} $}-3s//
	\glft`Ann will win' (lit. `is expected to')//\endgl\xe
	\item On a Kratzerian analysis, the \textsc{dec}$ _{T} $ marking seems to provide a modal base and the \texttt{exp$ ^{>} $} marker seems to provide an ordering source (see conditional examples on p 148.)
	\item Counterfactuals take \textit{-galuar} `\textsc{remote}' marking which seems to negate the requirement that $w'\in\boldsymbol{best}$
	\item[\textbf{Centering}] 
	\begin{itemize}
		\item Nominal centering marked by obviation ($3_\top;3_\bot$)
		\item modal centering marked by mood \textsc{dec/fct} and category/derivational suffix
		\item\mcom{Something similar is likely to be happenimng in Yolŋu also.} Warlpiri topic anaphor \textit{ngula-ju} is trebly ambiguous between nominal, modal and temporal readings so: 
		
		
		\ex\begingl\gla maliki-rli kaji-ngki tarlki-rni nyuntu ngula-ju kapi-rna luwa-rni//
		\glb dog\textsc{-erg} \textsc{cmp-}$3s>2s$ bite\textsc{-npst} 3s that\textsc{-top} \textsc{fut}-$1s>3s$ shoot\textsc{-npst}//
		\endgl\xe
		
		
		\begin{enumerate}[A.]
			\item That$ ^{\top} $ dog that bites you, I'll shoot \textit{it}$ _{\top} $
			 \item If$ ^{\top} $ a dog bites you, I'\textit{ll}$ _{\top} $ shoot it
			 \item When$ ^{\top} $ the dog bites you, I'\textit{ll}$ _{\top} $ shoot it
		\end{enumerate}
	

	\end{itemize}
\end{itemize}
	\item \textbf{The rest of the paper is an extremely dense formalism where B lays down an a typed, dynamic logic $\boldsymbol{UC_\omega}$ that makes use of information states consisting of ``top-bottom'' lists $\langle\top\bot\rangle$.\\
	I don't want to labour through this unless I need to. It seems like a powerful language but probably too baroque to be helpful. I suspect Tonhauser's formalism does most of the things that this one does through a different notation.\\Bittner's 2014 book \textit{Temporality} builds this up gradually by the look of it (although is probably huge by the time you get to the end.)
}


\end{itemize}


\section{Bochnak 2016 (Washo [\texttt{was}] optional past tense ref)}

\begin{itemize}
	\item Bochnak suggests that the \textsc{tensed/tenseless `dichotomy'} implied by work on (parametric) x-linguistic variation `does not follow from a purely semantic perspective'
	\item Prediction that languages ought to occur where `morphological tenses are not obligatory...where reference time can be identified through other (linguistic or contextual) means' (248)
	\item Washo has a nonparadigmatic/optional past suffix \textit{-uŋil} which Bochnak analyses w standard deictic semantics.
		\begin{itemize}
			\item matrix clauses: incompatible w non-past frame adv.
			\item 262: \textit{-uŋil} targets $t_r$ (as opposed to $ t_\varepsilon $) given interactions with \textsc{prosp}
			
			\pex \begingl\gla t'e:k'eʔ heyéʔem-\textbf{ašaʔ-uŋil}-a-š git-behúweʔ payʔ-ha-i//
			\glb much 3.win\textsc{\textbf{-prosp-pst}-dep-sr} \textsc{refl}-ticket 3.lose\textsc{-caus-ind}//
			\glft`He was going to win a lot of money, but he lost his lottery ticket.'//\endgl\xe
			
		\end{itemize}
	
	\item He uses this opportunity to argue against a \textsc{MaxPresupp}/covert \textsc{(nonfut)} tense analysis.
	\item Washo is canonically tenseless: \textit{-ašaʔ} \textsc{`prosp'} is used for matrix futures generally but tenseless (\textsc{plan}-type) futurates also occur without\mcom{\textbf{Is this gonna be true for \textit{dhu} as well?}}
	\begin{itemize}
		\item Conditional antecedents etc can receive an abs fut interpretation w/o any \textsc{fut} morphology.
		\item In embedded predicates (including att predicates), \textit{-uŋil} is subject to sequence of tense effects (i.e. morphological tense is not semantically interpreted in embedded predicates $\to$ ambiguity between ``simultaneous'' and ``back-shifted'' readings, explained on p 268.)
		\item Bochnak marshals these data as further evidence for the \textsc{tns} status of \textit{-uŋil} (as opposed to, say, \textsc{term} aspect)
		\begin{itemize}
			\item\mcom{Why SOT? This feels like it must have been an asked question. I suspect an constraints-on-learning-type account is probably the sort of one that we want for this!} an \textit{-uŋil}-marked clause embedded under a tenseless one \textbf{cannot} receive the simultaneous reading. (This probably lends some support to SOT.)
			\item An implementation for the back-shifted reading is (partially) spelled out on pp269-70: it involves a number of additional assumptions including \textit{res}-movement (?? Abusch 97, Heim 94), the ``upper limit constraint'' and the ``acquaintance relation''.
		\end{itemize}
	\item \textit{-uŋil} can optionally right-adjoin to the head where it introduces a standard \textsc{pst} presuppositional semantics
		\end{itemize}
	\item \textbf{Tenseless clauses in Washo}
	\begin{itemize}
		\item Untensed clauses are compatible with \textsc{pst, pres, fut} reference. Temporal adverbs constrain $t_r$
		\item Similarly to Guaraní (Tonhauser), untensed matrix clauses tend to be \textbf{incompatible} w future-oriented adverbs:
		
		\textit{`only ref times on the subsets of histories that are already setteld can be referred to in morphologically tenseless clauses'} (assuming a Thomasonian \textsc{pst/fut} asymmetry: branching times.)
		\item Tenseless absolute futures are available as • \textsc{plan}-type futurates, • with \textsc{seq} marking (complex sentences) and • in conditional antecedents.
		\item For B, these arise because of a covert modal in \textsc{plan} and \textsc{cond} (probably reasonable??) and in \textit{-ud} `\textsc{seq}' because of the existential binding of a $t_r$ (i.e. a different tense mechanism.)
		\begin{itemize}
			\item Copley's \textsc{plan} is a metaphysical modal operator:\\
$ c $ provides a director $ d $ s.t. it is \textbf{presupposed} that $ d $ can make it such that $ p(t)(w) $ and \textbf{asserted} that $ d $ is committed to $ p(t)(w) $
\item commitment TC formalised as: $$\llbracket\textsc{plan}\rrbracket^{g,c}=\lambda P\lambda t\lambda w.\forall w'[w'\in\textsc{best}(mb,ob_d,t,w)]\to\exists t'[t'\succ t\wedge P(t')(w)]$$
\item Copley claims that the `rules of the universe' can act as $ d $, permitting an inertial-type reading for things like \textit{the sun sets at 5.30 (tonight)}... (2002:37) B ignores this and marshalls the unavailability of tenseless `[it] rains tonight' in Washo as evidence of a \textsc{plan} operator. \mcom{The consequences of this convenient omission/prediction are unclear, but it's a bit dodgy of him... Would this need to be modelled as xling variation w/r/t the availability of ``the universe'' as a $d$?}
		\end{itemize}
	\item For antecedents, adopts Tonhauser's treamtent of \textsc{cond} as asserting `the existence of times, which can be in past, pres or fut of utt time' (273)
	\item \textit{-ud} `\textsc{seq}' existentially binds a time:
	$$ \llbracket\textit{-ud}\rrbracket^{g,c}=\lambda P\lambda t\lambda w.\exists t'[P(t')(w)\wedge t'\prec t]$$
	
\mcom{isn't this exactly what partee, klein a.o. say about discourse principles involving the free concatenation of eventive predicates?? Is it just a promotion to truth conditional meaning being proposed here?}	When composed with a \textsc{conj} operator, this is supposed to get you identity between the existentially bound time and the time of the following event description. 
	\end{itemize}
\item \textbf{§4.3.1 --- contra Matthewson 2006 for Washo}
\begin{itemize}
	\item Major advantage would be the \textsc{nonfut} type reading that tenseless clauses get in matrix clauses as described above without any additional machinery.
	\item The \textit{ud-} `\textsc{seq}' data are the main issue that a covert-\textsc{nonfut} would struggle to respond to (280)
	
\end{itemize}
\item {\bf §4.3.2 --- contra Tonhauser 2011 for Washo}
\begin{itemize}
	\item \textit{Claim}: there is no temporal pronoun/no Tº in Guaraní. (``Matrix tense rule'')
	\item But \textit{-uŋil} needs a home.
	\item Guaraní: no backshifted interpretation for tenseless clauses embedded under attitude predicates. The fact that this interpretation \textbf{is} avail for Washo argues for a temporal pronoun.
\end{itemize} 
\item \textbf{§5 --- Contra a \textsc{MaxPresupp} account}
\begin{itemize}
	\item If indeed \textit{-uŋil} is \textsc{pst}, then its optionality is a problem for the theoretically universal status of MP (following schlencker.)
	\item B gets around this by suggesting that a `paradigm[atic opposition]' is necessary for MP to apply: i.e. B imposes a condition on the formal elements of sentence alternates where $\alpha,\beta\neq\varnothing$. (Includes a discussion about (symmetry between) notions of `semantic' and `morphological' markedness)
\end{itemize}
\item \textbf{§6 Cessation implicatures}. Jacobsen 64 on Washo makes a claim that \textit{-uŋil} provides something like a discontinuous past.
\item B shows that this is too strong and too weak.
\item Beyond showing the non-TC status of cessational meaning, B doesn't venture to explain the distinct contribution of \textit{-uŋil}
\end{itemize}
\section{Temporal remoteness: metricality, relativity and gradability}\label{TRM}

\begin{itemize}
\item For Maya, \textbf{Bohnemeyer} models metricality (in the vpt. asp domain) as part of the at-issue content using a $D$ relation between topic and runtime (``cardinally quantify over the temporal distance bw $t_{top},t_{\varepsilon}$). (effectively $\mu$) (p.+20)\mcom{A $\mu$-relation was always gonna be the way to deal w/ metricality but modeling this as at-issue feels more contentious.} \textbf{See also 2018 SULA presentation.}
\begin{itemize}
	\item In effect these the are predicates of times `\textsc{be a long time}' (22 apud 1998, 2002, 2009).
	\item Similar to Cable 2013's account of TRMs in Givón's 1973 chiBemba doc: $\denote{TRM}=\Delta(t_\varepsilon,t_\textit{utt})$. Problem with this view is what do you do with $t_r$: is remoteness measured w/r/t other times? (yes, evidently Sesotho has \textsc{anaphoric metrical tense} where $\llbracket\textsc{trm}\rrbracket=\Delta(t_\varepsilon,t_r)$)
	\item[\textbf{Klecha \& Bochnak 2016}] inventorise Luganda's ternary metrical past tense system w measure functions and Kennedyne contextual standards. The distant past is the unmarked case which presumably is resolved by dint of Gricean inference once the temporal distance retrieved is not \textbf{close} or \textbf{$\neg$far}
	\item[\textbf{Cable 2013}] On the basis of Giyũkũ's 5-way distinction (documented by Mugane 1997), takes TRM to constrain $\Delta(\tau(\varepsilon),t_U)$ (i.e. not Kleinian tenses.)
	\end{itemize}
\item \textit{-uŋil} can but need not cooccur with TRMs (not discussed elsewhere in the paper), which according to Bochnak are evidence that tense markers can cooccur if their presupps don't clash.
\end{itemize}

\subsection{Cable 2013 \textit{NLS~21} (Gĩkũyũ \texttt{[kik]} graded tense)}
\begin{itemize}
	\item Cable sets his paper up as dealing with the option of \textbf{more than 3 tense distinctions} being grammaticalised, the logical converse to the existing literature on \textbf{tenselessness} (where fewer than 3 are)\mcom{Doesn't it feel like the whole question is exploded a little bit by the Arnhem/Yolŋu data?}. Marion Johnson 1977PhD, 1980, 1981 has also dealt with this question and these data.
	\item tense prefixing: 4 past tense grades (\textsc{cur, nr, rem}), 3 future tense grades in Gĩkũyũ.\\ChiBemba distinguishes \textsc{imm.past, today.pst, yest.pst, earlier.pst.}\begin{itemize}
	\item Per Cable's analysis the of the \textbf{cur} prefix actually differs (\textit{kũ-/$\varnothing$-}) depending on whether the verb receives \textsc{ipfv/pfv} suffixation (this is implicitly modelled as some type of ``aspectual agreement'' i guess?)
	\item Perf-suffixed verbs can be \textbf{unmarked} for TRM! (Cable argues against previous treatments of non-TRM-unmarked forms as pluperfects 227,§5)
	\item Then there's this kind of aberrant \textbf{imm.pst.pfv} prefix that replaces the need for a suffix ostensibly. This form is treated as a \textbf{homophone of the distant pst marker.} The viability of this analysis (228, Johnson 1980) is possible, although I'd love to see work on these markers' etymologies. 
	\item Current future (compositional) marks • same-day future and • psychologically 'close' (same-$i$ future), similar perhaps to present tensed/realis futurates (\textsc{plan}?, see 11b p229)???\\
	Based on previous descriptions from the mid-C20th, Cable suggests that the \textbf{\textsc{current future} was previously describable as \textsc{present imperfective} and \textsc{hodiernal future}. The \textsc{pres} use has been displaced by another form.}
	
	
%\begin{table}[]
	\begin{tabular}{l|llllllll}
		& Ø & \textcolor{blue}{\textbf{RemP}} & \textcolor{green}{\textbf{NrP}} & Cur & \textcolor{blue}{\textbf{Imm.PP}} & \textcolor{green}{\textbf{Prs.Ipfv}} & †NrF & RemF \\\midrule
		pst.ipfv     & * & a-   & ra- & kũ-& &          &      &      \\
		pst.pfv \textit{-ire} & * & a-   & ra- & Ø-&  &          &      &      \\
		perf \textit{-ĩte   } & Ø & a-   & ra- & kũ-& &          &      &      \\
		Ø            &  ? &      &     & kũ- & a- & ra-      & rĩ-  & ka- 
	\end{tabular}\mcom{I've made a few edits, I think there's some generalisations that, based on the first third of the Cable reading, he seems to be missing or ignoring (namely some sort of morphomic analysis.)}
	\item The other side of the same puzzle is that Cable and Johnson \textbf{both assume that the \textsc{prs.ipfv} and \textsc{nr.pst} prefixes (\textit{ra-}) are homophonous p231}. \textcolor{violet}{now \textbf{this is strikingly similar to the Yolŋu patterning and suggests that we're not looking at coincidental homohphony here!}}
\end{itemize}
	\item TRPs modelled as presuppositions that concern $\tau(\varepsilon)$ directly rather than a ``topic time''
	\item Interesting pragmatic datapoint where dance+\textsc{near.pst} seems to be restricted to hesternal, whereas die+\textsc{near.pst} permits of `past few days' readings. (224)
	\item Assumes, per Copley 2005 that \textsc{fut} is an aspectual head (inverse of \textsc{perf,} so effectively \textsc{prosp})
	\item Cable's analysis (probably uncontroversial in contemporary temp sems) is that TRMs are partial identity functions (just like tense: presuppositional operators)
	\item[\textbf{ignorance of temporal location:}] `when did you buy that TV?' or `they bought a new TV!' -- unmarked case takes \textsc{RemPst} marking, otherwise implicates strong suspicion of immediate recency (suggests a default or weaker semantics for \textsc{rem?} Cable thinks so.)\mcom{Alternatively it could come coupled with a default belief that things in the past weren't really immediate for the most part? I \textbf{don't} actually see how this differs to Bochnak's neogricean approach. In fact it seems like it makes \textbf{identical predictions.}} \begin{quote}
	\textit{Cable's generalization regarding the `Remote Past'}
	
	\textsc{RemP} is used when a speaker does not know whether an event occurred on the day of utterance, `recently', or some time prior to that.\hfill{(241,~ex~36)}
		\end{quote}
\item Similarly, partial-ignorance (as in today or otherwise v recently takes \textsc{NrPst} as would be expected). There's almost some evidential component: ``...because you were in touch with them yesterday, and so know the TV wasnt bought before then.'
\item Similarly again, unless it is known that something will happen today, the \textsc{current} is infelicitous in competition with a \textsc{Fut} marker.\mcom{In a summary on p245, Cable formulates \textit{informal semantic hypotheses} based on the \textit{generalizations} formed over the data seen so far.}
\item $\langle\textsc{RemPst,NrPst,CurrPst(,ImmPstPfv)}\rangle$\\$\langle\textsc{RemFut,CurrFut}\rangle$

Also demonstrated by coordinating TFAs (``today \textsc{and} yesterday'') is ill-formed with \textsc{CurrPst} but grammatical \textsc{NrPst}
\item[\textbf{Cable's ``empirical challenge''}] speakers deny the truth of weaker equivalents when a stronger is available (there ought to be an entailment relation). This could maybe be modelled as a conventionalised scalar implicature? (And really, in effect this is exactly what Cable's \textit{TRM specificity principle} reinvents. He explicitly denies this on p250:
\begin{itemize}
	\item\mcom{\textbf{Bullshit}: it's conventionalised/grammaticalised. This just means that like any grammatical distinction that is obligatory, the neogricean hearer-based implic. obtains:\textit{ if you say $W$ you're not in a position to say $S$.\\This responds to his second empirical problem too. TFAs definitionally are not grammaticalised. wtf is he on about?}} Maxim of quantity requires ``speakers to be as informative ``as is \textit{required} for the purposes of the exchange.'' Cable claims that the exact temp. location is not required for any of the examples he gives.
	\item Behaviour with \textbf{temporal adverbials in nonfinite clauses.} TFAs are all optional and can be as specific as S wants: \\
	\textit{Mwangi wanted to go to New York (yesterday (evening))}
\end{itemize}
\item The argument is that, as with Slavic \textsc{dl} \textit{vs.} \textsc{pl}, this should not be treated as a scalar implicature (anti-presupposition) derived from Dvořak \& Sauerland's (2006) \textsc{MaximizePresupposition} principle. \mcom{\textbf{It's unclear to me how this makes any differnt projections?} Havent most people cast MP in terms of Gricean reasoning \textit{anyway??} (Ask LArry)}


\item The semantics then are pretty basic:

 $\llbracket \textsc{cur}\rrbracket^{g,t}=\lambda e:\tau(e)\infty\text{ day surrounding }t.e$
 
 Plus a couple of $\mathcal{I\to I}$ functions:
 \begin{itemize}
 	\item \textsc{impst}$(t)=[t_1\hdots t_2)$ where $t_1,t_2\prec t$ and both lie within the day surrounding $t$.
 	\item \textsc{rec}$(t)=[t_1\hdots t_2)$ where $t_1\underset{\text{nontoday}}{\prec} t$ and $t_2$ is the end of the day surrounding $t$
 	\item These functions just take ther $t$ in the \textsc{imm} entry to give \denote{IMM} and \denote{NRP} respectively.
 	\item \denote{\textsc{rem}} then is just gonna be an identity function ranging over $e_j\in D_\varepsilon$ (syntactically in Spec,AspP).
 	\item The syntax is spelled out more pp. 256ff but I'm not convinced this is interesting.
 	\item He uses a $\infty$ `overlap' (although i think this is formally undefined??) instead of $\sqsubseteq$ `subint' relation because of situations in which an event may have started in the \textsc{rempst} but spanned into/across the \textsc{nrpst} ($\therefore \tau(e)\not\sqsubseteq\textsc{nrpst}(t)$). In this case \textsc{rempst} marking is infelicitous.
 	\item \mcom{Mmm, this is good for the point he's making right now... but couldn't his semantics predict general infelicity for i-level preds with the remote categories? Given that these have the weakest presuppositions?} Additionally statives as in `he was tall/from~africa' take \textsc{nrpst} when talking about someone who was recently met.
 \end{itemize}
\item For C the consequence of his analysis is that TRMs are \textbf{not} \textit{sensu stricto} `tense': `restricted id functions on times, which mod the T node of the cl and thereby restrict the TT' (264).
\item C's analysis (favoured) is that it is ET, not TT that is restricted -- this is testable by looking at the PERF and maybe FUT domains
\begin{itemize}
	\item Perfect sentences optionally take TRMs -- they seem to relate to the eventuality predicated given the contextual minimal quadruple on 266-7. (Modifying the time of the event of Mwangi's visit w Obama leads to differential TRM selection.)
	\item I think I need to go back and think about whether the diagnostics C uses in this section are doing what he says they are. (Given that tense is a relation between topic time and speechtime, it's not immediately clear why the tense analysis is sunk for him using these examples.)
	
	Additionally, Dahl 85:120 as cited in Bohnemeyer 2018 seems to have said the same thing thirty years earlier.
		\end{itemize}
	\item C's major typological contribution is the idea that TRMs are taken to modify \textbf{event variables }and are `in a local morphsyntactic relation with \textit{neither} tense \textit{nor} aspect' (274) although distributionally:
	\begin{itemize}
		\item obligatory w \textsc{pst.ipfv} and \textsc{pst.pfv} forms, \textit{optional} with \textsc{perf} and incompatible with \textsc{pres.ipfv}.\\\texttt{\textsc{imm.pst}} can only occur w \textsc{pst.pfv}.

\end{itemize}
\end{itemize}


\subsection{Bohnemeyer 2018 \textit{SULA}}
\begin{itemize}
	\item \denote{TRM}$=\Delta(E,S)$ as the ``traditional'' (Dahlian 85) view (reinvented perhaps by Cable.)
	\item The E,R vs R,S ambiguity question was also taken up by Dahl 85:121 (and Comrie 85: 86) by way of a Sesotho apparent anaphoric TRM distinction
	
	\pex\a\begingl\gla ha letatsi le-likela re-ne re-tsoa tloha Maseru//
	\glb when sun \textsc{prv}-disppear we-\textsc{pst} we-\textsc{immP} leave Maseru//
	\glft`At sunset, we had just left Maseru' //\endgl
	\a\begingl\gla ha letatsi le-likela re-ne re-tloh-ile Maseru//
	\glb when sun \textsc{prv}-disppear we-\textsc{pst} we-leave-\textsc{recP} leave Maseru//
	\glft `At sunset, we had left Maseru' //\endgl\xe
	
	\item `break with reichenbach': contemporary views (Kamp \& Reyle 93, Klein 94, Kratz 98) on tense construe it as a R between \textbf{topic time} and evaluation time which can be \textbf{either $t_{utt}$ OR some other $t_r$}. (Whereas viewpoint aspect constrains $R(t_t,t_\varepsilon)$)
	
	
	\item Unlike Cable's TRMs $\Delta(t_{top},t_U)$ and K\&B's TRMs $\Delta(t_{top},t_r)$, \textbf{Bohnemeyer treats TRMs in Yucatec as constraining} $\boldsymbol{\Delta(\tau(e),t_{top})}\Rightarrow$ they are therefore treated as a relative/anaphoric system accessing ev. time.
	
	
	\item \textit{Good summary of his tests showing the non-deictic-tense nature of TA markers}
\end{itemize}
\subsection{Klesna \& Bochnak}
\begin{itemize}
	\item Entirely anaphoric analysis, $t_{top}$ is the $t_r$ of the main verb in the matrix clause
	\item Evaluation time $u$ is $t_{utt}$ or some other contextually retrieved $t_r$
	\item TRMs just encode some measure relation between these two times $ t,u $
\end{itemize}
\part{Interactions with modality}
\begin{itemize}
\item \textit{para} (irrealis modal, future marker) in  in Chamorro (Chamorro 2012)
\item tonnhauser \textit{-ta} 2011 
\end{itemize}

\section{Krifka 2015/6 (Daakie \texttt{[ptv]} T\&M reference)}
\begin{itemize}
	\item Krifka makes the generalisation (over which data?) that for Vanuatuan languages\footnote{Melanesian and adjacent groupings are given as an example of a language linkage as opp. to family (cf. Lynch, Ross \& Crowley 2002). I suspect that if this contrast is a useful one then Yolŋu may also be considered a linkage.} TMA marking `is typically centred around a realis/irrealis distinction.'
	\item Five inflectional categories (`modality markers') that encliticise (?) to pronominal stems (or appear bare in the case of 3s subject.) Realis/potentialis as the ``basic distinction.''
	\begin{enumerate}
		\item \textsc{\textbf{Realis}/actualis} \textit{-m(we)}
			\begin{itemize}
				\item ongoing
				\item past eventualities
				\item generics
				\item `fictional worlds' (i.e. storytelling??)
			\end{itemize}
		\item \textsc{\textbf{Potentialis}/irrealis}
			\begin{itemize}
				\item\mcom{nb. morphotactically parallel to function of \textit{$\varnothing$\textasciitilde dhu}} $\varnothing-$: directives, hortatives, commissives
				\item \textit{a-}: futurates are marked with an additional prefix 
			\end{itemize}
		 \textit{-p\textasciitilde-b(we)}
		\item\textsc{Negation} \textit{(te)-re}
		
				realis negation `a modality in its own right' (and for Kripke's analysis of the \textsc{realis} to hold it must be treated as such)
		\item \textsc{`Dependent' negation} ($\approx$ irrealis negation in 2016?)
		

		\item\textsc{Distal}
	\end{enumerate}
\item This basic distinction may have been described as ±\textsc{fut} but reality status seems to be a better way of characterising the opposition. Note the similarity of this claim to \textbf{McLellan's treatment of Wangurri} (and also evidently Lichtenberk on Manam)

\item Krifka assumes: \begin{itemize}

\item  that Iº is a `modal marker'
\item that Cº contains an xistential closure operator $\exists$
\item a Forceº with speech act operators like \textsc{assert} \end{itemize}

Consequently, he has subject traces originating in spec,$vP$ which must raise to $I^o$ where they obligatorily left-adjoin a modal marker (and ostensibly right-adjoin a \textsc{fut} prefix if necessary)
\item further, he deploys a Dowtian branching times-type model where $I$ is a partially-ordered set of indices $\langle \mathcal{W\times T,\preceq}\rangle$
\begin{itemize}
	\item $h\subseteq I$ is a history $\iff\forall i,i^\prime\in h.i\sim i^\prime\wedge\forall i^{\prime\prime}.i\preceq i^{\prime\prime}\preceq i^\prime\to i^{\prime\prime}\in h$
\end{itemize}
\end{itemize}
\subsection{\textsc{Realis}}
\textsc{realis}$=\lambda c\lambda p \lambda i\lambda i^\prime\boldsymbol{:\exists i^\prime:i^\prime\preceq c.p(i^\prime)}\wedge: i^\prime\preceq i.p(i) $

\textbf{my paraphrase}: A realis marker uttered in a context $c$ presupposes the existence of an index $i^\prime$ which precedes (or is coextensive with) the utterance context. The event described in the prejacent holds at this time. It further presupposes that $i^\prime$ precedes (or is coextensive with) a reference index $i$. \textsc{realis} asserts that the proposition holds at this index.

\textit{i.e.} `presupposed that the host proposition is true at some $i$ at or before $c$ and that the event index $i^\prime$ is at or before the ref index $i$.

{\color{ochre} It's not actually completely clear to me yet how Krifka's formalism is supposed to get this, but so far he claims that realis contributes a number of restrictions:
	
	\begin{itemize}
		\item Speaker commitment to the truth of $\phi$ (i.e. $!\phi$), (which is taken to account for apparent infelicity with negated propositions, modelled as a presupposition failure 2016:569)
		\item That the eventuality predicated must precede the index of utterance/evaluation (see e.g. my work on \textit{bambai} and foregoing ideas.)
	
	\end{itemize}

\paragraph{Realis negation} contributed by a different morpheme \textit{tere} is taken to `[express] the condition that $\phi$ is not true at or before $c$. Note the weakness of existential closure.... $\neg\exists i^\prime\preceq c[\phi(i^\prime)]$ `It is not the case that there's an index preceding the utterance context such that $\phi$ holds at that index'

	}
	\begin{itemize}
		\item \textbf{Realis negation `expresses an antifactive presupposition {\color{ochre}[viz that there's no index that precedes utterance time at which $p$ holds?]} and negates the host proposition' (573)}
		
			\begin{itemize}
				\item Consequently this `modality' is not used to express \textbf{predictions or speaker preferences/(in)abilities} for non-instantiation. \textbf{This is the domain of `potentialis negation'} (see below)
			\end{itemize}
		
		\item Krifka acknowledges that `the range for which nonexistence [of an index] is claimed has to be pragmatically restricted]. 
		
		From what I can tell, he builds this caveat into his semantics by introducing a variable over histories and quantifying into this. (note that the entry for \textit{-re} asserts that the prejacent doesn't hold at $i^\prime$)
	\end{itemize}
	
His final semantics for the realis (\nextx) and its negation (\anextx) are:
\pex $\denote{$-m$}(c)=\lambda i\lambda i^\prime\lambda p\begin{Bmatrix}:h\preceq c\\
	:\exists i^\prime\in h[p(i^\prime)]\\
	\&\\
	:h\preceq i\\
	:i^\prime\in h\\
	.\boldsymbol{p(i)}
		\end{Bmatrix}$	\xe
\pex $\denote{\textit{-re}}(c)=\lambda i\lambda i^\prime\lambda p\begin{Bmatrix}:h\preceq c\\
:\neg\exists i^\prime\in h[p(i^\prime)]\\
\&\\
:h\preceq i\\
:i^\prime\in h\\
.\boldsymbol{\neg p(i^\prime)}
\end{Bmatrix}$
	

\subsection{\textsc{Potentialis}}

\begin{itemize}
	\item Unprefixed forms are performatives, future tense/predictive readings arise from prefixation of \textit{a} as in (\nextx).
	
	\pex\begingl\gla li\textdblhyphen malek \textbf{a-}na\textbf{-p} kuo \textbf{a-}na\textbf{-p} tinyam//
	\glb at\textdblhyphen night FUT-1SG-POT run FUT-1s-POT hide//
\glft	‘At night, I will run and hide.’//\endgl\xe


\item embeddable under \textit{ka}, predominantly non-factive environments (although also \textit{kiibele} `know (how)'$\approx$`be.able')
\begin{itemize}
	\item \textit{a} is derived from \textit{ka} (the irrealis complementiser which is still used in equivalent constructions in related Ambrym language Daakaka.) 
	\item \mcom{\textbf{(!!)} this seems dodgy as shit.} Krifka uses this (diachronic) fact to justify an analysis of \textit{a-} in Cº (575). 
	\item One nice consequence however is that whereas \textit{a} presupposes a \textsc{fut} relation between $i$ and $h$, \textit{ka} is underdetermined w/r/t the accessibility relation $\mathcal R$ `specified by the [lexical semantics of the] embedding verb'
\end{itemize}
\item `potentialis as expressing \textbf{that the host proposition is true at some later index} and as presupposing the same'...\\
`consequently, presupposes that the host proposition can be realized and that it cannot be negated' (therefore requires a special negation).

\pex 

$\denote{$-bo$}(c)=\lambda i\lambda i^\prime\lambda p
\begin{Bmatrix}
	:\exists i^\prime\\
	:i\prec i^\prime\\
	:p(i^\prime)\\
	\&\\
	:i\prec i^\prime\\
	.\boldsymbol{p(i^\prime)}
\end{Bmatrix}$	\xe
\item Meanwhile, the (syntactic complementiser?!) \textit{a-} is taken to quantify over possible post-$i$ timelines (``histories'').
\pex $\denote{a-}(c)=\lambda i\lambda p\forall h:\textsc{fut}(i)(h) \exists i^\prime h(i^\prime). p(i^\prime)$\\The utterance of \textit{a-} at some evaluation index $c$ requires a reference index $a$ and a host propsition $p$.\\
For all timelines $h$ in the future (\textbf{``inertial worlds''}) of $i$, there's some $i^\prime$   in those timelines at which $p$ holds. \xe
\item The performative uses (those without \textit{a-}) are taken to have a \textsc{pref} operator (rather than \textsc{assert}) in Forceº. (A semantics for this operator is given on pg. 576.)

\item Krifka says that he's built in some requirement to embedded (e.g. bouletic type) clauses where the embedded prejacent $\phi$ of the potentialis \textbf{can} become true after $i$. (Whereas counterfactual attitudes are expressed w/ realis negation or \textbf{distal} modalities).		\end{itemize}

\subsubsection*{\textsc{Potentialis negation}}
\begin{itemize}
	\item `Negative complementiser' \textit{saka} in Cº for negated preditions/preferences/``epistemic clauses'' (??).\mcom{(\nextx) maybe $\approx$`it won't be the case that he find me'?}
	\pex\a \begingl\gla sa\textdblhyphen{ka} \textbf{ne} lehe ngyo//
\glb	\textsc{c.neg} \textsc{\textbf{pot.neg}} see 1s//
\glft	‘He will not find me.’//\endgl

\a$\llbracket{saka}\rrbracket(c)=\lambda i\lambda p.\forall h:\mathcal R(i)(h) \neg\exists i^\prime h(i^\prime) p(i^\prime)$
\xe
A sentence w \textit{saka} in Cº, uttered at some index \textit{c} is evaluated in terms of another index \textit{i} and a proposition \textit{p}. For all timelines that are accessible to $i$ by some $\mathcal R$, there is no $i^\prime$ featuring in one of those histories at which $p$ holds.

(This is super weak. Krifka claims that it resolves a contradiction between $:\phi$ and $\phi$ that is introduced by the modal marker's falsity proposition, see below.)
\begin{itemize}
	\item  $\mathcal R$ is `pragmatically specified' (\textsc{fut, pref} or some ``epistemic relation'' (??))
\end{itemize}




	\item For dependent clauses the irrealis complementiser \textit{ka} appears again in the scope of \textbf{negated} (\textit{nare kiibele ka...} \texttt{1s.\textsc{neg} know C}) and \textbf{negative} (\textit{nam notselaane ka...} \texttt{1s.real be.mistaken C}) verbs.
	
	\item Krifka's treatment of this form is the same as the positive irrealis but with a \textbf{falsity presupposition.}
	
\ex	$\llbracket\text{-n(e)}\rrbracket(c)=\lambda i\lambda i^\prime\lambda p
	\begin{Bmatrix}
		:i\prec i^\prime\\
		:\neg p(i^\prime)\\
		.\boldsymbol{p(i^\prime)}
	\end{Bmatrix}$	\xe
	\begin{itemize}


	\item Note the presupposition failure: negation is then provided by the complementiser \textit{saka} (\blastx) above. The negation ``trickles down'':
	
	\item $\llbracket saka\rrbracket(c)(i)((\llbracket ne\rrbracket(p)(c)(i)=\lambda i\forall h:\mathcal R(i)(h)\neg\exists i^\prime:h(i^\prime):\begin{Bmatrix}
	i\prec i^\prime\\
	\neg\phi (i^\prime)\\
	\end{Bmatrix}.\phi(i^\prime)\\
	=\lambda i\boldsymbol{\forall h}:\mathcal R(i)(h)\forall i^\prime:h(i^\prime):\begin{Bmatrix}
		i\prec i^\prime\\
		\neg\phi (i^\prime)\\
	\end{Bmatrix}.\boldsymbol{\neg\phi(i^\prime)}$\hfill by def. $\exists,\forall$
		\end{itemize}
		\end{itemize}
	\subsection{Distal}
	\begin{itemize}
		\item `temporal scene setter either as an independent clause for the following discourse [\nextx]...or as an adjunct clause [\anextx].'
		\begin{itemize}
			\item In discourses where distal-marked clauses setting the temporal scene for future utterances, this is modelled as the (coreference) relation of two \textbf{topic times} $i^\prime$.
		\end{itemize}
		
		\pex\begingl\gla meerin témat la-t pwee//
		\glb long.ago zombies \textsc{3p-dst} be.many//
		\glft‘Long ago, there were many zombies.’//\endgl\xe
		\pex\begingl\gla yaa te van te pwet ti piipili mwe kuoli=mee tyenem//
		\glb sun \textsc{dst} go \textsc{dst} \textsc{prog} \textsc{dst} red \textsc{real} return=come home//
		 \glft‘When the sun became red (in the evening), he went home’//\endgl\xe


\item Also in counterfactual cases: claim that an attitude was held at an earlier time \textbf{implicates its falsity} (non continuation)

\item \mcom{Has this been predicted elsewhere as a non-occuring morphological form?} Krifka models the distal as \textbf{contributing no temporal information} (except some sort of \textsc{nonpresent} value?)\mcom{Would we also want to be building in some kind of restriction (minimal temporal distance?)}

\pex $\llbracket-t\rrbracket(c)=\lambda i\lambda i^\prime\lambda p:i\neq c.p(i^\prime)$\xe
\begin{itemize}
	\item In main clauses temporal adverbials (esp \textit{meerin} `long ago') anchors the event index. $\llbracket\textit{meerin}\rrbracket(c)(i)(i^\prime)=i^\prime\ll c$
\end{itemize}
\item In \textbf{embedded contexts} (e.g. under \textit{deme}, which lexically encodes a $\mathcal R$ as \textsc{dox}$(x)(i)(h)$ -- that relation that holds where nothing in $h$ runs counter to the beliefs of $x(i)$.)
\item Unlike \textit{kiibele} (which imposes \textsc{dox}$(x)(i)(i^\prime)$)-- beliefs of $x(i)$ about $i$-- there is no factivity presupposition.

\item \textbf{Conditional protasis} (i.e. antecedent, categorical??)
\item \textbf{Apodosis} (consequent) receives \textbf{potentialis marking in indicatives (those that \textbf{can} become true)} OR \textbf{distal marking in subjunctives} (without any presupposition of instantiation) as well as \textbf{the future prefix \textit{a-}} (which indicates that it's not a future marker but some type of viewpoint aspect.)

\item{\color{ochre}`While the distal is a relatively rare marker, it is semantically the least specified, and hence can be used in case the conditions for realis and potentials would not be met because they would introduce unwarranted presuppositions.'}
	\end{itemize}
\section{Matthewson 2010 \textit{S\&P3} `Variation in modality systems'}
\section{Rullman \& Matthewson 2018 \textit{Lang. 94(2)}}
\begin{itemize}
	\item restriction of modal domain by temporal factors as a ``perennial issue'' in NL modality (exemplified in \textbf{Condoravdi's (2002)}: epistemic v. metaphysical readings of \textit{might have})
	\item As in Condoravdi 2002 (?), \textbf{temporal perspective} is a function of some operator (e.g. \textsc{tns}) scoping over \textsc{mod}, \textbf{temporal orientation} a function of \textsc{asp} operators below. 
	\begin{description}
		\item[temp. perspective] $t$ at which conversational background is evaluated (w/r/t $t*$)
		\item[temp. orientation] relation between $t_{\textsc{tp}}$ and prejacent event	
	\end{description}
	\item Condoravdi's stipulation: `\textsc{epist} cannot scope under \textsc{pst} or \textsc{perf}' (282) potentially follows from syntactic hierarchy.\mcom{Cinque \textbf{still has no bloody explanatory power}; this just kicks the burden of explanation down the bloody road.}
	\item \textit{John might have won}
	\begin{tabular}{llll}
		$\boldsymbol f$ & \textsc{tp} & \textsc{to}& \textit{\textbf{reading}}\\\midrule
		\textsc{epist} & \textsc{pres} & \textsc{pst}& \textit{past epistemic}\\
		\textsc{circ} & \textsc{pst} & \textsc{fut}&\textit{metaphysical (counterfactual)}\\
	\end{tabular} 
	\item \textit{Contra} Condoravdi, R\&M claim that there are no grammatical restrictions imposed on/by TP by/on modal flavour. (The pragmatics inhibits/makes available these readings) (See Condoravdi 2002: 63 where she claims that modals with past TP only receive future TO, cited here on p 283.)
	\item TO is determined by \textsc{asp} operators and lexical aspect of predicates.
	\item {\color{violet}{\textbf{upshot:} the modal is ``atemporal'' under the H\&M's analysis.
			
			Temporal interpretation of modals ``is derived from the way [they interact] in a compositional fashion w independently motivated temporal operators''} (282)}
	
	\begin{itemize}
		\item Consequence is that ``all modals including epistemic ones scope under tense and therefore receive past TP iff the tense provides a past reference time'' (consonant w vF\&Gillies discussion of the availability of past epistemic interps)\mcom{This feels so correct.. how else to account for the felicity even of a se like \textit{i thought it might('ve) be(en) $P$}?}
			\end{itemize}
\item Basic type assumptions: (partially inspired by Kratzer (1998/\textsc{salt}8 `more structural analogies')). Denotations for T, \textsc{mod} and viewpoint (``ordering'' \& ``inclusion'') \textsc{asp} categories (sc. \textsc{(i)pfv}) are given on pp286-7.


	\Tree [ [. T$_i$ ] [ [.\textsc{mod}$_{\langle\langle i,st\rangle,\langle i,st\rangle\rangle}$ ] [ [.\textsc{Asp}ord$_{\langle\langle i,st\rangle,\langle i,st\rangle\rangle}$ ] [ [.\textsc{Asp}inc$_{\langle\langle\varepsilon,st\rangle,\langle i,st\rangle\rangle}$ ] [.vP$_{\langle\varepsilon,st\rangle}$ ] ] ] ] ]
	\begin{itemize}
		\item $\langle i,\langle s,t\rangle\rangle$ --- `property of times': given a time, returns all the propositions that hold at that time.
		\item ``ordering aspect'' -- perfect as overtly marked in Dutch/English, prospective as overtly market in Gitksan/St'át'imcets. So \textsc{perf, prosp} are modelled as functions from properties of times to properties of times; as e.g.:
		$$\denote{\textsc{perf}}^{g,t*,w*,f,h}=\lambda P_{\langle i,st\rangle}\lambda t\lambda w.\exists t'[t'\prec t\wedge P(t')(w)]$$\\So \textsc{perf} imposes a truth condition on a predicate (sc. an (untensed) event description that has been inflected for inclusion aspect): given a reference time $t$ and world $w$, the predicate is asserted to obtain subsequently to $t$ in $w$.
		
		The argument (which isn't explicated by R\&M) will be that there's a set of (inclusion-inflected) predicates: properties of times that are ordered linearly w/r/t the prejacent predicate. So the typing kinda checks out here. (ca. p. 287)
		
		Modal operators are modeled same type, but the imposed truth condition here involves quantification over worlds: the reference world is the input to the \textbf{cg} function.
	\subsection{Independence of modal flavour from temp perspective (§2)}
	
\item It's claimed here that temporal perspective is provided ``freely'' by some higher-scoping tense operator (300).
\item[\textbf{Dutch:}] past-tensed modals or \textsc{cfact} pluperfect $
\lozenge\to$ past epistemic readings.
\item[\textbf{Gitksan:}] all finite clauses are covertly inflected for \textsc{nfut}. Covert \textsc{nonprosp} (vs. \textsc{prosp} \textit{dim}), cooccur with covert tense marker.
	\item \textsc{prosp/nonprosp} has TO effects when a modal is present
	\item `due to the absence...instantaneous present-tense morpheme, both eventive and stative predicates can pick out $\mathcal E\circ t*$ without the need for \textsc{ipfv} marking.' (292)
	\item Epistemic modalities marked w 2nd pos. clitic \textit{\textdblhyphen imaa} (does not trigger dependent marking on predicate associate).
	\item Circ \textsc{pos} \textit{da'ḵhlxw} ``regular verb''. \textsc{neg} w subordinating pcl \textit{sgi}
	\item R\&M just run the data and show that all three of these lexical items admit of ``both past and present T[emporal ]P[erspective]s'' (294)
\item[\textbf{St'át'imcets}] similarly \textsc{nfut/fut} distinction, prospective marked by \textit{cuz'} `\textsc{asp}' or \textdblhyphen\textit{kelh} `mod'
	\item This subsection just rehearses the observations made of Gitksan.
\end{itemize}
\item R\&M lift a bunch of text corpus data to show that past epistemic readings of Eng semimodals exist (k...)
\item The (sensible) claim is that present access to an \textsc{epist} $f$ is more easily available because $t*$ is available by default for TP whereas other $t$ require rich contextual support (frequently as \textbf{free indirect discourse} (cf. Eckardt 2015))
\item Note that epistemic verbs (semimodals, propositional attitudes, \textit{seem}) contrast with modal auxiliaries in that they seem to freely admit of past TP by inflecting for tense. (Other contrasts include that they contribute to at-issue (rather than backgrounded) content.)
	\end{itemize}
\subsection{TO and Aspect (§3)}
\begin{itemize}
	\item TO is restricted by \textbf{modal flavour} \& Asp$_{view} $/Asp$_{ord}$
	\item The \textsc{diversity condition}: Condoravdi uses this special case of a ``general informativity constraint on assertions'' (300$ _{\texttt{fn23}} $)\footnote{This isn't uncontroversial; referenced in note 24 \textit{Jean a pu partir} perfectively inflected modal gives rise to actuality entailment.} to account for the unavailability of metaphysical flavours with nonfuture temporal orientations---because the modal must return a proposition contingent in the modal base i.e...
	
	\[ \llbracket\textsc{Mod}(P)(t)(w)\rrbracket^{g,t*,w*,f,h}\text{ is defined iff } \exists w',w''[w',w''\in\cap f(w,t)\wedge P(t)(w')\wedge P(t)(w'')] \]
	
	\item Lexical aspect distinction: statives permit of present TO whereas eventives are restricted to future orientns (she must sing$_{\text{fut TO}}$/be singing$ _{\textsc{nfut}} $) -- derivable from fact that eventives lack subinterval property, cannot be contained in instantaneous $t_{ref}$
	\item[\textbf{Gitksan}] : occurrence of \textsc{prosp} iff future TO (else nonfuture.) \textbf{A consequence of this is the obligatory cooccurrence of \textit{dim} `\textsc{prosp'} for circumstantial modals.} (predicted by C's diversity condition.)
	
	\item $\denote{\textsc{nonperf}}^i=\lambda P\lambda t\lambda w.\exists t'[t\preceq t'\wedge P(t')(w)]$ (i.e. $t_\varepsilon\succeq t_{\text{ref}}$)
	
	\item\begin{tabular}{lll}
		\textit{Language} & \textsc{Asp} &$ \leftrightarrow $\textbf{ TO}\\\midrule
		\textsc{ned/eng} & \textsc{perf} &$\leftrightarrow$ \textsc{pst }\\
		\textsc{git/lil} & \textsc{prosp} &$ \leftrightarrow $ \textsc{fut}
	\end{tabular}
	
\end{itemize}
\subsection{}
\begin{itemize}
	\item[\textbf{§4}] is just
	\item[\textbf{§5}] looks at modals/semimodals in English, comparing the TO/TP combinations each ``class'' seem to admit of.
	
	Only really interestng hypothesis here is that ``diachronically this instability...loss of productive (overt) tense inflection on the modal auxiliaries...tense feature to be lexicalised as part of the modals themselves...recruitment of perfect \textit{have} which normally marks past \textbf{TO} to mark past \textbf{TP} instead...'' (321) 
	\item ``...even in English \textbf{TO} is determined in a completely predictable way by the interaction of Aktionsart, aspectual marking...and the diversity condition''
	\item ``[no] cases...in which a modal is idiosyncratically specified in the lexicon as having, say, past \textbf{TO}''
	\item[\textbf{§6: cntra other analyses} (esp Hacquard)]
\item  on English: TP of \textsc{epist} is always the \textit{local time of eval} (=$t*$ for matrix clauses.)
		\item past epistemic tp only available \textbf{(a)} embedding under prop att vb, \textbf{(b)} FID, \textbf{(c)} adverbially explicated conv bkgrd, \textbf{(d)} elided \textit{because}. R\&M (decently convincingly) provide reasons why these constraints fall short.
	\item[\textbf{§7 \textsc{Conclusions}}] 
	\item Sem-syn interface: general tendency for epistemic lex items to scope higher --- contested of Eng epist modals as against epist adverbials here.
	\item\mcom{It could be that Dhuwal(a) at least be analysable as the \textsc{git/lil}-type?} R\&M rehearse their typological claim that languages \textbf{either} mark an overt \textsc{perfect} \textbf{or its dual}, an overt \textsc{prospective.} (Identical or similar/divergent systems elsewhere?)
\end{itemize}
\section{Verstraete 2006 \textit{AJL26} `Irreality in pst domain'}

\section{vanderblok 2012 \textit{TMA in Paciran Javanese}}
	\begin{itemize}
		\item Ch 5 on the semantics deploys storyboards and questionnaires (powerpoint administered) to test various modal flavours and forces and how these are lexicalised in Paciran.
		\item In Ch. 6 vdB lays down a \textbf{hypothesis for the typology of modals} \textit{languages can only vary along one axis for a modal domain }(i.e. `where there is one modal that stands for all modal meanings) (275ff)
		\item{\color{violet} This isn't obviously well defined (what's the future?) but could \textit{balaŋ} be an example of a completely underspecified modal? What range of meanings does it share with \textit{dhu}}

	\end{itemize}

\section{treatments of Negation}
	\subsection{Ramchand 2005 on two types of negation in Bengali}
\begin{itemize}
	\item Treatment of apparent asymmetry in negation: two negative markers, one incompatible with the perfect
	\item Nicely spells out assumptions for tense and asp
	\item R treats both as quantifiers (\textbf{no})
	\item \textit{na} quantifies over variables of times
	\item \textit{ni} quantifies over variables of events
\end{itemize}
\part{Notes from the Aspect literature}
\begin{itemize}
\item `Aspectual viewpoints function like the lens of a camera, making objects visible to the receiver'
\item \textsc{xnow} due to Bennett \& Partee (1972) McCoard (78), \textbf{Dowty (79)}, Richards \& Heny (1982), Iatridou, Angistorou \& Izvorski 2001...
\begin{itemize}
	\item XNOW has mostly seen use in the literature on the English (present) perfect. Cited in Klein (1992:532) as characteristic of ``current relevance''-based approaches to the \textsc{perf}, Dowty (1979:341) says that xnow covers:
	\begin{quote}
		the view that the perfect serves to locate an event within a period of time that began in the past and extends up to the present moment
	\end{quote}
\end{itemize}
\item Kamp \& Rohrer 2013, Partee i.a. on discourse progressing qualities of \textsc{pfv} preds.
\item \textbf{Klein (92) on the English Present Perfect}
\begin{itemize}
	\item Interpretation of \textsc{pres perf} utterance requires assigning a value to:
	\begin{description}
		\item[\textsc{tu}] (utterance time)
		\item[\textsc{fin}-time] (the time denoted by the finite part of the utterance, \textit{viz.} the tense inflected \textsc{perf} auxiliary)
		\item[\textsc{inf-}time] the time denoted by the uninflected event description, viz. the time of the eventuality.
	\end{description}
\item \textsc{fin-}time is the homologue of a Reichenbachian $ t_r $ -- here (p535), K introduces a (replacment) notion of $ t_{top} $
\item \textit{The door was open}: $ t_{top} $ is the time span that \textbf{verifies} the utterance. Note that $ \tau(\varepsilon) $ (K's $ t_{sit} $) is a superinterval of this --- \textit{viz. }the time frame at which the door \textsc{be} open.
\begin{itemize}
	\item Contrast against \textit{the door was wooden} -- indiv. level predication: $ \tau(\varepsilon) \sqsupseteq t_{utt}$. Past tense is invoked because a $ t_{top} $ is retrieved which is said to preced $ t_{utt} $
	\item English past (\textit{sc.} $ t_{top}\prec t_{utt} $) is taken to be neither \textbf{b}oundary- nor \textbf{p}osition-definite \textbf{whereas}
	\item English present \textbf{is p}osition-definite: it \textbf{must} include $ t* $ even if different \textsc{present} $ t_{top}s $ are sub/superintervals of one another. (536)
\end{itemize}
\item \textsc{\textbf{Aspect}}, then is concerned with the relation between $ t_{top} $ and $ \tau(\varepsilon) $:

\textsc{perf}$=t_{top}\succ\tau(e)$\\
\textsc{pfv}$= t_{top}\sqsupset(\textsc{end}\tau(e)) $\\
\textsc{ipfv}$= t_{top}\sqsubset\tau(e) $
\end{itemize}
\end{itemize}
\section{Krifka (89, 92) on analogies bw nominal ref \& temp constitution}
\begin{itemize}
\item $ \tau:\mathcal E\to\mathcal T $ originates from Krifka (1992: 33).

He notes that $ \tau $ is a homomorphism to his join operation $ \sqcup $ (i.e. $ \forall e,e'[\tau(e)\sqcup\tau(e')=\tau(e\sqcup e')] $).

This paper crucially notes the formal similarity between $D_e:D_i:D_\varepsilon$. 
\begin{itemize}
\item The main way that this fact is shown is:


\begin{tabular}{l|ll}
	R & $ \mathcal O $ & $ \mathcal E $\\\midrule
	\textsc{qua}ntized predicates & count-denoting NPs & telic (accomplishment-denoting) VPs\\
	\textsc{cum}ulative predicates & mass-denoting NPs & atelic (activity-denoting) VPs\\
\end{tabular}
\item Provides a formalism (p 35) that finds the terminal point $ TP $ of an event variable $ e $. Pred is telic iff $ P(e) $ entails a terminus --- $ STP(P) $
\item \textit{He read the letter} -- if atelic requires a (coerced??) \textsc{partitive} (47) or \textsc{iterative} (40) reading 
\item partitivity (at least in finnish) and progressivity are understood as two sides of the same coin (in the nominal and eventive domains respectively)
\end{itemize}
\item ref type of noun can affect temp contitution of vb and v.v.
\end{itemize}

\section{Perfectivity as definiteness (Ramchand 2008)}

\begin{itemize}
\item telicity/inner aspect ($\approx$ lex.) \textit{vs.} perfectivity/boundedness/outer aspect
\item \textit{contra} standar approach to slavic aspect, prefixes are \textit{not taken} to correspond to telicity (per Filip 1993 et seq)
\item `lexical' vs `superlexical' prefixes
\item \textbf{lexical: }small clause/event-structure decompositional analysis of prefixes, cf. phrasal verbs `B throw out the dog' = \textit{throw(dog)(B) $ \wedge $ out(dog)}
\item \textbf{superlexical:} do not introduce additinoal pred structure, add information about $ e $ (adverbial/measure/modificatory)
\end{itemize}
\section{`Relative tense' \textit{vs.} (viewpoint) aspect (Bohnemeyer 2007)}
\begin{itemize}
	\item Klein 1994 suggests (on the basis of English) that this is the same phenomenon (monosemy. \textit{contra} traditionalist (e.g. Reichenbach) disemous analyses)
	\item Minimal pair:\textbf{\textit{ B had arrived at 6.}}
\begin{itemize}
	\item  I arrived at 6 sharp and he was already half done with his meal (so must have arrived a decent amount earlier.) \textbf{\textsc{perf} in past} (viewpoint aspect: \textit{result state holds} at $ t_{top} $)
	\item (And) he had left again at 7. The inspector didn't get there til 8. \textbf{\textsc{pst} in past} (relative tense: \textit{result state doesnt hold at $ t_{top} $}))
\end{itemize}
\item See composition (ambiguity in the adverbial) on pg. 932. two functions: \textit{\textsc{clocktime}$ _{e} $} v \textit{\textsc{clocktime}$ _{t} $} which map events and intervals to $ \mathbb N\times\mathbb N $ (hh:mm) are responsible for the ambiguity.
\begin{itemize}
	\item Fails to predict infelicity of event time adverbials in present perfect (\textit{present perfect puzzle})
	\item Attempts to escape this by stipulating a ``positional definiteness constraint'' (\textit{p-def}) of the present tense.
\end{itemize}
\item `...viewpoint aspect in terms of the selection of a particular part of the event under description...such that the utterance concerns ... specifically this part' (923, `time-relational' (K 1995) or `frame selection' (C\&T 1985) approaches to viewpoint aspect, departs from Smith 91)
\item B claims that Reichenbachian $ t_r $ includes notions of $ t_{top} $ and `\textsc{perspective times'} (following Kamp \& Reyle 1993:593ff and Cover \& Tonhauser 2014).
\begin{itemize}
	\item `perspective times' potential relata of $ t_{top} $ in lieu of $ t_{utt} $ (hybrid abs-rel) or generalising over $ t_{top} $ \& anaphorically determined $ t_r $s (pure relatives - Jap/Kituba)
\end{itemize}
\item \textsc{perf} shouldn't combine with event-time adverbials the (\textit{Present Perfect Puzzle})
\begin{itemize}
	\item 
\end{itemize}
\item following B \& Swift 2004: \textsc{pfv} `is the default interp of asp unmarked dynamic verb forms	in English... Grice's second maxim of Quant.' (936)
\begin{itemize}
	\item B\&S distinguish between \textsc{telicity-dependent} aspectual ref
	\item telic-dependent (telic ipfv receive marking, atelic pfv receive marking)
	\item contrasts w/ stativity-sensitivity (predicts only dynamic ipfv receive overt marking) (277)
\end{itemize}
\end{itemize}

\section{Treatments of the \textsc{aorist}}
\subsection{\textit{Opérations énonciatives} (Culioli 1980)}

\begin{itemize}
	\item circ. modal component in negated georgian aorists (use of perfect for nonmodal negated pasts?) (184)
\end{itemize}
\subsection{Armenian (Donabédian 2016)}


\part{Treatments of (\& data from) Yolŋu Matha \& surrounds}
	\section{Wilkinson 1991 \textit{Djambarrpuyŋu}}
	{\large \color{red}{ Major source for description \& data in prospectus (TMA desc not currently rehearsed in this document)
	
	\normalsize
What follows is not so much theoretical lit. review but a rehearsal of Mel's description of subordination/complementation in \texttt{djr} }}

	\pex\begingl\glpreamble\textbf{SCl (nonfinite?), apparently with A-S coreference, embedded under speakingVb
}//
	\gla ga ŋunhi napurr ŋanya waŋa-ny birrka'yu-n-dja dhä-dhirr'yu-n-dja \textbf{[}balanyara-w-nydja ḻurrku'-ḻurrkun-gu-ny ŋorra-nhara-w\textbf{]}//
	\glb and \textsc{texd} 2p 3s\textsc{-acc} speak-I-\textsc{prom} think/test-1-\textsc{prom} mouth-stir-I-\textsc{prom} \textbf{[}such\textsc{-dat-prom} few\textsc{\textasciitilde red-dat-prom} lie-IV\textsc{-dat}\textbf{]}//
	\glft`and when we$_i$ requested him that a few (of us$_i$) sleep together'\trailingcitation{650}//\endgl\xe

\subsection{Complementation (Wilkinson \texttt{ms.})}
\begin{itemize}
	\item ``No complement clauses'' (in the Dixon 2006 sense)
	\item (Intrasentential/noncoordinative?) clause-linking by three means: \textbf{1} nominalization, \textbf{2} adjoining (with \textit{ŋunhi} or $\varnothing$), \textbf{3} ``a serial verb strategy''

\subsubsection{Nominalization}
\item  \textit{Intrinsically} devoid of TMA information 

\item ``Nominalisation'' happens by taking the \textbf{IV} form of a given verb and attaching an appropriate case suffix (\textsc{dat}, \textsc{assoc} etc... tabulated on \texttt{ms.}:33). Suffixes have different functions on the semantics of the nominalised complement (\textsc{dat} for purposives, \textsc{erg} for causal/instrumental/temporal....)
\begin{itemize}
	\item Also ``thematic affix'' \textit{-ra} (Lowe's ``long form'' of \textbf{IV}) before `single phoneme allomorphs of a case suffix' (for Gupapuyŋu this would have to be modified, maybe single mora allomorphs ought to get us the same result generalised across both varieties.)\mcom{\textbf{Thought bubble:} are these really nouns? Do we know this? And if so what is it about the \textbf{IV} marker that could have permitted a situation where it is multifunctional between \textsc{pst.pot} type semantics and a nominalising function?}
\end{itemize}

\item Minor clause: $[[verb]\alpha]_{nom}$ `and a single argument' (which occurs with the same peripheral-case suffix as the predicate.)
\begin{itemize}
	\item Coreference can be established between any (?) argument of the main predicate and (nominalized) complement predicate, It's likely that this is done via the pragmatics? See (\nextx)
	\pex\begingl\gla djamarrkuḻi' ga galkun mäḻu-w gondha-nhara-w//
	\glb children \textsc{ipfv} wait.\textbf{I} \textsc{fa-dat} collect-\textbf{IV}\textsc{-dat}//
	\glft`The children are waiting for their dad to fetch them'//\endgl\xe
\end{itemize}
\item ``Exception'' to this strategy is the use of locative-type \textsc{all} or \textsc{orig} marking on the complement (e.g. \textit{nhäma} selects for an \textsc{all} complement in these contexts)
\begin{itemize}
	\item This could cause structural (attachment) ambiguity in prinicple
	\item Though the main clause predicate (being some sort of propositional attitude or what have you) probably has a syntax/semantics that makes this infeasible!
	\item Does this suggest that resolving the locative semantics is more important than resolving GRs/thematic roles for interpretation??\mcom{W notes that \textit{mäwa'yun} `dream' cannot take a nominal complement, requires full sentential complement. To consider poss semantic motivations for this...}
\end{itemize} 



\subsubsection{Apposed finite clauses}
	\item `In \texttt{djr} \textit{ŋunhi} clauses are used to provide both a temporal or conditional grame for the main clause as well as elaboration about a particular argument in the main clause' (11, Wilkinson refers to other literature on Yolŋu/ALs that describe similarly polyfunctional complementisers.)
	\item \textit{ŋunhi} is a deictic (but one not accompanied by demonstration probably?) coding entities, intervals or ideas retrieved from utterance context/\textsl{cg}.
		\item `Most relevant strategy for complements to Dixon's \textbf{Primary-B} verbs' (2006: 10) -- that set of verbs that can take sentential complements. Semantically verbs of \textsc{attention, thinking, liking, speaking}
		\begin{itemize}
			\item Dixon (2006:11ff) distinguishes these from \mcom{these descriptions of dixon's categories are all my own paraphrases/interpretations to bring his typology in line with contemporary semantic thinking}\textbf{Primary-A} (just predicates taking nominal arguments) and \textbf{Secondary concepts} (which are dependent grammatical operators like \textbf{`A'} negation, modality and aspectual markers and \textbf{`B'} (same-subj) semimodals like `want/wish/hope (to/for)' and \textbf{`C'} (diff-subj) valence increasing operators like `make, cause, force...')
		\end{itemize}
		\item `most...identified by subord. lexemes found clause initially...any examples...no such marking' (1991: 655)
		\item formally similar to matrix clauses (cf. Hale 1976 `adjoined' clauses, not embedded)

		\item[\textbf{subordinators.}] \textit{ŋunhi} `\textsc{texd}' or interr/indef pronouns \textit{nhä, nhaku...}
		\item[\textbf{particles.}] \textit{bili/ḻinygu/ḻiŋgu} $\approx$`because'
		\item \textit{märr (ga)} `so that'
		\item \textit{yurr} `but, furthermore'
		\item \textit{ŋuli} (\textit{conditional protasis})
		\item[$\boldsymbol\emptyset.$] SCl apposition without any lexical subordinator (ms.:10)
		

		\pex\a\begingl\gla yaka ŋarra marŋgi wäŋa [\textbf{wanhal} ŋayi ga nhina yuwalk wäŋa-ŋur]//
		\glb \textsc{neg} 1s know place where.\textsc{loc} 3s \textsc{ipfv} sit.I true place\textsc{-loc}//
		\glft`I don't know the place where they really live'\hfill(1991:660)//\endgl
		
		\a\begingl\gla ~ ŋarra dhu dhä-birrka'yun ŋanya [\textbf{nhätha} \textbf{ŋunhi} dhu rom dhawar'yu-n]//
		\glb\rightcomment{(embedded Q?)} 1s \textsc{fut} ask-I 3s\textsc{-acc} when \textsc{texd} \textsc{fut} law finish-I//
		\glft`I'll ask him when the ceremony will finish'//\endgl
		
		\a\begingl\glpreamble \textit{Zero-complementizer}//
		\gla waŋi ŋanya nhe-ny ŋayi dhu räli marrtji]//
		\glb tell.II 3s\textsc{-acc} 2s-\textsc{prom} 3s \textsc{fut} \textsc{twds} go//
		\glft`You tell them that they're to come here'\hfill(1991:661)//\endgl\xe
		
		\item \textbf{\textit{pp662ff} for more CCls incl prop atts.}: ``it is not known to what degree other semitransitive predicates permit a finite clause complement but the following two examples show they are possible (exx. 931-2)\mcom{Does this have a relative clause inside the CCl? Bears further analysis perhaps.}
		\pex\a\begingl\gla ŋarrapi-ny gan ŋunhi märr-yuwalkthi-n [nyäl'yurr-a ŋayi gan ŋunhi dhäwu-ny lakara-ŋal]//
		\glb 1s.\textsc{emp-prom} \textsc{ipfv\textbf{.III}} \textsc{texd} believe-\textbf{III} [lie.\textbf{III}-\textsc{seq} 3s \textsc{ipfv.III} \textsc{texd} story\textsc{-prom} tell-III]//
		\glft`I believed that the story he told was untrue'//	\endgl
		
		\a\begingl\gla märr-ṉiṉ'thurr ŋarra nhanukal [ŋunhi mak ŋayi dhu rraku ŋunhi bäyŋu-n bäy-lakarama-ny]//
		\glb believe.\textbf{III} 1s 3s\textsc{.obl} [\textsc{texd} maybe 3s \textsc{fut} 1s\textsc{.dat} \textsc{texd} nothing\textsc{-seq} forgive.I\textsc{-prom}]//
		\glft`I believed of her that she would not forgive me'//				\endgl
		\mcom{For (c): Default past interpretation? Is there an implicature that he's changed his mind? (or the converse?)\\Note that there's no overt complementiser, the SCl seems to be the O of the predicate?}
		\a\begingl\gla ŋarraku wäwä-mirriŋu-y ga guyaŋa [bäyŋu-n gapu-ny guyiŋarr]//
		\glb 1s.\textsc{dat} brother-\textsc{kinprop-erg} \textsc{ipfv.I} think$_{tr_3}$-\textbf{I} [\textsc{negq-seq} \textsc{water-prom} cold]//
		\glft`My brother thought the water wasn't cold'\hfill(1991:663)//\endgl
		
		\a\begingl\glpreamble \textbf{interesting examples of sentential complementation apparently happening as a modifier to a nominal O}//
		\gla ŋarra-ny ga ŋunhi birrka'yurr [yanbi balaŋ ŋayi yaka-n ḏo'yu-na]//
		\glb 1s\textsc{-prom} \textsc{ipfv.I} \textsc{texd} think$_{tr_5}$.III [\textsc{cfact} \textsc{irr} 3s \textsc{neg-seq} arrive-\textbf{IV}] //
		\glft`I was thinking it mistakenly that they wouldn't come' (663)//\endgl
		
		\a\begingl\glpreamble nonspec ref \textit{ŋula} and \textsc{interr/indef}//
		\gla yaka ŋarra marŋgi [ŋula~nhaliy nhuna dhu marŋgi-kum]//
		\glb \textsc{neg} 1s know [\textsc{indef-erg} 2s\textsc{-acc} \textsc{fut} know-\textsc{trns}.I]//
		\glft`I don't know what makes you aware (of sth' (???))//\endgl\xe
		\subsubsection{Serialisation}
\item Share arguments, agree in inflection. No hard constraints on clause-internal contiguity/ordering etc.
\item Secondary predicates are probably not a well-defined category, but incl.
\begin{description}
	\item[Aspectual] \textit{marrtji} \textsc{ipfv?}, \textit{ŋurru'yirryun} `begin' ($<$ nose+motion/stance???), \textit{dhawar'yun} `finish' ($<$leg+\textsc{intrV})
	\item[Trying] \textit{birrka'yun} `try' ($<$`haphaz+\textsc{intrV}), \textit{baḏatjun} (miss, fail)
	\item[Else] \textit{bitjan, nhaltjan, mirithirr,}
\end{description}



		\subsection{Conditionals (667\textit{ff})}
		\item  \textit{ŋuli} introduces conditional protasis... unclear whether there's any systematicity to alternation with \textit{ŋunhi} for this same function' 
		

		

	\end{itemize}
	\section{van den Wal 1992 \textit{Gupapuyŋu}}
		\begin{itemize}
			\item Milingimbi (prim consultant: Mätjarra Garrawurra)
			\item `uninflecting psychological state verbs' \textit{djäl, marŋi, dhuŋa} `want, know, not.know' (\textsc{abs-dat}) frame. \textit{guya} `think' inflects. \item \textit{yanapi} `mistakenly think' doesn't inflect but takes a clausal arg
			
			\pex\a\begingl\gla ŋayi ga-na buyu guya-ŋa-na//
			\glb 3s \textsc{cont-III} biting~snake think-III\textsc{-foc}//
			\glft`he thought they were biting snakes'//\endgl
			\a\begingl\glpreamble\textbf{Context:} the children were fighting//
			\gla yanapi [ga-na djamarrkuḻi bul'-yurruna napurru nhä-ŋala]//
			\glb think.mistakenly \textsc{ipfv-III} children play-\textsc{intr.III} 1p\textsc{.excl} see-III//
			\glft`We wrongly thought that we saw children playing'//\endgl\mcom{I guess one of (probably the matrix?) subject is elided here? Probably unclear whether \textit{gana} associates with the matrix or embedded pred.}
			\xe
			\item[\textbf{§3.3 the semantics of Gupapuyŋu verbal inflections}]
			\item``it will become clear that the actual form of the predicate does not depend so much on the grammaticalisation of tense as on the grammaticalisation of mood and temporal reference together; the presence or absence of certain partciles...modal particles, determines the degree of salience of the predication rather than the exact temporal reference.'' (101)
			\item `both III and IV can be used for the same TempRef...depends on the mood of the sentence, not on degrees of remoteness, nor on the absence of preence of lexicalised TempRef...' (102-3)
			\pex\begingl\gla balaŋu napurru ŋuli nhä-nha yolŋu'\textasciitilde yulŋu-nha nhäŋa'\textasciitilde nhäŋala napurru bäyŋu//
			\glb \textsc{mod} 1p.\textsc{excl} \textsc{hab} see-\textbf{IV} person\textasciitilde\textsc{red-acc} see\textsc{\textasciitilde red.\textbf{III}} 1p\textsc{.excl} \textsc{neg}//
			\glft`We would have seen (irr.) people there, normally, but we saw nobody' //\endgl\xe 
			\item Merlan on Mangarayi relationship between subordination, focus, mood and \textsc{hab}
			\subsection{Realis}
			\item[\textbf{I --- \textsc{nonpast}}]... except running commentary \& performatives rarely does $t_\varepsilon\circ t_u$ (cf. Comrie 85:7). Notions of XNOW are pretty encoded in \texttt{guf}: \textit{dhiyaŋu bala} (\textsc{prox.erg} then/away) = now/imm.fut. versus \textit{dhiyaŋu bili} (\textsc{prox.erg} \textsc{modal}) = imm.pst.
			\begin{itemize}
				\item Lowe talks about \textit{bili} as a \textsc{completive} aspectual marker (`already, finished; because'), vdW treats it as a marker of \textbf{strong illocutionary force}: in conjunction with realis mood indicates \textbf{high certainty}, in conjunction with irrealis mood indicates \textbf{high uncertainty}
				\item Consequently she understands that \textit{bili} `enforces the actual realisation of the proposition contained in \textit{dhiyaŋu}...looks back \& establishes an event as having in fact occurred at a time prior to and relatively close to the absolute now from which \textit{dhiyaŋu} `(from) here' departs' (106)
			\end{itemize}
		\item[\textbf{Primary inflection w/ future reference (\textit{dhu/yurru})}]
		\item \textit{dhu/yurru} (look to be synonymous) have a modal function (can occur w epistemic \& deontic readings, 125-6)
		\item vdW takes the fact that \textit{dhu} occurs with \textbf{I} as evidence that \textit{contra} Lyons etc. the future is not considered an \textsc{irr} category in \texttt{guf} (``a certain degree of certainty that it will becom factual in the future'' (109-10))
		\item \textit{Contra} Lowe, Wilkinson, vdW claims that `for negation w future reference the realis form of the verb is used...it is jsust as valid to predict the affirmation of a proposition with future temporal reference as it is to predict the negation of such a propoisiton'' (110)
		\pex\begingl\gla ga yaka-na ŋarra yurru bulu roŋiyirri//
		\glb and \textsc{neg-foc} 1s \textsc{mod} again return.\textbf{I}//
		\glft`And I will not come back again.'//\endgl\xe
	\item[\textbf{Past time reference \& III}] 
	\item vdW makes confusing claims like `no different forms diff between abs pst temp ref and rel pst time ref (s Comrie 85) nor is there a dist bw ref to action in the rec pst or ref to action in the rem pst.' (110)
	
	\textbf{I suppose crucially to her analysis, the past temporal reference that is associated with the primary form is part of the XNOW?} (although how does this really account for its (\textbf{I}) cooccurrence with \textit{barpuru} etc.)
	
	\pex\a\begingl\gla mokuy-nha walala dharpu-ŋala; ga walala bitjarra waŋa-na ḏilkurru-wurru:...//
	\glb spirit-\textsc{acc} 3p spear-\textbf{III} and 3p thusly.\textbf{III} talk.\textbf{III} elder\textsc{-perl}//
	\glft``They had speared a person; and the old people said:...//\endgl
	\a\begingl\gla bala ŋayi ga-na barrtjurruna yolŋu mala//
	\glb then 3s \textsc{ipfv-III} spear.\textbf{III} person \textsc{pl}//
	\glft`After that he speared many people'//\endgl\xe

	\subsection{Irrealis}
	\item `negation of an event occurring \textbf{in the present} is always expressed with the irrealis form of the verb.' (as opposed to the future, which occurs with \textsc{rea} and the past which can do either.) (112, note 12)
	\item In order to preict something uncertain about the future in Gup, the irr form of the vebr is used in conj w the pcl \textit{yurru}. this sense of unc may be further reinforced by use of a mod adjct like \textit{maku} `maybe'
	\pex\begingl\gla maku limurru yurru boturru, nhamunha limurru yolŋu bak-thurruna//
	\glb maybe 1p\textsc{.inc} \textsc{mod} count how.many 1p\textsc{.inc} yolŋu break\textsc{-intr.III}//
	\glft `maybe we should count how many of us, Yolŋu, have died' (113)//\endgl\xe
	\item coocc with \textit{balaŋu} `\textsc{Modal}' indicates ``strong irrealis sense''
	\item In questions, irrealis forms also indicate ``the expected answer is a negative one'' (113)
	\item Imperatives 
	\pex\begingl\gla wäy! gurtha ŋunha, nhawi, ḏutji män-ŋu, bak-maraŋu//
	\glb hey! fire(wood) \textsc{rel.deic} what's.it firesticks get-\textbf{II} break\textbf{-II}//
	\glft`Hey! get that firewood, what's it, those firesticks, and break them.'//\endgl\xe
 	vdW claims that the use of the \textsc{irr} for \textsc{imper} `is interesting from a cultural PoV as it indicates that when a Yolŋu issues a command, it is up to the hearer to decide whether he/she will fulfill that command: this is in line with the soc-ling rule which underlies the fact that in y cult it is gen up to the H whether he/she will answer a q (cf. Harris 1980:150-1) (114).
 	\pex\a\begingl\gla dhuwala-nydja bäru-miriw wäŋa; muŋuna rra dhäwu ŋä-nha ga-nha; walala ga-nha ḻakara-nha//
 	\glb\textsc{deic-foc} crocodile-\textsc{priv} place otherwise 1s story hear\textbf{-IV} \textsc{cont\textbf{-IV}} 3p cont-\textbf{IV} tell\textbf{-IV}//
 	\glft`there were no crocs in this place; otherwise i would have heard (irr, PiP) about it. They would have told me about it' (114)//\endgl
 	\a\begingl\gla ga yaka-na ŋayi ga-nha ŋuyulk-thinya dhuwala dhäwu bili ŋayi gana dhäwu ŋä-kula manymak-nha//
 	\glb and not\textsc{-foc} 3s \textsc{cont-IV} reject-\textsc{tr.IV} \textsc{deic} story because 3s \textsc{cont-III} story hear-\textbf{III} good\textsc{-foc}//
 	\glft`And he didn't reject that story, bc he heard it was a good story''//\endgl\xe
 		\end{itemize}
 	\subsection{notes on aspect}
 	\begin{itemize}
	\item\mcom{\textbf{McLellan} also reports that \textit{ŋarra} `go/come' in \textbf{Wangurri} `can be used as an aspectual verb auxiliary meaning to keep on, persist' (p. 194)} Distinctions encoded by auxiliaries: \textit{ga} and \textit{marrtji} (for motion predicates?) as \textsc{cont} marking. (115, w/ (modalised) example)
	\item \textit{ŋuli} as \textsc{hab} marking (`every time, on every occasion').
	\begin{itemize}
		\item ``existential quality'': attention is not so much n the temp parameters of the action indidcated by the pred which is modified by \textit{ŋuli} but more on certain chars of circs around the action' (cf. \textit{bitjan bili})
		\item ``\textsc{hab} always occurs with the realis form of the vb'' (118) (this is also surprising, presumably she means that a habitual meaning arises in I/III inflected predicates that are \textit{ŋuli-}modified? This has got to be missing a generalisation (that one that McLellan tries to get at.))
	\end{itemize}
	
	\textit{yäna bili} `until' and \textit{bitjan bili} (`all the time, uninterruptedly' (\textsc{dur? cont?}))
	\pex\begingl\gla waṉḏina napurru gana yäna~bili wäŋa-ŋura napurru buna-na//
	\glb run.III 1p.\textsc{excl} \textsc{cont.III} until home\textsc{-loc} 1p.\textsc{excl} arrive-III//
	\glft`We ran 'til we arrived home'//\endgl\xe
	\item Pluractionality (and plurality) apparently can be (probably symbolically) encoded by final vowel lengthening.
 	\end{itemize}	
	\subsection{notes on modality}
\begin{itemize}
 	\item ``...the possible egrees of meaning expressed bt rhe interaction of the modal \textit{balaŋu}  realis and irrealis vb forms'' (121)
 	\item\mcom{It's probably worth verifying this with some basic modal description} modal adjuncts (Halliday 85:82) like \textit{balaŋu, warray, ŋula} also fn as tense \textit{dhu, yurru} and asp \textit{bili} (121; vdW claims on 123 that only this latter subset can function deontically)
 	\item ``in gupapuyŋu, epistemic and deontic modality are the only types of modality that are gram/lexd in the language. thus other types of modality like `facultative modality', expressing capacity and volition meanings (as in the Eng \textit{He can swim} (see Goosens 1985:204)), are not directly expressed in Gup.''
 	\item degree of force depends on interaction w verb...strong: \textit{dhu, yurru, bili, balaŋu} weak: \textit{maku, ŋula, warray}
 	\item \textit{\textbf{balaŋu}} may occur w any of the four verb forms (\textit{contra} Wilkinson?) -- the degree of ``factivity'' is conditioned by the mood of the verb fm. In (b) below, e.g. the combo of \textit{balaŋu} and \textbf{III} seems to create this counterfactual meaning by `convey[s] the notion that they certainly expected the croc to snap at them (which, considering the nature of crocs, would be a likely prediction)
 	\pex\a\begingl\gla ga balaŋu dhu maṉḏa buna, bäpa ga ŋäṉḏi dhiyaku,...//
 	\glb and \textsc{mod} \textsc{mod} 3d return\textbf{.I} father and mother \textsc{deic.dat}//
 	\glft``and when this one's fa and mo arrive...''(123-4)//\endgl
 	\a\begingl\gla dhirr'-thirr-yurruna; balaŋu ŋayi ŋuli dhan'-thurruna maṉḏaŋgu bäyŋu!//
 	\glb poke\textsc{-redup-intr.III} \textsc{irr} 3s \textsc{hab} snap.at\textsc{-intr.III} 3d.\textsc{dat} nothing//
 	\glft`They poked \& poked [the crocodile]; it might have snapped at them; but it didn't!'(124)//\endgl\xe
 	\item With irrealis seems to decrease the modal force??
 	\pex\a\begingl\gla nhe-nydja balaŋu gi bäna liya-ŋorri yanapi napurru gi ŋayathulu gaṯpurr yolŋu//
 	\glb 2s-\textsc{foc} \textsc{mod} \textsc{ipfv.\textbf{II}} even~though head-lie\textbf{.II} think.wrongly 1p.\textsc{excl} \textsc{ipfv.\textbf{II}} wounded man//
 	\glft`Even though you might mistakenly be thinking that we would be keeping a wounded man there'//\endgl
 	\a\begingl\gla ga bäy:nu balaŋu napurru ŋuli wäŋa-lili-nydja ŋäthili marrtji-nya//
 	\glb and nothing \textsc{mod} 1p.\textsc{excl} \textsc{hab} home\textsc{-all-foc} first go.\textbf{IV}//
 	\glft`We might have gone straight to our homes first, but we didn't'\hfill(125)//\endgl\xe
 	\item \textit{dhu} predominantly indicates reference to future w verbs in \textbf{I} (125). (Can also behave as a strong deontic modal apparently). vdW claims (126) that \textit{yurru} does the same but also appears frequently with \textbf{either I or II}.
 	\pex\a\begingl\gla yaka-na dhu limurru roŋiyirri Yirrkala-lili//
 	\glb \textsc{neg-foc} \textsc{mod} 1p\textsc{.inc} return\textbf{.I} Yirrkala\textsc{-all}//
 	\glft`We won't return to Yirrkala'//\endgl
 	\a\begingl\gla ga yaka-dhi walala dhu ga yatjun-dhi//
 	\glb and \textsc{neg-ana} 3p \textsc{mod} \textsc{cont.\textbf{I}} bad\textbf{.I}//
 	\glft`And they must not be disobedient'//\endgl\xe 
 	\item p.127: \textit{bili}...indicating a strong degree of probability/necessity. She differs from Lowe here who describes it as a completive (perfective) marker as in contexts like (\nextx b) below. Additionally it appears function as some sort of strengthener when cooccurring with \textbf{III/IV} verb forms (both in epistemic (c) and root(d) contexts).
 	\mcom{is all the flavour work done by \textit{bili} in (\lastx d)?}
 	\pex\a\begingl\gla yaka, ŋunha-na bili mari-nydja//
	 	\glb \textsc{neg} \textsc{rel.deic-foc} \textsc{mod} fight\textsc{-foc}//
	 	\glft`No that \textit{must} be a fight'//\endgl
 	\a\begingl\gla --- mänŋu djorra! --- Bili.//
	 	\glb --- get\textbf{.II} book --- \textsc{mod}//
	 	\glft --- ``Get [the] book!'' --- I already have (got it).//\endgl
	 \a\begingl\gla	ŋayi-pi-dhi bili Dhä-gapaṉ ŋorra-nha-na//
	 \glb 3s-\textsc{emp-ana} \textsc{mod} \textsc{name} sleep\textsc{-IV-foc}//
	 \glft`Dhä-gapaṉ probably went to sleep'//\endgl
	 \a\begingl \gla ga dhiyala bili ŋarra dhu dharpuma-nydja//
	 \glb and \textsc{deic.loc} \textsc{mod} 1s \textsc{mod} spear\textbf{.I}\textsc{-foc}//
	 \glft`and here I must spear him'//\endgl\xe
	
	 \item \textit{maku, ŋula, warray} all seem to indicate possibility (and as with\textit{dhu}, sometimes multiple will occur \textit{\textbf{ŋula maku }nhä maṉda bathana})
	 
	 \item There are also items like \textit{yuwalk} `yes' and \textit{ŋani} `\textsc{tag}' that occur which vdW analyses as having some modal import (although these seem more similar on first blush to the UC items in German with some sort of hedging-type perlocutionary force)
	 
	 \subsection{Other notes}
	 \item \textit{waŋa} and \textit{guya} always seem to embed free indirect speech/style (207).\mcom{x-ref Wilkinson? Slash test this claim?}
	 \item Basic most SOV with common stylistic fronting.

	\end{itemize}
	
\section{McLellan 1992 \textit{Wangurri}}
	\begin{itemize}
		\item[\textbf{On Djambarrpuyŋu, Djapu, Gupapuyŋu}] Forms $1_{NEU},2_{PFV},3_{HAB.PFV},4_{IRR},5_{IMPER}$ correspond to I, III,IV,II, annoyingly. Tabulated differences between djapu,dhuwal,dhuwala. Citing Lowe L41, \texttt{guf}'s cooccurrence of \textbf{III} and \textsc{neg} is `said to be Dhaŋu'mi influence'.
		\item `I suggest that the verb systems of \texttt{guf, djr \& djp} are actually an intersection of modal \& aspectual qualities' (86)
		\item `The \textsc{hab} covers that which is perceived to be true for both \textsc{rea} and \textsc{irr}.'
		\item Negative as an irrealis category (87)
		\item `the negative of the pfv' is expressed using this form of the vb [\textbf{IV}], expressing ``what might have been but wasn't'' (88)
		\item[\textbf{On Gälpu (Dhaŋu'mi)}] 
		\item Paradigm for \textit{ŋarru-} `go'
			\begin{tabular}{ll}
				\textit{ŋarruŋa} & \textsc{pres}, indef \textsc{fut}\\
				\textit{ŋarruŋan} & indef \textsc{pst}\\
				\textit{ŋarruŋay} & \textsc{neg.pres}, def \textsc{fut}\\
				\textit{ŋarriya} & \textsc{imper}\\
				\textit{ŋarruŋarra} & \textsc{pst.hab, pst.neg}, sometimes \textsc{pres.hab?}\\
				\textit{ŋarrunhara} & completive, nominalisation\\
				\textit{ŋarruŋany} & \textsc{dist.pst}\\
			\end{tabular}
			\item[\textbf{Wangurri}]
			\item  \textit{yaka} is given as the \textsc{cont}inuative auxiliary (!)
			\item Paradigm described as with 7 inflections (\textsc{\textbf{neu,irr,pfv,hab,imp},nom,refl}) (see Excel.) Some nonbasic... She claims (161) that there are \textbf{five forms} (referring to those bolded).
			\item \textsc{Temp}oral ``case'' (as identical to \textsc{erg/instr})
			\pex\begingl\gla ga waripu-yu-m g?muk-thu ŋanapu ŋarra...//
			and other-\textsc{\textbf{temp}-d} night-\textsc{\textbf{temp}} 1p.\textsc{excl} go.\textbf{I}//
			\glft`And another night, we went...' (146)//\endgl\xe
			\item \textbf{Four modal particles}: \textit{Ø} `\textsc{rea}', \textit{bayiŋ} \textsc{`hab'}, \textit{ŋarru} \textsc{`irr'} and \textit{warri} \textsc{`counterf'} (restricted to \textbf{Form 3})
			\item \textit{yawungu/barpuru} `recent past' (only \textbf{Form 1-Ø})
			\item \textit{barkthu} `near future' (only \textbf{Form 4})
			\item `The concern of Wangurri is not to locate a process in time but in reality...lack of distinction in the verbs between past and present. However, there is a distinction between that and what English would call \textit{future}' (153)
			\item \textit{goḏarr} `morning, next few days, soon-not-today'
			\item \textit{bayiŋ} `\textsc{hab:} this is the sort of thing that happens', with \textbf{Form 3} `this is the sort of thing likely to happen' (the \textsc{hab} as an ``intermediate modality'' 156, ``a characteristic situation which holds for both realis and irrealis'' 160))\\
			\textit{bayiŋ gayŋa} `\textsc{hab cont}' generates habitual aspect readings, otherwise McLellan (dubiously) describes the effect of \textit{bayiŋ} as modal:
			\pex\a\begingl\gla nhalpiyan nhunu bayiŋ dämba-m warkthun//
			\glb how 2s \textsc{hab} damper-\textsc{d} make\textsc{\textbf{-neu}}//
			\glft`How do you make damper?'//\endgl
			\a\begingl\gla nyamnyam' banha nhan bayiŋ gayŋa gapu-ŋa buwalun ya!//
			\glb fruit~sp. that 3s \textsc{hab} \textsc{cont} water\textsc{-loc} float\textsc{-neu} \textsc{excl}//
			\glft`\textit{Nyamnyamʔ} floats on water' (159)//\endgl 
			\item \textit{ŋarru} as a general modalising particle `indicates a yet unrealised situation' (160), used with apparent future meanings and with deontic readings (compatible with gayŋa). All examples given in \textbf{\textsc{neu}}
			\xe
		
		\end{itemize}
	\subsection{Descriptions of the 5 inflections}
	\begin{itemize}
	\item[\textbf{1 -- N\textsc{eu}tral}] most common, occurs with any auxiliary ex \textit{warri} \textsc{`counterf'}. Can fn as imperative. 
	\item[\textbf{2 --  \textsc{P}erfective}] Resists all mood particles. ``Completeness of a process ... includes a concept of boundedness'' (164). Analysis entails that past readings emerge as ``a secondary implication'' 
		\item `(realis implies that the event has happened or is happening... not available to an event which is happening' ``we could see there to the other side and we came along that (road) because they had taken him along the road'' (only `taken' here received \textbf{P} inflection.)
		\item Can occur with imperfective marking `drawing attention to the continuity or durativity of the activity which is now a completed unit'
		\pex\begingl\gla buku-nyena-n nhanguḻ gayŋa-n yolŋu-m warra manikay-ŋa-m, ne?//
		\glb head-sit-\textbf{P} 3s\textsc{.all} \textsc{cont\textbf{-P}} people\textsc{-d} \textsc{pl} song-\textsc{loc-d} \textsc{tag}//
		\glft`The people had been gathering to him in the songs, you see?' (165)//\endgl\xe
		\item \textbf{P} \textit{is compatible with negative marking in Wangurri.}
	\item[\textbf{3 -- Habitual-perfective}] only used in conjunction with \textit{bayiŋ} `\textsc{hab'} or \textit{warri} \textsc{`counterf'} -- functions as with habitual but introduces notion of perfectivity: `habitual activities which have now ended' (166). As with \texttt{djr} \textit{ŋuli balaŋ} these can cooccur). (McL claims that \textit{balaŋ} `\textsc{counterf}' ``always occurs with [\textit{ŋuli}]'' 167). Maybe this is only true \textbf{when \textit{balaŋ} occurs in a \textsc{counterf} context.}
		\item often referring to ``olden days''\mcom{\textbf{note that McL claims that \textit{bayiŋ} is homonymous with \textit{banha}+\textsc{erg/instr/temp}} (inflected \textsc{dem)}. This leads her to analyse \textit{warra} as doing the \textsc{counterf} work in the collocation. Cf. the counterfactuality of \textit{ŋuli} in \texttt{djr}? Also indicative conditionals occur w \textit{banha ŋarru} (168)}
		\item Syncretism leads to ambiguity between translations like \textit{we used to collect grass} and \textit{we probably would have gone to collect grass}. (167)
		\item \textbf{Indicative conditionals} do not take the \textbf{H} form, they receive \textbf{N\textsc{eu} inflection and occur as \textit{banha ŋarru} } 
	\item[\textbf{4 -- Irrealis}] mainly occurs with \textit{ŋarru}.
	\item Compared to uses of \textbf{NEU+\textit{ŋarru}} ($\approx$`should P'), `adds modal qualities of ``certainty'' or ``prediction''' ($\approx$`must/will P' ; 169)
	\item ``sometimes \textit{barkthu} is the verb particle, replacing \textit{ŋarru} in the clause'' \textbf{(what motivates this analysis?? Is there any semantic difference?)}
		\pex\a\begingl\gla nhunu barkthu gayŋiyi wukirrim dhäruk-ma ŋalaminy?//
		\glb 2s soon \textsc{cont-\textbf{ir}} write.\textbf{d} story.\textbf{d} 1p\textsc{.incl.acc}//
		\glft`Will you soon be writing our words?'//\endgl
		\a\begingl\glpreamble\textbf{Context:} A mother's suggestion of why her her son in his late 20s had a heart attack. She hypothesises that his habits of eating hurriedly and running back from football practice is to blame. (Some sort of circumstantial (weak) necessity?)//
		\gla ŋatha-wu reṯi nhäpa-n nyina'nyinayi-n guwaman ŋarray ŋatha-n//
		\glb food\textsc{-dat} ready filler sit\textasciitilde\textsc{redup\textbf{.ir}-d} eat\textsc{\textbf{-neu}} go\textsc{-\textbf{ir}} food.\textsc{d}//
		\glft`Ready for food, (you) must keep sitting around to eat food (170)'//\endgl\xe
		\item  McL claims that the \textit{ŋarray} can give rise to obligational (i.e. root nec) readings. She claims that there's evidence that this has just be elided (along w the subject pronoun) in the above datum. \textbf{(Dubious?, see also \textbf{C}6§2)}
		\item \textbf{Irrealis} form with \textit{ŋarru/barkthu/bayiŋ} gives `prediction, certainty, obligation' readings (i.e. strong modal force)
		\item[\textbf{5 - I\textsc{mp}erative}] (not present in \texttt{guf,djr},djapu, but present in Gumatj Dhuwala)
			\item Used in demands/commands rather than procedural instructions
			\item Incompat with \textsc{ipfv} marking.
	\subsection{Tense particles}
		\item \mcom{This contrasts perhaps to \textit{babalamirri} in \texttt{guf}?}\textit{barpuru/yawungu} `recent past'. ``does not relate to a specific day, but does to a specific event in time...`recently' [is] an unsatisfactory translation.''
		\pex\begingl\glpreamble\textbf{Context:} A speaker is talking about what she was doing while waiting for news of a medical evacuation that was introduced to the discourse in previous utterance.//
		\gla ga dhaya'thaya ŋaya barpuru batjiwarr-murru//
		\glb and stand\textsc{.redup.I} 1s recent road\textsc{-perl}//
		\glft`And I stood around yesterday/on.Wednesday(/$\boldsymbol{^\#}$sometime.recently) on the road'//\endgl\xe
	\subsection{Modal particles}
		\item 91\% of modal particles imm. follow Subj, 5\% imm. precede (subjunctive: McL seems to claim that this syntactic process is necessary for encoding a counterfactual antecedent??, p.184) \textbf{Basic word order \texttt{S$^\frown$Fin$^\frown$Asp}}
		\begin{itemize}
			\item\textcolor{violet}{Note that, implicit in this, is that the \texttt{Finiteness} category is associated for McL with \textbf{modal particles} \{\textit{Ø, ŋarru, bayiŋ...}\} rather than the verbal inflection.\\
			This is probably an empirical question that may be worth investigating}
		\end{itemize}
		\item There's a confusing discussion of how `subject [arguments] the verb particles form a unit to produce Mood in a clause' are  responsible for modalisation (doesn't seem to necessarily correlate to some judge function per se....) 185\textit{ff}
		\item Adopting Chung \& Timberlake's terminology: `[Wangurri encodes ]\textbf{Source through the Finite, and the Target through the Subject}' (191)
			\mcom{It remains to be seen how these formally/functionally are different categories: are there similar distributional restrictions between inflectional classes? Are they in parallel distribution? (certainly \textit{wilak ḻinygu} \textsc{`prob+term'} in collocation is given)}
		\item[\textbf{Modal adjuncts}] ``[relates] specifically to the mening of the finite verbal operator'' (196, citing Halliday 1985:82)
			\item \textit{wilak} `maybe', \textit{bitjan linygu} `always', \textit{yäna} `just/still/yet/only', \textit{ḻinygu/bili} `complete, already' (see vdW's treatment of \textit{bili}!)
			
			\item As well as this ``Mood element'', following what I guess is a systemic functional approach, the remainder of a clause is called \textit{the residue}. This is at least near-to-synonymous with the more general term `predicate.'
			
		\subsubsection{Conditionals}
		\item McL refers to these as \textbf{subjunctives §6.3.5}
			\pex\a\begingl\glpreamble\textbf{Indic condl}//
			\gla Banha ŋarru Galikaliyum dju'yun raki'murru, ŋayam ŋarru dhawurum ga rakaraman yolŋuny ŋangawulyinyara//
		\glb That Irr subsection-ERG-D send-NEU wire-PER lSgNOM-D Irr this\textsc{-abl} and tell\textsc{-neu-d} person\textsc{-acc} \textsc{neg}-become\textsc{-nmlzr}//
		\glft`Should Galikali call, I will tell from here (about) the people dying'//\endgl
		
		\a\begingl\glpreamble\textbf{Subjunctive condl}//
		\gla Banha warri nhän warraṯthuwarra wäyin guwattharanharami nhän warri liyuwarran bäwarraṉ//
		\glb that \textsc{counterf} 3s get-\textbf{H} meat kill\textsc{-nom}-having 3s \textsc{counterf} kill\textbf{-H.d} wallaby.//
		\glft`If he'd taken the gun, he would've killed a wallaby' (213)//\endgl
		\xe
	\item Although she does talk about anotehr ``subjunctive''-like construction where the counterfactual construction exists in a possibly matrix clause and indicates a wished/hoped.for eventuality. (214)
	


	\end{itemize}
\section{Waters 1986 \textit{Djinaŋ/Djinba}}
	\begin{itemize}{\color{gray}
		\item Wuḻaki as coverterm for Yirritjing varieties
		\item Lacks (lamino)dental series (lost from Proto-Yolŋu: shared or independent inovations?)
		\ex \vtop{\halign{%
				#\hfil&& \qquad #\hfil\cr
				\textsc{erg}&-dhi/-ri/-li\cr
				\textsc{acc}& \textit{-nyi}\cr
				\textsc{dat}&\textit{-Gi}\cr
				\textsc{orig}&\textit{-Bi}\cr
				\textsc{gen}&\textit{-\textsc{obl}-angi}\cr
				\textsc{loc}&\textit{-mirri, -ngi}\cr
				\textsc{all}&\textit{-li}\cr
				\textsc{abl}&\textit{-ngiri}\cr
				\textsc{perl}&\textit{-mirrpmi} (\textit{-pani} in djinba)\cr
			}}\xe
		\item most Yolŋu languages use \textsc{dat} to mark possession, Djinaŋ uses \textsc{gen}, Djinba uses \textsc{orig}. Waters claims that \textsc{gen} is the result of reanalysis of longer \textsc{dat} forms minus the \textit{-gu} suffix. \textsc{gen} therefore assumed part of the ``functional load'' of the erstwhile \textsc{dat} (31).}
	\item Three major conjugation classes \& twelve temporal `functions' over six inflections. (see Excel doc for more)
		\begin{itemize}
			\item[\textbf{Djinaŋ}] 
			\item\textbf{ YestPast, \textsc{Pres}ent continuous}
			
			Cognate with \textsc{unm} elsewhere in Yolŋu (185). In SCls tends to predicate characteristics of subject the men [who possess-\textsc{pres} dinghies] (186). Habitual and continuing properties of referent (instances when story is set in remote past, here this inflection is used to describe ipfv/cont properties.) See story 19 for examples of switches from RPa to YPA to discuss ipfv eventualities (described as a yolŋuwide prop).
			\item \textsc{Fut}ure
			\item \textsc{Imp}erative, \textsc{Pr}esentContIrrealis, YestPast,Irrealis
			\item TodayPst, RemotePst
			\item TodayPstCont, RemotePstCont
			\item TodayPastIrrealis, RemotePastIrrealis
			\item[\textbf{Djinba}] also has a `\textsc{pot}ential' inflection
			\item But lacks the synthetic/fusional \textsc{cont} categories
		\end{itemize}
	\item `Unlike other Yolŋu languages whch mark TAM ith a mix of vbl inflections and advbl p'cles, Djinang and Djinba use inflections almost exclusively' (166)
	\item `the system of TMA opps in Djinang has historically undergone significant reshaping...some categories that are appropriate for Dhuwal(a), Djapu, Rith are not well suited to a discussion of Djinang vb mphlgy
	\item `Djinang split \textit{*-nha(ra)} and \textit{*-na(ra)} protoforms into 2 distinct categories on analogy w Maningrida lanugages' (e.g. Rembarrnga \textsc{pst.cont} and \textsc{pst.punct}) -- this is generally done by way of auxiliary \textit{ga} in Dhuwal(a)
	\item ...turns out Ritharrŋu developed unique \textsc{fut} allomorphs in the same way as Djinang. (Cognate with Heath's \textsc{pst.potl}, cognate w Djinaŋ \textsc{imp, pri, ypi}) (176)
	\item[\textbf{Semantics of verbal inflection (§4.3, p177ff})] 
	
	
		\subsection{Notes on Djinaŋ}

		
			\item Waters' \textsc{Today Past} (semantically equivalent with III) receives a present state reading (cf \texttt{djr} etc...) He uses this to motivate the `unmarkedness' of his [$-$\textsc{cont}] feature (180).
			
			\pex\a\begingl\gla nyani galŋ-walŋi-ni//
			\glb 3s body-play-\textsc{TPa}//
			\glft`He's happy'//\endgl
			\a\begingl\gla ŋarri ŋal-but-tji-li//
			\glb 1s guts-loose-\textsc{themsr-TPa}//
			\glft`I'm hungry'//\endgl\xe
			
			
			
			\item ``With past time, \textsc{irr} forms may be used to express doubt or uncertainty as to whether an event obtained or not...hypotheticality...[\textsc{neg}] to assert the non-obtaining of an event.'
			\item In future contexts, this doubt is provided by context (as in it's kind of inherent to the future form)
			
			
			\item `no obligatory [subordinator]' (207). Sometimes \textit{miḏi} `like' is used at a clause boundary, or occasionally interrogative pronouns or \textit{ŋunu} `that' (=ŋunhi)
			
			\pex\begingl\gla djin rar-ki kiri, djin djaJtjibi kiri, yakirr inydji djingiri-ngili-ban//
\glb 3plERG knead-FUT PROG-FUT 3plERG lift PROG-FUT [sleepUNM]NOM RECIP complete-RPA-TF//
\glft`They would knead it, (then) they would lift it (from water) (when) a sleep had been finished (i.e. on the next day).'//\endgl\xe
			
			\item Hard to know exactly what's happening here (with that \textsc{fut/hab}), but the \textsc{rpa} (i.e. remote past/\textbf{III}) may be doing SoT kind of stuff?
			
			\item \textit{be} cognate in Djinang (=bala and other functions) described on p75
			\item p184: ``In subordinate clauses, the subordinate verb may take future, present or past inflections. However, the least marked of these is the FUT inflection. This inflection obtains whenever temporal or aspectual nuances do not need to be signalled within the subordinate clause; examples occur in (25), (291) and (312). Purposive subordinate clauses regularly take FUT inflection, though the event is not necessarily set in future time...''
			
			\pex\begingl\gla nginibi djining ingki djal nibi maḻ'rngirr-dji//
\glb 1plexcNOM [thisUNM]DAT NEG desire [lplexcNOM hear-FUT]DAT//
\glft We do not like this (which) we hear. (32:59)//\endgl\xe
			
		\subsection{Notes on Djinba}	
			
	
	\end{itemize}

\begin{framed}
	\textsc{hypothesis.} The \textsc{two} frames ((non-)today) emerge from the distinct between \textsc{two} discourse modes: ((non-)narrative). \textit{See} Waters on Djinaŋ precontemporary inflections, pg. 188. This also meshes with Albert's observation about the remote \textbf{III} coming up in `story' contexts.
\end{framed}
	\section{Heath 1980 \textit{Ritharrŋu} \texttt{[rit]}}
	\begin{itemize}
		\item p55 \textit{\textdblhyphen dhi} ``generally indicates that the referent or region is contextually definite. Demonstrative constructions with \textit{dhi} therefore normally \textit{refer} to an entity or region which has already been mentioned (or otherwise understood), rather than \textit{indicating} a new region or entity''
	\end{itemize}
	\section{Kabisch-Lindenlaub 2017 \textit{Golpa}}
	\begin{itemize}
		\item \textsc{irr} (and \textsc{pst.hab}) forms all but forgotten (164)
		\item KL decides that ``the modal realis-irrealis distinction is \textsc{not} grammatically expressed in [Golpa]'' (176)
		\item \textsc{neu} expresses present (optional \textit{ma} `\textsc{cont'}) and irrealis/future (with \textit{wurruku} `will, would')
		\item argues for a \textsc{±pst} distinction encoded in with the PST, PSThab categories
	\end{itemize}
		\section{Non-Yolŋu}
		\subsection{Green 1995 \textit{Gurr-goni}}
		\begin{itemize}
\item Glasgow oppositions table recapitulated on p 184, v also 308
\item suggests today/nontoday boundary is `after nightfall the previous night until the moment before speaking'
\item `I went last night' obligatorily takes \textsc{pre}, `I'm going home' takes \textsc{con}
\item p186, talking about past habits with \textsc{precon}, arrival of balanda in Maningrida takes \textsc{con} (relative recency)
\item No metricality in the future
\item Evidence of a (fading) metricality in the \textsc{recent past} (i.e. the pre-today \textsc{contemporary}) for some predicates (esp in ``Conj. 5'')
\item negation triggers irrealis (\textsc{pre:irr1::con:irr2})
\begin{itemize}
\item future distinction is neutralised in irrealis (marked as irr 2)
\item \textbf{irr1} conditionals (examples given: both antec- and conseq clauses)
\item \textbf{irr1} past conditionals
\item \textbf{irr2} conditional mirroring the \textbf{irr1} type is given in (4-36) (`otherwise' \textit{(djaluwu}) type)
\end{itemize}
\item Future tense in realis status is used to state the speaker's intentions and to make predictions about the behaviour of others; in using the future realis, speakers indicate that they consider the realisation of an event highly probable: it is only a matter of time before it happens. Nonprecontemporary irrealis, with future reference, on the other hand, is used where an event is merely possible.... (199)
\item Future realis is also often used in place of the imperative when telling people what to do
\item irrealis predicates (prop attitudes) embed realis predicates (non-subjunctive like behaviour)

\item \textit{njan} `if' can take future marked antec/conseqs (where negated still irr, see p 302)
\item irrealis clause can take realis relative? (5.238)
\item mistaken belief with \textit{mundjarra} (314)
\item \textit{Wulek, mangarraka, wurpu } prohibitives also trigger \textit{irr2}
\item 5.268 : thought-pre mundjarra saw-pre ghost there-pre

		\end{itemize}
		\subsection{\textit{Nunggubuyu}}
		\begin{itemize}
			\item \textbf{Hughes \& Healey 1971} Full set of portmanteau prefixes, defectively mark two sets of TM combinations
			\begin{tabular}{ccccc}
								& \textsc{pst} & \textsc{pres} & \textsc{Imper} &\textsc{Fut}\\
					\textsc{Pos} & A & A & B/A & B \\
					\textsc{Neg} & B & B/A & A & A \\
						\end{tabular}
			\item In addition to TM suffixes making a 6-way distinction for Pos/Neg \& Pst/Pres/Fut
		\subsection{\textit{Burarra}}										
\item[\textbf{K Glasgow 1968a `Frames of Reference'}] Owing to observations from Les Hiatt, Glasgow considers there to be two tenses: ``contemporary'' and ``remote'' that can occur inside two different \textbf{frames of reference} (\textit{today} vs. \textit{nontoday}.) (118.)

\pex \textsc{con} for ongoing props while \textsc{precon} past narr.
\a\begingl\gla abirri-ny\textdblhyphen yerranga \nogloss{[} marnnga jiny-bunggiya-Ø jiny-yorkiya-Ø \nogloss{]} abirri-ny-bamu-na//
\glb 3ua-f\textdblhyphen other sun 3mjin-fall-\textsc{con} 3mjin-always-\textsc{con} 3ua-fem-go.along-\textsc{precon}//
\glft`The other two went [to where] the sun sets.'\trailingcitation{Green 1987:87}//\endgl\xe

	\subsection{Eather 1990 \textit{Na-kara}}
	\item[Tense in the VC (pp164-)] Adopts a grid v similar to Glasgow 1968a (cited as 1964): \textit{(pre-)contemporary $\times$ (before) today}. Latter category is demarcated `by the event of sunrise' Replacement of \textit{remote} to cover ``earlier today.'' (166)
	\item[\textbf{irr}] Irrealis `non-occurrent status of an action or event...[:] indicative VCs with fut tns meaning, as well as all neg and sjctv clauses'
	\item Future virtually always unmarked (ex in like 14 vbs.). Precontemp is the marked choice, contemporary is marked only in closed classes (about most transitive verbs.)
	\item neg prefix slot preceding PP, irrealis immediately follwing PP, tense slot following vb stem (190, ex below 197)
	\pex\a\begingl\gla barrddjornanga keyarda//
	\glb 3uaf.go~back.\textsc{cont} home//
	\glft`They went back home'//\endgl
	\a\begingl\gla barriddjorna keyarda//
	\glb 3uaf-irr1-go.back-\textsc{fut} home~(\textit{recte})//
	\glft`The will go back home'//\endgl
	\a\begingl\gla korla kabarrddjorna keyarda//
	\glb \textsc{neg} \textsc{irr2/neg}-3uaf-go.back-\textsc{cont,fut} home//
	\glft`The didn't/won't go backk home'//\endgl\xe
	\item ʼ\textit{-(ndji)ya} is the \textsc{recip/refl} suffix.
	
	\item \textbf{existentials}: NCl-Q Noun (dem)
	\pex\a\begingl\gla kinkalangkaya kin-korrawa nardawa balbbala korla//
	\glb sandfly 3\textsc{m}f-many because wind \textsc{neg}//
	\glft`There's lots of sandflies (here) because there's no wind.//\endgl
	\a\begingl\gla kinmakkarra kin-korrawa ngarra korla nawonga//
	\glb shellfish~sp. 3\textsc{m}f-many and(new)  \textsc{neg} cockle//
	\glft`There's lots of sandflies (here) because there's no wind.//\endgl
	\xe

		\end{itemize}
	
	
	\section{Austin 1998 on tense marking in Australia}
	
	\begin{quote}
		
		
		
		In all Australian languages ‘point time’ words then have interval reference rather than strict point or punctual specification.
		
		[...]
		
		2. now versus past — all languages have a shifter whose core reference is an interval that includes the moment of speaking in contrast to situations that held in the past[...] Languages may also have a contrast of now versus future, but the term for future tends to have relatively immediate reference.
		
		\citep[147]{Austin1998}
	\end{quote}
\part{Anthropological \& ethnographic insights} \renewcommand\labelitemiii{\tiny$\square$}


	\paragraph{Keen 1995} `People believe that when a person dies, the spirit returns to waters in the person's country'
	\begin{itemize}
		\item `...thus the bäpurru relation of person to palce is thru the anc creation of gps, the ac origin of a pers's	being, the embodiment of ancs in country, and the reenactment of anc events in ceremonies in which ppl identify themselves as ancs.'
		\item\footnotesize The tropological structure of a belief like "such and such a hill is the fat of Kangaroo wangarr ancestor" is complex. The belief appears to rest, first, on an
analogical equivalence of the rocky substance in the hill to the fat of kangaroo; second, on the imaginative construction through narrative of a being with some characteristics of kangaroos, some of humans, and some peculiar to wangarr (such as the ability to transform the landscape on a large scale); and third, on a narrative describing or implying the transformation of Kangaroo wangarr into the hill. \textbf{Depending upon all the preceding stages, Yolngu employ ancestral narratives and their sung equivalents as rhetorical devices to comment allegorically on past or current relations and events}.' (511)\normalsize\end{itemize}
\paragraph{Mursharbash \textit{Time \& Boredom}}\begin{itemize}
\item ``..following [Ronald] Leach (1968) and examining the tension between two basic experiences \textbf{(1)} that certain phenomena of nature repeat themselves and \textbf{(2)} that life change is irreversible.'' (312-3)
	\end{itemize}

	\section{Hale's  \textit{World View \& Semantic Categories}}
	\begin{itemize}
		\item Distinguishing two notions of \textit{world view}: \textbf{1} philosophy/`postulates of how things are in the world' \textbf{2} Linguistic: effectively \textit{signifiant/signifié} relata.
		\item Warlpiri logics of • \textbf{\textit{eternity}} and \textbf{\textit{• complementarity}}.
		\begin{itemize}
			\item[\textbf{eternity}] cyclical perpetuity as opposed to `linear logic' 
			\item `persistence of entities through transformation' (``unity of actual and potential'' cf. O'Grady 1960: equations of firewood/fire and animal/meat)
			\item Untranslatability of \textit{make} in Warlpiri. \textit{ŋurrju-ma-ni} good+\textsc{caus+inch} ($\approx$`fix,repair') • \textit{yirra-rni} \textit{`put,place'$\approx$`endow'} • \textit{pakarni, pantirni, jarntirni} `chop/gouge/trim' -- verbs of modification/perfection
			\item Kinship recycling
				\item[\textbf{`complementarity'}] 
		\end{itemize}
	\item \textcolor{violet}{The article primarily focuses on an opposition which Hale claims is likely to be a universal and which is `observed with particular clarity and purity in the grammar of Warlpiri' and which `consitutes part of the mental structures which enable human beings to acquire the semantic systems of their native languages' (252).}
	\item locative cases (paradigmatic opposition of \textsc{loc/perl} and \textsc{all/el})
	\item directional enclitics (verbal derivation) (\textsc{dur/perl} vs. \textsc{centrip/centrif}$\approx$\textit{`hither, thither'})
	\item \textbf{complementizers}: central (\textit{kaji\textasciitilde kuja}) vs. noncentral (\textit{yungu})...
	
	Hale claims that \textit{yungu} (his `noncentral complementizer') occurs in `situations where the [eventuality] depicted in the dependent clause \textbf{precedes or follows} that of the main clause'
	\item `infinitival complementisers' (embedded predicate controlled by clausal Subj. or Obj. or either/causal or other versus \textsc{purp} and \textsc{serial}($\approx$coord)). As above the noncentral ones appear to encode a \textbf{sequentiality} relation between the predicated eventualities.
	\item Hale's most ambitious claim by his own reckoning is the extension of this semantic organising principle (\textit{sc.} \textbf{(non)central} dichotomy) to the \textbf{aspectual domain}
	\begin{itemize}
		\item noncentral $\emptyset$ (\textsc{`perf'}) vs. central \textit{-ka, -lpa} \textsc{`pres',`imperf'}
		\item In irrealis contexts Warlpiri's ±\textsc{pst} distinction is flattened.
		\item In irrealis contexts tense is interpreted as \textbf{nonpast} iff aspect is \textbf{central}\\
		Tense is interpreted as \textbf{past} iff aspect is \textbf{noncentral} (viz. perfective?)
	\end{itemize}
	\item different types of secondary predication (``depictive'' vs ``translative'' in Simpson's parlance). Translative case marker \textit{-karda} associates w noncentrality (otherwise secondary predicate appears in the same case as its controlling arg (e.g. \textsc{abs}))
	\end{itemize}
\begin{framed}
Yengoyan's 1980 \textit{AA} review of compendia/essays of Aboriginal mythology.
	\begin{itemize}
		\item  philosophies of life generated through cultural systems. Ontology implies that $\Phi$ of life is the basic/irreducible element of social life (on Hiatt \textit{ed. }1975: 841)
		\item ``myth and ritual are expressions of an irreducible ontological axiom'' (apud Stanner)
	\end{itemize}\end{framed}

\section{Malotki \textit{Hopi time}}

\begin{itemize}
	\item Taken to `conclusively [disprove] Whorf's contention that ``the hopi language contains no reference to `time', either explicit or implicit'' (carroll 56:58, p 629)
	\item not only each of man's individual actions, but indeed every spoken utterance is inextricably tied to a temporal situation' (629)
\end{itemize}

\part{TMA dissertations}
\section{Badiaranke (Cover 2010)}
\begin{itemize}
	\item `\textsc{pfv}' for past AND present events
	\item `\textsc{ipfv}' for ongoing, habitual AND future/epistemically-probable eventualities AND consequent clauses.
	\item \textsc{discontinuous past} marking (perspective time backshifted). Two formatives associated with \textit{realis} and \textit{irrealis}
	\item ``aspect is inextricably with modality'' (sic, p. 1) 
\end{itemize}

\end{document}