

%	\begin{abstract}
%
%	\end{abstract}
\chapter{\textit{bambai} as an apprehensional}\label{bambai.desc}



	`Apprehensional' markers are a nuanced, cross-linguistically attested grammatical category, reported to encode epistemic possibility in addition to information about speakers' attitudes with respect to the (un)desirability of some eventuality. Taking the meaning of Australian Kriol particle \textit{bambai} as an empirical testing ground, this paper provides a first semantic treatment of apprehensionality, informed by a diachronic observation (due to \citealp{Angelo2016}) in which apprehensional readings emerge from erstwhile temporal frame adverbials that encode a relation of temporal {\sc subsequentiality} between a discourse context and the eventuality described by the prejacent predicate.

To illustrate the issue, consider the contributions of \textit{bambai} in the Kriol sentence pair in (\ref{minpair}):

\pex\label{minpair}\textbf{Context:} \textup{I've invited a friend around to join for dinner. They reply:}
	\a\label{minpair.ssq}\begingl\glpreamble\textsc{Subsequential} reading of \textit{bambai}//
		\gla yuwai! \textbf{bambai} ai gaman jeya!//
		\glb yes! \textbf{\textit{bambai}} 1s come there//
		\glft ‘Yeah! I’ll be right there!’//\endgl
		
	\a\begingl\glpreamble \label{minpair.appr}\textsc{Apprehensional} reading of \textit{bambai}//
		\gla najing, im rait! \textbf{bambai} ai gaan binijim main wek!//
		\glb no 3s okay \textbf{\textit{bambai}} 1s {\sc neg.mod} finish 1s work//
		\glft ‘No, that’s okay! (If I did,) I mightn’t (be able to) finish my work!'\trailingcitation{[GT~20170316]}//\endgl
\xe


While the reading of \textit{bambai} in (\ref{minpair.ssq}) roughly translates to `soon, in a minute', this reading is infelicitous in (\ref{minpair.appr}), where \textit{bambai} is a discourse anaphor which contributes a shade of apprehensional meaning (\textit{i.e.}, indicates that the Speaker's hypothetically joining for dinner may have the undesirable possible outcome of him not finishing his work.) 

\section{Background}\label{bambai.intro}

Having entered into their lexicons predominantly via the contact pidgin established in colonial New South Wales (NSW) in the late eighteenth century \citep{Troy1994}, cognates of the English archaism \textit{by-and-by} are found across the English-lexified contact languages of the South Pacific. 


\pex[aboveglftskip=0ex] \textit{baimbai}, translated as `soon, eventually, (in the) \textsc{future}' in \citet{Troy1994}\deftagex{Troy-pidg}\footnote{\citet{Troy1994} collates a corpus of n texts, predominantly from settler journals. (\getref{Troy-pidg}a,c) are taken from \citet{Dawson1831} (Port Stephens) and (b) is taken from James Dredge's diary (Melbourne, 1839). Page numbers given in the example index Troy's (re)publication in the appendices to (and/or orthographically standardised in the body of) her doctoral thesis.}
\a\begingl\gla stopabit massa \textbf{baimbai} mi paiala dat agen aibliv//
%\glb
\glft`Wait, master, soon I'll speak to them again, I think.'\trailingcitation{(252, 571)}//\endgl
\a\begingl\gla \textbf{Baimbai} Potfilip blakfela Waworong blakfela kwambi ded olgon//
%\glb \textsl{\textbf{bambai}} \textsc{place} aboriginal Waworrong aboriginal quiet dead gone//
\glft`Soon Port Phillip ($ \approx $ Melbourne) Aboriginal people, the Waworrong, will be ``asleep'': dead and completely gone.'\trailingcitation{(697)}//\endgl
\a\begingl\gla Wool~Bill been choot him kangaroo; \textbf{by~and~bye} roast him//
\glft`Old Bill shot a kangaroo, then cooked it.'\trailingcitation{(575)}//\endgl
\xe
\marginnote{cite source for Troy \& promote footnotes?}

Additionally, \citet{Clark1979} describes \textit{by-and-by} as a particularly broadly diffused feature of the \textit{South Seas Jargon} that served as a predominantly English-lexified auxiliary means of communication between mariners of diverse ethnolinguistic backgrounds and South-Pacific islanders (21, cited in \citealt[262\textit{ff}]{Harris1986a} a.o.). The cognates across these contact languages have preserved the function of \textit{by-and-by} as encoding some relationship of temporal subsequentiality between multiple eventualities.\footnote{\textit{baimbai} (sic) is described as a `future tense marker' by \citet[112,418,711]{Troy1994} and \citet[268]{Harris1986a}. Indeed it appears to be a general marker of futurity in the textual recordings of NSW pidgin that these authors collate, although still retains a clear syntactic function as a frame adverbial. Their description of \textit{bambai} (along with \textit{sun, dairekli, etc}) as a tense marker is possibly due to the apparent lack of stable tense marking in the pidgins, although is likely used pretheoretically to refer to an operator that is associated with future temporal reference. This is discussed further in §~\ref{dataStfa} below.}$^,$\footnote{\citet[10-11]{Clark1979} lists cognates of \textit{bambai} (transcribed as \textit{baymbay} for Roper Kriol) in the contact languages of New Guinea, Solomon Islands, Vanuatu, Cape York, Norfolk Island and Hawai`i. According to \citet{Romaine1995}, in Tok Pisin \textit{baimbai} grammaticalised into a general future tense marker. On the basis of a corpus oof Pacific Jargon English, she also hypothesises emergent irrealis-type readings in admonitory contexts. (this claim is discussed further in Ch. \ref{bambai.prag}.) See also \citealt{Angelo2016} for further review of cognates of \textit{bambai} across other Pacific contact varieties.} Clark takes this shared feature (along with other cognates) to be a retention, evincing a shared history between these varieties (see also fn \ref{PN footnote} below.) 

As shown in \ref{minpair}, Australian Kriol (hereafter Kriol \textit{simpliciter}) has retained this function: below, in (\getref{ssq0}), \textit{bambai} serves to encode a temporal relation between the two clauses: the lunch-making event occurs at some point in the (near) future of the speaker's father's trip to the shop: \textit{bambai }might well be translated as `then' or `soon after'.
\pex\deftagex{ssq0}\begingl
\gla main dedi imin go la det shop ailibala \textbf{bambai} imin kambek bla gugum dina bla melabat//
\glb my father 3s\textdblhyphen{}{\sc pst} go {\sc loc} the shop morning \textit{\textbf{bambai}} 3s\textdblhyphen{}{\sc pst} come.back {\sc purp} cook dinner {\sc purp} 1p{\sc.excl}//
\glft`My dad went to the shop this morning, \textbf{then} he came back to make lunch for us.' \hspace*{\fill}[AJ~23022017]//
\endgl\xe
In addition to the familiar `subsequential' use provided in (\getref{ssq0}), \textit{bambai} appears to have an additional, ostensibly distinct function as shown in (\getref{app0}) below.\footnote{
	Note though that \citeauthor{Clark1979} also observes that the Pitkern cognate appears to have developed \textsc{lest/in~case}-type readings (\textit{i.e.}, an \gls{appr} reading) as in (\getref{clark-PN}′). Pitkern -- the variety spoken by \textit{Bounty} mutineers -- is generally described as an outlier among other Pacific contact varieties (\textit{i.e.}, not a descendant of the South Seas Jargon, see \citealp[48]{Clark1979}); this is likely to be an entirely independent innovation.\label{PN footnote}

\pex[aboveglftskip=0ex,exno=\getref{app0}′]\begingl\glpreamble Apprehensional-like cognate in Pitkern-Norfolk\trailingcitation{\citep[15]{Clark1979}}//
\gla kʌm dʌʊn \textbf{bɛmbɛǝ} ju  fɔl\deftagex{clark-PN}//
\glft`Come down, lest you fall.'//\endgl
\xe

}
\pex[nopreamble]\deftagex{app0}
\begingl
\glpreamble\textbf{Context:}  It's noon and I have six hours of work after this phonecall. I tell my colleague://
\gla ai\textdblhyphen{}rra dringgi kofi \textbf{bambai} mi gurrumuk la desk iya gin//
\glb 1s\textdblhyphen{\sc irr} drink coffee \textit{\textbf{bambai}} 1s fall.asleep {\sc loc} desk here {\sc emph}//
\glft `I'd better have a coffee \textbf{otherwise} I might pass out right here on the desk.'\hfill[GT~28052016]//
\endgl
\xe
In (\getref{app0}), the speaker asserts that if he doesn't consume coffee then he may subsequently fall asleep at his workplace. In view of this available reading, \citet{Angelo2016} describe an `apprehensive' use for Kriol \textit{bambai} --- a category that is encoded as a verbal inflection in many Australian languages and is taken to mark an `undesirable possibility' (256). In this case, \textit{bambai} is plainly not translatable as an adverbial of the `soon'-type shown in (\getref{ssq0}). Rather, it fulfills the function of a discourse anaphor like `otherwise', `or else' or `lest' \citep[see also][]{Webber2001,PhilKotek}.

This chapter proposes a diachronically-informed and unified semantics for Australian Kriol \textit{bambai}, concerned especially with the apparent emergence of \textsc{apprehensional} readings in this (erstwhile) temporal frame adverbial. 
The current chapter reviews and motivates the grammatical category of `apprehensional epistemics' as described in typological literatures (\S~\ref{typS}). Section \ref{dataS} describes the function and distribution of Kriol \textit{bambai}, both in its capacity as a subsequential temporal frame adverbial (§\ref{dataStfa}) and its apparent apprehensional functions (§\ref{dataSapp}). 

In the data we have seen so far, \textit{bambai} appears to connect two propositions. In Chapter \ref{bambai.prag}, we conside,r how \textit{bambai} is interpreted in view of the relationship between these two propositions: specifically how the prejacent of \textit{bambai} is \textbf{modally subordinate} to material accommodated in a discourse context (\S\ref{bambai.subord}). In view of these facts, we develop an account of the diachronic emergence of apprehensionality and the status of the expressive component of these items' meaning.

Finally, Chapter \ref{bambai.semx} comprises a proposal for a unified semantics for \textit{bambai}.% and discusses the grammaticalisation of apprehensional meaning while section \ref{conclS} concludes.



%T Beginning with a brief overview of ``apprehensionality'' as a linguistic category (§3.2), it: describes the distribution of these two readings (synchronically, when do apprehensional readings ``emerge'' in context, (§ \ref{dataS}), considers how apprehensionality emerges out of so-called ``subsequentiality'' markers diachronically (§ \ref{diaS}), and proposes a unified meaning component for the two readings (§ \ref{semS}).


\section{Apprehensionality cross-linguistically}\label{typS}
While descriptive literatures have described the appearance of morphology that encodes ``apprehensional'' meaning, very little work has approached the question of their semantics from a comparative perspective. Particles that encode negative speaker attitude with respect to some possible eventuality are attested widely across Australian, as well as Austronesian and Amazonian languages \citep[258]{Angelo2016}. While descriptive grammars of these languages amply make use of these and similar categories,\footnote{The terms \textsc{Timitive} and particularly \textsc{evitative}, a.o. are also used in these descriptive literatures.} \citet{Lichtenberk1995}, \citet{Angelo2016,Angelo2018} and \cite{Vuillermet2018} represent the few attempts to describe these markers as a grammatical category).\footnote{An edited collection on  \textit{Apprehensional constructions}, edited by Marine Vuillermet, Eva Schultze-Berndt and Martina Faller, is forthcoming via Language Sciences Press. The papers collected in that volume similarly seek to address this gap in the literature.}

\subsection{Apprehensionality as a semantic domain}\label{typ.appr}

In the first piece of published work dedicated to the properties of apprehensional marking (``apprehensional-epistemic modality''), \citet{Lichtenberk1995} claims that the To'abaita ({\tt[mlu]} Solomonic: Malaita) particle \textit{ada} has a number of functions, though generally speaking, serves to modalise (``epistemically downtone'') its prejacent while dually expressing a warning or otherwise some negative attitude about its prejacent. The symbol $ \blacklozenge $ is used throughout to signify these two `\textsc{apprehensional}' properties. Shown here in (\getref{mlu}), Lichtenberk distinguishes: (\getref{mlu.bare}) \textbf{apprehensive-epistemic} function, (\getref{mlu.fear}) a \textbf{fear} function and (\getref{mlu.aver}-\getref{mlu.link}) \textbf{precautioning} functions.

	\pex \textbf{Apprehensional marking in To'abaita: four uses of \textit{ada} `\gls{appr}'}\deftagex{mlu}
	\a\deftaglabel{bare}%% Begins a new example
	\begingl %% Begins a gloss
	%% IMPORTANT: use forward slashes WITHIN the gloss environment!
	\glpreamble \textbf{\textit{Apprehensive modal $\qquad \blacklozenge p $}}\\\ \textbf{\textsc{Context}.} Dinner's cooking in the clay oven; opening the oven is a labourious process.//
	\gla \textbf{ada} bii na'i ka a'i si `ako ba-na // 
	\glb \glem{appr} oven\_food this it:{\sc seq} {\sc neg} it{\sc:neg} be.cooked {\sc lim-}its//
	\glft `The food in the oven may not be done yet.'\hfill(295)//%\hfill(To'abaita {\tt[mlu]}: Solomonic, Lichtenberk: 295)//
	\endgl %% Ends a gloss
	\a\deftaglabel{fear}\begingl
	\glpreamble \textbf{\textit{Embedding under predicate of fearing $ \qquad \textsc{fear}(\blacklozenge p) $}}//
	\gla nau ku ma'u `asia~na'a \textbf{ada} to'an na'i ki keka lae mai keka thaungi kulu//
	\glb 1s \gls{fact} be.afraid very \textbf{\gls{appr}} people this \gls{pl} they:\gls{seq} go hither they:\gls{seq} kill 1p.\gls{incl}//
	\glft`I'm scared the people may have come to kill us.'\trailingcitation{(297)}//
	\endgl
	\a\deftaglabel{aver}\begingl\glpreamble\textbf{\textit{Precautioning} (``\textsc{avertive}'' reading)}$ \qquad \neg p\to\blacklozenge q $//
	\gla riki-a \textbf{ada} `oko dekwe-a kwade'e kuki `ena//
	\glb see-it \gls{appr} 2s:\gls{seq} break-it empty pot that//
	\glft`Look out; \textbf{otherwise} you may break the empty pot.'\trailingcitation{(305)}//\endgl
	\a\deftaglabel{link}\begingl\glpreamble \textbf{\textit{Precautioning} (``in-case'' reading)}$ \qquad \neg p\to\blacklozenge(\mathfrak{r}(q)) $//
	\gla kulu ngali-a kaufa \textbf{ada} dani ka `arungi kulu//
	\glb 1p{\sc.incl} take{\sc-pl} umbrella \glem{appr} rain it:{\sc seq} fall.on 1p{\sc.incl}//
	\glft `Let's take umbrellas \textbf{in case} we get caught in the rain'\hfill(298) //\endgl
	\xe


(\getfullref{mlu.bare}) functions as a possibility modal encoding negative speaker attitude vis-à-vis the eventuality described in its prejacent (i.e. opening the oven in vain). This reading also obtains under the scope of a predicate \textit{ma'u} `fear' in (\getfullref{mlu.fear}). Lichtenberk analyses this use of \textit{ada} as a complementizer, introducing a subordinate clause \citeyearpar[296]{Lichtenberk1995}. 

In each of (\getref{mlu.aver}-\getref{mlu.link}), meanwhile, \textit{ada} appears to link two clauses. In both cases it expresses negative speaker attitude with respect to its prejacent (the following clause), which is interpreted as a possible future eventuality, similarly to the English archaism \textit{lest}. On the \textit{avertive} reading $ p \textit{ ada } q$--- translated as `$ p $ otherwise/or else $ q $' --- a conditional-like interpretation obtains: if $ p $ doesn't obtain, then $ q $ may $ (\neg p \to\blacklozenge q) $. On ``in-case'' readings, while $ q $ is interpreted as a justification for the utterance of $ p $, there is no reasonably inferrable causal relation between the two clauses --- \citeauthor{Lichtenberk1995} is somewhat ambivalent about whether these whether these two readings constitute a single or multiple readings \citeyearpar[298-302]{Lichtenberk1995}. For \citet{AnderBois2020}, ``in-case'' uses involve some distinct ``contextually inferrable'' proposition $ r $ from which $ q $ follows $ (\mathfrak{r}(q)) $. Effectively, if $ p$  doesn't obtain, then some $ r $ (a consequence of $ q $) may. In (\getfullref{mlu.link}), the failure to take umbrellas $( \neg p) $ might result in getting wet $ (r) $ (should we get caught in the rain -- $ (q) $). They appeal to a number of pragmatic factors (reasoning about the plausibility of relations between $ p $ and $ q $) in adjudicating between these two readings. This treatment is discussed in some further detail below.

 Of particular interest for present purposes is the categorical co-occurrence of {\sc seq}-marking \textit{ka} in the prejacent to \textit{ada}. Lichtenberk notes that the sequential subject-tense portmanteau \textit{appears categorically in these predicates}, independent of their `temporal status.' He claims that this marking indicates that the encoded proposition `\textit{follows the situation in the preceding clause}' (296, emphasis my own). Relatedly, Vuillermet tentatively suggests that the Ese Ejja (\texttt{[ese]} Tanakan: SW Amazon) \textsc{avertive} marker (\textit{kwajejje}) may derive from a non-past-marked auxiliary with ``temporal subordinate'' marking \citeyearpar[281]{Vuillermet2018}. The analysis appraised in this chapter proposes a basic semantical link between the expression of the \textbf{temporal sequentiality} of a predicate and \textbf{apprehensional} semantics.



Subsequent typological work has concentrated on fine-tuning and subcategorising apprehensional markers. Notably, \citet{Vuillermet2018} identifies three distinct apprehensional items in Ese Ejja, which she refers to as realising an \textsc{apprehensive} \textit{(-chana)}, \textsc{avertive} \textit{(kwajejje)} and \textsc{timitive} \textit{(\textdblhyphen yajjajo)} function. These three apprehensionals respectively scope over: entire clauses (as a verbal inflection), subordinate clauses (as a specialised complementiser) and noun phrases (as a nominal enclitic). Similarly to Lichtenberk, Vuillermet suggests that these data provide evidence for a ``morphosemantic apprehensional domain'' (287).

Adopting this taxonomy, \cite{AnderBois2020} focus their attention on the ``adjunct'' uses of the A'ingae (\texttt{[con]} NW Amazon) apprehensional enclitic \textit{\textdblhyphen sa'ne}. That is, they model the contribution of \textit{\textdblhyphen sa'ne} in its functions as • a \textit{precautioning}/avertive marker, analysed as encliticising to (subordinate) clauses (\getfullref{con.aver}-b), compare To'abaita (\getfullref{mlu.aver}-d), in addition to • a \textsc{timitive} function, where the \textsc{appr} functions as a DP enclitic (\textit{e.g.}, \getref{con.tim}). Adapting treatments of the semantics of rationale/purposive clauses, they propose the core meaning given in (\getfullref{sa'ne}).


\pex Adjunct uses of apprehensional \textit{\textdblhyphen sa'ne} in A'ingae\deftagex{con}\trailingcitation{\citep{AnderBois2020}}
\a\begingl[glstyle=nlevel] 
\glpreamble \textsc{Avertive} use\deftaglabel{aver}\endpreamble
sema-’je\textdblhyphen{}ngi [work-\gls{ipfv}\textdblhyphen1]
dû’shû\textdblhyphen{}ndekhû [child\textdblhyphen \gls{pl}]
khiphue’sû\textbf{\textdblhyphen{}sa’ne }[starve\textdblhyphen{\textbf{\gls{appr}}}]
\glft‘I'm working \ul{lest} my children starve.'\trailingcitation{(381)}
\endgl
\a\begingl\glpreamble \textsc{in-case} use\deftaglabel{incase}//
\gla tsa’khû\textdblhyphen{}ma\textdblhyphen{}ngi guathian-’jen [ña yaya khuvi\textdblhyphen{}ma i\textbf{\textdblhyphen{}sa’ne}]//
\glb water\textdblhyphen{}\gls{acc}\textdblhyphen{}1 boil-\gls{ipfv} 1SG father tapir\textdblhyphen\gls{acc} bring\textdblhyphen\textbf{\gls{appr}}//
\glft‘I am boiling water \ul{in case} my father brings home a tapir.’\trailingcitation{(383)}//\endgl
\a\begingl\glpreamble \textsc{Timitive} use//
\gla anae'ma\textdblhyphen{ni}\textdblhyphen{ngi} phi [thesi\textbf{\textdblhyphen{sa'ne}}]//
\glb hammock\textdblhyphen{\gls{loc}}\textdblhyphen{1} sit jaguar\textdblhyphen{\textbf{\gls{appr}}}//
\glft`I'm in the hammock \ul{for fear of} the jaguar.'\trailingcitation{(374)}\deftaglabel{tim}//
\endgl
\xe

\pex \citeauthor{AnderBois2020}'s (2020:382) semantics for A'inge apprehensional adjunct uses of \textit{\textdblhyphen sa'ne} (on its avertive/\textit{lest}-like reading)\deftagex{sa'ne}


$ \llbracket \textit{\textdblhyphen sa'ne}\rrbracket=\lambda q.\lambda p.\lambda w:\exists i[\textsc{resp}(i,p)]	.p(w)\wedge \forall w'\in\textsc{goal}_{i,p}(w):\neg q(w')  $


Supposing that some entity $ i $ is the agent of $ p $, \textit{\textdblhyphen sa'ne} takes a proposition $ q $ as its input and outputs a propositional modifier, asserting that, in $ w $, both $ p $ holds and the (relevant) \textsc{goal} worlds of the agent $ i $ are those where $ q $ doesn't hold. 
\xe

\noindent For \citeauthor{AnderBois2020}, the semantics for this \textit{lest}-type usage can be extended to other precautioning (``in-case'') uses and timitive uses by appealing to an third, ``inferrable'' proposition $ r $. That is, on the \textsc{in-case} reading, all $ \textsc{goal}_{i,p}$-worlds are such that $\neg r(w') $ --- as they point out, on this analysis, \textsc{avertive} is a special case of the precautioning use where $ r\Leftrightarrow q $. On the \textsc{timitive} reading, \textit{\textdblhyphen sa'ne} takes an argument $ x\in\mathfrak D_e $ (instead of $ q\in \mathfrak D_{\langle s,t\rangle}$
%^{\mathfrak D_t} $
), now asserting that • $ x $ ``is involved in'' $ r(w') $ and that • $ \neg r(w') $.\footnote{\citet[15]{AnderBois2020} retain a lexical entry for \textit{\textdblhyphen sa'ne}$ _{\textsc{timitive}} $ distinct from the precautioning uses. They suggest that an alternative to avoid this polysemy would be to adopt a ``coercion'' style analysis or (less plausibly) an ellipsis one.
		
	A fourth possibility which they do not address would be to reanalyse the timitive DP as a (verbless) existential proposition (see Part \ref{NEC} of the current dissertation.) It is unclear whether this accords with available strategies of existential predication in A'ingae, although there is a reserved negative existential predicate (\textit{i.e.}, one not derived from a (positive) existential one) \textit{me'i} `\textsc{neg~pred}' \citep[according to][]{Hengeveld2018}. In this case, $ \textsc{exist}(x) = r$. Typological support for such a strategy might be found in Pitjantjatjara \gls{pjt}, where again, a single formative \textit{-tawara} `\gls{appr}' attaches to nouns and verbs. When functioning as a nominal suffix, \textit{-tawara} selects for a \gls{loc} marked noun. Pintjupi [\gls{piu}] deploys similar strategies \citep[16-9]{Zester2010}. Locative-marking of NPs is a strategy related to/often used in existential predication.
	}


 On the basis of the apparent loosening of morphosyntactic restrictions between each these three uses, the authors additionally predict that an implicational hierarchy of the form \textsc{avertive $\gg$ in-case $ \gg $ timitive} holds \citeyearpar[386-87]{AnderBois2020}, and provide some cross-linguistic data in support of this conjecture.
\footnote{Beyond the adjunct uses (\getref{con}) analysed in \citealt{AnderBois2020}, A'inge \textit{\textdblhyphen sa'ne}, \citet{Dabkowski} (forthcoming) additionally report uses corresponding to the \textsc{apprehensive} and \textsc{complementizer} uses described above. Examples are replicated below (\getref{con}′). It is not immediately clear what alterations to the semantics in (\getref{sa'ne}) would be needed to account for these uses.
	
The analysis of Kriol \textit{bambai} that follows shares a number of properties with this treatment of A'ingae apprehensive \textdblhyphen\textit{sa'ne} --- notably the (possibly) indirect relation between clauses connected by apprehensional morphology. As we will see, however, the numerous distributional and morphosyntactic differences between these two items (in addition to a number of diachronic concerns) will lead us down a somwhat different path.

\pex[glstyle=nlevel,aboveglftskip=.2ex,belowglpreambleskip=0ex,belowexskip=0ex,aboveexskip=1ex,exno=\getref{con}′] Non-adjunct uses of \textit{\textdblhyphen sa'ne}\deftagex{con.nadj} (\citealp{Dabkowski} forthcoming:3)
\a[label=d]\begingl\glpreamble\textsc{complementiser} use \endpreamble
tsai−ye\textdblhyphen\textbf{sa’ne} [bite−pass\textdblhyphen\textbf{\gls{appr}}]
\glft`You might get bitten.'
\endgl
\a[label=e]\begingl\glpreamble \textsc{apprehensive} use \endpreamble
tsama [but]
ña [1s]
dañu\textdblhyphen\textbf{sa'ne}\textdblhyphen khe [be hurt\textdblhyphen\textbf{\gls{appr}}\textdblhyphen thus]
dyuju−je\textdblhyphen ya [be afraid−\gls{ipfv}\textdblhyphen\textsc{verid}]
\glft`I was afraid I'd get hurt.'%\trailingcitation{()}
\endgl
\xe}



Finally, on the basis of a comparison with the neighboring Lau language (\texttt{[llu]} Solomonic: Malaita) and other SE Solomonic languages, Lichtenberk argues that the apprehensional functions of To'abaita \textit{ada} are a result of the grammaticalisation of an erstwhile lexical verb with meanings ranging a domain `see, look at, wake, anticipate' that came to be associated with warning and imprecation for care on the part of the addressee, before further developing the set of readings associated with the present day {\sc appr} marker \citeyearpar[303-4]{Lichtenberk1995}. According to Lichtenberk, Lau \textit{ada} admits of an \textsl{appr} reading while also functioning as a a fullly-inflected predicate. Its To'abaita cognate has lost this function, recruiting a new verb \textit{riki} `see, look', which apparently has shown signs of being recruited into apprehensional space (evincing a possible grammaticalisation cycle from perception verbs to apprehensionals.)




\subsection{Apprehensionality in the context of Australian Kriol}



\citet[171]{Dixon2002a} refers to the presence of nominal case morphology that marks the \textsc{aversive} as well as the functionally (and sometimes formally, see \citealp[44]{Blake1993}) related verbal category of apprehensionals as `pervasive feature of Australian languages' and one that has widely diffused through the continent.%\footnote{Dixon in fact attributes the paucity of work/recognition of this linguistic category to `grammarians' eurocentric biases' (171).}$ ^, $
\footnote{Aversive case is taken to indicate that the aversive-marked noun is ``to be avoided.'' This corresponds to the \textsc{timitive} for other authors \citep[\textit{e.g.},][]{Vuillermet2018,AnderBois2020}.} \citet[306]{Lichtenberk1995} marshalls evidence from Diyari (\texttt{[dif]} Karnic: South Australia) to support his claim about a nuanced apprehensional category, drawing from Austin's 1981 grammar. The Diyari examples in (\nextx) below are all adapted from \citet{Austin2011}, labelled for the apprehensional uses described in the previous section.

\pex Apprehensional marking in Diyari \a\begingl\glpreamble Avertive (precautioning)//
\gla \textbf{wata} yarra wapa\textbf{-mayi}, nhulu yinha parda-\textbf{yathi}, nhulu yinha nhayi-rna//
\glb {\sc \textbf{neg}} that~way go.\textbf{\gls{imp}}.\gls{emph} 3s{\sc.erg} 2s{\sc.acc} catch-\bfseries{{\footnotesize APPR}} 3s{\sc.erg} 2s{\sc.acc} see-{\sc ipfv$_{\gls{SS}}$}//
\glft `Don't go that~way or else he'll catch you when he sees you!'\hfill(230)//
\endgl
\a\begingl\glpreamble In-case (precautioning) //
\gla \textbf{wata} nganhi wapa-yi, karna-li nganha nhayi-\textbf{yathi}//
\glb {\sc \textbf{neg}} 1s\textsc{.nom} go\textsc{-pres} person-\textsc{erg} 1s\textsc{.acc} see-{\glem{appr}}//
\glft ‘I’m not going in case someone sees me.’\hfill(228)//
\endgl
\a\begingl\glpreamble Fear complementizer//
\gla nganhi \textbf{yapa}-li ngana-yi, nganha thutyu-yali matha\textasciitilde{}matha-thari-\textbf{yathi}//
\glb 1s{\sc.nom} \textbf{fear}{\sc-erg} be{\sc-pres} 1s{\sc.acc} reptile{\sc.erg} \gls{iter}\textasciitilde{}bite-\gls{dur}-\glem{appr}//
\glft `I'm afraid some reptile may bite me.'\hfill(228)//\endgl
\a\begingl\glpreamble Apprehensive use//
\gla nhulu-ka kinthala-li yinanha matha-\textbf{yathi}//
\glb 3s.\gls{erg}-\gls{deic} dog{\sc-erg} 2s{\sc.acc} bite\glem{-appr}//
\glft `This dog may bite you.'\hfill(230)//
\endgl
\xe

The sentences in (\lastx) shows a range of syntactic contexts in which Diyari apprehensional \textit{-yathi} `{\sc appr}' appears. The \textit{-yathi}-marked clause appears to be evaluated relative to a prohibitive in (a), a negative-irrealis predicate in (b) and predicate of fearing in (c), or alternatively occurs without any overt linguistic antecedent in (d).\footnote{Austin claims that these clauses are invariably `structually dependent' (230) on a `main clause' (\textit{viz.} the antecendent.) We will see in what follows a series of arguments (to some degree foreshadowed by Lichtenberk (1995: 307)) to eschew such a description.} In all cases, the predicate over which \textit{-yathi} scopes is \textbf{modalised} and expresses a proposition that the speaker identifies as `unpleasant or harmful' \citep[227]{Austin2011}. Little work has been undertaken on the grammaticalisation of apprehensionality.\footnote{\citet[171]{Dixon2002a} and \citet[44]{Blake1993} are partial exceptions although these both focus on syncretism in case marking rather than dealing explicitly with the diachronic emergence of the apprehensional reading.}

As we will see in the following sections, apprehensional uses of preposed \textit{bambai} in Kriol have a strikingly similar distribution and semantic import to the apprehensional category described in the Australianist and other typological literatures. \citet{Angelo2016} focus their attention on demonstrating the cross-linguistic attestation of a grammaticalisation path from (sub)sequential temporal adverbial to innovative apprehensional marking. They suggest that, for Kriol, this innovation has potentially been supported by the presence of like semantic categories in Kriol's Australian substrata. Note that for (almost all of) these languages, there are attested examples of the apprehensional marker appearing in both biclausal structures -- the \textbf{precautioning}-type uses described in the previous section $ (p \textsc{ lest } q) $, as well as ``apprehensive'' (monoclausal) ones (\textit{$\blacklozenge p$}). Data from virtually all attested languages of the Roper Gulf are shown in (\nextx).

\pex \textbf{Apprehensional/aversive marking in Roper Gulf languages}\deftagex{ropa}



\a\begingl\glpreamble\textbf{Wubuy}\deftaglabel{wub}//
\gla numba:-'=da-ya:::-ŋ gada, nama:='ru\textbf{-ngun-magi}//
\glb 2s$\scriptscriptstyle>$1s\textdblhyphen{}spear.for-go-\gls{npst} oops 1d$.\gls{incl}\scriptscriptstyle>\gls{anim}$\textdblhyphen{leave}-\textbf{\gls{appr}}-\textbf{\gls{appr}}//
\glft`Spear it! Ey! Or it will get away from us!'\trailingcitation{(\citealp[86]{Heath1980-nmet}, interlinearised)}//
\endgl



\a\begingl\glpreamble \textbf{Ngandi}\deftaglabel{nga}//
\gla a-d̪aŋgu-yuŋ ŋaṛa-wat̪i-ji, a-waṭu-d̪u aguṛa-\textbf{miliʔ}-ŋu-\textbf{yi}//
\glb \gls{ncl}-meat-\gls{abs} 1s$\scriptscriptstyle>$3s-leave-\gls{neg}:\gls{fut} \gls{ncl}-dog-\gls{erg} 3s$\scriptscriptstyle>$3s-\textbf{\gls{appr}}-eat-\textbf{\gls{appr}}//
\glft`I won't leave the meat (here), lest the dog eat it.'\trailingcitation{(\citealp[106]{Heath1978}, interlinearised)}//
\endgl




\a \begingl\glpreamble \textbf{Ngalakan}\deftaglabel{ngl}//
\gla garku buru-ye \textbf{mele}-ŋun waṛŋʼwaṛŋˀ-yiˀ//
\glb high 3ns-put \textbf{\gls{appr}}-eat.\gls{pres} crow-\gls{erg}//
\glft`They put it up high lest the crows eat it.'\trailingcitation{\citep[102]{Merlan1983}}//
\endgl

\a \begingl\glpreamble \textbf{Rembarrnga}\deftaglabel{rem}//
\gla ŋaran-\textbf{mǝʔ}-ɲamʔ ŋa-na laŋǝ ṛalk//
\glb 3s$\scriptscriptstyle>$1p.\gls{incl}-\textbf{\gls{appr}}-bite.\gls{pres} 1s$\scriptscriptstyle>$3-see.\gls{pst} claw big//
\glft`He might bite us! I saw his big claws.'\trailingcitation{\citep[182]{McKay2011}}//
\endgl



\a\begingl\glpreamble\textbf{Ritharrŋu}\deftaglabel{rit}//
\gla gurrupulu rranha nhe, \textbf{wanga} nhuna rra buŋu//
\glb give.\gls{fut} 1s.\gls{acc} 2s \textbf{\textit{or else}} 2s.\gls{acc} 1s hit.\gls{fut}//
\glft`Give it to me, or else I'll hit you.'\trailingcitation{(\citealp{Heath1980a}, interlinearised \& standardised to Yolŋu orthography)}//\endgl



\a\begingl\glpreamble \textbf{Marra}\deftaglabel{mar}//
\gla wu-ḷa ṇariya-yur, \textbf{wuniŋgi} ŋula ṉiŋgu-way//
\glb go-\gls{imp} 3s-\gls{all} \textit{\textbf{lest}} \gls{neg} 3s$\scriptscriptstyle>$2s-give.\gls{fut}//
\glft`Go to him, or else he won't give it to you.'\trailingcitation{(\citealp[187]{Heath1981}, cited also in A\&SB:284)}//\endgl
 
\a\begingl\glpreamble \textbf{Mangarayi}\deftaglabel{mang}//
\gla bargji $\varnothing$-ṇama \textbf{baḷaga} ña-way-(y)i-n//
\glb hard 2s-hold \textsl{\textbf{lest}} 2s-fall-\textsc{mood}-\gls{pres}.//
\glft`Hold on tight lest you  fall!'\trailingcitation{(\citealp[147]{Merlan1989}, cited also in  A\&SB:284)}//
\endgl


\xe



As shown in (\lastx), there is a diversity of formal strategies deployed (or combined) in these languages to realise apprehensional meaning: suffixation inside the verbal paradigm (\getfullref{ropa.wub}-\getref{ropa.nga}), prefixation to the verb stem (\getfullref{ropa.nga}-\getref{ropa.rem}) and a separate apprehensional particle (\getfullref{ropa.rit}-\getref{ropa.mang}).\footnote{Nominal suffixes are also reported in Australian languages, often described as \textsc{evitatives}, \textsc{aversives, adversatives} in the Australian descriptive literature (\citealp[9]{Zester2010}, \citealp[][]{Browne} forthcoming).} While detailed work on the expression of apprehensionality in these languages (including the syntactic status of apprehensional clauses) is not currently available,\footnote{Although see \citet{Zester2010} for a typology and \citet{Browne} (forthcoming) for an overview of apprehensional morphosyntax in Australian languages. The latter includes a detailed description of the variety of strategies deployed across the Ngumpin-Yapa family --- \textit{viz.} nominal marking, specialised complementisers and apprehensional auxiliaries. They argue that the precautioning-type apprehensional constructions in these languages are syntactically coordinate.} a number of generalisations can be made on the basis of the data in (\lastx). In all cases, the apprehensional appears to modify a fully-inflected (finite) clause, in most cases, ostensibly linking two (the $ p $ \textsc{lest} $ q $-type usage, see discussion above) predicates, each completely inflected for agreement/TMA information. Conversely, the Rembarrnga datum in (\getref{ropa.rem}) provides an example of an apprehensive (monoclausal/$ \blacklozenge p $) type use. It is unclear at this stage whether/for which languages the apprehensional-marked clauses invite an analysis as syntactically subordinate, althouogh in all cases, the prejacent to \textsc{appr} can be shown to be modally subordinate to information in the discourse context (often constrained by $ p $, see Ch. \ref{bambai.prag}). 

 In view of better understanding the semantical unity of these categories and the mechanisms of reanalysis which effect semantic change in \textit{bambai} and its TFA counterparts in other languages, the distribution and meaning of the `subsequential' and apprehensional usages of \textit{bambai} are described below.
%  This chapter proposes an account of this polysemy.
 
\subsection{Temporal frame adverbs and apprehensionality}\label{dataS}

\citet{Angelo2016,Angelo2018} provide convincing cross-linguistic evidence of the apparent lexical relationships between temporal frame adverbs and apprehensional markers. This can be taken, prima facie, to provide evidence of markers of temporal relations for recruitment as lexicalised modal operators. Table \ref{etyma} (partially adapted from \citet{Angelo2016,Angelo2018}) summarises examples from a number of languages where temporal frame adverbials also appear to display a robust apprehensional reading. Further, \citet[288]{Angelo2016} additionally suggest that there is some evidence of apprehensional function emerging in the \textit{bambai} cognates reported in Torres Strait Brokan,	 \texttt{[tcs]}, Hawai'ian Creole \texttt{[hwc]} and Norf'k (see fn \ref{PN footnote}).

\begin{table}[h!]\centering
	\caption{Etyma and polysemy for apprehensional modals} \label{etyma}
	\begin{tabular}{llll}
		Language & Adverbial & Gloss\footnotemark & Author (grammar)\\\midrule
		Std Dutch \texttt{[nld]} & \textit{straks} & soon & \citet{Boogaart2009,Boogaart2020}\\
		Std German \texttt{[deu]} & \textit{nachher} & shortly, afterwards&A\&SB (2018)\\
		Marra \texttt{[mec]}& \textit{wuniŋgi} & further & \citet{Heath1981}\\
		Mangarayi \texttt{[mpc]} & \textit{baḷaga} & right now/today & \citet{Merlan1989}\\ 
		Kriol \texttt{[rop]} &  \textit{bambai} & soon, later, then& \\\bottomrule
	\end{tabular}\end{table}
	\vspace{.25cm}
\footnotetext{This isn't to suggest that the semantics of those words provided in the `{\sc gloss}' column in the table above ought to be treated as identical: the definitions seek to capture a generalisation about sequentiality. A prediction that falls out of this generalisation is that TFAs like `later, soon, afterwards, then' might be best interpretable interpretable as subsets of this category.}


% Additionally, as \citet{Angelo2018} show, apparent lexical relationships between markers of subsequentiality and apprehensionality hold in some of these languages. Examples of a single formative apparently encoding readings are given in (\nextx). 
 

% Compare, for example, the uses of Marra \textit{wuninggi} and Mangarrayi \textit{barlaga} in  (\nextx) to those in (\getref{ropa.mar},\getfullref{ropa.mang}) above. Here, the availability of apprehensional and subsequential readings appears to echo that of Kriol.
  
 Compare these uses of Mangarrayi \textit{baḷaḷaga\textasciitilde{baḷaga}} in (\getref{mang}) to (\getfullref{ropa.mang}) above. In (\getfullref{mang.a}), \citet[138]{Merlan1989} notes that the temporal frame uses of \textit{baḷaḷaga}---while often translated as `today'---appears to correspond to `right now' (she also notes that ``Pidgin English informants use [...the reduplicated form] \textit{today-today} to mean `now' as well as `today' in the English sense''). In all of these Mangarayi data, \textit{baḷaga} appears to indicate that the event described in the clause that it introduces obtains (or may obtain) subsequently to some time established in the previous clause.\footnote{Note that \textit{baḷaga} is glossed by Merlan as `before' in the imperative sentences (\getfullref{mang.imp1}-\getref{mang.imp2}). In both cases, the speaker appears to indicate that event described in the following clause is imminent (note that in declarative contexts this might be translated as `then').} %Additionally, another subsequentiality marker \textit{wuṛay} -- glossed as `later' -- also appears to function apprehensionally in (\getfullref{mang.wuray}).
 
 
\pex\textbf{Mangarayi}\deftagex{mang}
\a\begingl\gla ḏayi ŋa-yirri-wa-ya-b gurrji, \textbf{baḷaḷaga} ga-ŋa-wa-n\deftaglabel{a}//
\glb \gls{neg} 1s$\scriptscriptstyle>$3s-see-\gls{aug}-\gls{pneg} long.ago \textbf{today} 3-1s$\scriptscriptstyle>$3s-go.to.see-\gls{pres}//
\glft`I hadn't seen it before, today I'm seeing it.'\trailingcitation{(\citealp[138]{Merlan1989}, cited also in  A\&SB 2018:13)}//\endgl
\a\begingl\gla galaji ŋanʔ-ma \textbf{baḷaga} yag//
\glb quickly ask-\gls{imp} \textbf{before} go//
\glft`Ask him quick before he goes.'\trailingcitation{(\citealp[147]{Merlan1989}, cited also in  A\&SB: 284)}\deftaglabel{imp1}//\endgl
\a\begingl\gla a-ŋaḷa-yag \textbf{baḷaga} miḷiḷitma//
\glb \gls{hort}-1p.\gls{incl}-go \textbf{before} sunset//
\glft`Let's go before the sun sets.'\trailingcitation{\citep[147]{Merlan1989}}\deftaglabel{imp2}//\endgl


\a\begingl\gla bargji nama \textbf{balaga} iia-way-(y)i-n//
\glb hard 2s.hold.\gls{imp} \textbf{\textsl{lest}} 2sf all-\gls{mod}-\gls{pres}//
\glft`Hold on tight lest you fall!'\trailingcitation{\citep[147]{Merlan1989}}//\deftaglabel{imp3}\endgl


\a\begingl\gla ŋiñjag ŋaḷa-bu-n guṛuugguṛug-bayi, \textbf{wuṛay} ḍoʔ a-ŋayan-ma//
\glb \gls{proh} 1p.\gls{incl}-kill-\gls{pres} white.people-\gls{foc} \textbf{later} shoot \gls{irr}-3s$\scriptscriptstyle>$1p.\gls{incl}-\gls{aux}//
\glft`We can't kill white people. Later on they might shoot us.'\trailingcitation{\citep[147]{Merlan1989}}\deftaglabel{wuray}//\endgl
\xe



\citet[147]{Merlan1989} glosses \textit{baḷaga} as `\textsc{evitative/anticipatory}', commenting  that these two notions are ``sometimes indistinguishable.'' She also notes the formal (reduplicative) relation to frame adverbial \textit{baḷaḷaga} `right now, today', commenting on the shared property of ``immediacy'' that links all these readings.\footnote{A common derivational process in Australian languages \citetext{\citealp[113,209]{Dineen1990}; \citealp[201]{Dixon2002}}, Mangarayi reduplication frequently functions as an property intensifier \citep[166-7]{Merlan1989}. In this sense, \textit{baḷaḷaga} `imminently/right now' can be read as an intensified form of \textit{baḷaga} `soon, later.'} 
Note additionally the apparently apprehensional use of \textit{wuṛay} `later' in a prohibitive context in (\getfullref{mang.wuray}). While Merlan makes no mention of any conventionalised ``evitative/anticipatory'' uses of this adverb, this type of use context is a likely source for the type of apprehensional and causal/elaboratory inferences invited by temporal frame adverbials. A similar patter in attested in Marra (\getref{mar}):



\pex \textbf{Marra \textit{wuniŋgi}}\trailingcitation{(\citealp[360]{Heath1981}, interlinearised)}\deftagex{mar}
\a\begingl\glpreamble Subsequential use//
\gla wayburi jaj-gu-yi \textbf{wuniŋgi}: gaya bayi gal-u-jingi//
\glb southward chase-3s$\scriptscriptstyle>$3s.\gls{pst} \textbf{more} there in.south bite-3s$\scriptscriptstyle>$3s-did//
\glft`Then [the dingo] chased [the emu] a bit more in the south.'\deftaglabel{ssq}//\endgl
\a\begingl\glpreamble Apprehensional use (see also \getfullref{ropa.mar} above)//
\gla ŋa-naŋgu-wa, \textbf{wuniŋgi} ṛag-ṇiŋg-anjiyi//
\glb  2s$\scriptscriptstyle>$1s-give.\gls{imp} \textbf{\textsl{lest}} hit-1s$\scriptscriptstyle>$2s-\gls{aux}(\textsc{evit})//
\glft `Give it to me, otherwise I'll hit you!'\deftaglabel{appr}//
\endgl
\xe

Per \citeauthor{Heath1981}'s analysis (1981:308), Marra has an inflectional apprehensional category (his `\textsc{evitative}') which is realised only in positive \textit{lest}-type clauses (\getfullref{mar.appr}). These frequently co-occur (in elicitation) with the adverbial \textit{wuniŋgi} `farther along, furthermore, in addition' (common in text translations.) Heath suggests that negative \textit{lest}-clauses are ``conveyed by the future negative along with \textit{wuniŋgi}'' (187). He explicitly notes the similarity between this strategy/apparent polysemy between subsequential-type TFAs and apprehensionals in neighbouring languages, incuding Kriol \textit{bambay} (\textit{sic}; 187, 308). Further discussion and a diachronic account of this apparent polysemy is given in \S~\ref{bambai.dia}.

\section{The distribution of Kriol \textit{bambai}}\reversemarginpar
This section (informally) describes the distribution and meaning of both temporal-frame and apprehensional readings of \textit{bambai} in the data.  The Kriol data cited here draws from \citeauthor{Angelo2016} ([\textit{A\&SB}], 2016) and the Kriol Bible ([\textit{KB}], \citealp{TheBibleSocietyinAustralia2007}) in addition to elicitations from, and conversations with, native speakers of Kriol recorded in Ngukurr predominantly in 2016 and 2017 (see Ch. \ref{IntroCh}). Figure \ref{bb-dist} represents a coarse taxonomy of the readings available to \textit{bambai}, cross-referenced for the subsection in which each is discussed\footnote{As we will see, uses corresponding to Lichtenberk's \textsc{fear} function (discussed above) and co-occurrences of \textit{bambai} with \textit{if}-clauses are taken to be subsumed under the \textit{bambai}'s \textsc{\textit{apprehensive}} function.}

\begin{figure}[h]\caption{Range of functions for \textit{bambai}}\centering\label{bb-dist}
\begin{tikzpicture}[align=center]
	\tikzset{level distance=75pt}
	\Tree [.\textit{\textbf{bambai}} [.{\textsc{subsequential}~(\S~1)\\`then'} ] [.{\textsc{apprehensional}~(\S~2)\\$ \blacklozenge $} {\textit{\textsc{Precautioning}}~(\S~2.1)\\`lest, otherwise'} {\textit{\textsc{Apprehensive}}~(\S~2.2)\\`possibly'} ] ]
\end{tikzpicture}\end{figure}


	\subsection{Temporal frame reading}\label{dataStfa}
\textit{Temporal frame adverbials} (TFAs) are linguistic expressions that are used to refer a particular interval of time, serving to precise the \textit{location} of a given eventuality on a timeline. As an example, TFAs include expressions like \textit{this morning} or \textit{tomorrow}, which situate the eventuality that they modifies within the morning of the day of utterance or the day subsequent to the day of utterance respectively \citep[see][307]{Binnick1991}. 

As shown in Chapter \ref{IntroCh}, formally, we can model the contribution of temporal expression by assuming a set (chain) $\mathcal T$  of points in time which are all strictly ordered with respect to each other chronologically. This is represented by a \textsc{precedence relation} $ \boldsymbol\prec $ (where $t_1\prec t_2\leftrightarrow$ $t_1$ precedes $t_2$). A TFA like \textit{today}, then, is a predicate of times: it picks out a temporal \textit{frame} for the predicate --- that is, all the points in time between the beginning and the end of the day of utterance. In the sentence \textit{Mel ate today}, the TFA restricts the instantiation time of the eating event $(t_e)$ to this interval. That is, \textit{Mel ate today} is true iff Mel ate at $t_e$ and $\underset{\text{start-of-day}}{t_1}\prec t_e\prec\underset{\text{end-of-day}}{t_2}$. This can be represented using an interval notation as $ t_e\in[t_1,t_2] $. %\footnote{This }
%A Dowtyan treatment of the semantic contribution of \textit{today}, then takes the contribution of \textit{today} as in \nextx:\pex 
	
	
As mentioned in §~\ref{bambai.intro}, Kriol \textit{bambai} is derived from an archaic English temporal frame adverbial, \textit{by-and-by} `soon', a lexical item with some currency in the nautical jargon used by multiethnic sailing crews in the South Pacific in the nineteenth century. The general function of \textit{by-and-by} has been retained in contemporary Kriol, namely to temporally advance a discourse, much as Standard Australian English uses expressions of the type `soon/a little while later/shortly after(wards)' or `then.' These expressions represent a subset of `temporal frame adverbials': clause modifiers that delimit the temporal domain in which some predicate is instantiated. In this work, I refer to the relevant set of TFAs as \textit{subsequentiality} ({\sc`subseq'}) adverbials. The motivation for describing this as a semantic subcategory (a special case of the prospective) is the robust intuition that, in addition to temporally advancing the discourse (\textit{i.e.}, marking the instantiation of the prejacent predicate posterior to a given reference time), {\sc subseq} TFAs give rise to a salient, truth-conditional expectation that the predicate which they modify obtain in non-immediate sequence with, but in the \textbf{near future} of a time provided by the context of utterance. This general function of \textit{by-and-by} is attested in the contact varieties (\textit{i.e.}, pidgins) spoken in the nineteenth century in Australia; this is shown in (\ref{ntpidgin}).

\ex\label{ntpidgin}\begingl\glpreamble \textup{An excerpt from a (diagrammatic) explanation of betrothal customs and the genealogy of one couple as given to T A Parkhouse by speakers of a Northern Territory pidgin variety from the Larrakia nation in the late nineteenth century. \trailingcitation{(\citealt[4]{Parkhouse1895}, also cited in \citealp[299]{Harris1986a}.}\trailingcitation{My translation, incl. subscript indexation)}}\vspace{.3em}//
	\gla ... that fellow lubra him have em nimm. +	\textbf{by-and-by} him catch him lubra, him have em nimm. +	Him lubra have em bun-ngilla.
	\textbf{By-and-by} girl big fellow, him nao`wa catch him, him méloa have em bun-ngilla. +	\textbf{By-and-by} nimm big fellow, by-and-by bun-ngilla big fellow, him catch him.//
	\glb ~ that {\sc attr} woman 3s have {\sc tr} boy
	\textit{\textbf{bambai}} 3s catch \textsc{tr} woman 3s have \textsc{tr} boy 3s woman have \textsc{tr} girl \textit{\textbf{bambai}} girl big \textsc{attr} 3s husband catch 3s 3s pregnant have \textsc{tr} girl \textit{\textbf{bambai}} boy big \textsc{attr} \textit{\textbf{bambai}} girl big \textsc{attr} 3s catch 3s//
	\glft `...That woman$_h$ had a son$_i$. Later, he$_i$ got a wife and had a son$_j$. This woman$_k$ had a daughter$_\ell$. Then, when the girl$_\ell$ had grown up, her husband got her$_\ell$ pregnant, she$_\ell$ had a daughter$_m$. Then, when the boy$_j$ was grown and the girl$_m$ was grown, he$_j$ got her$_m$.'//\endgl
\xe

Note that, according to Parkhouse, (\ref{ntpidgin}) constitutes a description of the relationship history of one couple; each sentence is past-referring. There is no tense marking in the Pidgin narrative.  In each of the \textit{by-and-by} clauses in (\ref{ntpidgin}), the speaker asserts that the event being modified is \textit{subsequent} to a reference time set by the previous event description. In this respect, \textit{by-and-by} imposes a temporal frame on the event description that it modifies. 

As we have seen above (\textit{e.g.}, \getref{ssq0}), the {\sc subseq}-denoting function of \textit{bambai} shown here has been retained in Kriol. This reading is shown again in the two sentences in (\getref{ssq}). The schema in (\getfullref{ssq.schema}) provides an informal representation of this context-dependent, ``subsequential'' temporal contribution.

\pex\a\deftagex{ssq}\deftaglabel{fladwoda}	\begingl	\glpreamble \textbf{Context:} During a flood a group of people including the speaker have moved to a dry place up the road//
	\gla mela bin ol mub deya na, jidan deya na, \textbf{bambai} elikopta bin kam deya na, detlot deya na garra kemra//
	\glb 1p{\sc.excl} {\sc pst} all move there now sit there now \textbf{\textit{bambai}} helicopter {\sc pst} come there now {\sc det:pl} there now have camera//
	\glft `We all moved there, \textbf{then} a helicopter came, the people there had cameras'\\\hspace*{\fill}[A\&SB: 271]//
	\endgl

\a\begingl	\deftaglabel{ib}\glpreamble\textbf{Context:} Eve has conceived a child.//
	\gla \textbf{Bambai} imbin abum lilboi//
	\glb \textit{bambai} 3s.{\sc pst} have boy//
	\glft `Subsequently, she had (gave birth to) a boy'\hfill[KB:~Jen~4.1]//
	\endgl
	\a\deftaglabel{schema}\textbf{Instantiation  for subsequential reading}\hfill\textit{(to be revised)}\\\begin{tikzpicture}[grow=right]\large
	\tikzset{level distance=100pt,sibling distance=18pt}
	\tikzset{execute at begin node=\strut}
	\Tree [.$t_r$ [.$\boldsymbol{t_e}$ \edge[dotted]; $t^+$ ] ]
	\end{tikzpicture}
	
	%In the reference world $w_0$:
	 The eventuality described by the predicate is instantiated at some time $ t_e $ in the future of a reference time $t_r$. $ t_r $ is contextually determined---by an antecedent proposition if present---or otherwise established by the discourse context. Further, \textit{subsequential} TFAs impose a requirement that $ t_e $ obtain within some constrained interval subsequent to $ t_r $ (that is, before $ t^+ $).
			\xe

As shown in (\getfullref{ssq.fladwoda}) above, the arrival of the helicopter (and its associated camera crew) is modified by \textit{bambai qua} TFA. This has the effect of displacing the instantiation time forward with respect to the reference time provided by the first clause. 	\textit{Bambai} has the effect of displacing the instantiation of helicopter-arrival forward in time with respect to the reference time provided by the first clause (\textit{sc.} the time that the group had moved to a dry place up the road).

 Similarly, (\getref{ssq.ib}) asserts that the eventuality described by the prejacent to \textit{bambai} (namely the birth of Cain) is instantiated in the near future of some reference time $t_r$ provided contextually, albeit not by a linguistically overt antecedent clause. That is, Eve gave birth at some $ t_e \in \{t'_e:t_r\prec t'_e\prec t^+\}$.\footnote{This is not to suggest the referability of some `latest bound' reference time $t^+_r$. The latter merely represents a (vague) contextual expectation by which the event described by the prejacent had better have obtained for the whole sentence to be judged true. This device is described in more detail in \S~\ref{bambai.subseq}.}  The subsequent verse: \textit{Bambai na Ib bin abum najawan lilboi} (KB Jen 4:2) `Soon after \textit{that}, Eve had another boy' further forward-displaces the birth event of Abel.  Subsequential TFAs are distinguished by this `near future' restriction, underpinned by a set of conversational expectations over reasonable degrees of ``soonness.''
 
 \paragraph{Narrative cohesion} \textit{bambai} additionally occurs with an undoubtedly related endophoric use (along with the apparently phatic discouse particle \textit{na} < `now').\footnote{There are 455 tokens of clause-intial \textit{Bambai na} in the Kriol Bible.} This function is particularly frequent in the Kriol Bible and can be taken to rely on a metonymic relationship between the structure of time and the structure of a text/discourse (compare English \textit{now then} or \textit{so next}).
 
 
 \pex \textbf{Discourse cohesion uses of \textsc{subsequential} \textit{bambai}}
 \a\begingl\gla Wal deya na deibin jidan longtaim. \textbf{Bambai na} wen imbin brabli olmen, Tera bin dai.//
 \glft`So they lived there for a long time. And then, when he was very old, Terah died.'\trailingcitation{[KB~Jen.~11.32]}//\endgl
 \a\begingl\gla Longtaim God bin meigim det pramis garram Eibrahem, en imbin tok im garra kipum det pramis. \textbf{En bambai na} 430 yiyastaim God bin gibit det lowa langa Mosis.//
 \glft`Long ago, God made a covenant with Abraham and said that he would keep the promise.\textbf{ Now then,} 430 years later, God gave Moses the laws...'\trailingcitation[{KB~Gal.~3:17]}//\endgl
 
 \xe
 
 

In this subsection, we have seen an overview of the semantic contribution of \textit{bambai} in its capacity as a `subsequential' TFA. A discussion of apprehensional uses follows. %Additionally, sentences linked by these adverbials appear to invite an inference of an additional nontemporal (etiological) dimension to the relation between the two clauses.


	\subsection{Apprehensional reading}\label{dataSapp}



In his survey of `apprehensional epistemics' (reviewed in \S \ref{typ.appr} above), \citeauthor{Lichtenberk1995} describes apprehensionals like To'abaita \textit{ada} as having a dual effect on their prejacents (``mixed modality''):
\begin{itemize}\item  \textit{epistemic downtoning} --- \textit{i.e.}, `signal[ling] the [speaker's] relative uncertainty [...] about the factual status of the proposition' --- and \item  (a shade of) \textit{volitive modality} --- `the fear that an undesirable state of affairs may obtain.' \trailingcitation{\citep[295-6]{Lichtenberk1995}}\end{itemize} While we are not at this stage committed to Lichtenberk's metalinguistic labels, a modal semantics for Kriol \textit{bambai} is suggested on the basis of the data below. We will see how this use diverges from the subsequential/temporal frame readings described above, broadly dividing \textit{bambai}'s apprehensional contribution into two main subtypes that align with the \textit{avertive} (§ \ref{lest}) and \textit{apprehensive} (§ \ref{ep.adv}) functions identified in previous literature \citep{Lichtenberk1995,Vuillermet2018} and described above.




\subsubsection[Precautioning uses]{$ \boldsymbol{p \textit{ bambai } q}$ : the precautioning/conditional use}\label{lest}

The ``precautioning'' uses of apprehensional morphology are characterised by serving to ``connect a clause encoding an apprehension-causing situation to a preceding clause encoding a precautionary situation'' \citep[298]{Lichtenberk1995}. The data provided below show \textit{bambai}'s function in conditional-like constructions, where it precedes both indicative and counterfactual consequent clauses.\footnote{Given the availability of these counterfactual \textsc{lest}-type uses of \textit{bambai}, Lichtenberk's ``precautioning'' label may be less appropriate. Lichtenberk doesn't provide evidence of counterfactual uses for To'abaita \textit{ada}, although his discussion of colloquial Czech \textit{aby} `\gls{appr}' shows that this item is apparently compatible in counterfactual contexts \citeyearpar[309]{Lichtenberk1995}. In any case, I continue to describe all \textsc{lest}-type uses as \textit{precautioning} given this term has been adopted by other authors \citep{Vuillermet2018,AnderBois2020}.}\label{lests}

\paragraph{Indicative `nonimplicationals.'}

Apprehensional \textit{bambai} occurs in situations where the speaker identifies some undesirable eventuality as a potential outcome of the discourse situation. \citet[272\textit{ff}]{Angelo2016} observe that these readings may or may not constitute ``admonitory'' speech acts --- \textit{i.e.}, can serve as direct warnings or threats (directive illocutionary force in \getfullref{appr1.motika}-\getref{appr1.chase}), or merely as predictions of a negative outcome for the subject  (\textit{e.g.}, \getfullref{appr1.meds}).

The sentence data in (\getref{appr1}) demonstrate how \textit{bambai}-sentences are used to talk about undesirable possible future eventualities. Extending the model introduced above to modelling this (following the ``possible worlds'' semantic framework introduced in chapter \ref{IntroCh}), we postulate a set $\mathcal W$ of \textit{possible worlds}. On standard assumptions, a ``proposition'' $(p\in\mathcal W\times\{\mathbb{T,F}\})$ is a set of possible worlds, namely those in which it is true (\citealp[\textit{e.g.},][]{Stalnaker1976,Kripke1963,Kratzer1977}, a.o.) 

Generally speaking, the ``precautioning'' construction --- \textit{i.e.}, \textit{p bambai q} on its apprehensional reading --- appears to convey converse nonimplication between $ p $ and $ q $: `if some situation described in $p$ doesn't obtain in $w$, then the (unfortunate) situation described in $q$ might' --- \textit{i.e.}, $\neg p(w)\to\blacklozenge q(w)$.
% That is, it functions similarly to \textit{otherwise} as discussed in Ch. \ref{otherwise} above, in addition to apparently implying that the speaker is negatively disposed to the eventuality described by the prejacent. 
% Additional data showing these uses is shown in (\getref{appr1}) below.

\pex[interpartskip=1ex]
\a\deftagex{appr1}\deftaglabel{motika}\begingl
\glpreamble \textbf{Context:} Two children are playing on a car. They are warned to stop.//
\gla Ey! \textbf{bambai$_1$} yundubala breikim thet motika, livim. \textbf{bambai$_2$} dedi graul la yu//
\glb Hey! \textbf{\textit{bambai}} 2d break {\sc dem} car leave \textbf{\textit{bambai}} Dad scold {\sc loc} 2s//
\glft `Hey! You two might break the car; leave it alone. Otherwise Dad will tell you off!'\trailingcitation[A\&SB: 273]//
\endgl
\a
\begingl\deftaglabel{chase}
\gla yu stap ritjimbat mi na \textbf{bambai} ai kili yu ded en mi nomo leigi meigi yu braja jeikab nogudbinji//
\glb 2s stop chase.\gls{ipfv} 1s {\sc emp} {\bf bambai} 1s kill 2s dead and 1s \gls{neg} like make 2s brother jacob unhappy//
\glft`Stop chasing me or I'll kill you and I don't want to upset your brother Jacob (\textit{sic})'\trailingcitation{[GT 22062016-21', retelling KB 2Sem~2.22]}//
\endgl
\a\deftaglabel{meds}\begingl\gla ai garra go la shop ba baiyim daga, \textbf{\textit{bambai}} ai (mait) abu no daga ba dringgi main medisin//
\glb 1s {\sc irr} go {\sc loc} shop {\sc purp} buy food \textit{\textbf{bambai}} 1s (\gls{mod}) have no food \gls{purp}//
\glft `I have to go to the shop to buy food \textbf{otherwise} I may not have food to take with my medicine.'\hspace*{\fill}[AJ 23022017]//
\endgl
\a\begingl\gla ai\textdblhyphen{rra} gu la det \textup{airport} ailibala, \textbf{bambai} mi mis det erapein//
\glb 1s\textdblhyphen\gls{irr} go \gls{loc} the airport early \textit{\textbf{bambai}} 1s miss the æroplane//
\glft`I'll go to the airport early, \textbf{otherwise} I could miss my flight.'\trailingcitation{[GT~16032017-21']}//\endgl
\xe

In (\getfullref{appr1.motika}), there are two tokens of apprehensional \textit{bambai}. The second \textit{(bambai$_2$)} appears to be anaphoric on imperative \textit{livim!} `leave [it] alone!' Notably, it appears that the Speaker is warning the children she addresses that a failure to observe her advice may result in their being told off: $\neg \textit{ (livim) }\to\blacklozenge \textit{ (dedi graul)}$. Unlike the uses of \textit{bambai} presented in the previous subsection, \textit{bambai} here is translatable as `lest/otherwise/or~else.' \textit{bambai$_1$}, the first token in (\getfullref{appr1.motika}), appears to have a similar function, although has no overt sentential antecedent.\footnote{In reconstructing this sentence context, a consultant unprompted introduced an explicit antecent: \textit{gita burru det mutika, bambai yu breigim im} `get off the car! Otherwise you might break it!' [GT~20170316]} In this case, the Speaker is issuing a general warning/admonition about the children's behaviour at speech time. In uttering the \textit{bambai$_1$} clause, she asserts that, should they fail to heed this warning, an event of their breaking the car is a possible outcome. (\getfullref{appr1.chase}) shows a similar use.

(\getfullref{appr1.meds}) provides an example of an apprehensional/{\sc lest}-type reading occurring in a narrative context (that is a representational/predictive-type illocutionary act). Here, the Speaker identifies a possible unfortunate future situation in which she has no food with which to take her medicine. Here, in uttering the \textit{bambai} clause, she asserts that such an eventuality is a possible outcome should she fail to go to the shop to purchase food: $ \neg(\textit{go.shop})\to\blacklozenge(\textit{foodless}) $. This reading is robustly attested in contexts where the antecendent is modified by some irrealis operator. For example, in (\getref{app0rp}) -- repeated here from (\getfullref{app0}) above -- \textit{bambai} makes a similarly modal claim: if $\kappa$ is a set of worlds in which I drink coffee at $t^\prime$ (and $\overline{\kappa}$ is its complement), then an utterance of (\getref{app0rp}) asserts that $\exists w\in\overline{\kappa}:\text{I sleep by }t^+\text{ in }w.$

\pex\a\deftagex{app0rp}\deftagpage{app0rppage}\deftaglabel{kofi}\begingl
\glpreamble\textbf{Context:}  It's noon and I have six hours of work after this phonecall. I tell my colleague://
\gla ai\textdblhyphen{}rra dringgi kofi \textbf{bambai} mi gurrumuk la desk iya gin//
\glb 1s\textdblhyphen{\sc irr} drink coffee \textit{bambai} 1s fall.asleep {\sc loc} desk here {\sc emph}//
\glft `I'd better have a coffee otherwise I might pass out right here on the desk'\trailingcitation{[GT~28052016]}//
\endgl\vspace{.5cm}
\a \textup{\textbf{Instantiation schema for \textit{apprehensional} reading in (\getref{app0rp.kofi})}}


$\boldsymbol{\hspace{10pt}t\!*\hspace{40pt}  t^\prime \hspace{40pt}  t^+ }$\vspace{.3cm}

\leavevmode\vadjust{\vspace{-\baselineskip}}\newline\begin{tikzpicture}[grow=right]
%\node(appr) at (-9:-5) {$t_0$};
%\node(appr) at (-5:1.2) {$t^\prime$};
%\node(appr) at (-5:2.5) {$t^+$};

\tikzset{level distance=50pt,sibling distance=8pt,edge from parent/.append style={dashed}}
{\large\Tree [.$w*$ [.${\kappa}$ [.$w_{\kappa_4}$ ] [.$w_{\kappa_3}$ ] ] [.$\bar{\kappa}$ [.$w_{\bar{\kappa_2}}$ ] [.$w_{\bar{\kappa_1}}$ ] ] ]
}
\end{tikzpicture}

In the reference world $w*$ at speech time $t*$, the Speaker establishes a partition over possible futures: they are separated into those in which, at time $ t' $, he drinks coffee $ \{w'\mid w'\in\kappa \} $ and those in which he doesn't $ \{w'\mid w'\in\overline\kappa \} $. In those worlds where he fails to drink coffee, there exist possible futures ($w_{\bar\kappa_1}\vee w_{\bar\kappa_2}$) by which he's fallen asleep by some future time $t^+$.
\xe

Of particular note is this behaviour where \textit{bambai} appears to be anaphoric on \textbf{the negation} of a proposition that is calculated on the basis of a linguistically represented antecedent (that is, the preceding clause.) Demonstrated in (\getref{muvi}), This appears to be categorical. where a {\sc subseq} reading of \textit{bambai} --- \textit{viz.} $^\# \textit{watch.movie}(t_2) \wedge \textit{sleep}(t_3) $ --- is infelicitous. That is: only an \textsc{apprehensional} reading is available: watching a film is a measure taken to avert asleep $ \boldsymbol\neg\textit{(watch.movie)}\to\blacklozenge \textit{(sleep)} $.
%the prejacent \textit{mi gurrumuk} `I fall asleep' is interpreted as a possible outcome of \textbf{not} watching a film.
\pex
\begingl\glpreamble\textbf{Context:} The Speaker is experiencing a bout of insomnia\deftagex{muvi}//
\gla\ljudge{$^\#$}airra wotji muvi bambai mi gurrumuk//
\glb 1s\textdblhyphen{\sc irr} watch film \textit{bambai} 1s fall.asleep//
\glft $ ^\# $\textbf{Intended:} `I'll watch a film, then I'll (be able to) fall asleep.'\\
\textbf{Available reading:} `I'll watch a film, otherwise I may fall asleep.'\trailingcitation{[AJ~23022017]}//
\endgl\xe

The relationship between the antecedent clause and the context on which (apprehensional) readings of \textit{bambai} is anaphoric is further discussed below in chapter \ref{bambai.prag}.
\paragraph{Counterfactual `nonimplicationals'}

\textit{bambai} similarly receives an apprehensional reading in subjunctive/counterfactual contexts: those where an alternative historical reality is considered.\footnote{See \citealp{VonFintel2012} for a general overview of counterfactual conditionals.} The occurrence of apprehensionals in these contexts is little-reported cross-linguistically (described as ``rare'' in \citealp{Angelo2018} for German \textit{nachher}.)



In (\nextx), the Speaker identifies that in some alternative world (say $w^\prime$) in which he behaved differently to the way in which he did in the evaluation world (${w'\not\approx_{t*} w*}$)\footnote{A definition and further discussion the $ \boldsymbol{\approx}$-relation (``historical alternative to'') is given in (\getref{histaltdef}). A formal account is further developed below.} --- namely one in which the event described in the antecedent failed to obtain --- there is a (significant) possibility that he would have slept at work. Consequently, and comparably to the example (\lastx) above, \textit{bambai} modalises its prejacent: it asserts that $\exists w'[w'\notin\kappa\wedge\text{ I sleep by }t^+\text{ in }w^\prime$].

\pex
\a\deftagex{sjv}\deftaglabel{A}\deftagpage{sjvApg}\begingl
\gla ai\textdblhyphen{}bin dringgi kofi nairram \textbf{bambai} ai bina silip\textasciitilde silip-bat la wek//
\glb 1s\textdblhyphen{\sc pst} drink coffee night \textit{\textbf{bambai}} 1s {\sc pst:irr} sleep{\sc\textasciitilde dur-ipfv} {\sc loc} work//
\glft`I had coffee last night \textbf{otherwise} I might have slept at work' \trailingcitation{[AJ~23022017]}//\endgl


\vbox{\a \textbf{Instantiation schema for apprehensional reading in (\getref{app0rp.kofi})}


$\boldsymbol{\hspace{10pt}t_0\hspace{40pt}  t^\prime  \hspace{50pt} t^+ }$\vspace{.3cm}


\begin{tikzpicture}[grow=right]

	\tikzset{level distance=50pt,sibling distance=8pt}
	\large\Tree [.$w*$ \edge[thick]; [.$w*$ \edge[thick]; [.$w*$ ] ] \edge[dotted]; [.$\bar\kappa$ \edge[dotted]; [.$w_{\bar\kappa}^{\prime\prime}$ ] \edge[dotted]; [.$w_{\bar\kappa}^{\prime}$ ] ] ]
\end{tikzpicture}}

Here, the Speaker considers a set of worlds that historically diverge from the evaluation world $w*$, namely the set of worlds where, unlike the evaluation world, the Speaker did not drink coffee at $t^\prime$ --- $\{w'\mid w'\in\overline{\kappa}\}$. The Speaker asserts that there are some possible near futures to $\langle t^\prime,w_{\bar\kappa}\rangle$ in which he falls asleep by some time $t^+$, posterior to $t^\prime$.

\xe

The Kriol apprehensional data described so far is intuitively unifiable insofar as it bears some resemblance to more familiar conditional constructions --- \textit{(i.e.}, that of an ``infixed'' two-place relation between two propositions.) Unlike \textit{if... then}-conditionals, in all the apprehensional data, we have seen so far, \textit{bambai} introduces a predicate describing some eventuality which construes as undesirable for the speaker. It appears to that this eventuality is a \textit{possible, foreseeable} future consequence of some other contextually provided proposition --- in the examples discussed so far, this proposition is often interpreted as that of the non-instantiation of $ q $ (see Ch. \ref{bambai.prag}). 

The `indicative' and `counterfactual' uses presented here can be unified by appealing to the notion of ``settledness'' presuppositions \citep[\textit{e.g.,}][82, \textit{passim}]{Condoravdi2002}. In all sentences of the form $ p \textit{ bambai }q$, a reference world and time are provided by some (perhaps modalised) antecedent proposition. In those contexts where $ q $ is understood to be being asserted of a future time ($t_e\succ t*$) or a different world ($w^\prime\not\approx_{t*} w*$); 
%--- those where the Speaker could not possibly have access to a determinate set of facts --- the Speaker \textbf{\textit{R}}-implicates \citep[see][]{Horn1984} that they are making a prediction;
 the entire proposition construes as modalised. This intuition will be spelled out in detail in Ch~\ref{bambai.semx}).  %Of additional interest is the fact that, in the examples we have seen so far, $ p $ --- the proposition modiified by \textit{bambai} --- appears to be a potential consequence of the non-instantiation of $ q $ (the antecedent.) 

In effect, the contribution and distributional properties of \textit{bambai} examined in this subsection --- the conditional-like or so-called \textit{precautioning} uses, in \citeauthor{Lichtenberk1995}'s typology --- resembles that of English \textit{otherwise} (and parallels that of \textit{lest}.)
All of these observations are further spelled out in chapters \ref{bambai.prag} and \ref{bambai.semx} below. 

We turn first, however, to a description of additional ``apprehensive'' uses of \textit{bambai}.

\subsubsection[Apprehensive uses]{\textit{bambai} as a modal adverbial: the \textsc{apprehensive} use}
%\paragraph{Epistemic adverbial}
\label{ep.adv}

In contrast to the `nonimplicational' or \textsc{precautioning} (\textit{i.e. }{\sc lest}/`in case'-type) readings presented above (\S~\ref{lests}), \textit{bambai} also functions as an epistemic adverbial with apprehensional use conditions; a usage corresponding to  \citeauthor{Lichtenberk1995}'s \textit{`apprehensional-epistemic'} function and to \citeauthor{Vuillermet2018}'s \textit{apprehensive} (proper).\footnote{The first token of \textit{bambai} in (\getfullref{appr1.motika}) also represents an apprehensive use like this.} As we will see, this function of \textit{bambai} arises in monoclausal contexts in addition to within conditional constructions. Note that this distributional fact can be taken as evidence that \textit{bambai} is \textbf{not} a (syntactic) subordinator: that is, it doesn't introduce a dependent clause (unlike other purposive/apprehensional expressions cross-linguistically.)\footnote{See, \citealp[\textit{e.g.},][]{Bluhdorn2008,Cristofaro2005} for overviews of subordination.} Consider first an elaboration of (\getref{app0rp}), provided as (\getref{kofi2}) below. Here there is no explicit linguistic antecedent for \textit{bambai}, whereas its prejacent encodes an unfortunate future possibility.


\pex[everylabel=\sc]\textbf{Context:} Grant's heading to bed. Josh offers him a cuppa.\deftagex{kofi2}
\a[label=j]\begingl \gla yu wandi kofi muliri?//
\glb 2s want coffee \textsc{kinship.term}//
\glft`Did you want a coffee, \textit{muliri}?'//\endgl
\a[label=g]\begingl \gla	najing, im rait muliri! \textbf{bambai} ai kaan silip bobala! Ai mait weik ol nait... garram red ai...//
\glb no 3s okay \textsc{kinship.term} \textit{\textbf{bambai}} 1s \textsc{neg:irr} sleep poor 1s might awake all night \textsc{poss} red eye//
\glft`No it's fine \textit{muliri}! \textit{bambai} I might not sleep, I could be awake all night... be red-eyed (in the morning)...'\trailingcitation{[GT~16032017~17']}//\endgl\xe

\noindent Similarly, in the exchange in (\getref{hanting}) below, \textsc{\textbf{b}} deploys \textit{bambai} to the same effect in two single-clause utterances; each encoding an unfortunate future possibility --- namely an unsuccessful trip ($ \blacklozenge \textit{no.meat} $) in the event that the two \textit{gajin}s permit their young relative to join in.


\pex[everylabel=\sc]\textbf{Context:} Two relatives \textsc{(a, b)} are planning a hunting trip; a younger relative wants to join.\deftagex{hanting}
\a\begingl\gla im rait, yu digi im then gajin.//
\glb 3s okay 2s take 3s then \textsc{kinship}  bambai 3s \textsc{neg.irr} give \textsc{loc} 1d.\textsc{incl}//
\glft`It's fine, bring him along poison-cousin'//
\endgl
\a\begingl\gla  \textbf{Bambai} yunmi gaan faindi bip//
\glb \textbf{\textit{\textit{bambai}}} 1d\textsc{.incl} \textsc{neg.irr} find meat //
\glft`But then we may not be able to find meat'//
\endgl
\a[label=a]\begingl\gla Yunmi garra digi im//
\glb 1d\textsc{.incl} \textsc{irr} take 3s//
\glft`We'll take him'//
\endgl
\a[label=b]\begingl\gla \textbf{bambai} im gaan gibi la yunmi.//
\glb \textit{\textbf{bambai}} 3s \textsc{neg.irr} give \textsc{loc} 1s\textsc{.incl}//
\glft `But then [the country] may not provide for us.'\trailingcitation{[DW 20170712]}//\endgl
\xe


Finally, (\ref{bos}) below provides a clear example of Lichtenberk's \citeyearpar{Lichtenberk1995} ``epistemic downtoning'' function for apprehensionals. Here, \textit{bambai} clearly behaves as an epistemic possibility modal $ (\textit{bambai }q=\underset{\textsc{epist}}{\blacklozenge} q) $. In this case, where the speaker doesn't \textit{know} who's at the door, she makes a claim about how---in view of what she \textit{does} know and might expect to be happening---the (present-tensed) situation described in the prejacent is a distinct possibility (and a distinctly undesirable one at that.)



	\ex\label{bos}\begingl\glpreamble\textup{\textbf{Context:} Speaker is at home to avoid running into her boss. There's a knock at the door; she says to her sister:}//
	\gla Gardi! \textbf{Bambai} im main bos iya la det dowa rait~na//
	\glb Agh \textit{\textbf{bambai}} 3s my boss here \textsc{loc} the door right~now//
	\glft`Oh no! That could be my boss at the door.'\hfill[AJ 02052020]//\endgl
\xe



%\pex\begingl\gla Maitbi bushfaia kamin. Bambai ollot get ben, big bushfaia kamin//
%\glb Maybe a bushfire is coming. Then everything might get burned, big bushfire coming.//
%\glft`A bushfire may be coming. Then everything might get burned, a big bushfire's coming'[ESB p.c.	]//
%\xe


\noindent In these apprehensional-epistemic occurrences, \textit{bambai} has entered into the functional domain of other epistemic adverbials (notably \textit{marri\textasciitilde{maitbi}} `perhaps, maybe'.) Note that the availability of apparently epistemic readings to linguistic expressions with future-orientation is well-attested in English cross-linguistically (e.g., \textit{the bell just rang, it'll be Hanna/that's gonna be Hanna}, \citealp[see also][]{Condoravdi2003,Winans2016,Werner2006}.) \citet{Giannakidou2018}, for example, defend an analysis of that unifies future tense morphology with epistemic modality, appealing to data like the English epistemic future and its corollaries in Greek and Italian to argue that future markers in these languages in fact always encode epistemic necessity (\textit{sc.} that its \textit{epistemic modals} that perform the work of signalling predictive illocutionary force.) We will have further observations to make on these facts in the chapters that follow (ch. \ref{bambai.prag} for a discussion of pragmatic competition with \textit{marri} and ch. \ref{bambai.semx} for presentation of an analysis that unifies these uses.)

\paragraph[Counterfactual]{Apprehensive counterfactual}

The relation between the counterfactual prejacent to \textit{bambai} and the content of the preceding clause appears to diverge from the patterns of data described in the previous subsection. As with the epistemic adverb uses above, in (\getref{cfact.adv}), \textit{bambai} appears to introduce a modalised assertion and expresses negative speaker affect. Its interpretation doesn't appear to be restricted by the preceding question. Similarly to the uses shown above, \textit{bambai} appears to behave here as an apprehensive modal insofar as it encodes an unfortunate possible eventuality. Unlike the above examples, however, the prejacent (\textit{viz.} one of the Philistines committing adultery with Rebekah) is taken to describe a counterfactual event in view of Isaac's deception.

\pex\begingl\deftagex{cfact.adv}\glpreamble\textbf{Context.} Abimelek (king of the Philistines) chides Isaac for having earlier identified his wife Rebekah as his sister.//
 \gla Wotfo yu nomo bin jinggabat basdam, \textbf{bambai} ola men bina silipbat garram yu waif? Yu bina meigim loda trabul blanga melabat//
\glb why 2s \gls{neg} \gls{pst} think before, \textbf{\gls{appr}} all man \gls{pst}:\gls{irr} sleep.\gls{ipfv} with 2s wife 2s \textsc{pst:irr} make much trouble \gls{dat} 1p.\gls{excl}//
\glft`Why didn't you think [to say something] earlier? The men might have slept with your wife! You could have caused many problems for us!'\trailingcitation{[KB~Jen 26.10]}//\endgl
\xe


\paragraph[cond]{Apprehensives with \textit{if-}restrictors}\label{ifs}
Contrasting with the `nonimplicational' (\textit{i.e.}, {precautioning/\sc lest}-type) readings in \S~\ref{lests} above, Kriol also forms conditional sentences using an English-like \textit{if...(then)} construction. The two sentences in (\nextx) give examples of an indicative and subjunctive \textit{if-}conditional, where \textit{bambai} modifies the consequent clause (the ``apodosis.'') 


\pex\a\deftagex{if1}\deftaglabel{ind}\begingl %\glpreamble \textbf{and in if construction w neg consq}//
\gla if ai dringgi kofi \textbf{bambai} mi $^\#$(nomo) gurrumuk//
\glb if 1s drink coffee \textit{\textbf{bambai}} 1s $^\#$({\sc neg}) sleep//
\glft `If I drink coffee then I might not sleep'\trailingcitation{[AJ~23022017]}//
\endgl
\a\deftaglabel{sjv}\begingl\gla if ai\textdblhyphen{}ni\textdblhyphen{}min-a dringgi det kofi \textbf{bambai} ai($^\#$\textdblhyphen{}ni)\textdblhyphen{}bin-a gurrumuk jeya//
\glb if 1s{\sc\textdblhyphen{}neg\textdblhyphen{}pst-irr} drink the coffee \textit{\textbf{bambai}} 1s{\sc($^\#$\textdblhyphen{}neg)\textdblhyphen{}pst-irr} be.asleep there//
\glft\textbf{Intended:} `If I hadn't drunk coffee then I may well have fallen asleep there'\\(This reading is available if \textit{\textdblhyphen no(m)o} `\textsc{neg}' is omitted) \trailingcitation{[GT~16032017]}//
\endgl\xe




The contrast between (\getfullref{if1.ind},\getref{if1.sjv}) and their \textit{if-}less counterparts in (\getfullref{app0rp.kofi} and \getfullref{sjv.A}) respectively (\textit{pp.} \getref{app0rppage}-\getref{sjvApg}), evinces some restriction that \textit{if}-clauses apparently force on the interpretation of \textit{bambai}. Whereas the \textit{if}-less sentences presented previously assert that a particular eventuality may obtain/have obtained just in case the antecedent predicate \textbf{fails}/failed to instantiate (\textit{i.e.}, the {\sc lest} readings), the sentences in (\getref{if1}) diverge sharply from this interpretation. That is, each of the \textit{if $ p $, bambai $ q $} sentences in (\getref{if1}) asserts a straightforward conditional $p\to\blacklozenge q$: should the antecedent proposition hold (have held), then $q$ may (have) obtain(ed). 

In this respect, \textit{bambai} appears to be behaving truth conditionally as a modal expression encoding possibility --- \textit{sc.} a modal adverbial --- similarly to the monoclausal uses presented above in this subsection. The \textsc{modal base} (\textit{i.e.}, those worlds over which bambai quantifies) is explicitly restricted by the (syntactically subordinate) \textit{if}-clause, whose sole function can be taken to involve the restriction of a domain of quantification \citep[cf.][]{Kratzer1979,Lewis1975,VonFintel1994,Roberts1989,Roberts1995}. Additional argumentation to this effect is included in ch. \ref{bambai.prag}.



\subsection{Summary}

In the preceding sections, we have seen clear evidence that \textit{bambai} has a number of distinct readings. Nevertheless, we can draw a series of descriptive generalisations about the linguistic contexts in which these readings emerge. These are summarised in (\nextx).%able (\ref{readingstable}).

%	\begin{table}[h]
%	\caption{
		\pex[everylabel=\bf\it]\textbf{Semantic conditions licensing readings of \textit{bambai}.}
			\a	\textit{bambai} is interpreted as a \textbf{subsequential temporal frame} when the state-of-affairs being spoken about is \textbf{settled}/the same as the actual world $(w'\approx_{t*} w*)$  (\textit{i.e.}, in \textbf{factual}, \textbf{nonfuture}  contexts).
			
			Consequently, \textit{bambai}'s prejacent generally contains past marking \textit{(bin)} in subsequential contexts
				
			\a	In other (\textbf{nonfactual/future}) contexts (that is, in predications that fail to satisfy \textsc{settledness}) apprehensional readings ``emerge''. 
				
			\a	In apprehensional contexts, precautioning (\textsc{lest}-type) readings occur in a $p \textit{ bambai } q$ construction. That is, in a sentence of the form $p \textit{ bambai } q$ is interpreted as an admonition that $ \boldsymbol\neg p\to\blacklozenge q $ \xe%}\label{readingstable}\centering
%	\begin{tabular}{llrccc}\toprule
%		&  \denote[w*]{\textit{ bambai } q} && $w^\prime\simeq w*$ & $p,\textit{ bambai }q$      \\\midrule\midrule
%		\multicolumn{2}{c}{\textsc{subseq}}  &       & \checkmark      & \xmark        \\\midrule
%		\multirow{2}{*}{\textsc{appr}} & {\footnotesize\textit{LEST}} &$\neg p\to\blacklozenge q$ & \xmark           & \checkmark     \\
%		& \textsc{epist} &$(p\to)\blacklozenge q$& \xmark            &\xmark\\\bottomrule
%	\end{tabular}
	
%\end{table}


As discussed in the preceding sections, \textbf{nonfactual} utterances are those in which  (a) a predicate is understood to obtain in the future of evaluation time $t*$/\textbf{\textit{now}} or (b) the predicate is understood as describing some $w^\prime$ which is not a historic alternative to the evaluation world $w*$. It is in exactly these contexts that \textit{bambai} gives rise to a modalised reading. In Kriol, a number of linguistic operators (which we have seen in the data presented above) appear to ``trigger'' predication into an unsettled timeline. A selection of these is summarised in Table \ref{triggers} below.\footnote{This is not intended to suggest that these operators are in any way semantic primitives, Table \ref{triggers} is to be read as a non-exhaustive list of linguistic devices that appear to associate with nonfactual mood.}

\begin{table}[H]\caption[Semantic operators]{Semantic operators co-occurring with modal (apprehensional) readings of \textit{bambai}}\label{triggers}\centering\small
	\begin{tabular}{lll}\toprule
		\textsc{\textbf{Gloss}} & \textbf{Form} & \textbf{\textit{Example}} \\\midrule\midrule
		\textsc{irrealis} & \textit{garra} &\specialcell[l@]{\textit{ai\textbf{rra} dringgi kofi \textbf{bambai} mi gurrumuk}\\`I'll have a coffee or I might fall asleep'}\\\midrule
		
		\textsc{\gls{neg} irrealis} & \textit{kaan} &\specialcell[l@]{\textit{ai \textbf{kaan} dringgi kofi \textbf{bambai} mi nomo silip}\\`I won't have a coffee or I mightn't sleep'}\\\midrule
		
		\textsc{c'factual} & $\underset{\textsc{pst:irr}}{\textit{bina}}$ &\specialcell[l@]{\textit{aibin dringgi kofi nairram \textbf{bambai} ai\textbf{bina} gurrumuk}\\`I had a coffee last night or I might've passed out'}\\\midrule
		
		\textsc{imperative} & $\varnothing$ & \specialcell[l@]{\textit{yumo jidan wanpleis \textbf{bambai} mela nogud\footnotemark}\\`Youse sit still or we might get cross'}\\\midrule
		\textsc{prohibitive} &  $\underset{\textsc{impr}}{\varnothing}$[\textit{nomo}] & \specialcell[l@]{\textit{\textbf{nomo} krosim det riba, \textbf{bambai} yu flodawei}\\`Don't cross the river or you could be swept away!'}\\\midrule
		
		\textsc{generic} & $\varnothing$ &\specialcell{\textit{im gud ba stap wen yu confyus, \textbf{bambai} yu ardim yu hed}\\`It's best to stop when you're confused; you could get a headache'}\\\midrule
		
		\textsc{\gls{neg} generic} & $\underset{\textsc{gen}}{\varnothing}$[\textit{nomo}] & \specialcell{\textit{ai \textbf{nomo} dringgi kofi enimo, \textbf{bambai} mi fil nogud}\\`I don't drink coffee anymore or I'd feel unwell'}\\\midrule\midrule
		{\sc conditional}&\textit{if}&\specialcell[]{\textit{\textbf{if} ai dringgi kofi, \textbf{bambai} ai kaan silip}\\`If I have coffee, then I mightn't sleep'}\\\bottomrule
		
		
	\end{tabular}
\end{table}
\footnotetext{This example due to Dickson (\citeyear[168]{Dickson2015} [KM~20130508]).}



\chapter{An apprehensional pragmatics}\label{bambai.prag}



Chapter \ref{bambai.desc} provided a detailed account of the distribution of the Kriol adverb \textit{bambai}, the numerous syntactic environments in which it surfaces and the numerous interpretations that it appears to license. The current chapter proposes a way of understanding the synchronic relationship that holds between these different uses and readings of \textit{bambai}, crucially interrogating the relationship between clauses of the type \textit{bambai $ q $} and the context in which they're embedded/their ``matrix discourse'' (\S~\ref{bambai.subord}).

In developing this understanding of the crucial role of context in the interpretation of \textit{bambai}, \S~\ref{bambai.dia} proposes an account of the diachronic emergence of apprehensional expressions from temporal frame adverbials (\textit{sc.} devices that encode \textsc{subsequentiality}.) Deploying insights from the diachronic semantics literature, we will see that this apparent meaning change arises from the conventionalisation of a (subtype) of \textit{post hoc ergo propter hoc}-type conversational implicatures.



In contemporary Roper Kriol --- due to the developments described in this chapter (and the distribution described in ch. \ref{bambai.desc}) --- \textit{bambai}, the erstwhile TFA, can be shown to function as a modal adverb. Consequently, it has entered into the functional domain of other possibility adverbials, notably \textit{marri} `perhaps.' Incidentally, the competition between \textit{marri} and apprehensive \textit{bambai} provides a frame to investigate the attitudinal component of apprehensionality, the key distinguishing feature of this category. \S~\ref{bambai.expr} compares Kriol data with that of other apprehensionals and proposes a treatment of the ``undesirability'' component of apprehensional meaning as \textit{use-conditional} or \textit{expressive} content.


%We have seen throughout that \textit{bambai} can give rise to readings of implicational relations between the two propositions. §\ref{modSems} defended an analysis of \textit{bambai} that claims that the TFA and modal uses emerge follow from reasoning about the speaker's information state with respect to the realisation (\textit{sc}. settledness) of the predicate it modifies. The following discussion sketches a way to reconcile these observations.\marginnote{in the diaichronic aection talk about romaine 1995' claims about Pac. Jargon, app appr uses}


\section{A modal subordination account}\label{bambai.subord}


The first examples presented in Chapter \ref{bambai.desc} are repeated below in (\nextx):

\pex\deftagex{minpair}\textbf{Context:} \textup{I've invited a friend around to join for dinner. They reply:}
\a\deftaglabel{seq}\begingl\glpreamble\textsc{Subsequential} reading of \textit{bambai}//
\gla yuwai! \textbf{bambai} ai gaman jeya!//
\glb yes! \textbf{\textit{bambai}} 1s come there//
\glft ‘Yeah! I’ll be right there!’//\endgl

\a\begingl\glpreamble \deftaglabel{appr}\textsc{Apprehensional} reading of \textit{bambai}//
\gla najing, im rait! \textbf{bambai} ai gaan binijim main wek!//
\glb no 3s okay \textbf{\textit{bambai}} 1s {\sc neg.mod} finish 1s work//
\glft ‘No, that’s okay! (If I did,) I mightn’t (be able to) finish my work!'\trailingcitation{[GT~20170316]}//\endgl
\xe

As we have seen, an important way in which the range of uses of \textit{bambai} are united is in the fact that they appear to modify the proposition that they precede (the \textsc{prejacent}), crucially relating it to some component of the discourse context. For clarity, paraphrases and schemata for (\getfullref{minpair.seq}-\getref{minpair.appr}) are provided below. 



\pex[exno=\getref{minpair}]
\a[label=a′]  The prejacent (that the subject comes to dinner) is taken to hold at $ i_e $, \textsc{subsequently} to (\textit{i.e.}, in the near future of) some contextually-specified reference time ($ i_r $= speech time $ i* $ in this case.)\deftaglabel{sec.para}

{\centering
\begin{tikzpicture}[grow=right]\large
	\tikzset{level distance=100pt,sibling distance=18pt}
	\tikzset{execute at begin node=\strut}
	\Tree [.$i_r$ [.$\boldsymbol{i_e}$ \edge[dashed]; $\succ$ ] ]
\end{tikzpicture}
}

\a[label=b′] In (\getfullref{minpair.appr}), the prejacent (the subject's failure to complete his work) is taken to represent a possible outcome (\textit{e.g.}, at $ i_e $) of (the negation of) some contextually-supplied proposition (\textit{e.g.}, the subject's not declining their addressee's dinner invitation at $ i_r $.)\deftaglabel{appr.parar}

{\centering\begin{tikzpicture}\tikzset{level distance=100pt,sibling distance=5pt,edge from parent/.append style={dashed},grow=right}
{\large\Tree [.$i$ [.$i*$ [.$i_3$ ] [.$i_2$ ] ] [.${i_r}$ [.$i_1$ ] [.$\boldsymbol{i_e}$ ] ] ]}\end{tikzpicture}}


\xe






 Craige \citet[663]{Roberts1995} draws an explicit connection between the retrieval of a ``Reichenbachian reference time'' and the retrieval of a reference ``situation'', both of which she identifies as ``species of domain restriction on an operator'' (over intervals/possible worlds respectively.) She therefore analogises the logical structure of temporal and modal (incl. conditional) operators to other types of quantifiers (\ref{rob-domrest}).
 
 \pex\label{rob-domrest} The logical structure of quantificational expressions in natural language\\ \texttt{[Operator,~Restriction, Nuclear Scope]} following \citet[665]{Roberts1995}\footnote{This terminology likely due in part to \citeauthor{Heim} (1982: \textit{e.g.} 89) although the idea of quantifiers as second-order relations appears to stem from Aristotle's syllogistic logic \citep[see][]{Westerstahl2019}.}
 	  $$ \lambda \mathit{Q}[\textsc{Operator}\ \mathcal R\ \mathit{Q}]$$
 	  $ \mathit{Q} $ represents the nuclear scope of some quantificational \textsc{Operator}. The first argument $ \mathcal R$ represents a ``restrictor clause'' -- a free variable that is furnished by context and restricts the domain of the quantificational operator.
\xe



We have clear evidence, then, that the interpretation of \textit{bambai} is constrained by and dependent on elements of the foregoing discourse that, crucially, \textbf{need not be linguistically explicit}/overt. The phenomenon of interest is that of \textit{discourse anaphora} and the observation that particular linguistic expressions (incl. lexical items) ``specify entities in an evolving model of discourse'' \citep[see][]{Webber1988}. The uses of \textit{bambai} in \ref{minpair} exhibit this property: this lexical item apparently an intensional operator whose domain is restricted by entities (prima facie of different types) in its \textsc{subsequential} (temporal entities?) and \textsc{apprehensional} uses (eventive? propositional entities?)


In order to account for these types of anaphor phenomena (particularly in the modal domain), \citet{Roberts1989,Roberts1990,Roberts2020} develops the notion of \textsc{modal subordination}, defined in (\ref{modsub-def}):

\pex\label{modsub-def}\textsc{modal subordination} is a phenomenon wherein the interpretation of a clause $ \alpha $ is taken to involve a modal operator whose force is relativized to some set $ \beta $ of contextually given propositions.\hfill\citep[718]{Roberts1989}


\xe





%Alluded to in the previous chapter, a fruitful way of conceiving of conditionals is as a type of modality, where the quantificational domain of the modal is explictly restricted. This is achieved by intersecting a (contextually-retrieved) modal base with a proposition (\textit{viz.} that proposition denoted by the conditional antecedent) (\citealp{VonFintel1994,Kratzer2012}).


In \textit{bambai}'s `\textsc{avertive'}-type uses (\textit{sc.} those of the form \textit{$ p $ bambai $ q $}, described in \S~\ref{lest}), \textit{bambai $ q $} often functions to introduce an eventuality which is interpreted as a possible consequence of the antecedent subject's failure to attend to some situation which is described in the antecedent clause --- what we had above represented as $ \neg p(w)\to\blacklozenge q(w)$. In other words, these uses of \textit{bambai} have usually been translated as, and strongly resemble, uses of the English adverb \textit{otherwise} (albeit with possible differences in modal force and the conventionalised expressive (apprehensional) content described in \S \ref{bambai.expr}.) \citet{PhilKotek} provide an account of the interpretation (and meaning contribution) of utterances of the form $ p  $ \textit{otherwise} $ q $, where \textit{otherwise} is analysed as a discourse anaphor that triggers modal subordination. In the subsections below, their (our) analysis of \textit{otherwise} as (1) invoking modal subordination and (2) sensitive to information structure is adapted to account for analogous components of the behaviour of \textit{bambai}.


\subsection{Accommodation and restriction}\label{modsub}

As introduced above (and informally defined in (\lastx)), the notion of \textsc{modal subordination} captures the idea that a modal operator scoping over a clause has visibility of elements of the foregoing discourse.\footnote{Much of the content of this subsection draws on the presentation of a similar analysis for \textit{otherwise} in \citet{PhilKotek2018,PhilKotek}, available at \href{https://ling.auf.net/lingbuzz/004800}{\texttt{lingbuzz/004800}}. The arguments in this analysis are summarised and modified in view of accounting for \textit{bambai}'s different properties. The introduction to Discourse Representation Theory and modal subordination are particularly close to the text in \citeyear[§4]{PhilKotek}.} Roberts's schematisation of this type of relation is reproduced in (\getref{modsub-schem}) and a classic operationalisation is given in (\getref{wolf}).

\pex[labeltype=numeric,everylabel=\small] The general logical form of a modal subordination relation --- given two (syntactically independent) clauses $ \mathit{K_1,K_2} $ --- where the prejacent to a modal operator (\textsc{mod}$ _2 $) is ``modally subordinate'' to the content in the scope of \textsc{op}$ _1 $, another (intensional) operator \citep{Roberts2020}.
\deftagex{modsub-schem} $$ \big[_{\mathit{K_1}}\hdots \textsc{op}_1[\hdots \mathtt{X}\hdots]\hdots\big]\hdots\big[_{\mathit{K_2}}\hdots\textsc{mod}_2[\hdots\mathtt{Y}\hdots]\hdots\big] $$
\a\texttt{Y} is a presupposition trigger and only the content \texttt{X} (under the scope of~\textsc{op}$ _1 $) would satisfy this presupposition.
\a \textsc{mod}$ _{2} $ is a modal operator scoping over \texttt{Y}.
\a The constituent in $\mathit{K_2} $, headed by \textsc{mod}$ _2 $, has an interpretation wherein part of its restriction consists of $ \mathtt{X} $.
\xe


\pex[everylabel=\bf\sc] An example of modal subordination in discourse.\trailingcitation{\citep[1]{Roberts2020}}\\
\textbf{\textsc{context.}} Hansel \& Gretel are arguing about whether to lock the door.\deftagex{wolf}

\a[label=g] \begingl\gla A wolf \textbf{might} come in. It \textbf{would/will} eat you first!//
%\glb ~ ~ $ \lozenge $ ~ ~ ~ $ \square $//
\glft $ \boldsymbol{\underset{\,\textsc{op}_1}{\lozenge}}\exists x\big[\text{Wolf}(x)\wedge\text{Come.in}(x)\big]\ \&\ \boldsymbol{\underset{\,\textsc{mod}_2}{\square}}\ \text{Eat.you}(y) $//\endgl
\xe


\noindent This schema is straightforwardly reflected in Gretel's two sentence utterance in (\getref{wolf}) below, where crucially:\begin{itemize} \item the domain of \textsc{mod}$ _2 $ is somehow restricted to those worlds in which `a wolf come[s] in' (\textit{sc.} the proposition in the scope of $ \mathit{K_1} $'s possibility modal---\textsc{op}$ _{1} $) and \item the presuppositions associated with the pronoun \textit{it} in $ \mathit{K_2} $ are satisfied by the (hypothetical) wolf bound, existentially bound in $ \mathit{K_1} $ (\textit{i.e.}, $ y=x $).\end{itemize}

\noindent That is, in (\getref{wolf}), $ \mathit{K_2} $ is \textbf{modally subordinate} to $ \mathit{K_1} $ (and material in $ \mathit{K_1} $ is consequently accessible to $ \mathit{K_2} $.) According to \citet{PhilKotek}, the English adverb \textit{otherwise} is a discourse anaphor and sentences containing this lexical item are taken to rely on a similar logic. Given that the \textsc{avertive} uses of \textit{bambai} are taken to have a similar meaning contribution to \textit{otherwise}, pertinent details of \citet{PhilKotek}'s analysis are adapted here (which in themselves are an implementation of Craige Roberts's extended DRL for modal subordination.) An overview of the basic assumptions of this version of Discourse Representation Theory (DRT) are given in \S~\ref{sec-MDRL}, which are then used to model the contribution of \textit{bambai} in the subsequent sections.


\subsubsection{A modal discourse representation language}\label{sec-MDRL}

Discourse Representation Theory \citetext{originating simultaneously with \citealt{Kamp1981} and the related system of \citealt{Heim}} is a framework for modelling the development of participants' ``mental representations'' of a given situation as a discourse unfolds \citep[see][]{Geurts}.\footnote{While these frameworks are often described as empirically equivalent, Heim's \textit{File Change Semantics} differs crucially insofar as it denies or makes no claim about mental representation and or the ``procedural aspects'' of interpretation \citetext{\citealp[102]{Kamp1988}, this property also addressed in \citealp[\S~6]{Geurts}.} Nothing in the current work hinges on commitment to a particular dynamic semantics/pragmatic framework.} 
Because it models the accretion of information over the course of a discourse, \textsc{discourse representations} --- effectively ``pictures of the world [$ \approx $ partial models] described by sentences that determine them'' --- are the basic meaning-bearing units in a discourse, mediating between syntactic units (\textit{i.e.}, sentences) and the determination of truth.


For a given \textsc{Discourse Representation Structure} (DRS) $ \boldsymbol K $, $ K $ denotes a pair $ \langle X_K,C_K\rangle $, where $ X $ represents a \textit{local domain} -- a finite set of variables that represent discourse objects relevant in the context (including participants, eventualities, and times etc.); and $ C $ is a finite set of `satisfaction conditions' that eventually determine the truth value of a given proposition. For diagrams where a DRS $ K $ is represented as a box, the top of the box lists the variables $ X_K $ and the bottom represents the satisfaction conditions $ C_K $. 

For a simple discourse as in (\nextx), we provide a DRS below. Notice that the indefinite is treated as a variable here, and is eventually existentially closed \citep{Heim}: any variable that is not locally bound by another operator is assumed to be existentially bound by a global operator that applies to variables that remain free by the end of the derivation. DRT allows us to model continued reference to a variable introduced earlier in a discourse as long as it is still accessible. The first sentence of \getref{quack} introduces a discourse referent and condition set, represented as (a), expanded in the second (b).\footnote{These representations are somewhat abbreviated in subsequent diagrams. See \citet{Kamp1993} for further detail.}


	\pex A duck entered the room. It quacked.\deftagex{quack}\\
		a. \drs{x}{duck(x)\\entered-room(x)} \hspace*{1in}	b. \drs{x\quad y}{duck(x)\\entered-room(x)\\ x=y\\quacked(y)}

	\xe
	

\vspace{1em}

A given DRS $ K $ contains atomic conditions of the form $ P(x_{i_1}...x_{i_n}) $ (where $ P $ is an $ n $-place predicate). In a given model $ \mathcal M $, if a world/variable-assignment pair $ \langle w,f\rangle $ \textbf{satisfies} $ (\underset{\scriptscriptstyle{\mathcal M}}{\boldsymbol{\Vdash}}) $ all of the conditions in $ K $, then that pair \textbf{verifies} $(\underset{\scriptscriptstyle{\mathcal M}}{\boldsymbol{\vDash}}) \;K$. Additionally, DRSs are recursively closed under the operations $ \neg,\bigvee,\Rightarrow,\square,\lozenge $. That is, if $ K_i,K_j $ are DRSs and $ \circ $ is one of these (2-place) operators, then $ K_i\circ K_j $ can represent a \textit{complex condition} in $ K $. This complex condition needs to be satisfied by $ w $, if $ K $ is to be verified in $ w $.\footnote{The semantics and interpretation of these operators is further discussed below, though \citet[714]{Roberts1989} provides formal satisfaction conditions for all condition types that she defines. See also the appendix to this paper for some additional detail.} (\ref{donkey}) is an example containing a possibility modal, illustrating that the variable $x$, which is introduced in the box to the left of the operator, remains accessible in the box on the right: 


\ex \label{donkey} \label{drs-if} If a duck is hungry, Hanna may feed it. \\
\drs{~}{
	\drs{x}{duck(x)\\hungry(x)}~{\Large$\lozenge$}~\drs{y}{Hanna(y)\\
		feed(y, x)}
}\xe

Crucial to the theory is the notion of an ``accessible domain'' $ A_{K_i} $ -- a superset of the local domain $ (X_{K_i}) $ for any given $ K_i $. As a discourse proceeds, the set of objects that can be referred to expands. The notion of `accessibility', then, allows us to predict which objects can be referred to at a given stage in a discourse. 

\pex The accessible domain $ A_{K_i} $ contains all the variables that occur: 
\a In $ K_i $'s local domain $ (X_{K_i}) $
\a In the domains of all DRSs that graphically \textit{contain} $ K_i $
\a If $ K_i $ is the right element of a (binary) modal condition $(\Rightarrow,\square,\lozenge)$, $ A_K $ also contains all the elements of the antecedent's (the DRS on the left's) local domain.\\\textit{I.e.} $ K_\ell\,\square\,K_i\, \longrightarrow K_\ell\leqslant K_i$ where `$ \leqslant $' reads ``is accessible from.''\xe

In (\blastx), the consequent box of the conditional makes reference to a variable introduced in the antecedent. Furthermore, the entire conditional statement is embedded inside a larger discourse, so that we are not committed to the existence of any dog in the context: the \textit{feeding}-worlds are a subset of \textit{hungry-dog}-worlds.

Based on the assumptions introduced in (\lastx), a given DRS $ K $ that is interpreted in the scope of a modal operator can be \textit{modally subordinate} to those DRSs whose domains it has access to. Example (\ref{john}) illustrates such a case, from \citet[701]{Roberts1989}. Here, the consequent clause is \textit{modally subordinate} to the antecedent \textit{in a given conversational background}. That is, the entire conditional is taken to assert that the speaker predicts that `John will be at home reading a book' in those worlds \textit{(that best conform with the speaker's expectations)} in which he bought a book. Like in (\lastx), we need not be committed to the fact that John bought a book in the actual world; in other words, the entire statement is not a part of the matrix DRS $ K $; it is further embedded.


%\needspace{3\baselineskip}
\ex \textit{A \textsc{drs} illustration of modal subordination in a conditional sentence:} \label{john}\\
If John bought a book, he'll be at home reading it by now.\\
\hspace*{.25\textwidth}$\scriptstyle K $\\
\drs{}
{\hspace{1em}{\footnotesize$ K_i $\hspace{10em}$ K_j $}\\		\drs{x y}
	{
		john(x)\\
		book(y)\\
		bought(x,y)
	}
	~$ \boldsymbol{\LARGE\square} $		 \drs{~}{reading(x,y)}
}\xe

\noindent In (\ref{john}), the DRS representing the consequent clause $ (K_j) $ is \textit{modally subordinate} to its antecedent $ K_i $ and, as a result, can access the discourse entities introduced in $ K_i $ (i.e. $ K_i\leqslant K_j $). Moreover, both $K_i$ and $K_j$ are subordinate to the matrix DRS $K$ (i.e. $ K\leqslant K_i\leqslant K_j $); had any variables been introduced in $K$, they would have been accessible to both $K_i$ and $K_j$. 

\subsubsection[Discourse representation of the precautioning construction]{\textit{\textbf{p} bambai \textbf{q}} and discourse representation}



On the basis of this framework, we can propose an account for the apparent clause-linking (avertive/precautioning) uses of \textit{bambai}, representing each clause as a discourse representation structure (\textsc{drs}) --- \textit{sc.} $ \mathit{K_1} \textit{ bambai } \mathit{K_2} $. On the basis of the description given in chapter \ref{bambai.desc}, (\nextx) enumerates some key properties of these uses.

\pex[labelformat=\bf\it A]\deftagex{precaut-conds}\textbf{In sentences of the form $ \boldsymbol{\mathit{K_1} \textbf{ bambai }\mathit{K_2} } $ :}
\a \textit{bambai} functions as an intensional operator encoding a type of conditional modality; it asserts that -- in a set of worlds (according to some criterion), some condition holds $ (q) $. 
\a The (modal) domain of \textit{bambai} is restricted to some nonfactal proposition derived from $ \mathit{K_1}  $: that is, the \textbf{negation} of a ``basic proposition'' (which may be in the scope of another other modal operator.)\footnote{Operationalised in the discussion of (\getref{modsub}) below, where some sentence $ \mathit{K_1} $ is of the form \textsc{op}$_1 \varphi $ (\textit{i.e.}, headed by a modal operator), the corresponding \textit{basic proposition} (prejacent) is $ \varphi $.} 
\a The speaker asserts $ \mathit{K_1} $.% is asserted.% as true in the evaluation world.% We believe that these properties lend themselves to a dynamic account; one concerned with the development of participants' information states across the discourse.%, which naturally explains the non-commutative aspect of \textit{otherwise}. % (although, as discussed, the process we called ``modal weakening'' allows for consideration of what happens if it were not.)
\xe


\noindent For clarity, the three sentences in (\getref{modsub}) illustrate these interpretation conventions for precautioning uses of \textit{bambai} and different relations between the syntactic antecedent $ \mathit{K_1} $ and the prejacent to \textit{bambai} $ \mathit{K_2} $, recalling (\getref{modsub-schem}), the modal subordination schema from \citet{Roberts2020}.

\pex[interpartskip=2ex,nopreamble=false]\deftagex{modsub}\textbf{Modal subordination with \textit{bambai}}
\a\begingl\glpreamble The negation of $ \mathit{K_1} $ restricts the domain of \textit{bambai}\deftaglabel{kofi.cfact}//
\gla \nogloss{{[$ _\mathit{{K_1}} $}} ai\textasciitilde{}bin dringgi kofi nairram \nogloss{]} \textbf{bambai} ai \textbf{bina} silip\textasciitilde{}silip-bat la wek//
\glb 1s\textdblhyphen{\sc pst} drink coffee night \textit{bambai} 1s {\sc pst:irr} sleep\textasciitilde\gls{ipfv} {\sc loc} work//
\glft`I drank coffee last night otherwise I would have fallen asleep at work'\\
$ \approx $ `If I hadn't had coffee, I might've fallen asleep'\trailingcitation[AJ 23022017]//
\endgl
%\glpreamble\textbf{Context:}  It's noon and I have six hours of work after this phonecall. I tell my colleague://
\a\begingl\glpreamble The negation of the proposition in the scope of \textit{garra} `must, will' restricts the domain of \textit{bambai}\deftaglabel{kofi.pot}//
\gla \nogloss{{[$ _\mathit{{K_1}} $}} ai\textbf{\textdblhyphen{}rra} dringgi kofi  \nogloss{]} \textbf{bambai} mi gurrumuk la desk iya gin//
\glb 1s\textdblhyphen{\sc irr} drink coffee \textit{bambai} 1s fall.asleep {\sc loc} desk here {\sc emph}//
\glft `I'll/ought to have a coffee; otherwise I might pass out right here on the desk'\trailingcitation{[GT~28052016]}\\
$ \approx $ `If I don't have coffee, I might fall asleep'\\
$ \not\approx\ ^\#\!{} $ `If I need not have a coffee, I might fall asleep'//\endgl

\a\begingl\glpreamble \textit{kaan} $ \varphi $ `won't/can't/mustn't $ \upvarphi $' has the logical form $ \square\big[\neg[\varphi]\big] $. The negation of the proposition in the scope of $ \square $ restricts the modal.\deftaglabel{shop}//
 \gla \nogloss{{[$ _\mathit{{K_1}} $}} yu \textbf{kaan} gu la shop  \nogloss{]}  \textbf{bambai} yu spendim yu manima//
\glb 3s \gls{irr}.\gls{neg} go \gls{loc} shop \textsl{bambai} 2s spend 2s money//
\glft `You mustn't go to the shop; (otherwise) you could end up spending all your money.'\trailingcitation{[AJ~23022017]}\\
$ \approx $ `If you don't not go to the shop, you might spend all your money.'\\
$ \not\approx\ ^\#\!{} $ `If it's not the case that you mustn't  go to the shop...'//
\endgl
\a\begingl\deftaglabel{konfyus}\glpreamble The negation of the (generic) complement of a propositional attitude// \textit{\textsc{bi} gud} `be good to' restricts the domain of \textit{bambai}//
\gla \nogloss{[$ _\mathit{K_1} $} im gud ba stap wen yu konfyus \nogloss] \textbf{bambai} yu ardim yu hed//
\glb 3s good {\sc purp} stop when 2s confused \textsl {bambai} 2s hurt your head//
\glft`It's best to stop when you're confused, (otherwise) you'll get a headache!'\\
$ \approx $If you don't stop when you're confused, you might get a headache!.'\\
$ \not\approx\ ^\#\!{}$If it's not best to stop when you're confused, then you might get a headache!.'//%\trailingcitation{[AJ~\textcolor{red}{find elicitation!}]}//
\endgl



\xe




As the infelicitous paraphrases in (\getfullref{modsub.kofi.pot}-\getref{modsub.konfyus}) make clear, $ \mathit{K_1}\textit{ bambai }\mathit{K_2}$ doesn't have a straightforward conditional semantics. It is \textbf{not} the negation of $ \mathit{K_1} $, but rather material under the scope of some modal (or otherwise intensional) operator within $ \mathit{K_1} $ (\textit{viz.} \textsc{op$ _1 $}) whose negation ends up being accommodated.



Again, following the analysis laid out in \citet{PhilKotek}, the possible sets of propositions that are available to constrain the interpretation of \textit{``bambai $ K_2 $''} are calculated on the basis of those discourse representations which \textbf{have access to} (\textit{i.e.}, are contained within) the pronounced antecedent to \textit{otherwise}, which will refer to throughout as $ \boldsymbol{K_1} $.
A new operator over DRSs $ \boldsymbol\circleddash$ (and hence the complex condition $ K_i\circleddash K_j$) will represent the (truth-conditional) contribution of \textit{bambai}:%\footnote{Again, a formal treatment of this proposal (\textit{sc.} an extension of the DRL to include conditions of the type $ K_i\boldsymbol\ominus K_j $) is spelled out in the appendix.}

\ex \label{bb-complex} \textit{Proposal: A dynamic semantics for \emph{bambai}}\\
$ K_i\circleddash K_j\iff (K_i) \wedge (\neg K_{i_{\text{sub}}}\,\lozenge\,K_j) $\nobreak\\
\uline{In words}: $ K_{i}\circleddash K_j$ is satisfiable iff both $ C_{K_i} $ and $ (\neg K_{i_{\text{sub}}}\lozenge K_j)$ are satisfiable, where $K_{i_{\text{sub}}}$ is some DRS that is contained within $K_{i}$.\footnote{More precisely, these  conditions will be satisfied by the same set of world-assignment pairs $ \langle w,g\rangle $. See below for more discussion of the determination of $K_{i_{\text{sub}}} $.\label{caveat-type}}\xe

\noindent This proposal can be paraphrased as the claim that: ``the conditions of $K_i$ hold; however, in case (some of) these conditions --- those of $K_{i_{\text{sub}}}$ --- do not hold, the conditions in $K_j$ may then hold.'' Notice that this treatment takes precautioning apprehensionals to be akin in their (logical) structure to a conditional as 

Notice additionally that we employ the possibility operator $ (\lozenge) $ from Roberts' DRL \citeyearpar[695, 715]{Roberts1989}, building on the observation throughout that apprehensionals (incl. \textit{bambai}) involve a modal (possibility) component. A primary contribution of \citealt{Roberts1989} is an expansion of the ontology of the discourse representation theory of \citealt{Kamp1981} to include possible worlds, in view of modeling modality. In effect, $ \lozenge$ is an existential quantifier which also builds in ``conversational backgrounds''---sets of propositions: a modal base $ m $ and ordering source $ o $---in order to capture the observations made by \citet[\S2.7]{Kratzer1981} regarding different ``flavors'' of modality. 

A complex condition of the form $K_i\,\underset{m,o}{\lozenge}\,K_j$  then, is satisfiable iff $ K_j $ can be verified in some worlds in the conversational background (as determined by $ m,o $) in which $ K_i $ can be verified. Consequently a DRS containing the condition $ K_i\underset{m,o}{\lozenge}K_j $ can be instructively rewritten as in (\nextx):\footnote{See Chapter \ref{IntroCh} for a definition of \textsc{best} and a brief overview of ordering semantics.}$ ^, $\footnote{\citet{Roberts1989} in fact equivalently defines the satisfaction conditions for `possibility (in view of)' $ K_i\underset{m,o}{\lozenge}K_j$ as the dual of `necessity (in view of)' $ \neg(K_i\underset{m,o}{\square}\neg K_j)$. Relevant adjustments are made here. Mentioned in the previous section, satisfaction (verification) is a property that holds between a 4-tuple: a model, world, assignment and set of conditions (DRS). This is simplified here for perspicuity.}

\ex\deftagex{poss-sat}\emph{Satisfaction conditions for Roberts' possibility operator $ \lozenge $ as an existential quantifier, given a world $ w $:\\} 
$K_i\,\underset{m,o}{\lozenge}\,K_j \Longleftrightarrow\exists w'\big[w'\in\underset{o(w)}{\textsc{best}}\big(\bigcap[m(w)\cup\{w''\vDash K_i\}]\big) \wedge w'\vDash K_j\big]$\\[6pt]
\underline{In words}: The condition $K_i\,\underset{m,o}{\lozenge}\,K_j$ is satisfied in \textit{w} if there's some world $ w' $ in the ``best worlds'' (according to $ o $) within $ m $ and verifying $ K_i $ which also satisfies the conditions of $ K_j $.\xe

%Parsing out the components of (\getref{poss-sat}), note that each of the three expressions on the right-hand side can be understood as representing a set of worlds which verify a given condition set: (a) $w'$ is among the `best' worlds according to some contextually-determined criteria, (b) the conditions in $K_i$ are satisfied in $w'$, and (c) the conditions in $K_j$ are satisfied in $w'$. 
\subsubsection{Modal subordination in action}

Described above, the second (\textit{bambai}) clause of (\getfullref{modsub.kofi.pot}) is interpreted as \textit{modally subordinate} to antecedent material. Following the discussion of the previous subsection, its discourse representation structure can be diagrammed as in (\nextx). In (a), $ K_1 $ is asserted. In (b), the content \ul{in the scope of \textsc{op}$ _1 $} (\textit{viz.} $ \mathit{K_{1_{\text{sub}}}} $) is accommodated; its negation restricts the domain of the possibility modal encoded in \textit{bambai}.

\pex[labeltype=numeric,labelformat=$\mathit K_A.$]{} Discourse representation structure for (\getfullref{modsub.kofi.pot})\\
\begingl\gla{} \textup{[$ _\mathit{K_1} $}airra dringgi kofi \textup{]} bambai mi gurrumuk//
\glft`I'll have a coffee, otherwise I may (fall a-)sleep.'//\endgl\deftagex{appr.drs}\\
\a \begin{minipage}[t]{.3\textwidth}\textsc{drs} for first clause\\


\xdrs{$ \square $\xdrs{{\footnotesize I drink coffee}}}
\end{minipage}$ \mathit{K_2} $.\quad\begin{minipage}[t]{.3\textwidth}\textsc{drs} for full sentence\\
	 	
	 	 \xdrs{$ \square $\xdrs{{\footnotesize I drink coffee}}\\\rule{0pt}{.1em}~\\
	 	\xdrs{$ \neg $\xdrs{{\footnotesize I drink coffee}}}$ \lozenge $\xdrs{\footnotesize{I~\textsc{be}~asleep}}}
	 \end{minipage}


\xe


\noindent Crucially, when \textit{airra dringgi kofi} `I'll have a coffee' is asserted, its prejacent is presumed unsettled at speech time (that is, the sentence presupposes that at the relevant (future) time, the subject's drinking coffee (or failure to do so) is not a settled fact of the world (Roberts's \textsc{nonfactual} mood.) Because of this, \textsc{neg}(`I drink coffee') is
available as a restrictor to \textit{bambai} --- in other words $ \mathit{K_2} $ is \textbf{modally subordinate} to $ \mathit{K_1} $. Similarly, in (\getref{modsub.shop}), it is presumed unsettled that the addressee go to the shop (again at some future time, retrieved from context). The negation of the prejacent of the modal --- \textsc{neg}(`You don't go to the shop') --- restricts the domain of \textit{bambai}.


The second clause of (\getfullref{modsub.kofi.cfact}) is interpreted as a counterfactual (while it has past temporal reference, \textit{bina} explicitly marks its nonfactual status.) Consequently, \textit{bambai} needs a nonfactual antecedent and the negation of the foregoing proposition is accommodated to restrict its domain. Reminiscent of standard treaments of counterfactuals \citetext{\textit{i.e.}, where worlds in a nonrealistic proposition are ranked by their ``similarity'' to the actual world, see \citealp{Lewis1973,Kratzer1981,VonFintel2001,VonFintel2012}}. This is represented in (\getref{cfact.drs}) below: the first clause (coffee-drinking) is asserted as actual, the second a nonrealised possible outcome had the coffee-drinking not obtained.


\pex[labeltype=numeric,labelformat=$\mathit K_A.$] Discourse representation structure for (\getfullref{modsub.kofi.cfact})\\
\begingl\gla{} \textup{[$ _{\mathit{K_1}} $} aibin dringgi kofi \textup{]} bambai aibina silip//
\glft`I had a coffee, otherwise I might've slept.'//\endgl\deftagex{cfact.drs}\\
\a \begin{minipage}[t]{.25\textwidth}\textsc{drs} first clause\\
	
	
	\xdrs{{\footnotesize I drank coffee}}
\end{minipage}$ \mathit{K_2} $.\quad\begin{minipage}[t]{.3\textwidth}\textsc{drs} full sentence\\
	
	\xdrs{{\footnotesize I drank coffee}\\\rule{0pt}{.1em}~\\
		\xdrs{$ \neg $\xdrs{{\footnotesize I drank coffee}}}$ \lozenge $\xdrs{\footnotesize{I~\textsc{pst}~asleep}}}
\end{minipage}


\xe



%While $ K_1 $ is asserted as factual, $ K_2 $ is under the scope of both \textit{bambai} \gls{appr} and \textit{bina} \gls{pst}:\gls{irr} ($ \doteqdot $\gls{cfact}). This clause is marked for nonfactuality and needs a nonfactual antecedent. The negation of $ K_i$ \textsc{neg}(`I had coffee last night') is false in $ w* $ and can therefore restrict the domain of \textit{bambai}.


Unlike \textit{otherwise} (as examined in \citealp{PhilKotek}), possible antecedents appear to be predictably constrained by the form of the foregoing linguistic material. The \textit{``Red Light''} sentence pair described in that work is translated in (\nextx); accommodation of the entire conditional as an antecedent appears to be infelicitous (that is \textit{bambai} is not available to translate \textit{otherwise} on the reading presented in (\getfullref{redlight.kipgon}) \citealp[\textit{cf.}][]{PhilKotek,Webber2003,Kruijff-Korbayova2001}.)\footnote{These judgments have only been tested on a single speaker and bear confirmation of a negative judgment/further investigation. Of course the felicity of (\getfullref{redlight.kipgon}) would also be predicted to be independently degraded without establishing negative speaker attitude vis-à-vis the prejacent.} A DRS for (\getfullref{redlight.ticket}) is additionally provided in (\getref{redlight.DRS}).


	\pex \textit{bambai} accommodates the smallest antecedent: the \textit{Red Light} examples \deftagex{redlight}
\a\begingl
\gla If det lait im redwan, stap; \textbf{bambai} yu gaji tiket.\deftaglabel{ticket}//
\glb if the light 3s red stop \textit{\textbf{bambai}} 2s catch ticket//
\glft`If the light's red, stop; \textbf{otherwise} you might get a ticket.'//
\endgl
\a\begingl\gla	If det lait im redwan, stap; \textbf{if najing}, kipgon.//
\glb if the light 3s red stop \textbf{if no} {\sc cont}//
\glft`If the light's red, stop; \textbf{otherwise} continue.'\deftaglabel{kipgon}\trailingcitation{[GT~19032017]}//
\endgl\xe
\marginnote{need to introduce RL sentences}

In both Red Light sentences, the \textit{bambai}-clause is modally subordinate to a conditional imperative `If the light's red, stop!' As with the other precautioning uses analysed above, the ``simple'' satisfaction conditions (\textit{i.e.}, the conditions of $ K_i $ \ul{stripped of its own modal restrictions} (\textit{viz.} the conditional modality)
%$ \underset{\mathit{road rules}(w)}{\textsc{best}}(\cap\{m(w)\cup\{w'\mid w'\vDash\textit{red light}\}) $) 
 are accommodated as the restrictor to \textit{bambai}.% \textit{viz.} $ \langle w,f\rangle\Vdash K\leftrightarrow \textsc{stop}(\mathit{Addressee}) $) 

\ex DRS for (\getfullref{redlight.ticket})\hfill $\mathit{K_i\circleddash K_j\Leftrightarrow K_i\wedge K_{i_\text{sub}}\lozenge\ K_j}$\deftagex{redlight.DRS}



\footnotesize\it\xdrs{\xdrs{red.light}$ \square $\xdrs{stop}\\\rule{0pt}{.1em}~\\
\xdrs{$ \neg $\xdrs{\textit{stop}}}$ \lozenge $\xdrs{ticket}
}
\hfill\begin{minipage}[t]{.25\textwidth}
	\textbf{summary.}\\
	$ K_i =\mathit{red\ light\ \square\ stop}$\\
	$ K_{i_{\text{sub}}} = \textit{stop}$\\
	$ K_j = \textit{get ticket}$
\end{minipage}
\xe




In this subsection, we have considered the relation between the two clauses involved in ``precautioning'' uses of \textit{bambai} --- that is, those uses occurring in \textit{p bambai q} `$ p $, otherwise $ q $' contexts. Crucially, we have considered evidence that $ q $ --- \textit{bambai}'s prejacent --- is \textbf{modally subordinate} to material in the foregoing discourse. As shown in \citet[\S~2.2]{Roberts1989}, this operation involves a process which she calls ``accommodation (of the missing antecedent)'', that is, given a non factual assertion (\textit{i.e.}, $ [_{\mathit{S_2}}\ \textsc{mod}_2 \hdots\mathtt Y\hdots ] $), an antecedent $ \mathtt{(X)} $ that determines the modal domain must be found among accessible discourse referents (\textit{i.e.}, $ [_\mathit{S_1}\ \textsc{op}_1\hdots\mathtt{X}\hdots] $). 

In this chapter, I defend an analysis that treats all \textsc{apprehensional} uses of \textit{bambai} invariably as a modal operator that takes a single, nonfactual propositional argument $ (q) $.\footnote{Additionally, a proposal for unifying \textit{bambai}'s range of apprehensional uses with its subsequential use is detailed in Ch. \ref{bambai.semx}.} When (as in \textit{precautioning} contexts) \textit{bambai $ q $} immediately follows a (conjunct) sentence $ p $, it accommodates the negation of the basic proposition associated with that sentence (that is, the prejacent of an imperative or modal operator/the content of $ p $, stripped of any mood/modal information.)


The next subsection (\S~\ref{infstr}) contains a discussion of the pragmatic mechanisms by which an antecedent is selected.



\subsection{Information structure}\label{infstr}



In the previous subsection, we saw how (when it is interpreted as nonfactual), $ p $ --- the prejacent to \textit{bambai} --- is obligatorily modally subordinate to some antecedent proposition. Again following the proposal of \cite{PhilKotek}, and modulo the constraints in precautioning uses described above, ``accommodation of the missing antecedent'' operates on a pragmatic basis with reference to prior discourse  and the content of the prejacent.\footnote{This claim bears some similarity to the notion of an ``anaphorically-derived contextual parameter'' that features in the analysis of \citet[14]{Webber2001}.}$ ^, $\footnote{Relatedly, \citet{Corblin2002} notes the possibility of \textit{negative accommodation} without \textit{otherwise} in \textit{I didn't buy the car. I wouldn't have known where to put it (otherwise)} and \textit{I should have accepted. I wouldn't have been fired.} (author's translation: 256, 258).\label{corblin-modsub-note}}

By deploying information-structural notions developed in \citet{Carlson1983} and \citeauthor{Roberts1996a} (\citeyear{Roberts1996a}/\citeyear{Roberts2012}), we can conceptualize of \textit{otherwise} as representing a \textsc{discourse move} ($ m_n $ : in effect, a temporally-ordered stage in a given discourse), which adds to the \textsc{Question under Discussion (QuD)} in a given discourse context $ \mathcal D $.

\pex An information structure for $ \mathcal D $ (\textsc{InfoStr}$ _\mathcal D $) includes:
\a The \textbf{common ground} is a set of mutually assumed background information. The $cg$ is often modeled as a set of propositions, \textit{i.e.},  a set of sets of possible worlds (\textit{e.g.}, \citealt{Stalnaker1978} \textit{a.o.}, also introduced in \S~\ref{BT-review}). \label{common-ground}
\a A totally ordered set of discourse moves $ m\in\mathbf{M} $, partitioned into questions (setup moves) and answers (payoff moves). A subset of \textbf{M} is \textbf{Acc}epted in $ \mathcal D $.
\a The \textbf{\textsc{QuD}} is a partially structured set of questions which discourse participants are mutually committed to resolving at a given point in time. It is often modeled as a stack, consisting of ordered subsets of accepted question moves, the answers to which are not entailed by the $cg$ (\textit{i.e.}, the \textsc{QuD} is a set of ``open'' questions at a given stage $ m $ in $ \mathcal D $)\xe 


\noindent An important consequence of the conceit of a \textit{\textsc{QuD} stack} is that its structure and management are governed by \textit{strategies of inquiry} \citep{Roberts1998,Roberts2004,Roberts2012}. A (segment) of $ \mathcal D $ is associated with a \textit{discourse question} \textsc{(dq)} (or ``Big Question.'') Subsequent discourse moves (including additional questions) are appropriate iff they are taken to ``constitute a reasonable strategy of inquiry'' for answering the \textsc{dq} \citep*{Simons2017}.

These concepts provide a way of representing the `flow' of information and changes in the interlocutors' information states over time. Again beginning with \textit{bambai}'s \textit{precautioning} uses, take an utterance $ p $ \textit{bambai} $ q $ to consist of (at least) three discourse moves. A discourse anaphor, \textit{bambai} represents a ``setup'' move with the effect of adding to the \textsc{QuD}. 

\ex \emph{Proposal: the pragmatics of \emph{bambai}}\\
\textit{bambai} represents a discourse ``setup'' move with the effect of adding to the \textsc{QuD} stack a question about the \textsc{complement} of a set of worlds calculated on the basis of the discourse in which a \textit{bambai} sentence is uttered	.\deftagex{bambai-prag}\xe

\noindent The role of this information-structural aspect to the interpretation of \textit{bambai} is shown in (\getref{is-kofi}). Crucially, this treatment takes the role of \textit{bambai} to be the ``introduction of a question'' into the discourse (\getref{is-kofi}-$m_{\getref{is-kofi.bb}}$): an approach that converges with observations of formal and conceptual links between conditionals, interrogatives and ``topichood.'' That is: an utterance $\boldsymbol{ q\textit{ \textbf{if} }p }$ links the assertion of $ \denote{q} $ to the raising of a question $\denote{ ?p} $' \citep[36]{Starr2010}. This fact is especially clear when considering ``advertising conditionals'': \textit{e.g.}, \textit{Single? You haven't visited Match.com}, where an affirmative answer to the question is ``supposed'', much as a conditional antecedent would be \citep[4]{Starr2014a}.\footnote{For discussion of these links, see especially \citet{Starr2010,Starr2011,Starr2014a}, containing a proposal for a unified (dynamic/inquistive) semantics for conditional and interrogative-embedding uses of \textit{if}. Relatedly, the ``conditional question under discussion (\textsc{cQuD})'' in \citealp{Ippolito2013a}, following insights from \citeauthor{Isaacs2008}' 2008 dynamic treatment of conditional questions. These accounts similarly take a conditional antecedent/\textit{if}-clause to induce a temporary restriction over the common ground---``the answer to the question is an answer to the modally subordinated question'' \citep[200]{Ippolito2013a}. These observations are picked up again in \S~\ref{bambai.dia}}

The information-structural analysis of $ \boldsymbol{p \textit{ \textbf{bambai }}q} $ in (\getref{is-kofi}) provides a heuristic to capture some of these insights on functional similarities between conditionals and questions.


\pex[labelformat=$\boldsymbol m_A$,pexcnt=105] \deftagex{is-kofi} \textsc{InformationStructure}$ _\mathcal D $ and precautioning \emph{bambai}

[\textit{airra dringgi kofi}]$_{m_i}$, \textbf{bambai}$ _{m_j} $ [\textit{mi silip!}]$_{m_k}$


%\begin{itemize}%[label={\textit m}{\tiny\arabic*}.]
	\a This is the pronounced antecedent. It represents a modalized assertion: the addressee has a coffee in all worlds in some unspecified conversational background (here, potentially some teleological ordering source containing the subject's work goals / expected office behaviour at the Ngukurr Language Centre --- \textit{e.g.}, $ \underset{\textit{tel(w)}}{\textsc{best}}\big(\cap\underset{\textsc{circ}}{m(w)}\big)  $
	$$\forall w'\big[w'\in\underset{\textit{tel(w)}}{\textsc{best}}\big(\cap\underset{\textsc{circ}}{m(w)}\big) \to \textsc{have.coffee}(w')\big]$$
	\a \deftaglabel{bb} Per (\getref{bambai-prag}) and the discussion that follows, \textit{bambai} can be understood to encode an instruction to consider the \textsc{complement} of some set of worlds $ \boldsymbol p $ that has been made contextually discourse-salient. This set-up move can be thought of as signalling the addition of a question to the \textsc{QuD} stack of the form:\footnote{As in the previous chapter, I use the $ \overline{\text{overline notation}} $ to denote a function that maps a set of worlds to its complement.}
	
	 $$\text{what could (unfortunately) happen next in } w\in\overline{\boldsymbol p}\text{?}$$
	
	 In this case, a plausible candidate is: what if we are in a world s.t. the speaker doesn't have a coffee in that world? 
	
	\a The second clause -- \textit{bambai}'s \textbf{prejacent} -- is necessarily interpreted as proffering a (partial) answer to the \textsc{cq} (current \textsc{QuD}, a reflex of the maxim of \textsc{relevance}.)\footnote{For Craige Roberts, the notion of \textit{Relevance} --- a derivative of the Gricean maxim --- she defines it as follows (boldface added):
	
\begin{quote}
	A [discourse] move $\boldsymbol m $ \textbf{is \textsc{relevant} to the question under discussion} $ \boldsymbol q $ (\textit{i.e.}, to the last $ \textsc{QuD}(m) $), \textit{iff} $ m $ either \textbf{introduces a partial answer to $ \boldsymbol q $ ($\boldsymbol m $ is an assertion)} or is part of a strategy to answer $ q $ ($ m $ is a question).\trailingcitation{\citep[216]{Roberts2004}}
\end{quote}


\label{rob-rel}} Here, the speaker predicts that he may pass out as his desk in $ \overline{p} $: the set of worlds made available to \textit{bambai}. In this case, $ \overline p $ is the complement of the set of worlds in which he has a cup of coffee.
	%todo -- old formalism $${\forall w^{\prime\prime}\!\big[w''\!\in\textbf{\textsc{compl}}\big(\textsc{eat(\textit{Addressee})\big)}\to\neg\textsc{grow}(Addressee)(w^{\prime\prime})}\big]$$
$$\exists w^{\prime\prime}\big[w''\!\in\ \underset{\mathit{s'typ}(w)}{\textsc{best}}\big(\cap\{ \underset{\textsc{circ}}{m(w)}\cup\overline{\textsc{drink.coffee}(w^{\prime\prime})}\}\big)\ \wedge\ \textsc{sleep}(w^{\prime\prime})\big]$$
	
	

	
	%		Pragmatically, as Inkova-Manzotti observes (2001), the \textit{otherwise} clause canonically (\textit{``l'emploi prototypique''}) appears to encode a situation that is undesirable for the interlocutor. Indeed, the retrieval of a bouletically ordered conversational background to the modal in $p$ appears to be informed by the probable implication of the undesirability of the prejacent.
\xe


\subsection{Apprehensive domain restriction}

So far, this section has focussed on theorising the relationship between the two clauses in \textit{precautioning} uses of \textit{bambai} --- utterances of the form $ p\textit{ bambai }q $ are interpreted as $ p\wedge\blacklozenge q $. § \ref{modsub} showed that the assertion of $ \blacklozenge q $ (in utterances of the form is interpreted relative to (\textit{sc.} modally subordinate) to some antecedent derived from $ p $. \S~\ref{infstr} has shown how appeal to information-structural notions (\textit{viz.} the \textsc{QuD}) is helpful in understanding how this antecedent is accommodated. Here, the accommodation analysis is extended to other apprehensional uses described in Chapter \ref{bambai.desc} (\textit{e.g.}, Figure \ref{bb-dist}), again by appealing to pragmatic notions.


%The description of the information structure of an utterance of the form 

%The framework outlined above spells out the contribution and the information structure of these precautioning uses of \textit{bambai}. Described in the preceding subsection, an utterance of the form \textit{$ p $ bambai $ q $} asserts both $ p $ and $ \blacklozenge q $ (where $ q $ is modally subordinate to a proposition derived from $ p $.) 


In describing her notion of relevance -- introduced in (\lastx-$ m_k $) \& fn \ref{rob-rel} above -- \citeauthor{Roberts2012} additionally notes that, just as assertion moves are felicitous iff they constitute a (partial) ``answer'' to the \textsc{QuD}: ``a question can only be accepted \textbf{if it furthers answering those [questions] to which the interlocutors are already committed}'' \citetext{\citeyear[21]{Roberts2012}, emphasis added}. The \textit{apprehensive} uses of \textit{bambai}, are distinguished insofar as there need not be an explicit, pronounced $ p $ to constrain the option space for an antecedent to $ \blacklozenge q $.\footnote{The Robertsian model permits for ``[q]uestions [to be] raised explicitly, with interrogatives; implicitly, by question-introducing assertions; or by real world goals'' \citep[200]{Simons2017}.\label{adding}}
 Consider again, for example, (\getref{hanting}) from \S~\ref{ep.adv}, repeated here as (\getref{hanting.rpt}).


\pex[everylabel=\bf\sc]\textbf{Context:} Two relatives \textbf{\textsc{(a, b)}} are planning a hunting trip; a younger relative (say, \textsc{\textbf{c}}) wants to join.\deftagex{hanting.rpt}
\a\begingl\gla im rait, yu digi im then gajin.//
\glb 3s okay 2s take 3s then \textsc{kinship}  bambai 3s \textsc{neg.irr} give \textsc{loc} 1d.\textsc{incl}//
\glft`It's fine, bring him along poison-cousin'//
\endgl
\a\begingl\gla  \textbf{Bambai} yunmi gaan faindi bip//
\glb \textbf{\textit{\textit{bambai}}} 1d\textsc{.incl} \textsc{neg.irr} find meat //
\glft`But then we may not be able to find meat'//
\endgl
\a[label=a]\begingl\gla Yunmi garra digi im//
\glb 1d\textsc{.incl} \textsc{irr} take 3s//
\glft`We'll take him'//
\endgl
\a[label=b]\begingl\gla \textbf{bambai} im gaan gibi la yunmi.//
\glb \textit{\textbf{bambai}} 3s \textsc{neg.irr} give \textsc{loc} 1s\textsc{.incl}//
\glft `But then [the country] may not provide for us.'\trailingcitation{[DW 20170712]}//\endgl
\xe

In each of \textsc{\textbf{b}}'s utterances in (\getref{hanting.rpt}), there is no ``pronounced antecedent.'' In view of our account of \textit{bambai} as adding to the \textsc{QuD} stack and the (relevance) constraints on felicitous question moves (\textit{i.e.}, any additional questions must form part of a strategy of inquiry for a given \textit{discourse question}), accommodation is guided by pragmatic principles in concert with salient extralingusitic context.

\pex\textbf{Context.} The speaker is looking at a high-end stereo in an electronics store.\\
\textit{My neighbors \textbf{would} kill me}\trailingcitation{\citep[5-6]{Stone1997}}
\xe

\noindent While likely uninterpretable in an ``out of the blue''-type context, note that the modal proposition in (\lastx) is felicitous on a reading where the speaker's neighbours would be furious in the event that the speaker bought an expensive stereo and played it sufficiently loudly (compare fn \ref{adding}).


Similarly, the uses of \textit{bambai} are interpretable in (\getref{hanting.rpt}) in view of pragmatic calculations on the basis of the development of each speakers' information state through this dispute ($ \mathcal D_{(\getref{hanting.rpt})} $). In this context, the \textsc{dq} is $ \langle $ Should \textbf{\textsc{c}} accompany \textbf{\textsc{a \& b}} on their hunting trip? $ \rangle $. Additionally, the perspective of each speaker has been established --- \textit{i.e.}, \textbf{\textsc{a}} favours a situation where their younger relative accompanies them on the hunt, \textbf{\textsc{b}} disfavours this eventuality and both are arguing in favour of these domain goals \citep[compare][215]{Roberts2004}. As a consequence of this, both of \textbf{\textsc{b}}'s utterances are likely to be interpreted as justifications for his perspective: that is, in both instances \textit{bambai $ q $} is modally subordinate to a sentence similar in content to: `we shouldn't permit \textbf{\textsc{c}} to accompany us.' This is spelled out in (\nextx).

\pex[labelformat=$\boldsymbol m_A$,pexcnt=106] \deftagex{is-kofi} \textsc{InformationStructure}$ _{\mathcal D_{(\getref{hanting.rpt})}} $ and apprehensive \emph{bambai}

\textbf{bambai}$ _{m_j} $ [\textit{im gaan gibi la yunmi!}]$_{m_k}$

\a As shown previously, \textit{bambai} signals the addition of a question: \textit{what could (unfortunately) happen next in $ w\in\overline p $?} to the \textsc{QuD} stack. Per Roberts' felicity condition on questions, admissible questions have to contribute to a ``strategy'' to answering the questions to which the speakers are already committed'' --- viz. \textit{Should \textbf{\textsc{c}} come hunting?}

That \textbf{\textsc{b}} is opposed to this idea (\textit{sc.} the proposition \textit{\textbf{\textsc{b}} believes that \textbf{\textsc{c}} should not come hunting}) is in the common ground.

\a As previously argued, the prejacent is interpreted as a response to the current \textsc{QuD} (\textsc{cq}). Here the speaker predicts that a unsuccessful hunting trip (``the country may not provide) in $ \overline p $. In this case $ \overline p $ is the complement of the set of worlds in which \textbf{\textsc{c}} \ul{does not} join the hunting expedition.
$$\exists w^{\prime\prime}\big[w''\!\in\ \underset{\mathit{s'typ}(w)}{\textsc{best}}\big(\cap\{ \underset{\textsc{circ}}{m(w)}\cup\overline{\neg\textbf{\textsc{c}}.\textsc{comes.hunting}}(w^{\prime\prime})\}\big)\ \wedge\ \textsc{hunting.failure}(w^{\prime\prime})\big]$$
\xe

Ultimately, this section has sought to demonstrate that an appeal to modal subordination (particularly the accommodation of an antecedent) and information structural notions (the relevance of the \textsc{QuD}) allows for a unified account of the pragmatics of apprehensional uses of \textit{bambai} -- that is, in all cases, \textit{bambai $ q $} represents a modal claim --- $ \blacklozenge q $ --- against a predictive conversational background restricted by (the negation of) some salient proposition accommodated from the (explicit or implicit) discourse context. 

The following section provides a diachronic perspective on the relationship between $ p $ and $ q $ in view of better understanding the relationship between these apprehensional uses and the subsequential (temporal frame) meaning from which they are understood to have arisen.
%A possible implication of the discussion in §\ref{diaS} is that a sentence of the form $ p $ \textit{bambai} $ q $ is understood as introducing a negative eventuality which is a possible consequence of a failure of the antecedent subject to attend to some situation described in the antecedent clause \textit{i.e.} $\neg p(w)\to\lozenge q(w)$ (\textit{i.e.} if $p$ is false in $w$ then $q$ is possibly true) --- a truth condition very similar to that which is frequently given for \textit{if...(then)}-type clauses in English. In this case, the modal's premise set (\textit{i.e.} cogetnversational background) is restricted to a subset of the worlds in the modal base, \textit{viz.} those worlds in which an antecedent proposition does not hold true.






\section[Apprehensionality emerging]{Apprehensional readings emerge in subsequential TFAs}\label{bambai.dia}



\begin{quote}\small
	Of course borderline cases can arise because language changes. Something	that was not originally employed as a means of expressing a thought may eventually come to do this because it has constantly been used in cases of the same kind. A thought which to begin with was only suggested by an expression may come to be explicitly asserted by it.\trailingcitation{\citetext{\citealp{Frege1897}/1979, cited in \citealp[241]{Horn2013}}%\footnote{Thanks to Larry Horn for this reference.}
	}
\end{quote}


\noindent Here I consider a number of linguistic factors that appear to have contributed to the emergence of apprehensional readings of TFAs. As shown in \S~	\ref{dataS}, this meaning change pathway (and the apparent synchronic polysemy between temporal and apprehensional uses) has been observed by a handful of other authors \citep{Angelo2016,Angelo2018,Boogaart2020} on the basis of data including analyses of German \textit{nachher} and Dutch \textit{straks}, in addition to Kriol \textit{bambai} \citep[see also ][427-8]{Kuteva2019}. Parallels between \textit{bambai} and \textit{straks} are shown in (\getref{straks1}) below for example, where the contrast between a subsequential (\getref{straks1.ssq}) and apprehensional (\getref{straks1.appr}) reading is apparent.

\pex \textbf{Subsequential and apprehensive readings of the \textit{straksconstructie} in Dutch} %\citep[see also][]{Boogaart2020}
\deftagex{straks1}\a\begingl\glpreamble\textbf{Context.} It's 3.30, the shop closes at 4. I tell my friend:\deftaglabel{ssq}//
\gla de winkel is straks gesloten//
\glb the shop is \textit{straks} closed//
\glft `The shop will be closed soon.'//\endgl
\a\begingl\glpreamble\textbf{Context.} It's 4.10, the shop closes at either 4 or midnight, I'm unsure which. I say to my friend:\deftaglabel{appr}//
\gla straks is de winkel gesloten!//
\glb \textit{straks} is the shop closed//
\glft`The shop may (already) be closed!'\trailingcitation{[Mireille L'Amie,\textit{ p.c.} 20200130]}//\endgl\xe


\subsection{Temporal sequence \& conditional modality}
Many authors (\citealp[\textit{e.g.},][]{Stukker2012,Schmerling1975,Harder1995,Culicover1997,Klinedinst2012,Bluhdorn2008} a.o.) have investigated the semantic dependencies that often obtain between clauses that are \textit{syntactically coordinate}. These include the ``conditional readings'' of \textit{and} and \textit{or}, in addition to asyndetic constructions of the type: \textit{Matt comes, I leave.} In these cases, although there is no explicit conditional morphology, it is \textsc{r}-implicated that the second sentence should be interpreted as modally subordinated to the first: that is, my departure is a consequence of John's arrival. As mentioned above in fn \ref{corblin-modsub-note}, \citet[256-258]{Corblin2002} additionally notes the possibility of \textit{negative accommodation} in coordinate sentences:

\pex[aboveglftskip=0pt]\textbf{Negative accommodation of a modal antecedent}\deftagex{corblin}\a\begingl\gla Je n'ai pas achetée la voiture. Je ne~\textbf{saurais} pas où la mettre.//
\glb I \textsc{}have \gls{neg} bought the car I \textsc{neg}~\textbf{know.\gls{cond}} \textsc{neg} where it put//
\glft`I didn't buy the car. I wouldn't have known where to put it.'//\endgl
\a\begingl\gla J'aurais dû accepté. On ne~m'\textbf{aurait} pas viré.//
\glb I.have.\gls{cond} ought accepted one \gls{neg}~me.\textbf{have.\gls{cond}} \gls{neg} fired//
\glft`I should have accepted. I wouldn't have been fired.'//\endgl\xe

\noindent Crucially, the second sentence in each of (\getfullref{corblin}a-b) contains a modal operator (realised as a conditional inflection, \textsc{cond$ _2 $}). The (nonfactual) \textbf{negation} of a proposition contained in the previous clause is accommodated as the restrictor for \textsc{cond$ _2 $}.\footnote{Note that while the first sentence is not under the scope of a modal operator, its \textbf{negation}---which is accommodated to restrict the domain of \textit{saurais}---is interpreted as nonfactual making available a modal subordination reading.}

In \S \ref{infstr}, we considered the formal and conceptual links between conditional and interrogative clauses. It was claimed that a functional motivation for these appears to be that conditional apodoses (consequent clauses) can be understood as answering a ``question'' posed by the antecedent/protasis. The illocutionary effect of both interrogatives and conditionals is often taken to be the ``supposition'' of a proposition: that is, adding a proposition to the common ground (or partitioning contextual possibilities, \citealp[see][]{Starr2010}).
These conceptual parallels have clear linguistic reflexes, shown clearly for Danish, \textit{e.g.} by \citet[100-2]{Harder1995}, replicated in (\getref{harder}) below.

\pex[aboveglftskip=0pt]\textbf{Conditionals as ``telescoped'' discourse}\trailingcitation{\citep{Harder1995}}\a\textup{ A two-participant discourse}\trailingcitation{(101)}\deftagex{harder}\beginsubsub\b{\textsc{\textbf{a.}}}\begingl\gla Kommer du i aften?//
\glft Are you coming tonight?//\endgl
\b{\textbf{\textsc{b.}}}\begingl\gla \textit{ja}//
\glft Yes//\endgl
\b{\textbf{\textsc{a.}}}\begingl\gla\textit{ Så laver jeg en lækker middag}//
\glft Then I'll cook a nice dinner.//\endgl\endsubsub
\a\begingl\deftaglabel{harder.cond}\gla Kommer du i aften, (så) laver jeg en lækker middag//
\glft `If you're coming tonight, (then) I'll cook a nice dinner.'\trailingcitation{(101)}//\endgl
\xe

\noindent\citet[101]{Harder1995} suggests that ``the conditional can be seen as a way of \textit{telescoping a discourse sequence into one utterance} so that \textbf{\textsc{b}} has to respond not only on the basis of the present situation, but also on the basis of a possible future.''






In view of the data presented in (\getref{corblin}-\getfullref{harder}), consider the discourses in (\getref{car}-\getref{sinek}) below.


\pex[everylabel=\bf\sc,aboveexskip=1pt]\textbf{Context:} A child is playing on a car and is told to stop.\deftagex{car}
\a	\begingl 	\gla \rightcomment{[compare (\getfullref{appr1.motika})]}gita la jeya!//
\glb get~off {\sc loc} there!//
\endgl
\a[label=\textcolor{gray}{b}]\begingl\gla \textcolor{gray}{ba~wani?}//
\glb \textcolor{gray}{why?}//
\endgl
\a[label=\textbf{\textsc{a}}]\begingl
\gla bambai yu breigim motika//
\glb \textbf{\textit{bambai}} 2s break car//
\glft `Get off of there [...why?...] You're \textbf{about to} break the car!'\trailingcitation{[GT~16032017]}//
\endgl\xe
\pex[everylabel=\bf\sc,aboveexskip=1pt]\deftagex{flodawei} \textbf{Context:} It's the wet season and the Wilton River crossing has flooded.
\a\begingl
\gla nomo krosim det riba!//
\glb {\sc neg} cross.{\sc tr} the river//
\endgl
\a 	\begingl\gla \textcolor{gray}{ba wani?}//
\glb \textcolor{gray}{why?}//
\endgl
\a[label=\textbf{\textsc{a}}]\begingl\gla bambai yu flodawei!//
\glb \textbf{\textit{bambai}} 2s float~away//
\glft`Don't cross the river [...why not?...] You're \textbf{about to} be swept away!'\trailingcitation{[GT~16032017]}//\endgl\xe
\pex[everylabel=\bf\sc,aboveexskip=1pt]\deftagex{sinek}
\textbf{Context:} A snake slithered past \textit{A}'s leg.
\a\begingl
\gla det sineik bin bratinim mi!//
\glb the snake \gls{pst} frighten{\sc.tr} me//
\endgl
\a \begingl\gla \textcolor{gray}{ba wani?}//
\glb \textcolor{gray}{why?}//
\endgl
\a[label=\textbf{\textsc{a}}]\deftagex{bambai}\begingl\gla \textbf{bambai} imina baitim mi!//
\glb \textbf{\textit{bambai}} 3s.\gls{irr}:\gls{pst} bite\textsc{.tr} 1s//
\glft`The snake scared me [...why?...] It might've been \textbf{about to} bite me!'\trailingcitation{[GT~01052017]}//\endgl\xe



In each of the short discourses above, the translation provided elucidates: \textbf{(a)} that each of these dialogues can be ``telescoped'' onto a single utterance, and that \textbf{(b)} the capacity of the temporal properties of \textit{bambai} \textit{qua} sequential TFA to implicate additional nontemporal properties of the relation between the clauses it links --- that is, the \textit{bambai} clause is modally subordinate to the content of \textbf{\textsc{a}}'s first utterance. In each of the examples, \textsc{\textbf{a}}'s response identifies an eventuality that might obtain in the near future (of the speech-time for (\getref{car}-\getref{flodawei}) and of the slithering\slash frightening-time for (\getref{sinek}). 

Further, in all three cases, this \textit{bambai} clause is obligatorily interpreted as nonfactual. In the first two cases it describes an eventuality that is posterior to a possible future event (the one described by the previous imperative and one that is therefore only felicitous if it is presumed unsettled.) In (\getfullref{sinek}), the \textit{bambai} clause has explicit irrealis marking, indicating its coounterfactual status: it expresses that \textbf{\textsc{a}}'s psychological state at the event time was such that biting was an unsettled, possible future.

Via pragmatic strengthening \textit{(viz.} an inference of the form \textit{post hoc ergo propter hoc)}, \textit{bambai} can be understood to assert that there exists some type of logical (\textit{e.g.}, etiological) relation between the predicate contained in the first proposition and the  eventuality described in \textit{bambai}'s prejacent: the second clause. In (\getref{car}), for example, the child's failure to comply with \textsc{\textbf{a}}'s (precautioning) instruction could contribute causally to the car's breaking. Inferencing-based theories of meaning change will hold that, while there is no lexical item that encodes causality, in many contexts, reasoning about informativity and relevance ``invite'' the \textit{propter hoc} inference \citep[\textit{e.g.},][564]{Geis1971}.

This type of implicature is well-documented in cross-linguistic studies of meaning change \citep[see also][403]{Kuteva2019}; the extension of English \textit{since} (\textit{siþþan}) from encoding subsequentiality (they report ostensibly similar shifts in numerous other language) to causality (particularly when talking about past events) is discussed by \citet{Traugott1991}:

\pex\a I have done quite a bit of writing \textbf{since} we last met\hfill (temporal)\deftagex{TK-since} \a \textbf{Since} Susan left him, John has been very miserable\hfill (temporal, causal) \a \textbf{Since} you are not coming with me, I will have to go alone \hfill(causal)
\a \textbf{Since} you are so angry, there is no point in talking with you\hfill(causal)
\xe

\noindent Traugott \& König go on to say

\begin{quote}
	{\small With \textit{since}, when both clauses refer to events, especially events in the past, the reading is typically temporal, as in [\getref{TK-since}a] When one clause refers to a non- past event or to a state, the reading is typically causal, as in [\getref{TK-since}c] and [\getref{TK-since}d], but the causal reading is not required, as [\getref{TK-since}b] indicates. The contrastive readings in [\getref{TK-since}b] signal polysemy, i.e. \textbf{conventionalized meanings, not just conversational}.\trailingcitation{\citetext{\citealp[195]{Traugott1991}, emphasis added}}}
\end{quote}

%\footnote{There is an extensive, contemporary literature investigating the pragamatics of clausal connection/concatenation e.g. Schmerling 1979, Stukker \& Sanders 2012 a.o.}

It appears, then, that precautioning type uses of \textit{bambai} arise from a related inference, namely the conventionalisation of an inference that emerges on the basis of reasoning about relevance: ``if \textsc{\textbf{a}} is alerting me that a possible event $ e_1 $ may be followed by another possible event $ e_2 $, it's likely that they're drawing a causal connection between these two possible events'' ($ e_1 $ causes $ e_2 $). \S~\ref{bambai.expr} below further investigates this process in view of the expressive/speaker attitude component of \textit{bambai}'s conventional meaning.

\subsection{\textit{Conventionalized...not just conversational}}

\textit{Subjectification} --- associated especially with related concepts from the work of Elizabeth Traugott (\citeyear[\textit{e.g.},][]{Traugott1989}; \citealp{Traugott2002}) and Ronald Langacker \citeyearpar[\textit{e.g.},][]{Langacker1989} --- refers to an observed meaning change tendency whereby linguistic expressions diachronically come to encode increasingly ``subjective'' meanings --- those concerning the private beliefs and attitudes of the speaker in a given context. Subjectivity as a relevant linguistic notion has been construed in a number of ways (\nextx).

\pex[labeltype=numeric] \textbf{The loci of \textsc{subjectivity} in according to \citet[4]{Finegan1995}} are

\textsc{a locutionary agent's:}
\a \textsc{perspective} as shaping linguistic expression;
\a expression of \textsc{affect} towards the propositions contained in utterances;
\a expression of the \textsc{modality} or epistemic status of the propositions contained in utterances.
\deftagex{def.sbjn}
\xe

\noindent To my knowledge, at the time of writing, no work has explicitly interrogated the role of (inter)subjectification as a force in meaning change from a formal perspective \citetext{\citeauthor{Eckardt2006} acknowledges this in her \citeyear{Eckardt2006} monograph (239).}\footnote{\citet{Jucker2012} (cited in \citealt[562]{Traugott2012}) expresses skepticism that a ``cognitive-inferential conceptualization'' (what he refers to as the ``Anglo-American'' approach, apparently including (neo)-Gricean theories) is capable of accounting for these types of phenomena, which apparently invite a ``performance-based''/``socio-interactional'' pragmatics (which he associates with European research programs.) It is not clear that this is a thoroughly fair assessment (see \textit{e.g.}, discussions of the social motivations for \textbf{R}-based implicata in \citealp{Horn1984,Horn1984a,Horn1993,Horn2007a} a.o.) \citet[43]{Eckardt2006} does also suggest a role for semanticisation of implicatures in apparently subjectivisation-driven changes.} As a driver of meaning change, \textit{subjectification} has been evoked especially in view of explaining the development of modal readings of verbal and adverbial elements, where these expressions come to encode the epistemic status of a speaker vis-à-vis a given proposition \citep{Finegan1995,Traugott1989,Traugott1995,Traugott2006,Traugott2003}.\marginnote{more? lexicalisation/grammaticalisation of social knowledge} Apparent connections between ``non-challengeability''/\textsc{not-at-issue}ness and \textit{subjectivity}, however, are implicit in recent formal work, particularly as this relates to the \textsc{evidential} and \textsc{expressive} domains (\citealp[\textit{e.g.},][]{Korotkova2016,Faller2002,Murray2014,Korotkova2020} a.o.)\footnote{\citeauthor{Korotkova-ms} (\texttt{ms}) explicitly suggests links between ``nonchallangeability'' and subjectivity on the basis of linguistic reflexes of `first-person authority' (that is the ``immunity'' of ascriptions of self-knowledge to correction.)}


 The meaning change pathway that \textit{bambai} has traced --- \textit{i.e.}, the trajectory from temporal frame adverbial to (multifunctional) apprehensional modal --- clearly can be characterised as conforming with generalisations about subjectification in meaning change in each of the criteria in (\getref{def.sbjn}).

  In chapter \ref{bambai.semx}, a unified lexical entry for \textit{bambai}'s temporal and apprehensional uses is proposed. This proposal relies on the ``emergence'' of modal readings in \textbf{nonfactual contexts} as a function of reasoning about discourse context, a reflex of what I've called the ``omniscience restriction'' (a component of the asymmetry of past and future/the ``problem'' of future contingents: outlined in \S~\ref{BT-review}.) This condition is described in  (\getref{omni1}) and resembles the epistemic constraints identified in \citet{Kaufmann2002}, to be further discussed in Ch. \ref{bambai.semx}.
  
  \ex\textbf{The omniscience restriction}\deftagex{omni1}
  
Predications of subsequentiality (near-future instantiation, see ch. \ref{bambai.semx}) are interpreted as carrying predictive illocutionary force (\textit{i.e.}, modalised or ``epistemically downtoned'') when they are presumed unsettled.
  \xe
  
In view of this general pragmatic principle, when a \textit{bambai} clause is interpreted as making an unsettled claim --- that is, some future-oriented claim that the discourse participants know that the speaker cannot possibly know the truth of --- a modal (predicted possibility) interpretation is invited. This implicature can be understood as resulting from reasoning on the part of language users: discourse participants mutually understand that the \textit{bambai} predication is unsettled and therefore must represent a prediction.\footnote{A related account might appeal to \citeauthor{Eckardt2009}'s \textsc{Avoid Pragmatic Overload} principle \citeyearpar{Eckardt2009}, where, faced with an utterance that carries an unaccommodable presupposition (\textit{pragmatic overload}), a (charitable) hearer/reader surmises that the speaker has ``used words or phrases in a sense that were formerly unknown to the hearer'' (22) and ``hypothesize[s] a new meaning...for the item that gave rise to the problematic presupposition'' (35). In the present case, \textit{bambai $ q $} asserts $ q $ in the future of some presupposed reference index (see ch. \ref{bambai.semx}). Given the infelicity of making non-modal assertions about nonactual events, the domain accessible to \textit{bambai}, pragmatic overload is ``avoided'' by expanding the modal domain of \textit{bambai}.}%\marginnote{i feel like i'm missing sth here... the future orientation thing being ripe for modal reanalysis "that's gonna be south of crockett'' (Trau 95:35)}


 More specifically, given the apparently frequent use of \textit{bambai $ q_{\text{nonreal}} $} in directive contexts and under fear predicates, encoding an ``apprehension-causing situation'' \citep[298]{Lichtenberk1995} and the justification for an utterandce of $ p $ \textit{bambai} has come to be associated with \textit{admonitory} predictions. Similarly, \citet[285]{Angelo2016} propose that:
 \begin{quote}
 	{\small
 		The conventionalisation of the implicature of undesirability may come about through frequent use of a clausal sequence in which the first clause has the illocutionary force of a directive and the second is introduced by the temporal marker.}\end{quote} 
 
\noindent The status and emergence of this ``undesirability implicature'' is further investigated directly below, in \S~\ref{bambai.expr}.


In this section, I have proposed that the apparent subjectification of \textit{bambai} is unifiable with observations about the diachronic conventionalisation of conversational implicature \citetext{\citealp[\textit{e.g.},][273\textit{ff}]{Cole1975} and especially Traugott's invited inferencing theory of semantic change (\citeyear{Traugott1980} \textit{et seq.})} The frequent occurrence of \textit{bambai} in admonitory contexts and consequent generalisation and conventionalisation of these \textbf{R}-implicatures\footnote{That is, implicatures following from conversational principles of relevance and avoidance of ``overinformativeness'' \citetext{\citealp{Horn1984} \textit{et seq.}}} is the source of \textit{bambai}'s apparent (epiphenomenal) subjectification trajectory and present day ``lexically denoted information.''

\section[Use conditions]{\textit{bambai} and apprehensional expressive content} \label{bambai.expr}


Of course, a crucial, characterising meaning component for apprehensionals is that they express information about the Speaker's attitude vis-à-vis their prejacent. This contrast is demonstrated by the minimal pair in (\getref{expr}), where the utterance in (\getref{expr.pos}) is not ``expressively correct'' \citep[\textit{cf.}][]{Kaplan1999} because the conditions on speaker attitude are not satisfied --- that is, \textit{bambai} is felicitous in negative-purposive (apprehensional) contexts, not positive purposive ones.



\pex \textbf{Apprehensional use conditions for \textit{bambai}}\deftagex{expr}
\a\begingl%\glpreamble\textbf{\textit{but under negation...}}//
\gla mi \textbf{nomo} wandi gu la mataranka \textbf{bambai} mi luk la main banjimob.//
\glb 1s \textbf{\textsc{neg}} want go \textsc{loc} Mataranka \textbf{bambai} 1s look \textsc{loc} my cousin.\gls{assoc}//
\glft`I \textbf{don't} want to go to Mataranka, \textbf{(because) then }I might see my cousins.'//\endgl
\a\begingl\gla\ljudge{$^{??}$}mi wandi gu la mataranka \textbf{bambai} mi luk la main banjimob.//
\glb 1s want go {\sc loc} mataranka \textbf{bambai} 1s look \textsc{loc} my cousin.\gls{assoc}	//
\glft\textbf{Intended:} `I want to go to Mataranka so/then I'll see my cousins.'\trailingcitation{[AJ~072017]}\deftaglabel{pos}//\endgl\xe





 

As suggested above \citep[see also][]{Angelo2016}, the apprehensional reading frequently occurs embedded under a predicate of fearing or in conjunction with a directive (prohibitive) antecedent: corresponding to Lichtenberk's \textsc{fear} and \textit{precautioning} uses respectively (shown in exx. \getref{car}-\getref{sinek} above).

Relatedly, \citet[192\textit{ff}]{Boogaart2020} suggests (of Dutch) that it is the ``sense of immediacy'' of this class of adverbials that associates with notions of ``urgency'' and that this is the source of the ``expressive nature'' of subsequential TFAs. Consequently, we might hypothesise that the frequent association of sequential TFAs with these discourse contexts (situations of urgent warning) has resulted in the \textbf{conventionalisation} of apprehensional use-conditions for $\textit{bambai}\,q.$ 

In contemporary Kriol, then, the selection of an erstwhile subsequential TFA when making some unsettled predication (instead of a different epistemic adverbial)  conventionally implicates that the Speaker is negatively disposed to the event described in the prejacent.

\subsection{The status of apprehensional ``attitude conditions''}
%todo existing Dutch judgments/elicitatitons to be added?? 
Marshalling cross-linguistic evidence of this path of change for German and Dutch respectively, an utterance \textit{nicht jetzt, nachher!}/\textit{niet nu, straks!} `not now, later' is reported to involve a higher degree of intentionality and immediacy than the less specialised \textit{nicht jetzt, später!}/\textit{niet nu, later!} `not now, later.'\footnote{See also \citealt{Angelo2018} for these observations and insightful comments about the properties of these adverbials in Kriol and German. Related observations are made for Dutch by \citet{Boogaart2020}.} What's more, tracking the facts for \textit{bambai} presented above, these TFAs appear to have encroached into the semantic domain of epistemic/modal adverbials, where they are reported to encode negative speaker affect with respect to their prejacents (relative to the other members of these semantic domains.)\footnote{Compare also the colloquial English expression \textit{(and) next thing you know, $q$} As with the other subsequential TFAs we have seen, it appears that this adverbial tends reads less felicitously (or indeed invites an ironic reading) when $q$ is not construed as an undesirable proposition.)% This is shown in \ref{stew} below, attributed to Jon Stewart.) %is given by     as an example of a comedic \textit{reverse}: ``a device that adds a contradictory tagline to to the opening line of a standard expression or cliché''(\begin{exe}
	\ex~[exno=i]\label{field}\textit{The fields dried up, \textbf{and the next thing you know} our fleet dropped from 68 drivers to six in the matter of a few months.\hfill[Google result]}\xe
	\ex~[belowexskip=0pt,exno=ii]\label{stew}\textit{The Supreme Court ruled that disabled golfer Casey Martin has a legal right to ride in a golf cart between shots at PGA Tour events. Man, \textbf{the next thing you know,} they're going to have some guy carry his clubs around for him.}\trailingcitation{[Jon Stewart]}\xe}

As with \textit{straks} (\textit{e.g.}, \getref{straks1}), \textit{nachher} appears to have a similar distribution to \textit{bambai},\footnote{Although see \citet[30]{Angelo2018} for a discussion of distributional differences between these two items.} shown by its felicity in the discourse in (\getref{nachher}) where it represents an alternative to \textit{vielleicht} `perhaps.'
% below, where, tracking $\langle$\textit{marri, bambai}$\rangle$, \textit{nachher} appears to have encroached into the semantic domain of \textit{vielleicht} `perhaps.' 
 In these contexts, \textit{nachher} asserts negative speaker attitude with respect to its prejacent.\footnote{Thanks to Hanna Weckler and Mireille L'Amie for discussion of German and Dutch intuitions respectively.}

\pex[everylabel=\bf\sc,labelformat=A,interpartskip=0pt,*=$^\#$]\deftagex{nachher} 
\textbf{German apprehensional \textit{nachher} and the not-at-issueness of speaker attitude}\trailingcitation{[H. Weckler, \textit{pers. comm.}]}

\textbf{Context.} A two-participant discourse in German
\a\begingl
\gla ich hoffe, dass es heute nicht regnet//
\glb I hope {\sc comp} it today {\sc neg} rain//
\endgl
\a 	\begingl\gla \textcolor{gray}{warum?}//
\glb \textcolor{gray}{why?}//
\endgl
\a[label=a$_2$]\begingl\gla nachher wird die Party noch abgesagt!//
\glb \textbf{\textit{nachher}} {\sc inch} the party \textit{noch} cancelled\deftaglabel{nachher}//
\glft`I hope it doesn't rain today [...why?...] Then the party might be \mbox{cancelled}!'//\endgl
\a[label=b$_2$]\begingl\gla nein, das ist nicht möglich//
\glb no, that is not possible\deftaglabel{infel}//
\endgl
\a[label=b$^{\prime}_2$]\begingl\gla\ljudge{$^\#$}nein, das wäre gut!\deftaglabel{infel}//
\glb no, that would.be good//
\endgl
\a[label=b$^{\prime\prime}_2$]\begingl\gla ja, das ist möglich aber das wäre nicht so schlimm!//
\glb yes, that is possible but that would.be {\sc neg} so bad!//
\endgl\xe


Similarly to the Kriol data, German \textit{nachher}, a TFA encoding imminence or ``subsequentiality'', has developed the characteristics of an apprehensional epistemic, a likely consequence of frequent embedding in the discourse contexts discussed above (\S~\ref{bambai.dia}). Crucially, the contrast between the possible responses (in particular the infelicity of \getfullref{nachher.infel}) illustrates that, while the use of \textit{nachher} in \getref{nachher.nachher} does commit the speaker to the proposition `I am negatively disposed to the possibility of rain today', this commitment has the status of a conventional implicatum (not-at-issue).\footnote{\textit{I.e.,} ``there is no simple way to indicate just the rejection of something that is conventionally implicated \citep[14]{Karttunen1979}.}


Following insights from the the literature on expressive content and use-conditional semantics \citetext{\citealp[\textit{e.g.},][]{McCready2010,Kaplan1999,Potts2007,Gutzmann2015}, ostensibly developing \citeauthor{Karttunen1979}'s \citeyear{Karttunen1979} proposed extension to PTQ}, it is fruitful to model the `negative speaker attitude' component of the meaning of apprehensionals as a conventional implicature, inhabiting a second semantic ``dimension''---connected to but distinct from the truth conditional contribution (see ch. \ref{bambai.semx}). The infelicity of (\getfullref{nachher.infel})'s utterance shows that negation cannot target this component of Speaker meaning: an argument for the treatment of this component of its semantics as non-truth-conditional/not-at-issue component. These proposals (variants of a ``logic of conventional implicature'' $ \mathcal L_{CI} $) develop a formalism that conceives of the semantic information contained in a given linguistic expression as a pair of truth- and use-conditional content.




\citet{Gutzmann2015} proposes a compositional ``hybrid semantics'' that is capable of handling these ``two dimensions'' of meaning --- \textit{viz.} distinct truth- and use-conditional content. On this type of account, the semantics of a lexical item like \textit{bambai} might be modelled as a ``mixed use-conditional item'' --- a lexical item whose meaning can be represented as a pair of metalinguistic formulæ.
The previous section discussed the truth-conditional contribution of \textit{bambai}, providing the lexical entry in (\getref{bb-semx}) above. Following the proposal in \citet{Kaplan1999} where a ``use-conditonal proposition'' is understood to denote a set of contexts, Gutzmann \citeyearpar{Gutzmann2015}, appeals to a model with parallel types, interpretation functions (\textit{i.e.}, $ \denote[t]{•} $ and $ \denote[u]{•} $) and composition rules for both truth- and use-conditions that allow for the interaction of these condition types while distinguishing these two ``dimensions'' of meaning.\footnote{This system closely resembles the proposal of \citet{Karttunen1979}, which these authors attribute (their fn 17) to the ``two-dimensional logic'' apparently discovered by \citet{Herzberger1973}.}


 Borrowing the informal ``fraction notation'' deployed by some of these authors, we can tease apart the implicated and asserted meaning components of the \textit{bambai} clause in (\getref{sinek}) -- this is given in (\nextx).%\mcom{this is the wrong place for this, I don't know exactly where the UC part should go if it sticks around, but at the moment it's formulated in such a way that it presupposes the TC semantic analysis}


\pex\a\begingl
\gla \textbf{Bambai} imina {baitim} mi!//
\glb \textsl{\textbf{bambai}} 3s.\gls{pst}:\gls{irr} bite 1s//
\glft`...It might've bitten me!'\trailingcitation{[GT~01052017]}//\endgl
\a$\dfrac{\mathit S\text{ is worried about/negatively disposed to snake bites}}{\mathit `S\text{ might have been about to be bitten by a snake}}$
%{S \text{ gets bitten by a snake in $w^\prime\in\textbf{best}(f,g,t*,w*)$ at $t^\prime:\mu(t*,t^\prime)<c_s$}}$
\xe

If this mode of thinking about the speaker attitude implications of \textit{bambai $q$} is on the right track, then, in addition to asserting $ \lozenge q $, a speaker's utterance of \textit{bambai $ q $} at $ t $ in $ w $ can be thought of as creating an updated context in which `it registers that [they regard $q$] negatively somehow' \citep[175]{Potts2007}.
The use-conditional contribution of \textit{bambai} can then be informally stated as (\nextx).\footnote{This use condition is comparable to the condition proposed by \citet{AnderBois2020}: $ \forall w'\in\textsc{goal}_{i,p}(w):\neg q(w') $ (I.e. that some proposition $ p $ is performed/caused by $ i $ in order to achieve the speaker's goals (in which $ \neg q $ holds))}

\pex \textbf{A use-condition for \textit{bambai}}

$ \denote[u]{\textit{bambai }q}=\{c:c_s\text{ is negatively disposed to $ q $ in }c_W\} $\\
\textit{bambai} $ q$ is expressively correct in a context where the speaker $ c_s $ is negatively disposed to $ q $ in $ w* $
\xe


\noindent In this sense, \textit{bambai} $ p $ can be taken to conventionally implicate a proposition of the form given in (\lastx).





 I propose a formal analysis of both of these components of \textit{bambai}'s semantics (\textit{sc.} the asserted and the conventionally implicated content) in the following section.




\subsection{Competition in the modal-adverb domain}

A predicted consequence of this meaning change --- that is the ``encroachment'' of \textit{bambai} into the modal adverbial domain --- is that \textit{bambai} enters into competition with other modal adverbs. 

One arena in which this is made particuarly clear is in \textit{bambai}'s \textit{apprehensive function} (\S~\ref{ep.adv}) --- that is, where it realises a possibility modal whose domain can be restricted by the presence of an \textit{if}-clause. In these contexts \textit{bambai} has entered into the semantic domain of other Kriol lexical items including \textit{marri/maitbi} `maybe'.  The examples in (\getref{smok}-\getref{doda}) below show the perseverance of apprehensional expressive content in these syntactic frames. In (\getfullref{smok.B}), consultants reported that apprehensive \textit{bambai} gives rise to an implication that the speaker may not go on holiday, where the minimally different (\getref{smok.M}) fails to give rise to this implication.

\pex<smok>\textbf{Context.} I'm planning a trip out to country but Sumoki has taken ill...
\a\deftaglabel{B}\begingl\gla if ai gu la holiday, \textbf{bambai} main dog dai//
\glb if 1s go {\sc loc} holiday \textit{\textbf{bambai}} 1s dog die //
\glft `If I go on holiday, my dog may die'\hfill$\boldsymbol\leadsto$\hfill I'm likely to cancel my holiday//
\endgl
\a\deftaglabel{M}\begingl
\gla if ai gu la holiday, \textbf{marri} main dog (garra) dai//
\glb if 1s go {\sc loc} holiday \textbf{perhaps} 1s dog {\sc(irr)} die//
\glft `If I go on holiday, my dog may die'\hfill$\boldsymbol{\not\leadsto}\hfill$ I'm likely to cancel my holiday\\[.2em]
\textsc{\textbf{speaker comment.}} \textit{Tharran jeya im min yu garra gu la holiday}\\`That one means you'll go on your holiday.'\trailingcitation{[AJ 04082017]}//
\endgl
\xe
Here, the contrast between (\getref{smok.B}) and (\getref{smok.M}) is attributable to the expressive content of \textit{bambai}. That \textit{bambai} licenses an implicature that the Speaker is considering cancelling her holiday to tend to her sick pet, an inference that isn't invited by neutral epistemic counterpart \textit{marri} provides strong evidence of the semanticisation of \textit{bambai}'s expressive content (similar to `sincerity'- or `use-conditions' for a given lexical item.) The extent of this process is further evinced in (\nextx) below, where the selection of \textit{marri} instead of \textit{bambai} gives rise to a conventional implicature that the Speaker's utterance of (\nextx) ought not be interpreted as the expression of a desire to prevent her daughter's participation in the football game.
\pex\begingl
\glpreamble\textbf{Context:} I am cognizant of the possibility that my daughter injures herself playing football.\\\textbf{$^\#$Context:} I am uncomfortable with the likelihood of my daughter injuring herself playing football.//
\deftagex{doda}
\gla if im pleiplei fudi, marri main doda breigi im leig//
\glb  if 3s play footy \textsl{perhaps} my daughter break her leg//
\glft `If she plays footy my daughter may break her leg'\hfill$\not\leadsto$ [so she shouldn't play]\trailingcitation{[AJ 04082017]}//
\endgl\xe

Based on this evidence, we may conclude that the ostensible encroachment of \textit{bambai} into the domain of modal/epistemic adverbials has given rise to a dyad (\textit{i.e.}, ``Horn scale'', \citealp{Horn1984}) with the form $\langle\textit{marri~p, bambai~p}\rangle$ --- selection of the ``weaker'' expression \textit{marri~p} \textbf{\textit{Q}}-implicates that the Speaker was not in a position to utter its stronger (more specific) scalemate, \textit{bambai p}. That is, the meaning of the `weaker' expression comes to represent the relative complement of the stronger in a given semantic domain. In this case, use of the neutral modal adverb \textit{marri} comes to conversationally implicate \textbf{non-apprehensional} readings/modalities.


\pex \textbf{Competition in the modal adverbial domain} $$\denote{\textit{marri}}\approx\lozenge\setminus\denote{\textit{bambai}}$$

Situations in which \textit{marri} is felicitous are those in modal possibility claims in which \textit{bambai} is inappropriate/expressively incorrect.

\xe

\section{Summary}


This chapter has considered a number of crucial issues relating to the interpretation of apprehensional \textit{bambai}, particularly as it relates to the role of context in the synchronic interpretation and the diachronic reanalysis of this lexical item. In view of the emergence of \textit{bambai}'s modal readings, \S~\ref{bambai.subord} developed an account of the interpretation of \textit{bambai} clauses as involving modal subordination to some accommodated antecedent. Appealing to basic principles of communication (\textsc{relevance} and the implementation of this notion as the \textsc{question under discussion}), \textit{bambai}'s prejacent is taken to encode a response (specifically a prediction) to a question about a salient eventuality.


In §~\ref{bambai.dia}, we saw how the development of apprehensional readings of \textit{bambai} (both its modal and expressive content) appears to be a result of its (as with subseq\-uential-TFAs in other languages) frequent occurrence in contexts of ``preacautioning'' and fearing. These contexts gave rise to inferences creating the conditions for the renalysis of \textit{bambai} as conventionally encoding apprehensional meaning. The reanalysis of \textit{bambai} as a modal adverb permits for the set of uses that correspond to its \textit{\textsc{apprehensive}} function.

Further developing these observations, the final section --- \S~\ref{bambai.expr} --- considered data from other two other languages in which a subsequential TFA appears to have undergone similar functional change, developing apprehensional expressive content (\textit{viz.} Dutch \textit{straks} and German \textit{nachher}). These data support an analysis of the distinctive negative attitude reading that is associated with apprehensionals as \textsc{not-at-issue content}. As with the diachronic emergence of modal readings of erstwhile TFAs, this expressive content/use condition is understood to have arisen as a result of the conventionalisation of an implicature arising under certain frequent (\textit{sc. }future-oriented + admonitory) discourse contexts.

This chapter has shown that the interpretation of \textit{bambai} is highly context-dependent. Where $ q $ isn't presumed settled in a discourse context $ \mathcal D $, an utterance of the form \textit{bambai $ q $}  asserts that $ q $ could happen (in a $ \mathcal D $-provided modal base and as a consequence of the non-obtention of some $ \mathcal D $-salient eventuality) and conventionally implicates that $ q $ would undesirable. Drawing on these observations, chapter \ref{bambai.semx} proposes a lexical entry for \textit{bambai} which unifies its two distinct readings --- \textit{viz.} \textsc{subsequentiality} and \textsc{apprehensionality}.
%In (\getref{nachher}), we saw how the expressive content of \textit{nachher} appears to be not-at-issue: Pott's ``nondisplaceability'' criterion for identifying use-conditional semantic content.






\chapter{A semantics for \textit{bambai}}\label{bambai.semx}
This section seeks to provide a semantics for Kriol \textit{bambai} that unifies the available \textsc{subsequential} and \textsc{apprehenensional} readings discussed above and explains how a given reading is privileged in particular linguistic contexts. Figure \ref{bb-dist} is repeated here for reference.


\begin{figure}[h]\caption{Possible readings of \textit{bambai}}\centering
	\begin{tikzpicture}[align=center]
		\tikzset{level distance=75pt}
		\Tree [.\textit{\textbf{bambai}} [.{\textsc{subsequential}\\`then'} ] [.{\textsc{apprehensional}\\$ \blacklozenge $} {\textit{\textsc{Precautioning}}\\`lest, otherwise'} [.{\textit{\textsc{Apprehensive}}\\`possibly'} ] ] ]
\end{tikzpicture}\end{figure}



 In order to settle on a unified semantics, we assume a version of a Kratzerian treatment of modal operators  (\textit{e.g.}, \citet{Kratzer1977,Kratzer1981} \textit{et seq.}, an overview provided in \S~\ref{LitRev} above.) The primary insight of Kratzer's treatment is that modal expressions are lexically underspecified for modal ``flavour''; different readings emerging as a consequence of a contextually-provided conversational background \citetext{see also \citealp[1490\textit{ff}]{Hacquard2011} for an overview.}

	\section{Subsequentiality}\label{bambai.subseq}
	
	In §~\ref{dataStfa}, we saw how Kriol has retained the temporal frame uses of \textit{bambai} derived from archaism `by-and-by.' For \citet{Dowty1979,Dowty1982}, time adverbials are taken to denote predicates of times/sets of temporal intervals --- that is, the set of all those intervals that intersect with the interval specified by the adverb (\nextx).
	
	\pex A lexical entry for the (indexical) TFA \textit{today} (adapted from \citealt[328]{Dowty1979}, cited in \citealp[43]{Ogihara1996})
	
$ 	\denote[c]{\textit{today}}=\lambda \mathit P_{\langle\imath,t\rangle}\exists t_\imath[t\subseteq\text{today}'\wedge \mathit P(t)] $	

\textit{today} holds of some property of times \textit{P} $ \in\mathcal D_{\langle\imath,t\rangle} $ if there there is some time $ t $ at which \textit{P} holds which is a subinterval of the day-of-utterance ($ \mathrm{today}' $ is an interval supplied by context --- \textit{viz.} the timespan of the day in which utterance time ($ t* $) is located.)

	\xe
	
	 A frame adverbial, then, takes a predicate and says that its instantiation is contained within a given temporal interval.\footnote{The term ``temporal frame adverbial'' due to \citealp{Bennett}, and  equivalent to \citeauthor{Kamp1993}'s ``locating adverbial'' \citeyearpar[613]{Kamp1993}.} Following assumptions made by \citet[238\textit{ff}]{Kamp1971} and \citet[115]{Johnson1977}, \citet[29\textit{ff}]{Dowty1982} sees fit to appeal to a notion of truth which is relativised to an index containing two intervals of time. These roughly correspond to the notions of \textit{reference time} and \textit{speech time} familiar from \citet{Reichenbach1947}. I will use $ t* $ and $ t_r $ to refer to each of these.
	
	
	 As we saw, the function of (what I have referred to as) the {\sc subsequentiality} class of frame adverbials is to effect the constrained forward-displacement of the reference time of their prejacents with respect to some contextually-provided reference time. (\getref{ssqIsem}) represents a proposal to capture this relation.
	
	
	\pex \deftagex{ssqIsem}\textbf{\textsc{Subsequential instantiation}} \\$\text{{\sc subseq}}(p,t_r,w)\underset{\text{def}}=\exists t^\prime:t_r\prec t'\wedge P(t^\prime)(w)\wedge\mu(t_r,t^\prime)\leq s_c$
	
	A subsequentiality relation {\sc subseq} holds between a predicate $P$, reference time $t_r$ and reference world $w$ iff $P$ holds in $w$ at some time $t^\prime$ that follows $t_r$.\\Additionally, it constrains the temporal distance $\mu(t_r,t^\prime)$ between reference and event time to some value below a contextually-provided standard of `soon-ness' $s_c$.\xe
	

The relation between a contextually-provided standard and measure function $\mu(t_1,t_2)$ analysis builds in a truth-condition that captures variable intuitions about the falsity of subsequential claims in context (\nextx-\anextx).\footnote{Given that $\mathcal T$ is isomorphic with $ \mathbb R $, formally $\boldsymbol\mu:\mathcal T^2\to \mathbb  R$ represents a Lebesgue measure function that maps any interval $ [t_1,t_2] $ to its length $t_2-t_1$.}

\pex\a The birth of Cain succeeded Eve's pregnancy by some contextually inappropriate length of time (\textit{e.g.,} ninety years.)\\
$ ^{\mathbb F} $\textit{Eve fell pregnant then shortly afterwards gave birth to a son}
\a\begingl\glpreamble\textbf{Context.} Dad went to the shop on Monday and returned to make lunch the following week.//
\gla \ljudge{$ ^{\mathbb F} $}main dedi bin go la det shop, \textbf{bambai} im\textdblhyphen{in} gugum dina//
\glb my father {\sc pst} go {\sc loc} the shop \textit{\textbf{bambai}} 3s\textdblhyphen{}{\sc pst} cook dinner {\sc purp} 1p{\sc.excl}//
\glft`My dad went to the shop, \textbf{then} he made lunch' \hspace*{\fill}[AJ~23022017]//
\endgl\xe


\noindent That is, the category of ``subsequential'' TFAs makes explicit reference to a time provided by the discourse context (e.g., identified with the instantiation time of a previous clause.) The assertion of a relation between this reference time and the instantiation of the prejacent is a component of these items' semantics.

 An additional advantage is that, in appealing to a pragmatically retrieved standard for subsequentials, we allow for faultless disagreement between interlocutors, in case speaker and addressee retrieve divergent standards of soonness from the discourse context (as in (\getfullref{futurama}) below).\footnote{The term \textit{faultless disagreement} due to \citet[53-4]{Kolbel2004}, where the nature of the disagreement does not concern a matter of fact. That is, two participants \textsc{a,b} are in a situation where \textsc{a} believes (judges) $ p $ and \textsc{b} believes $ \neg p $ yet neither has made a mistake (is ``at fault''.)}

\pex[labelwidth=3.5em] \textbf{\textsc{context.}} Glurmo is leading the Planet Express Crew on a tour of the Slurm (a popular beverage) factory. Fry is thirsty and inquires about when he'll be able to get a drink.
\a[label=\textbf{Fry}] When will that be?\deftagex{futurama}
\a[label=\textbf{Glurmo}] Soon enough.
\a[label=\textbf{Fry}] That's not soon enough.\trailingcitation{(\href{https://youtu.be/tTQn0OwlAgA}{`Fry and the Slurm Factory', \textit{Futurama 1e}13})}
\xe

In (\getref{futurama}), Fry's utterance is compatible with a situation in which he and Glurmo agree on the event time (\textit{e.g.}, $ t_e  =$ \textsc{that evening at 8pm}, at which the party with Slurms McKenzie will begin). The source of their disagreement appears to be the value of the contextual standard $ (s_c) $ that each of them retrieves, and whether the distance between utterance time and $ t_e $ gets to count as `soon'.
	
	In its capacity as a TFA then, \textit{bambai} can be thought of as realising a subsequential instantiation relation, as shown in (\nextx) below.
	
\pex \textbf{Lexical entry for \textit{bambai} (\textsc{tfa})}

	$\denote[c]{\textit{bambai}}\underset{\text{def}}{=}\lambda P.\text{{\sc subseq}}(P,t_r,w)$

\textit{bambai} asserts that the property described by its prejacent $ (P) $ stands in a {\sc subseq} relation with a time and world provided by the discourse context.
\xe
	
	
	\section{`Settledness' \& intensionalisation}\label{modSems}
	
A primary motivation for the current work is to better understand the linguistic reflex that underpins the availability of apprehensional/apprehensive-modality readings of \textit{bambai}. The TFA treatment formalised in the subsection above fails to capture this readings, although, as I will show, provides an essential condition for understanding \textit{bambai}'s synchronic semantics and diachronic trajectory.

In §~\ref{LitRev} above, the notion of \textbf{settledness} was introduced, as deployed by \citet{Condoravdi2002} (and \citealp{Kaufmann2005}) using $ \mathcal{W\times T} $ frames, where it is cast as derived from the concept of \textit{historical necessity} \citep{Thomason1970}. 

%a timeline --- formally a world-time pair $\langle w,t\rangle\in\mathcal{W\times T}$ -- is settled just in case it is a fact in the evaluation world that a particular inquiry resolves at a given time. As was shown in §\ref{dataS}, for present purposes this is when \textit{bambai} is making a claim about the instantiation of its prejacent in either (a) the future of speech time ($t_e\succ$ \textbf{now}) or (b) an alternative history ($w_p\not\simeq w*$).

Settledness/historical necessity is normally expressed in terms of \textbf{historical alternatives}. This refers to the notion of equivalence classes $\boldsymbol{(\approx_{t}}\subseteq\mathcal{W\times W}$ of possible worlds: those worlds which have identical `histories' up to and including a reference time $t$. The properties of the \textit{historical alternative} relation are given in (\nextx) and, on the basis of this, a formal definition of settledness is given as (\anextx).

\pex\textbf{Historical alternatives }$\boldsymbol\approx\,\subset\mathcal{T\times W\times W}$\deftagex{histaltdef.bb}
\a $\forall t\in\mathcal T[\approx_t\text{ is an equivalence relation}]$\\
All world-pairs in $\approx_t$ (where $ t $ is an arbitrary time) have identical pasts up to that time.\\Their futures may diverge.\\
The relation is symmetric, transitive and reflexive (\textit{i.e.}, an equivalence relation).
\a\textbf{monotonicity}\deftaglabel{mono}

$ \forall w,w',t,t'\big[(w\approx_t w'\wedge t'\prec t)\to w\simeq_{t'} w'\big]$\\
Two worlds that are historical alternatives at $t$ are historical alternatives at all preceding times $t'$.\\That is, they can only differ with respect to their futures.\trailingcitation{\citep[146]{Thomason1984}}
\xe

Formally then, the truth value of proposition $ p $ is settled at $ t $ iff it is uniformly true or false at all historical alternatives to $ w $ at $ t $. Also shown in \S~\ref{BT-review}, Condoravdi and Kaufmann \textit{i.a.} additionally derive a related property, \textit{viz.} \textsc{presumed settledness/decidedness} repeated here as (\getref{presumption-bambai}). The presumption of settled is effectively understood to be a relation between a discourse context and a predicate (or proposition). Following standard pragmatic assumptions, the \textit{common ground} (\textbf{\textit{cg}}) represents the set of propositions taken to be mutually understood by participants in a discourse context (see \getfullref{CondoravdiSett.cg}). The intersection of these propositions $ (\cap cg )$ --- the \textit{context set} --- is modelled as the set of worlds that is compatible with the \textit{cg} (those worlds in which all propositions in the common ground are true.)

\ex \textbf{Presumption of settledness for \textit{P}}\deftagex{presumption-bambai}.


$\forall w^\prime:w^\prime\in\cap cg,\forall w^{\prime\prime}:w^\prime\approx_{t*}w^{\prime\prime}:\\
\textsc{at}\big([t*,\_),w',\mathit{P}\big)\leftrightarrow \textsc{at}\big([t*,\_),w'',\mathit{P}\big)$
\trailingcitation{\citep[82]{Condoravdi2002}}


A property \textit{P} is presumed settled if it uniformly holds or does not hold in all historic alternatives to worlds compatible with the discourse participants' beliefs.%\footnote{The $AT$ relation holds between a time, world and an eventive property iff $\exists e[P(w)(e)\&\tau(e,w)\subseteq t]$ --- \textit{i.e.}, if the event's runtime is a subinterval of $t$ in $w$ (Condoravdi 2002:70). This can accomodate stative and temporal properties with minor adjustments (see \textit{ibid.}). For the sake of perpescuity, I abstract away from (davidsonian) event variables in this section.}

\xe

As indicated in \S~\ref{bambai.dia}, in this dissertation I defend a claim that the modalised meaning component of apprehensional \textit{bambai} arises as a consequence of a dia\-chronically-conventionalised implicature where \textbf{a claim that \textsc{subseq} holds of a predicate} encodes a \textbf{prediction} when that predicate is interpreted as nonfactual (compare \S~\ref{subseq-sync}). This explains the \textit{``epistemic downtoning''} function which characterises apprehensionals on \citeauthor{Lichtenberk1995}'s description \citeyearpar{Lichtenberk1995}.

Specifically, given notions of \textsc{relevance} (\textit{e.g.}, Horn's $ \mathcal R $-principle \textsc{``Say no more than you must''} \citeyearpar[13]{Horn1984}, an utterance of \textit{bambai $P$} licenses the (speaker-based) implicature that the Speaker is basing a predication (specifically an premonitory one, \textit{cf.} §~\ref{bambai.dia}) about some unsettled eventuality on its possible truth in view of (perceived compatibility with) a the set of facts that they know of the world. The locus of this implicature is that the Speaker can rely on her hearer's knowledge of the world to reason that an unsettled subsequentiality predication has the valence of a prediction.


Appealing to a Kratzerian framework, we can modalise our entry for \textit{bambai} in order to capture the ``epistemic downtoning'' effect associated with apprehensionals. A principal component (and advantage) of Kratzer's treatment of modals \citeyearpar{Kratzer1977,Kratzer1981,Kratzer2012} lies in the claim that the interpretation of modalised propositions relies on `conversational backgrounds': that they quantify over sets of worlds retrieved by an `accessibility relation' which is \textit{contextually} made available. The entry in (\nextx) gives an intensionalised (modal) semantics for \textit{bambai}.%todoo \marginnote{dregli, streidaway}

\pex \textbf{\textit{bambai} includes a modal expression}

$\denote[c]{\textit{bambai}}=\lambda P.\exists w\,^\prime\big[w'\in\underset{o(w)}{\textsc{best}}(\cap m(w))\wedge\text{{\sc subseq}}(P,t_r,w\,^\prime)\big]$

\textit{bambai} asserts that there exists some world $w^\prime$ in a set of worlds that are optimal with respect to a contextually-determined modal base $m$ and ordering source $o$ in the reference context $c=\langle t*,t_r,w*\rangle$. It additionally asserts that the {\sc subsequential instantiation} relation (as defined in (\getref{ssqIsem}) above) holds between that world $w^\prime$, the prejacent $P$, and a reference time provided by the utterance context $t_r$.\deftagex{bb-semx}

\xe


With the entry in (\lastx), we can formalise the intuition that, when (and only when) \textit{bambai $p$} is understood as making a nonfactual predication, it constitutes a prediction of a possible --- but unverified/unverifiable --- subsequential state-of-affairs; that is, one that is presumed unsettled. 


As a consequence, the apparent subsequential/apprehensional polysemy exhibited by \textit{bambai} is modelled as deriving form a single core meaning, where different contexts make different conversational backgrounds available \citep[\textit{cf.}][55\textit{ff}]{Kratzer2012}. We can conceive of this in terms of a pragmatically-enforced \textsc{omniscience restriction.}
%We model this by claiming that, \textit{bambai} is compatible with a range of conversational backgrounds.


\section[The omniscience restriction]{A pragmatic ambiguity:\\\it The omniscience restriction}


 Crucially, in the apprehensional cases we've seen, \textit{bambai}'s prejacent is understood to encode a predication about an unsettled state of affairs. That is, it involves reference (by means of existential quantification) to either • some time  succeeding utterance time $ t'\notin\cap{\preccurlyeq_{t*}} $ (the indicative cases) \textsc{or} • some world that is not a historic alternative of the actual world $w'\notin\cap{\approx_{t*}}w* $ (the subjunctive cases.) These two types of contexts can be unified as involving a \textsc{non-actual}/\textsc{nonfactual} predication --- one without the presumption of settledness. Recalling the discussion of branching-time models in \S~\ref{BT-review}, the non-actual property can be easily stated over indices as $ \{i'\mid i'\not\preccurlyeq i*\} $.\footnote{See also the \citeauthor{Rumberg2016a}/\citeauthor{VonPrince2019} partition in (\getref{trichot}).} In Kriol, the prejacent of \textit{bambai} is interpreted as actual if{f} \textit{bin}/past marking is present (and \textit{bina}/explicit counterfactual marking is absent.) These contexts were summarised in Table \ref{triggers} (\textit{p.}~\pageref{triggers} above.)
 
% (perhaps by one of the operators presented in Table \ref{triggers} (\textit{p.}\pageref{triggers} above)
 

The \textit{omniscience restriction}, also described in (\getref{omni1}) is a pragmatic principle implementing the \textsc{actual/nonactual} distinction to explain the distribution of \textsc{subsequential} vs. \textsc{apprehensional} \textit{bambai}.
\pex \textbf{The omniscience restriction.}
Predications of subsequentiality (posterior instantiation) are interpreted as carrying predictive illocutionary force (\textit{i.e.}, modalised or ``epistemically downtoned'') when they are presumed unsettled.
\xe

The idea here is that a speaker who makes a predication about the temporal properties of a non-settled eventuality cannot reasonably make an assertion that appears to presume its settledness.
Such an operation would require the participants to be able to retrieve all propositions that are true in and characteristic of worlds with respect to a vantage point in the future or to be able to calculate all the ramifying consequences of eventualities that might have obtained in the past (in the case of counterfactual uses.)


This restriction reflects a pragmatic reflex of Condoravdi's \citeyearpar[83]{Condoravdi2002}  diversity condition\footnote{That is, a property holding between properties \textit{P} and modal bases $ m:\mathcal{W\times T\to\wp(\mathcal W)} $ that they be unsettled w/r/t the instantiation of \textit{P} \citep[83]{Condoravdi2002}:
$$\exists w\big[w\in\textit{cg}\wedge\exists w',w''[w',w''\in m(w,t)\wedge\textsc{at}\big([t,\infty),w',P\big)\wedge\neg \textsc{at}\big([t,\infty),w',P\big)]\big]$$} and the twin epistemic constraints on the relations between doxa and settledness given in \citealt{Kaufmann2002,Kaufmann2005,Kaufmann2006} (\textit{viz.} historicity/lack of foreknowledge), axioms which guarantee that ``only what is settled can already be known'' \citep[101]{Kaufmann2006}. Consider again the truth conditions of \textit{bambai} in (\getref{bambai.bt}) with the \textsc{subseq} relation spelled out. The entry in (\getref{bambai.bt}) is translated into a branching-times formalism in order to draw the parallel treatment of ``indicative'' and ``subjunctive'' uses of \textit{bambai}. The relevant modelling assumptions were introduced in \S~\ref{BT-review}.

\pex\deftagex{bambai.bt}
$ \denote[c]{\textit{bambai}}=\lambda\textit{P}.\exists b
\big[b\in\underset{o(i)}{\textsc{best}}(\cap\!\approx_i^+)\wedge \exists i'_b[i'\succsim i_r\wedge P(i')\wedge\mu(i_r,i')\leq s_c]\big] $
\textit{bambai} asserts that $ \mathit P $ is instantiated at some index $ i' $ which is \textbf{posterior} (temporally subsequent) to some contextually-retrieved reference index $ i_r $ according to some branch that is metaphysically accessible from $ i $.\footnote{Additionally there may be contextually-derived additional restrictions on the modal base, hence $ \boldsymbol{\approx^+} $, following the notational convention ($ f^+(w) $) introduced by \citet{Kratzer1981} in modelling conditionals.}

\xe

\begin{table}\centering		\def\arraystretch{1.15}
	\caption{\textit{bambai} clauses relate three semantical indices: the instantation time of the prejacent $ (i') $, the utterance index $ (i*) $ and a contextually-retrieved reference index $ (i_r) $. \textit{bambai} requires that $ i_r\prec i* $}\label{indices}\marginnote{something's up here right, r can't really be a branchmate of i', it's just some salient anterior index}
	\begin{tabular}{ll>{$}l<{$}>{\small}ll}
		\hline
		\multicolumn{2}{c}{\textsc{\textbf{function}}} & \textit{\textbf{relations}} & \multicolumn{1}{c}{Text} \\
		\hline\rowcolor{gray!10}
		\multicolumn{2}{c}{a.~\textsc{subseq}} &i_r\prec\color{Green}\boldsymbol{i'\preccurlyeq i*}&`I had coffee$ _{i_r} $ \textbf{then} fell asleep'$ _{i'\prec i*} $&\\
		\parbox[t]{2mm}{\multirow{2}{*}[-1em]{\rotatebox[origin=c]{90}{\textsc{a  p  p  r}}}}& \multirow{1}{*}{b.~\textsc{indic}} &i*\prec i_r\prec i'&`I'll have coffee$ _{i_r} $ \textbf{otherwise} may fall asleep'$ _{i'\succ i*} $&\\\\
		& \multirow{1}{*}[1em]{c.~\textsc{sbjv}} &\begin{tikzpicture}
			\path [decorate,decoration={text along path,
				text={≺ {$ i' $} }}]
			(.25,.2) sin (1,.4) ;
			\path [decorate,decoration={text along path,
				text={≺ {$i* $}}}]
			(.25,-.25) sin (1,-.4) ;
			\node (0,0) {{$i_r$}} ;
		\end{tikzpicture}&\multirow{1}{*}[1em]{`I had coffee$ _{i_r} $ \textbf{otherwise} may've fallen asleep'$ _{i'\nprec i*} $}&\\
		\hline
	\end{tabular}
\end{table}



  





This condition allows us to unify the modalised and non-modalised readings of \textit{bambai} --- in view of the constraints discussed above, retrieval of a proper reading for \textit{bambai} in a given context is a function of the relation between evaluation indices. Summarised in table \ref{indices}, a subsequential reading obtains \textit{only if} the instantiation of the prejacent is \textsc{actual} w/r/t the utterance index --- that is \textit{bambai} receives its \textit{subsequential} reading/\textit{apprehensionality} ``fails to emerge'' when $ i'\preccurlyeq i* $.

Conversely, if the prejacent's instantiation index $( i' )$ is understood to be \textbf{\textit{posterior }}to $ i* $, a subsequentiality claim is subject to the omniscience restriction.

\marginnote{worth giving a version of (\getref{histaltdef.bb}) recast for BTs here? A version is given in ch.1}This can be modelled by assuming that context provides a species of \textit{metaphysical} (circumstantial) modal base. Recall, among the ontological metaphysical assumptions reflected in branching-times structures is \textit{left linearity} (\getref{ll-def}) --- representing historical necessity --- and \textit{right branching}, reflecting the problem of future contingency. It will be a property, then, of all metaphysical conversational backgrounds, that all branches undivided at $ i_n $ will also be undivided at $ i_{n-1} $ ${(\therefore\mathtt{B_\mathnormal{i_n}\subseteq B_{\mathnormal{i_{n-1}}}})}$\footnote{See \citet[79-80]{Rumberg2016a} for a proof of this theorem.}

\pex \textbf{The structure of the modal base}
\a \textit{Undividedness-at-$ \boldsymbol i $} \trailingcitation{\citep{Rumberg2016a,Muller2014}}

$\boldsymbol{ b \equiv_i b' }\triangleq \exists i'[i'\succ i\wedge i'\in b\cap b']$\\
\ul{That is:} two branches are undivided at some index $ i $ iff they both run through some successor index $ i' $.
\a A metaphysical modal base ($ \boldsymbol{\approx} $) contains all metaphysically possible propositions at an evaluation index $ i $.
\a Metaphysical modal bases therefore assume actualness/fixity of the past. 

$ \forall i,j\big[i\succcurlyeq j\to\forall b,b'[b\underset{i}{\approx} b'\to b\underset{j}{\equiv} b']\big] $\trailingcitation{(compare \getref{histaltdef}/\getfullref{histaltdef.bb.mono})}

\ul{That is:} metaphysically-accessible branches are undivided at any evaluation index $ i $ and at all indices preceding that evaluation index.
\xe

Shown above (\textit{e.g.}, table \ref{indices}), subsequential readings of \textit{bambai} are limited to contexts where instantiation time is taken to precede utterance time. Against a metaphysical modal base then, the instantiation of the prejacent is presumed settled at utterance time (\nextx).

\pex Assuming that the past morphology restricts instantiation of \textit{P} (\textit{e.g.}, that property described in \textit{bambai}'s prejacent) to $ \{i'\mid i'\prec i*\} $:

$ \forall b\in\cap\textit{cg}_{i*}\Big[i'\in b\wedge\textit{P}(i')]\to\forall b'\big[b'\underset{i*}{\approx} b\to[i'\in b'\wedge\textit{P}(i')]\big]\Big]$

All branches $ b $ that are compatible with the common ground are such that if \textit{P} at $ i' $ is true, then it is metaphysically necessary (\textit{i.e.}, holds at all historical alternatives to $ b $.)
\xe

Conversely, in the absence of past morphology, no such restriction is made on the instantiation index of \textit{P}: the modal base can therefore be \textit{diverse}: the truth (or falsity) of \textit{P$ (i) $} is contingent/unsettled with respect to $ P(i) $ --- that is, the common ground is compatible with branches at which \textit{P} is settled differently (\textit{i.e.}, (\lastx) is not valid if $ i'\not\preccurlyeq i* $).%s-- \textit{i.e.}, $ \neg(\cap\!\approx\to\textit{P}(i') $.

%\marginnote{table summarising the effects of $ i $ orientation on the modal base? Or a BT figure?}

This is implemented more precisely in the following sections.

\section{Deriving the subsequential reading}
What we've called the \textit{subsequential} (TFA) use of \textit{bambai} follows from general norms of assertion: given that the speaker is making a predication about a property that is presumed settled, her context set is understood as veridical and the assertion is taken to be factual --- \textit{cf.} Grice's (super)maxim of quality: ``try to make your contribution one that is true'' \citeyearpar[27]{Grice1991}.


As shown above, given the notion of historical necessity/the left-linearity of branching models of time, an evaluation index is associated with a unique past.

%In these cases the intensional contribution of \textit{bambai} can be captured by claiming that it quantifies (trivially) over a \textit{metaphysical} modal base and an empty ordering source \citep[see ][]{Kratzer2012}.)\footnote{In her treatment of Marathi present tense marking, \citet{Deo2017} makes similar appeal to veridical vs. nonveridical conversational backgrounds to capture ostensible polysemy associated with these (present-tense) forms.}

\pex \textbf{A veridical conversational background:\\ \textit{bambai}'s subsequential reading}
\a \textit{A metaphysical modal base} $ \boldsymbol{\underset{\text{meta}}{m}/\approx }$

A metaphysical modal base $ \approx$ is a function from indices to a set of propositions that are \textbf{consistent} with metaphysical assumptions about the state of the world at a given index $ i $.

Consequently, the intersection of these propositions: $ \cap\!\approx_i$ returns the set of \textbf{historical branching alternatives} to $ i $ --- a set of branches that share $ i $'s history and branch into its future (while according with metaphysical notions of possibility.s)
\a %The empt
$ o_{\text{empty}}(w)=\varnothing$\\
An empty ordering source $ o_\text{empty}$ contains no content (propositions) and hence induces no ordering over the modal base.
%\{p\mid w\in p\}$\\
%A totally realistic ordering source $ g_{\text{real}_{\text{total}}} $ is a set of propositions that uniquely characterise $ w $.

%$ g_{\text{real}_{\text{total}}}$ then induces an ordering $\leqslant_{g(w)}$ on the modal base:\mcom{it could just be empty right? this doesn't really add anything in particular?}

%\hspace{-.45cm}$\small\forall w^\prime,w^{\prime\prime}\in\bigcap f_\text{meta}(w)(t)\mid w^\prime\boldsymbol{\leqslant_{g(w)}}w^{\prime\prime}\leftrightarrow\big\{p:p\in\{w\}\wedge w''\in p\big\}\subseteq\big\{p\mid p\in\{w\}\wedge w^\prime\in p\big\}$

%A world $ w' $ is ``better than'' $ w'' $ according to $ g_{\text{real}_\text{total}}(w) $ is more of the propositions that characterise $ w $ are true in $ w' $ than in $ w'' $.


\a Because the ordering source is empty, the function \textbf{\textsc{best}}$_{\varnothing}\big(\cap\!\approx_i\big)$ simply returns $ \cap\!\approx_i $: the set of historic/branching alternatives to $ i $.
\xe

By the (BT-adaptation of \citeauthor{Thomason1970}'s) definition in (\getref{histaltdef.bb}), historical alternatives have ``identical pasts'' to one another and to the evaluation index $ i* $. In the relevant sense, then, the quantification is trivial. With/respect to some $ i':i'\prec i* $, all branches in the modal base are undivided-at-$ i' $. This is shown in the shaded portion of the BT diagram of $ \cap\!\approx_{i*} $ in fig. \ref{ssq-BT}.


\begin{figure}[h]\centering
	\caption{A possible representation of $ \cap\!\approx_{i*} $: a ``subtree'' of $ \mathfrak T $.\\[.25em]\textit{\textbf{shaded portion. }}All metaphysically accessible branches are undivided at indices preceding $ i* $.}\label{ssq-BT}
		
	\begin{tikzpicture}
	[scale=2,level distance=9mm,
	every node/.style={fill=black,circle,inner sep=1.5pt},
	level 1/.style={sibling distance=10mm},
	level 2/.style={sibling distance=8mm},
	level 3/.style={sibling distance=4mm},
	level 4/.style={sibling distance=2mm},
	edge from parent/.style={draw}]
	\node[label=above:$ i_r $] {} [grow=right]
	child {node[label=above:$ i' $] {} edge from parent[thick, double]
		child {node [style={fill=red},label=above:$ \boldsymbol{i*} $] {}  edge from parent[thick, double]
			child {node {} edge from parent[densely dashed]
				child {node[label=right:{\scriptsize$ b_1 $}] {} edge from parent[thin, tips ,->] }
				child {node[label=right:{\scriptsize$ b_2 $}] {} edge from parent[thin, tips ,->] } }
			child {node {} edge from parent[densely dashed]
				child {node[label=right:{\scriptsize$ b_3 $}] {} edge from parent[thin, tips ,->]}
				child {node[label=right:{\scriptsize$ b_4 $}] {} edge from parent[thin, tips ,->] }}}};
	\fill[very nearly transparent] (-.2,-.5) rectangle (2,.5);
\end{tikzpicture}\end{figure}




\noindent This is derived for (\nextx) below (the sentence simplified from (\getfullref{ssq0}) above). The derivation is further explicated below.

\pex\textbf{Deriving the subsequential reading}\\
\deftagex{ssq-deriv}\begingl
\gla main dedi bin go la det shop, \textbf{bambai} im\textdblhyphen{in} gugum dina//
\glb my father {\sc pst} go {\sc loc} the shop \textit{\textbf{bambai}} 3s\textdblhyphen{}{\sc pst} cook dinner {\sc purp} 1p{\sc.excl}//
\glft`My dad went to the shop, \textbf{then} he made lunch' \hspace*{\fill}[AJ~23022017]//
\endgl
\deftagex{meta}
%\a\textsc{pst}(\denote[c]{\textit{main dedi go la det shop}})$\leftrightarrow \exists t'\prec t\!*\wedge \textsc{go.shopping}(t')(w) $\\
\a \textbf{Taking \textit{bin }\textsc{`past'} to restrict $\boldsymbol i $ to before speech time $\boldsymbol{ i\!*}$}\deftaglabel{pst}

 $\denote[c]{\textit{bin}}= \lambda\textit{P}\lambda i.i\prec i*\wedge\textit{P}(i)$\\
 %\textsc{pst}=\lambda i:i\prec i\!*.i$\\
 \textit{bin} realises `\textsc{pst}' --- a past tense operator which restricts the instantation time to some index $ i $ that precedes the speech index $ i* $.
%A partial identity function over indices, valued only for indices anterior to $ i $. 
 

\a \textbf{Meaning of the first clause}\deftaglabel{p}

\begin{align*}
\denote[c]{\textit{bin}}(\denote[c]{\textit{main dedi go la det shop}})=&\lambda\textit{P}\lambda i.i\prec i*\wedge\textit{P}(i)\big(\lambda i'.\textsc{dad.go.shopping}(i')\big)\\
\denote[c]{\textit{main dedi bin go la det shop}}=&\lambda i.i\prec i\!*\wedge \textsc{dad.go.shopping}(i)%\lambda i.\textsc{go.shopping}(i) \\
\end{align*}
%Defined only if $ i\prec i*  $, the first clause asserts that the event of Dad's trip to the shop occurs at some contextually-retrieved index $ i $.
$ i $ is then existentially bound \citep{Dowty1979,Stump1985,Ogihara1996}. The first clause, then, asserts that the event of Dad's trip to the shop occurs at some index that precedes the utterance index --- I'll call this index $ j $.
$$ \denote[c]{\textit{main dedi bin go la det shop}}=\exists j[j\prec i\!*\wedge \textsc{dad.go.shopping}(j)] $$




\a \textbf{Meaning of \textit{bambai} \& assignment of $ \boldsymbol{i_r} $}\deftaglabel{r}
$$	\denote{\textit{bambai}}=\lambda\textit{P}.\exists b
	\big[b\in\underset{o(w)}{\textsc{best}}\big(\cap m(i*)\big)\wedge \exists i'_b[i'\succ i_r\wedge P(i')\wedge\mu(i_r,i')\leq s_c]\big]$$
	$ j $ is assigned to $ i_r $, per standard assumptions about temporal anaphora \citetext{\textit{e.g.}, \citealt{Hinrihcs1986,Partee}, these insights have been implemented in DRT frameworks \S~\ref{bambai.subord}, see chapter 5 of \citealt{Kamp1993}.}
$$	\denote[c]{\textit{bambai}}=\lambda\textit{P}.\exists b
	\big[b\in\underset{o(w)}{\textsc{best}}\big(\cap m(i*)\big)\wedge \exists i'_b[i'\succ j\wedge P(i')\wedge\mu(j,i')\leq s_c]\big]$$
\a\deftaglabel{q} \textbf{Meaning of the second clause (\textit{bambai}'s prejacent)}

%\begin{align*}\denote[c]{\textit{im gugum dina}}(\textsc{pst})=&\lambda i.\textsc{make.lunch}(i') \\	
$\denote[c]{\textit{imin gugum dina}}=\lambda i.i'\prec i\!*\wedge\ \textsc{dad.make.lunch}(i')$
%\end{align*}
% \a\deftaglabel{q} 
% $\textsc{pst}(\denote[c]{\textit{im gugum dina}})\leftrightarrow\exists t''\prec t\!*\wedge \textsc{make.lunch}(t'')(w) $
\a \textbf{Substitution of prejacent (\getref{meta.q})}\deftaglabel{bbq}
\begin{align*}
 	\denote[c]{\textit{bambai \textup{(\getref{meta.q})}}}=&\exists b\big[b\in\underset{\varnothing}{\textsc{best}}(\cap\!\approx_{i*})\wedge \exists i'_b[i'\succ j\wedge \lambda i'.i'\prec i*\wedge\textsc{make.lunch}(i')\wedge\mu(j,i')\leq s_c]\big]\\
	=&\exists b\big[b\in\underset{\varnothing}{\textsc{best}}(\cap\!\approx_{i*})\wedge \exists i'_b[i'\succ i_r\wedge \textsc{subseq}\Big(\lambda i.i'\prec i*\wedge\textsc{dad.make.lunch}(i'),j\Big)]
\end{align*}\marginnote{the binding is not working here, moving on. i probably need to substitute something in, maybe bambai should still be a relation bw indices and \textbf{propositions} instead of temp abstracts.}

%\a \textbf{substitution of conversational backgrounds $\boldsymbol{m,o}$}
%\begin{align*}
%\denote[c]{\textit{bambai imin gugum dina}}&=:t''\prec t*.\exists w'\big[w'\in\textbf{best}_\varnothing(m_{\text{meta}},t_r,w\!*)\\&\wedge\textsc{subseq}\Big(\big(\textsc{make.lunch}(t'')(w)\big),t_r,w  \Big)\big]
%\end{align*}
%
%
%\textsc{make.lunch} is in the \textsc{subseq} relation with $ t_r $ in $ w' $ in a historical alternative$ _{t*} $ to $ w* $.
%\mcom{I really don't know what to put in an index and what to lambda-bind and what if any diff preds this makes. what's clear is that $ \box\lozenge t\neq t_r $}


%\a \textbf{Spelling out the \textsc{subsequential instantiation} relation (cf. \getref{ssqIsem})}\deftaglabel{inst}
%\begin{multline*}\denote[c]{\textit{bambai imin gugum dina}}=:t''\prec t_r.\exists w'\big[w'\in\textbf{best}_{\varnothing}(m_{\text{meta}},t\!*,w\!*)\\\wedge\exists t''[t_r\prec t''\wedge\textsc{make.lunch}(t'')(w')\wedge\mu(t_r,t'')\leq s_c]\big]\end{multline*}
%
%The \textsc{subseq} component of \textit{bambai}'s meaning further restricts the instantiation time $ (t'') $ of \textsc{make.lunch}: it asserts • that a contextually-retrieved reference time $ t_r $ precedes $ t'' $ and • that the temporal distance between those two times is below some contextual standard (``soonness'').


\xe


\noindent In (\getref{meta.p}-\getref{meta.r}]), the mechanism responsible for establishing the interclausal anaphoric relation between \textit{im} and \textit{main dedi} is similar to that which equates of $ i_r $ with the index at which Dad's \textsc{shopping} trip was instantiated: \textit{viz.} $ j $. As described in \S~\ref{bambai.subord}, in the Kampian/DRT terms \citep[\textit{e.g.},][Ch. 5]{Kamp1993} -- also adopted in, \textit{e.g.} \citealp{Partee} -- this relies on the notion of an expanding universe of discourse: modelled as sets of assignments.


Shown in (\getref{meta.bbq}), \textsc{make.lunch} is instantiated prior to the utterance index $ i* $; the modal component of \textit{bambai} involves quantification over a totally realistic conversational background. That is, given that the prejacent is predicated of a preceding index $ i'\prec i* $, all branches in the metaphysical modal base are undivided at $ \{i\mid i\preccurlyeq i*\} $ (fig. \ref{ssq-BT}). Because the \textsc{subseq} predication involves branchmates of $ i* $, it is interpreted as factual.


\section{Deriving the apprehensional reading}

In unsettled contexts, \textit{bambai}'s metaphysical modal base gives rise to a nonfactual/nonveridical conversational background. In view of pragmatic principles (the ``omniscience restriction''), the metaphysical alternatives are sorted by a ``stereotypical ordering source'' \citetext{\citealp[\textit{e.g.},][37\textit{ff}]{Kratzer2012} \textit{i.a.}.} 




%\mcom{It is essential to find some way of intersecting the (negation of???) the antecedent with the modal base otherwise this is literally just $\lozenge$. What we have here however does give \denote{bambai $P$}, we also need $\denote{Q bambai P}$}



\begin{figure}[h]\centering
	\caption{A possible representation of $ \cap\!\approx{i*} $: a ``subtree'' of $ \mathfrak T $.\\[.25em]\textit{\textbf{shaded portion. }}Multiple accessible branches (metaphysically ``possible futures'') succeeding $ i* $.}\label{ind-BT}
	
	\begin{tikzpicture}
		[scale=2,level distance=9mm,
		every node/.style={fill=black,circle,inner sep=1.5pt},
		level 1/.style={sibling distance=10mm},
		level 2/.style={sibling distance=8mm},
		level 3/.style={sibling distance=4mm},
		level 4/.style={sibling distance=2mm},
		edge from parent/.style={draw}]
		\node {} [grow=right]
		child {node {} edge from parent[thick, double]
			child {node [style={fill=red},label=above:$ \boldsymbol{i*} $] {}  edge from parent[thick, double]
				child {node {} edge from parent[densely dashed]
					child {node[label=right:{\scriptsize$ b_1 $}] {} edge from parent[thin, tips ,->] }
					child {node[label=right:{\scriptsize$ b_2 $}] {} edge from parent[thin, tips ,->] } }
				child {node[label=above:{$ i_r $}] {} edge from parent[densely dashed]
					child {node[label=right:{\scriptsize$ b_3 $}] {} edge from parent[thin, tips ,->]}
					child {node[label=right:{\scriptsize$ b_4 $},label=above:{$ i' $}] {} edge from parent[thin, tips ,->] }}}};
		\fill[very nearly transparent] (2,.65) rectangle (4,-.5);
\end{tikzpicture}\end{figure}



\pex\textbf{conversational background: \textit{bambai}'s modal-apprehensional reading}

%\a $f_\text{epist}(w)(t)=\{w^\prime\mid w^\prime\text{ is compatible with what S knows in $w$ at $t$\}}$
\a% $\bigcap \underset{\text{meta}}{m}(i)=\{i^\prime\mid i^\prime\approx i\}$
(As above) a metaphysical modal base $ \approx $ is a function that retrieves the of metaphysically possible branches from a given index.

\a $\mathfrak s(i)=\{p\mid p\text{ will hold in the `normal' course of events at }i\}$.

A sterotypical ordering source is a set of propositions that are ``normally true'' in $ w $/can be taken to hold in the ``normal course of events'' in $ w $ \citetext{\citealp[295]{Kratzer1981}, see \citealt{Yalcin2010} for discussion.}

\a A set of propositions $\mathfrak s(w)$ then induces an ordering $\leqslant_{{\mathfrak s}(w)}$ on the modal base:

%{\color{gray}\hspace{-.45cm}$\forall w^\prime,w^{\prime\prime}\in\bigcap f_\text{epist}(w)(t):w^\prime\boldsymbol{\leqslant_{g(w)}}w^{\prime\prime}\leftrightarrow\{p:p\in g(w)\wedge w^{\prime\prime}\in p\}\subseteq\{p:p\in g(w)\wedge w^\prime\in p\}$
%\hspace{.35cm}}

\begin{multline*}
\forall b',b''\in\cap\approx_i:b'\leqslant_{\mathfrak s(i)}b''\longleftrightarrow\\
\{p'\mid p'\in\mathfrak s(i)\wedge i'[i'\in b'\wedge p'(i')]\}\\
\text{\rotatebox{310}{$ \supseteq $}}\\
\{p''\mid p''\in\mathfrak s(i)\wedge i''[i''\in b''\wedge p''(i'')]\}
\end{multline*}

That is, $ b' $ is more normal (stereotypical) than $ b'' $ iff $ \mathfrak s(w) $ -- the propositions ``normally true given $ i $'' that are true of indices along  $ b' $ are a superset of those true of indices along $ b'' $.

%{\color{gray}For any worlds $w^\prime$ and $w^{\prime\prime}$, $w^\prime$ is `at least as close to an ideal' than $w^{\prime\prime}$ with respect to $_\text{s'typ}(w)$ (\textit{i.e.} it is at least as close `normal course of events') if all the propositions of $o(w)$ true in $w^{\prime\prime}$ are also true in $w^\prime$.}
\a $\underset{\mathfrak s(i)}{\textbf{\textsc{best}}}(\cap\!\approx_i)$ then returns just that subset of metaphysical alternative branches that are closest to what is judged to be a ``normally-unfolding course of events'' at $i$.
\xe


%\marginnote{I've written to cleo and have a number of things to work out/add on the choice of epistemic modal base, especially given the apparent problems this wi	ll pose for counterfactuals. This draws largely from \cite{Giannakidou2018}, while trying to harmonise this with observations made at the end of \citet{Condoravdi2002} (21feb email exch.)}


%\footnote{\textit{I.e.} \textit{`Do not say that for which you lack adequate evidence'} (Grice 1991: 27, a.o.)}. 
\noindent Armed with these assumptions, we can now derive the proper semantics for a ``precautioning'' use of \textit{bambai}, as in (\getref{app0}), repeated here as (\getref{app-deriv}).

\pex\deftagex{app-deriv}\textbf{Deriving the apprehensional reading}\\
\begingl
\gla ai\textdblhyphen{}rra dringgi kofi \textbf{bambai} mi gurrumuk (la desk iya gin)//
\glb 1s\textdblhyphen{\sc irr} drink coffee \textit{bambai} 1s fall.asleep {\sc loc} desk here {\sc emph}//
\glft `I'd better have a coffee otherwise I might pass out (right here on the desk)'\trailingcitation{[GT~28052016]}//\endgl

\a \textbf{\textit{(ga)rra} as a necessity modal}

Let's take \textit{garra} to instantiate the abstract (untensed) modal particle \textsc{woll}.\footnote{Semantics for \textsc{woll} adapted from \citet[71]{Condoravdi2002}).
	
	A satisfactory analysis of the semantics of \textit{garra} (glossed here as `\gls{irr}') is beyond the scope of this work. It is treated by \cite{Schultze-Berndt} as polysemous between a future and ``obligation'' marker, although I have also elicited tentative evidence of epistemic necessity readings. Abstracting away from these questions of modal flavour, it is treated here as a species of necessity modal and glossed as \gls{irr}.}

$ \denote{\textit{garra}} =\lambda\mathit P\lambda i\forall b\big[b\in\underset{o(i)}{\textsc{best}}\big(\cap\!m(i)\big)\to\exists ^bi'[i\succ i\wedge\mathit P(i')])$

\textit{garra} takes a predicate $ \textit{P} $ and an evaluation index $ i $ and asserts that $ \textit{P} $ holds at some successor of $ i $ in all of the best-according-to-\textit{o} worlds in the modal base.



\a \textbf{Meaning of the first clause}\deftaglabel{p} 

Without explicit tense marking, the (evaluation) index variable for $ i $ is identfied as the utterance index (this is represented as a covert \gls{npst} morpheme below, the alternative to \textit{bin} in \getfullref{meta.pst})


\begin{align*}
&\begin{aligned}\denote[c]{\textit{garra}}(\denote[c]{\textit{ai dringgi kofi}}) =&\lambda\mathit P\lambda i.\forall b'\big[b'\in\underset{o(i)}{\textsc{best}}\big(\cap\!m(i)\big)\\&\to\exists i'[i'\in b'\wedge i'\succ i\wedge\mathit P(i')]\big]~~\big(\lambda i'.\textsc{i.drink.coffee}(i')\big)\end{aligned}\\
&\begin{aligned}\textsc{npst}(\denote[c]{\textit{airra dringgi kofi}})=\lambda\textit{P}\lambda i.i=i\!*\wedge\textit{P}(i)\\
	\big(\lambda i.\forall b'\big[\underset{o(i)}{\textsc{best}}\big(\cap\!m(i)\big)&\to\exists i'[i'\in b'\wedge i'\succ i\wedge\textsc{i.drink.coffee}(i')]\big]\big) \end{aligned}\\
&\begin{aligned}	\denote[c]{\textit{airra dringgi kofi}}=\forall b'\big[\underset{\textit{tel}(i*)}{\textsc{best}}\big(\cap\underset{\textsc{circ}}{m(i*)}\big)&\to\exists i'[i'\in b'\wedge i'\succ i*\wedge\textsc{i.drink.coffee}(i')]\big]\end{aligned}
%\denote{\textit{garra}} =&\lambda\mathit P\lambda i\forall b'\big[b'\in\underset{o(i)}{\textsc{best}}\big(\cap\!m(i)\big)\to\exists i'[i'\in b'\wedge i\succ i\wedge\mathit P(i')])
\end{align*}

\textit{airra dringgi kofi} is true in a context $ c $ iff all branches in the modal base that conform best with some ordering source (in $ c $, likely a teleological background, consisting of the speaker's goals) contain some index in the future of utterance time at which the speaker drinks coffee.\marginnote{maybe an issue here: $ i' $ isn't a successor of the coffee index, rather a posterior non-successor...}

\a \textbf{Meaning of \textit{bambai} \& substitution of  (\getfullref{app-deriv.p}-$\boldsymbol{i'\ (=i_\kappa)}$ for $ \boldsymbol{i_r} $}
$$\begin{multlined}
	\denote{\textit{bambai}}=\lambda\textit{P}.\exists b
	\big[b\in\underset{o(w)}{\textsc{best}}\big(\cap m(i*)\big)\wedge \exists ^bi'[i'\succ i_r\wedge P(i')\wedge\mu(i_r,i')\leq s_c]\big]\end{multlined}$$
$$\begin{multlined}	\denote[c]{\textit{bambai}}=\lambda\textit{P}.\exists b
	\big[b\in\underset{o(w)}{\textsc{best}}\big(\cap m(i*)\big)\wedge \exists ^bi'[i'\succ i_\kappa\wedge P(i')\wedge\mu(i_\kappa,i')\leq s_c]\big]
%	\denote[c]{\textit{bambai mi gurrumuk}}=&\lambda m\lambda o.\exists w'[w'\in\textbf{best}_o(m,t,w)\\&\wedge\textsc{subseq}(\textsc{pass.out},t_r,w')]	
\end{multlined}$$
As in (\getfullref{meta.r}), the ``reference time'' $i_r$ is assigned to the existentially-bound index $ i' $ from (\getref{app-deriv.p}) --- here notated as $ i_\kappa $ (coffee-drinking time).
\a\textbf{Meaning of the second clause}\deftaglabel{q}

$ \denote[c]{\textit{mi gurrumuk}} = \lambda i.\textsc{pass.out}(i)$%mcom{is it kosher just to say that these variables get existentially closed at the end of the derivation (that is what the SSQ condition is meant to assert. Or would it be better to not lambda-bind those variables)}

Temporal abstract \textit{mi gurrumuk} denotes a set of indices at which the speaker passes out.



\a \textbf{ (\getref{app-deriv.q}) saturates \textit{bambai}'s \textit{P} argument; temporal abstract is existentially bound}
\begin{align*}
	&\begin{aligned}
	\denote[c]{\textit{bambai}}(\mathrm{\getref{app-deriv.q}}^c )&=\lambda\textit{P}.\exists b	\big[b\in\underset{o(w)}{\textsc{best}}\big(\cap m(i*)\big)\\&\wedge \exists^bi'[i'\succ i_\kappa\wedge P(i')\wedge\mu(i_\kappa,i')\leq s_c]\big]\big(\lambda i.\textsc{i.pass.out}(i)\big)\end{aligned}\\
&\begin{aligned}\denote[c]{\textit{bambai \textup{(\getref{app-deriv.q})}}}&=\exists b\big[b\in\underset{\mathfrak s(w)}{\textsc{best}}(\cap\!\approx^+_{i*})\wedge \exists^bi'[i'\succ i_\kappa\wedge\textsc{i.pass.out}(i')\wedge\mu(i_\kappa,i')\leq s_c]\big]\\
	&=\exists b\big[b\in\underset{\mathfrak s(w)}{\textsc{best}}(\cap\!\approx^+_{i*})\wedge \exists^bi'[i'\succ i_\kappa\wedge \textsc{subseq}\big(\textsc{i.pass.out}(i'),i_\kappa\big)\big]\end{aligned}
\end{align*}




%\a \textbf{Spelling out the \textsc{subsequential instantiation} relation (cf. \getref{ssqIsem})}
%\begin{multline*}\denote[c]{\textit{bambai mi gurrumuk}}=
%	\exists w'\big[w'\in\textbf{best}_{\mathcal S}(m_{\text{meta}},t\!*,w\!*)\\\wedge\exists t'[\textsc{pass.out}(t')(w')\wedge\mu(t_r,t')\leq s_c]\big]\end{multline*}
%\reversemarginpar
The \textsc{subseq} component of \textit{bambai}'s meaning asserts that • the speaker's \textsc{passing out} obtains at some index $ (i') $ preceded by a contextually-retrieved reference time $ i_\kappa $ \textsc{drink.coffee} and • the temporal distance between those two times is below some contextual standard (``soonness'').\marginnote{ what to put in an index and what to lambda-bind and what if any diff preds this makes. what's clear is that $ \box\lozenge t\neq t_r $}[-1in]


\xe


In the context of (\getref{app-deriv}), $ i*\prec i_\kappa\prec i' $. Given that $ i_\kappa $ (and therefore $ i' $) is in the \textbf{future of speech time}, the modal base $ \approx_{i*} $ is \textbf{diverse with respect to the \textsc{subseq} property} --- that is: $ \textsc{subseq}\big([\lambda i'.\textsc{pass.out}(i')],t_\kappa\big) $ is \textbf{not presumed settled at $ \boldsymbol{i*} $} (compare fig. \ref{ind-BT}.)


On this analysis, then, the crucial property that distinguishes the pure (actualised) subsequential reading from the apprehensional one is that the property described by the prejacent is presumed \textbf{settled at $ \boldsymbol{i*}$} (or alternatively, by $ t* $ in $ w* $.) In all historical alternatives to the evaluation world, the event described by \textsc{make.lunch} in $ c_{(\getref{ssq-deriv})}$ holds at $ i' $. Conversely, in (\getref{app-deriv}), the context ($ c_{(\getref{app-deriv})} $) \textbf{fails to satisfy} settledness for \textsc{pass.out} because the relation between modal base and predicate here satisfies the \textit{diversity condition} --- that is, there are metaphysical alternatives branching from $ i* $ which booth verify and falsify \textsc{pass.out}$ (i') $ \citep[\textit{cf.}][83]{Condoravdi2002}:

\pex \textbf{Diversity of the common ground at $ i* $ w/r/t prejacent in (\getref{app-deriv})}\deftagex{divcon}

%$ \exists w',w''\in\cup\simeq_{t*} w*:\textsc{at}\big((t*,\infty],w',\textsc{pass.out})\wedge\neg \textsc{at}\big((t*,\infty],w'',\textsc{pass.out}) $

$$ \begin{multlined}\exists b\in\cap\textit{cg}\wedge\exists b',b''[b,b''\in\underset{\mathfrak s(^bi*)}{\textsc{best}}(\cap\!\approx^+_{^bi*})\\\wedge\textsc{subseq}\big(\textsc{i.pass.out}(^{b'}i'),{}^{b'}i_\kappa\big)\wedge\neg\textsc{subseq}\big(\textsc{pass.out}(^{b''}i''),{}^{b''}i_\kappa\big)]\end{multlined}$$ 

\marginnote{are the types here ok/is that a licit intersection?}There are metaphysical alternatives branching from $ i $ where the event described by the prejacent to \textit{bambai} in (\getref{app-deriv}) holds and others where it doesn't hold.
\xe

\noindent Finally, following the discussion and interpretation conventions discussed in \S~\ref{bambai.subord}, the accommodation of an antecedent (the ``apprehension-causing situation'') is intersected with the modal base --- that is, it is from that subset of metaphysical branching futures to $ i* $ \ul{in which the speaker doesn't have coffee} that $ \underset{\mathfrak s(i*)}{\textsc{best}} $ selects a domain to be quantified over.% (and diverse w/r/t the speaker's passing out at $ i' $).

\pex[exno=\getref{app-deriv}]\a[label=f]\textbf{Modal subordination}\deftaglabel{ms}


$$\begin{multlined} \denote[c]{\textit{bambai mi gurrumuk}} = \exists b\big[b\in\underset{\mathfrak s(w)}{\textsc{best}}(\cap\llparenthesis\approx_{i*}\cup\overline{\denote{\textit{ai dringgi kofi}}(i_\kappa)}\rrparenthesis)\\\wedge \exists^bi'[i'\succ i_\kappa\wedge \textsc{subseq}\big(\textsc{i.pass.out}(i'),i_\kappa\big)\big]\end{multlined}$$


The modal base is intersected with a (negated) proposition derived from the discourse context. \textit{bambai} signals that \textit{mi gurrumuk} is \textbf{modally subordinate} to the proposition \textit{ai dringgi kofi} `I drink coffee (at $ i_\kappa $).
\xe

\noindent The meaning of the sentence (\getref{app-deriv}), then, is the conjunction of (\getfullref{app-deriv.p}) and (\getref{app-deriv}f). The ``dynamic'' interpretive conventions (\textit{i.e.}, the update of $ c $) are clearly vital in terms of retrieving the relevant parameters of interpretation and the subordinative relation between the propositions in a precautioning-apprehensional (\textit{p bambai q}) usage of \textit{bambai}.

\subsection{The semantics of a counterfactual apprehensional}

Subjunctive/counterfactual uses (\textit{e.g.}, ex. (\getref{sjv}) or table \ref{indices}) are assumed to be derivable in much the same way as above. That is, the modal reading emerges as a consequence of a (conventional) implicature that the relation between the common ground and the \textsc{subseq} relation meets the diversity condition/is presumed \textbf{un}settled.\footnote{A precondition for diversity to be satsfied is that ``the common ground must be compatible with their being some past time at which [the truth of the prejacent is unsettled]'' \citep[85]{Condoravdi2002}.} A complete derivation is not provided, although truth conditions can be composed for (\getref{sjv}, repeated below as \getref{sjv-deriv}) drawing on standard treatments of counterfactuals. That is a nonrealistic modal base where alternative branches are ordered by their similarity to $ i* $---\textit{i.e.}, a \textit{totally realistic ordering source} --- \citealp[\textit{cf.}][]{Kratzer1981a,Kratzer2012,Lewis1973,Lewis1981} a.o.)

Described previously, in these ``subjunctive'' uses, \textit{bambai} marks a counterfactual apprehensional proposition. In (\getref{sjv-deriv}), the subject may have fallen asleep subseqently to a (nonrealistic/counterfactual) \textbf{non}instantiation of the coffee-drinking event.

The \textit{counterfactual} \textit{bambai} construction is similar to the \textit{subsequential} use insofar as the reference time and antecedent upon which \textit{bambai} is anaphoric are past marked $ (i_r\prec i*) $. Crucially though, as in other apprehensional uses, the common ground is nonveridical/\textbf{diverse} with respect to \textit{bambai}'s prejacent. \textit{Bambai}'s diverse quantificational domain is represented by the shaded region in Figure \ref{sbjv-BT}.





%\pex \textbf{Settledness of the prejacent.}
%\a The prejacent to \textit{bambai} in \getref{ssq-deriv} is settled fact of $ w* $ by $ t_r $ --- that is, it holds at all historical alternantiv
%%\begin{multline*}\forall w^\prime\Big[w^\prime\in cg,\forall w^{\prime\prime}\big[w^\prime\simeq_{t_r}w^{\prime\prime}\to\\ AT\big([t_r,\infty),w^\prime,\textsc{make.lunch}\big)\leftrightarrow AT\big([t_r,\infty),w^{\prime\prime},\textsc{make.lunch}\big)\big]\Big]\end{multline*}
%
%\a 
%
%\xe




%-- diversity condition as informing the omniscience one? hist alts retrieved at some t?

%()-- insubordination and the adverbial uses? MAybe in the antecedent section)

%\section{Use conditions}
%\marginnote{Verstr06 for a ref on why only in modal contexts att. UCs ``emerge''}


\begin{figure}[h]\centering
	\caption{A possible representation of $ \cap\approx_{i_0} $: a ``subtree'' of $ \mathfrak T $.\\[.25em]\textit{\textbf{shaded portion. }}$\underset{\{p\mid i*\in p\}}{\textsc{best}}(\cap\neg\kappa)$\\
		Multiple accessible branches (possible developing counterfactuals) succeeding $ i_0$ (the greatest lower $ \prec $-bound of $ i* $ and $ i' $.)}
		\label{sbjv-BT}
	
	\begin{tikzpicture}
		[scale=2,level distance=9mm,
		every node/.style={fill=black,circle,inner sep=1.5pt},
		level 1/.style={sibling distance=16mm},
		level 2/.style={sibling distance=4mm},
		level 3/.style={sibling distance=3mm},
		level 4/.style={sibling distance=1.5mm},
		edge from parent/.style={draw}]
		\node[label=above:{$ i_0 $}] at (0,0) {} [grow=right]
		child {node[label=below:$ i_r' $] {} edge from parent[double,thick,dotted] 
				child {node[label=below:$ i' $] {} edge from parent[thick,dotted] 
					child {node[label=right:{\scriptsize$ b_1 $}] {} edge from parent[thin, tips ,->] }
					child[sibling distance=2mm] {node[label=right:{\scriptsize$ b_2 $}] {} edge from parent[thin, tips ,->] } }
				child {node {} edge from parent[thick,dotted] 
					child[sibling distance=2mm] {node[label=right:{\scriptsize$ b_3 $}] {} edge from parent[thin, tips ,->]}
					child {node[label=right:{\scriptsize$ b_4 $}] {} edge from parent[thin, tips ,->] }}}
		child {node[yshift=-16mm,label=above:$ i_r $] {} edge from parent[thick, double]
			child {node [style={fill=red},label=above:$ \boldsymbol{i*} $] {}  edge from parent[thick, double]	
				child[sibling distance=4mm] {node {} edge from parent[densely dashed] 
					child[sibling distance=3mm] {node[label=right:{\scriptsize$ b_1' $}] {} edge from parent[thin, tips ,->] }
					child[sibling distance=2mm] {node[label=right:{\scriptsize$ b_2' $}] {} edge from parent[thin, tips ,->] } }
				child {node {} edge from parent[densely dashed]
					child[sibling distance=2mm] {node[label=right:{\scriptsize$ b_3' $}] {} edge from parent[thin, tips ,->]}
					child[sibling distance=3mm] {node[label=right:{\scriptsize$ b'_4 $}] {} edge from parent[thin, tips ,->] }}}};
		\fill[very nearly transparent,rounded corners=2ex] (1,-1.35) -- (.3,-.7) -- (.4,-.4) -- (3,-.3) -- (3,-1.4) ;
\end{tikzpicture}\end{figure}
 
\pex\deftagex{sjv-deriv}\begingl
\gla ai\textdblhyphen{}bin dringgi kofi nairram \textbf{bambai} ai bina silip\textasciitilde silip-bat//
\glb 1s\textdblhyphen{\sc pst} drink coffee night \textit{\textbf{bambai}} 1s {\sc pst:irr} sleep{\sc\textasciitilde dur-ipfv} {\sc loc} work//
\glft`I had coffee last night \textbf{otherwise} I might have slept [at work.]' \trailingcitation{[AJ~23022017]}//\endgl
\a \textbf{The syntactic antecedent} ($ \kappa $ stands for the predicate `\textsc{i drink coffee}')

$ \denote[c]{\textit{aibin dringgi kofi nairram}}=\exists i'[i'\prec i\!*\wedge i'\in\mathit{last~night^c}\wedge\kappa(i')] $


\ul{That is}, \textsc{i drink coffee} holds some index $ i' $ preceding speech time and contained within the interval denoted by \textsl{last night}.

\a \textbf{The prejacent}\deftaglabel{q}

\textit{bina} `\gls{pst}:\gls{irr}' is taken to be a composite auxiliary: in effect a modal with back-shifted temporal perspective (compare treatments of English \textit{would}.)\footnote{This observation, supported by a number of synchronic distributional facts about the Kriol IP has diachronic origins, see \citet[45]{Phillips2011} for a discussion of evidence that \textit{bina} is the result of fusion of \textit{bin} and \textit{wandi} `\textsc{desiderative'} $ < $ \texttt{\textit{AEng}}. semimodal `wanna.' According to \citeauthor{Verstraete2006}, ``formally composite'' counterfactuals are frequently occurring in Australian languages \citeyearpar[72]{Verstraete2006}.} Compare with the present (\gls{npst}) perspective reading derived in (\getfullref{app-deriv.p}).

Let \textit{\textsc{s}} stands for the predicate `\textsc{i be sleeping (at work)}.'

$$\begin{multlined} \denote[c]{\textit{ai bina silipsilipbat}}=\lambda i'.i'\prec i\!*\wedge\forall b\big[b\in\underset{\mathfrak{s}(i')}{\textsc{best}}(\cap \underset{\textsc{circ}}{m(i')})\\\to\exists{}^bi''[i''\succ i'\wedge\textit{\textsc{s}}(i'')]\big] \end{multlined}$$

\ul{That is}, along all branches best conforming with circumstances/expectations at some past index $ i' $, \textsc{i be sleeping} holds at some index $ i'' $ that is a successor of $ i' $.

\a \textbf{Application to \textit{bambai}}

Here, $ \underset{i'}{\mathrel\square}\textit{\textsc{s}}(i'') $ will be used to abbreviate the truth translation of the prejacent given in (\getref{sjv-deriv.q}) above.

$\begin{multlined} \denote[c]{\textit{bambai ai bina silipsilipbat}}=\exists b\big[b\in\underset{\mathfrak s(i_0)}{\textsc{best}}(\cap\llparenthesis\approx_{i_0}\cup\{b'\mid \kappa(i_\kappa)\notin b\}\rrparenthesis)\\\wedge\exists^bi'[i'\succsim i_\kappa
	\wedge \textsc{subseq}\big(\underset{i'}{\mathrel\square}\textit{\textsc{s}}(i') ,i_\kappa\big)\big]\end{multlined}$

\ul{That is}, there's some branch $ b $ which was a metaphysical alternative of $ i_0 $ along which the speaker didn't have coffee at $ i_\kappa $ ($ \kappa(i_\kappa) $). In $ b $, there's an index $ i' $, posterior to $ i_\kappa $ at which \textsc{s} holds.
\marginnote{serious composition problems here. probably subseq can harmlessly be changed to specify that $ i'\succsim i_r $ rather than $ \succ $ though.}
\xe

\subsection{The ``epistemic apprehensive'' use}

The discussion above has shown how the core meaning of \textit{bambai} involves a predication of a \textsc{subseq} relation between a predicate and a reference interval, where predictive force/apprehensionality emerge iff the predicate's instantiation is presumed unsettled. From this standpoint, apparently epistemic uses like (\ref{bos}, \textit{p.~\pageref{bos}}), repeated here as (\getref{bos.rpt}) are perhaps surprising.


\ex\deftagex{bos.rpt}\begingl\glpreamble\textup{\textbf{Context:} Speaker is at home to avoid running into her boss. There's a knock at the door; she says to her sister:}//
\gla Gardi! \textbf{Bambai} im main bos iya la det dowa rait~na//
\glb Agh \textit{\textbf{bambai}} 3s my boss here \textsc{loc} the door right~now//
\glft`Oh no! That could be my boss at the door.'\hfill[AJ~02052020]//\endgl
\xe

This type of use is not reported elsewhere and these is their acceptability status remains to be confirmed, however the emergence of an apprehensive reading even in a context where the predicate (\textit{i.e.}, the speaker's boss's arrival at the door) is \textit{presumed settled} is perhaps compatible with approaches to future meaning suggested by \citeauthor{Bennett} (\citeyear[100]{Bennett}/1978), \textit{sc.} that it could be(come) known (in the future) that AJ's boss is at the door (now.)

This proposal, which represents a plausible way of extending the analysis presented here, to ostensibly epistemic uses of \textit{bambai} is not further examined here.



\subsection{Possibly pessimistic}

A surprising consequence of the above proposal is the bifurcation of uses of \textit{bambai} into subsequential (interpreted as purely temporal) and apprehensional readings. This section has predominantly been concerned with the emergence of modal (possibility) readings from a temporal frame adverbial. In \S\S~\ref{bambai.dia}--\ref{bambai.expr}, we investigated the diachronic emergence and synchronic status of \textit{bambai}'s speaker-attitude/expressive character. This component (\textit{viz.} that \textit{bambai} expresses that the Speaker is apprehensive about or somehow disfavours the instantiation of the prejacent) is modelled as a conventional implicature.

We have seen how a branching-time semantics provides insights into how a single meaning can capture \textit{bambai}'s modal behaviour in contexts where the instantiation of the prejacent is presumed unsettled --- \textit{sc.} by modelling \textit{bambai} as a quantifier over metaphysical alternatives. But we have had nothing to say about why the use-conditional component ``emerges'' only (and exactly) in this set of contexts.

Here, there are again clues from the diachronic account provided above. As discussed in \S~\ref{bambai.dia}, both characterising components \textit{apprehensionality} (its modal and its expressive character) are taken to have developed simultaneously in view of the conventionalisation of implicatures emerging in admonitory contexts. Given that these admonitions obligatorily concern eventualities which are presumed unsettled, the associated expressive content is attached to these ``irrealis'' uses of \textit{bambai}, presumably extending into counterfactual uses via this abductive meaning change process. In a perhaps related observation, \citeauthor{Verstraete2006} suggests that subordinate purposive and apprehensional clauses can be conceived of as unsettled given that the doxastic state of the \textit{subject} (rather than speaker) is diverse with respect to the states of affairs they describe (``non-actualized and inherently unknowable from the agent's perspective'' \citeyear[71]{Verstraete2006}).


 From a functionalist perspective, this association in unsurprising, given that speaker (or other agent's) attitude is likely to be more discourse relevant when discussing a potential or a hypothetical state of affairs (\textit{i.e.}, describing an eventuality without committing to its truth, \citealp[see also][74-76]{Verstraete2006}.)

\subsection{Apprehensionalisation and the synchronic system}\label{subseq-sync}

In this chapter, I have claimed that the emergence of \textsc{apprehensional} readings of \textit{bambai} is predictable in context: \textit{i.e}, apprehensionality ``emerges'' when \textit{bambai}'s prejacent is not presumed settled.

\citet{Angelo2016} present a number of examples of \textit{bambai} used to modify predications about unsettled states of affairs. Notably, these uses are virtually always constrained to clause-final occurrences of \textit{bambai} and with distinct prosodic properties.\footnote{Recalling the mention of TFA \textit{baimbai}'s grammaticalisation in Tok Pisin, \citet{Romaine1995} distinguishes clause-initial/connectiv4e uses of \textit{baimbai} from preverbal \textit{bai} `\gls{fut}'
\citep[see also][271]{Bybee1994}.} In these cases, \textit{bambai} likely performs a related narrative cohesion function rather than behaving a (discourse anaphoric) modifier function as described here. Dutch \textit{straks} displays similar restrictions (\getref{straks1}). Negative judgments in (\getfullref{modsub.kofi.pot}) and elsewhere furnish further evidence of this complementary distribution. Otherwise, unsettled predications of (mere temporal) subsequentiality are encoded with other TFAs, including \textit{dregli} $ < $ `directly' or \textit{streidaway} $ < $ `straightaway.' An example is given in (\nextx).

\pex\begingl\gla  Wal deibin larramgo wi braja Timathi fri brom det jeil, en if im kaman langa mi \textbf{dregli}, wal minbala garra kaman en luk yumob.//
\glft `So they've let our brother Timothy out of jail. If he comes to me in time, then we'll come to see youse.'\trailingcitation{[KB~Hibrus 13:20]}//\endgl
%\a\begingl\gla Bat if Jisas Krais det Bos blanga wi wandim mi blanga kaman deya, wal ai garra kaman deya dregli. En ai garra faindat wanim detlot praudbala pipul toktok...//\xe
\xe

Above, apprehensionality is effectively understood as an epiphenomenon of a implicature that subsequential predications have predictive force iff they represent an unsettled property. Whereas this implicature is short-circuited ($ \doteqdot $ conventionalised) in the case of \textit{bambai}, it is suspended in the context of other (less frequent) subsequential TFAs (compare the similar, well-documented phenomenon in the (indirect) speech act literature, \textit{e.g.}, \citealt[29-31]{Horn1984} and \citealt{Morgan1978}.)


%\marginnote{no account given of apparently epistemic readings like (\getref{bos}) -- \textit{p.}44}
\section{Conclusion}\label{bambai.concl}
%%%%%%%%% REWRITE CONCLUSION

Part \textbf{\ref{bambai}} of this dissertation has proposed a formal account for the emergence of apprehensional epistemic markers from temporal frame adverbs, based on the central descriptive observation of Australian Kriol \textit{bambai} made in \cite{Angelo2016}. A meaning change trajectory documented in other literature \citep{Kuteva2019a,Kuteva2019,Angelo2018}, this analysis shows the potential of formal semantic machinery for better understanding the conceptual mechansims that underpin  meaning change (in the spirit of much the emergent tradition appraised in \citealt{Deo2015}) as applied to the modal domain. %Further work may additionally extend the formal treatment of the expressive component of apprehensional (and other apparently use-conditional) items.


These three chapters have attempted to elucidate the mechanisms through which temporal frame adverbs that originally encoded a relation of temporal sequency come to encode causality, possibility and speaker apprehension by way of semantic reanalysis performed by language users, driven by the generalisation, conventionalisation and semanticisation of conversational implicatures. The existence of this ``pathway'' of grammaticalisation provides further evidence of the conceptual unity of these linguistic categories and sheds light on the encoding of (and relationship between) temporal and modal expression in human language. Of particular note is the salient role played by (presumptions of) ``settledness'' (\textit{cf.} \citealp{Condoravdi2002,Kaufmann2005} a.o.) in adjudicating the available readings of relative temporal operators (here exemplified in subsequential TFAs.) That is, the apparent polysemy of \textit{bambai} reported by \citet{Angelo2016} can be unified by assuming that this item uniformly quantifies over accessible metaphysical  alternatives and asserts the instantiation of its prejacent in one such alternative.

As shown, the apprehensional reading of \textit{bambai q} ``emerges'' when that set of metaphysical alternatives is understood to be diverse with respect to the instantiation of the eventuality described by \textit{q}. A \textsc{branching time} semantics for temporal and modal operators perspicuously captures this property of metaphysical alternatives: namely the presumed settledness of a given index's unique past in contradistinction to branching future and counterfactual possibilities. Reasoning about settledness -- and the proper interpretation of \textit{bambai} sentences -- crucially involves the retrieval of particular referents (temporal/propositional) from the broader discourse context, whether or not these are syntactically overt. On the basis of this, \textit{bambai} is understood uniformly as a temporomodal operator, triggering modal (but \textit{not} syntactic) subordination of its prejacent: a finding that can likely be applied to related devices in other languages (\textit{e.g.}, apprehensionals and purposives in addition to other discourse anaphors.)

Further, the apparent cross-linguistic relationship between subsequentiality  and the semanticisation of apprehensional use-conditions (\textit{i.e.}, the generalisation of implicatures about speaker attitude previously associated with ``admonitory'' discourse contexts) likely has implications for our understanding of the development of linguistic markers which express speaker affect and the relation of these \textsc{subjective} experiences to predication about non-actual states of affairs.

%An open issue that demands further consideration is that of better understanding the relation between the proposition on which the \textit{bambai} clause is anaphoric and which is interpreted as the restrictor of the modal base in apprehensional contexts and the antecedent clause to which it is syntactically linked. A satisfying answer to this question likely lies at the semantics-pragmatics interface. A successful analysis may have ranging implications for understanding the interplay of factors that contribute to the proper interpretation of discourse anaphors.


